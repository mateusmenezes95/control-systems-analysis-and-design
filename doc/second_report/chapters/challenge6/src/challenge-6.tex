\section{Desafio I - Controle Multivariável - Espaço de Estados}

\subsection{Motivação}

Com o avanço da tecnologia e consequentemente com avanço das aplicações e
técnicas de controle, os Engenheiros e Engenheiras de Controle passaram a ter
uma gama de técnicas para resolução de problemas de controle. Uma delas e que é
de extrema importância no campo de controle moderno, é a modelagem de processos
por uma realização em espaço de estados. A realização em espaço de estados se
aproveita adicionalmente da teoria de Álgebra Linear para solucionar diversos
problemas de controle. Se destaca no conjunto de problemas o uso desta técnica
para lidar com controle de processos multivariáveis, que é tema deste desafio.
Não obstante, a realização em espaço de estados permite modelar observadores de
estados que são de extrema importância em problemas em que a dinâmica interna do
processo não pode ser facilmente medida ou até mesmo podem ser utilizados para
diminuir o custo com a aquisição de sensores em processos que ainda sim podem
ter sua dinâmica interna mensurada. Diante disso, é imprescindível que
profissionais que atuam na resolução de problemas de controle tenham
conhecimento das técnica de controle por realização em espaço de estados.  

\subsection{Simulações realizadas}

Para aplicação das técnicas de controle via realização em espaço de estados, foi
utilizado o modelo multivariável de dois tanques interconectados conforme a
função de transferência $\mathbf{G}(s)$ dado pela Equação
\ref{eq:matriz-de-fts}. A partir da conversão deste modelo para a representação
em espaço de estados em tempo continuo e discreto, as seguintes análises foram
realizadas:

\begin{itemize}
    \item Avaliação das variáveis do espaço de estados considerando o processo
    na sua condição de equilíbrio;
    \item Projeto de um controlador discreto por realimentação de estados para
    regulação dos estados ao valor de equilíbrio desejado;
    \item Projeto de um controlador discreto por realimentação de estados e com
    ação integral para assegurar seguimento de referência e rejeição de
    perturbações constantes do tipo degrau; e
    \item Projeto dos dois controladores anteriores com o uso do observador de
    estados de Luenberger.
\end{itemize}

\begin{equation}
    \label{eq:matriz-de-fts}
    \mathbf{G}(s) =
    \begin{bmatrix}
        \frac{2}{(10s+1)}         & \frac{0,8}{(10s+1)(2s+1)} \\
        \frac{0,6}{(10s+1)(2s+1)} & \frac{2}{(10s+1)}
    \end{bmatrix}
\end{equation}

Como as análise realizadas se basearam em técnicas de controle de sistemas
amostrados, todos os resultados foram obtidos considerando que dinâmica da
planta foi simulada utilizando a aproximação de Euler com um passo de integração
de 0,025s.

Além disso, para melhor compreensão os vetores, linha e coluna foram
representados por letras minúsculas em negrito ($\mathbf{v}$) e as
matrizes por letra maiúscula também em negrito, exemplo $\mathbf{A}$. Para
distinguir as matrizes do tempo continuo das do tempo discreto, optou-se por
usar o til nas matrizes relacionadas ao tempo discreto, como
$\mathbf{\tilde{A}}$ para a matriz de dinâmica em tempo discreto. Já as matrizes
relacionadas a realização aumentada de espaço de estados foram representadas com
uma barra superior, isto é $\mathbf{\bar{A}}$. Uma exceção a essa regra foi a
adoção do símbolo $\mathbf{\bar{0}}$, que significa uma matriz de zeros. Esta
distinção foi para evitar confusão com o vetor nulo definido como $\mathbf{0}$.

\subsection{Resultados obtidos}
\label{sub:resultados-obtidos-desafio6}

\subsubsection{Cálculo do ponto de equilíbrio}
\label{subsub:calculo-do-ponto-de-equilibrio}
Dado que o sistema multivariável é caracterizado por \ref{eq:matriz-de-fts}, a
saída do sistema é $\mathbf{y} = [Y_1(s) \thickspace Y_2(s)]^{T}$ dado o vetor de
sinais de controle $\mathbf{u} = [U_1(s) \thickspace U_2(s)]^{T}$, temos então que
$\mathbf{y} = \mathbf{G}\mathbf{u}$ ou

\begin{equation}
    \label{eq:relacao-entrada-saida-de-g}
    \begin{bmatrix}
        Y_1(s) \\
        Y_2(s)
    \end{bmatrix}
    =
    \begin{bmatrix}
        \frac{2}{(10s+1)}         & \frac{0,8}{(10s+1)(2s+1)} \\
        \frac{0,6}{(10s+1)(2s+1)} & \frac{2}{(10s+1)}
    \end{bmatrix}
    \begin{bmatrix}
        U_1(s) \\
        U_2(s)
    \end{bmatrix}.
\end{equation}

Quando aplicado o sinal de controle $\mathbf{u}_{eq} = \lim_{t \rightarrow
        \infty }\mathbf{u}(t)$, a saída do sistema chega ao equilíbrio $\mathbf{y}_{eq}
    = \lim_{t \rightarrow \infty }\mathbf{y}(t)$. Aplicando o teorema do valor
final em \ref{eq:relacao-entrada-saida-de-g} dado uma entrada em degrau para
$\mathbf{u}(t)$, temos que $\mathbf{u}_{eq} = \mathbf{G}(0)^{-1}\mathbf{y}_{eq}$
se $det(\mathbf{G}(0)) \neq 0$. Como,

\begin{equation}
    \label{eq:ganho-estatico-de-g}
    \mathbf{G}(0)
    =
    \begin{bmatrix}
        2,0 & 0,8 \\
        0,6 & 1,5
    \end{bmatrix},
\end{equation}logo, det($\mathbf{G}(0)$) $\neq 0$ e $\mathbf{G}(0)^{-1}$ existe.
Dado a saída de equilíbrio desejada $\mathbf{y}_{eq} = [2 \thickspace 1]^\top$,
o sinal de controle de equilíbrio

\begin{equation}
    \label{eq:vetor-do-controle-de-equilibrio}
    \mathbf{u}_{eq} =
    \begin{bmatrix}
        0,8730 \\
        0,3175
    \end{bmatrix}
\end{equation}

\subsubsection{Espaço de Estados no domínio de tempo continuo}
\label{subsub:espaco-de-estados-no-dominio-de-tempo-continuo}

A utilização da função \textit{ss} do Octave para transformar o modelo
\ref{eq:matriz-de-fts} dado no domínio de Laplace para realização em espaço de
estados

\begin{subequations}
    \label{eq:espaco-de-estados-continuo}
    \begin{align}
        \mathbf{\dot{x}}(t) & = \mathbf{A}\mathbf{x}(t) + \mathbf{B}\mathbf{u}(t)
        \label{eq:derivada-do-vetor-de-estados}                                   \\
        \mathbf{y}(t)       & = \mathbf{C}\mathbf{x}(t) + \mathbf{D}\mathbf{u}(t)
        \label{eq:saida-do-sistema-em-espaco-de-estados}
    \end{align}
\end{subequations} resulta nas matrizes $\mathbf{A}_{4\times 4}$,
$\mathbf{B}_{4\times 2}$ e $\mathbf{C}_{2\times 4}$, cujo elementos estão
definidas em \ref{eq:matrizes-do-espaco-de-estados}. Ressalta-se entretanto, que
$\mathbf{D} = \mathbf{\bar{0}}$ para o sistema deste desafio já que não há
transmissão direta do sinal de controle para a saída. Observa-se das dimensões
destas matrizes, que para o sistema dado existem 4 variáveis de estados.

\begin{subequations}
    \label{eq:matrizes-do-espaco-de-estados}
    \begin{align}
        \mathbf{A} & =
        \begin{bmatrix}
            \label{eq:matriz-a}
            0,0500 & 0       & 0,1164  & 0       \\
            0      & 0,0500  & 0       & 0,1164  \\
            0,7089 & 0       & -0,6500 & 0       \\
            0      & -0,7089 & 0       & -0,6500
        \end{bmatrix},  \\
        \mathbf{B} & =
        \begin{bmatrix}
            \label{eq:matriz-b}
            -0,0849 & -0,2333 \\
            -0,3111 & -0,1131 \\
            0       & 0,3007  \\
            0,4010  & 0
        \end{bmatrix}, \\
        \mathbf{C} & =
        \begin{bmatrix}
            \label{eq:matriz-c}
            0 & 0 & 0      & 0,4988 \\
            0 & 0 & 0,4988 & 0
        \end{bmatrix} 
    \end{align}
\end{subequations}

Com estas matrizes é possível calcular quais são os estados do sistema quando a
saída atinge seu valor de equilíbrio. Quando isto acontece, $\mathbf{\dot{x}}(t)
= \mathbf{0}$ e, portanto, a partir de \ref{eq:derivada-do-vetor-de-estados},

\begin{equation}
    \label{eq:calculo-do-vetor-de-estados-de-equilibrio}
    \mathbf{x}_{eq} = -\mathbf{A}^{-1}\mathbf{B}\mathbf{u}_{eq}.
\end{equation} Vale reforçar que isto é verdadeiro se $\mathbf{A}$ é invertível
ou det($\mathbf{A}$) $\neq 0$. De \ref{eq:matriz-a}, verifica-se que
det($\mathbf{A}$) $\neq 0$. Assim, substituindo
\ref{eq:vetor-do-controle-de-equilibrio}, \ref{eq:matriz-a} e \ref{eq:matriz-b}
em \ref{eq:calculo-do-vetor-de-estados-de-equilibrio}, tem se que

\begin{equation}
    \label{eq:estados-de-equilibrio}
    \mathbf{x}_{eq} =
    \begin{bmatrix}
        x_{eq_{1}} \\
        x_{eq_{2}} \\
        x_{eq_{3}} \\
        x_{eq_{4}}
    \end{bmatrix}
    =
    \begin{bmatrix}
        -1,7038 \\
        -3,1831 \\
        2,0050  \\
        4,0100
    \end{bmatrix}\text{.}
\end{equation}

\subsubsection{Espaço de Estado no domínio de tempo discreto}
\label{subsub:espaco-de-estados-no-dominio-de-tempo-discreto}

A versão em tempo discreto para a representação em espaço de estados definido em
\ref{eq:espaco-de-estados-continuo} é dado da seguinte forma

\begin{subequations}
    \label{eq:espaco-de-estados-discreto}
    \begin{align}
        \mathbf{x}[k+1] & = \mathbf{\tilde{A}}\mathbf{x}[k] + \mathbf{\tilde{B}}\mathbf{u}[k]
        \label{eq:estimativa-do-vetor-de-estados}                                             \\
        \mathbf{y}[k]   & = \mathbf{\tilde{C}}\mathbf{x}[t] + \mathbf{\tilde{D}}\mathbf{u}[k]
        \label{eq:saida-do-sistema-em-espaco-de-estados-discreto}
    \end{align}
\end{subequations} que segundo \citeonline{Chen2006}, a relação entre as
matrizes de tempo continuo e discreto é dado como

\begin{equation*}
    \mathbf{\tilde{A}} = e^{\mathbf{A}T},
    \thickspace
    \mathbf{\tilde{B}} = \int_{0}^{T}e^{\mathbf{A}\tau}d\tau \mathbf{B},
    \thickspace
    \mathbf{\tilde{C}} = \mathbf{C}
    \thickspace \text{e} \thickspace
    \mathbf{\tilde{D}} = \mathbf{D},
\end{equation*} em que $T$ é o período de amostragem do sistema de controle
amostrado. Observa-se portanto que é necessário definir um período de amostragem
adequado para ter a realização do modelo \ref{eq:matriz-de-fts} em espaço de
estados de tempo discreto.

O critério para definição do tempo de amostragem foi igual ao utilizado no
desafio 4 de projeto de controladores no tempo discreto (seção
\ref{sub:simulacoes-realizadas-desafio4}). Entretanto, o modelo em questão é
multivariável. Considerando então, que $\mathbf{W}$ é uma matriz cujo elementos
$w_{ij}$ é a frequência de corte de cada função de transferência $g_{ij}$ de
$\mathbf{G}(s)$, optou-se por aplicar o critério na máxima frequência de corte,
ou seja $w_a = 30\times\max(\mathbf{W})$, em que $w_a$ é a frequência de
amostragem em rad/s. Esta abordagem é conservadora, pois os sistemas cuja a
frequência de corte $w_c < \max(\mathbf{W})$ por si só irão atenuar sinais de
alta frequências, e então estes sinais não contribuirão significativamente em
suas respostas. Ou de forma matemática, se $w_a = 30\times\max(\mathbf{W})$,
logo o critério de $30\times$ será válido para as demais plantas. Assim,
obteve-se a matriz $\mathbf{W}$ conforme
\ref{eq:matriz-das-frequencias-de-corte}. Percebe-se que $w_{21} = 0$ pois
$|g_{21}(0)| < -3$dB e esta atribuição não compromete o critério da
definição do período de amostragem.

\begin{equation}
    \label{eq:matriz-das-frequencias-de-corte}
    \mathbf{W} =
    \begin{bmatrix}
        0,2646 & 0,0516 \\
        0      & 0,1871
    \end{bmatrix}
\end{equation}

Portanto, $\max(\mathbf{W}) = 0,2646$ rad/s, o que resultou, após
arrendondamento, num período de amostragem $T_a = 0,75$s.

Novamente com o auxílio da função \textit{ss} do Octave, mas agora convertendo
as matrizes para realização em espaço de estados de tempo discreto utilizando a
função \textit{c2d} passando como argumento período de amostragem definido,
chegou-se nas seguintes matrizes:

\begin{subequations}
    \label{eq:matrizes-do-espaco-de-estados-discretizadas}
    \begin{align}
        \mathbf{\tilde{A}} & =
        \begin{bmatrix}
            \label{eq:matriz-til-a}
            1,0179 & 0       & 0,069961 & 0      \\
            0       & 1,0179 & 0      & 0,069961 \\
            -0,42613 & 0       & 0,59712 & 0      \\
            0       & -0,42613  & 0     & 0,59712
        \end{bmatrix}\text{,} \\
        \mathbf{\tilde{B}} & =
        \begin{bmatrix}
            \label{eq:matriz-til-b}
            -0,0644 & -0,1686  \\
            -0,2248 & -0,0859  \\
            0,0146  & 0,2173   \\
            0,2897  & 0,0195
        \end{bmatrix}\text{,} \\
        \mathbf{\tilde{C}} &= \mathbf{C} \\
        \mathbf{\tilde{D}} &= \mathbf{D}
    \end{align}
\end{subequations}

Com a definição destas matrizes é possível definir os estados de equilíbrio dado
o vetor do sinal de controle em equilíbrio
\ref{eq:vetor-do-controle-de-equilibrio}. Salienta-se entretanto que para o
tempo discreto, o estado de equilíbrio é alcançado quando o estado atual é igual
ao estado anterior, ou de forma algébrica, a partir de
\ref{eq:estimativa-do-vetor-de-estados},

\begin{equation}
    \label{eq:calculo-dos-estados-de-equilibrio-em-tempo-discreto}
    \mathbf{\tilde{x}}_{eq} = \mathbf{\tilde{A}}\mathbf{\tilde{x}}_{eq} + \mathbf{\tilde{B}}\mathbf{u}_{eq}
    \Rightarrow
    \mathbf{\tilde{x}}_{eq} = (\mathbf{I}-\mathbf{\tilde{A}})^{-1}\mathbf{\tilde{B}}\mathbf{u}_{eq}.
\end{equation} Portanto, substituindo \ref{eq:vetor-do-controle-de-equilibrio},
\ref{eq:matriz-til-a} e \ref{eq:matriz-til-b} em
\ref{eq:calculo-dos-estados-de-equilibrio-em-tempo-discreto}, chegou-se ao mesmo
resultado de tempo contínuo conforme \ref{eq:estados-de-equilibrio}. O mesmo
aconteceu para a saída $\mathbf{\tilde{y}}_{eq} =
\mathbf{\tilde{C}}\mathbf{\tilde{x}}_{eq}$, o que é coerente já que
$\mathbf{\tilde{x}}_{eq} = \mathbf{x}_{eq}$ e $\mathbf{\tilde{C}} = \mathbf{C}$.

\subsubsection{Regulação em torno do ponto de equilíbrio}
\label{subsub:regulacao-em-torno-do-ponto-de-equilibrio}

As seções anteriores abordaram a análise do estado de equilíbrio assumindo que o
sistema já estava na sua condição de equilíbrio. Entretanto, para levar o sistema
ao equilíbrio é preciso aplicar uma ação de controle de regulação em torno do
ponto de equilíbrio, dada por

\begin{equation}
    \label{eq:acao-de-controle-em-torno-do-ponto-de-equilibrio}
    \mathbf{u}(kT_a) = -\mathbf{\tilde{K}}[\mathbf{x}(kT_a) - \mathbf{x_{eq}}]+\mathbf{u}_{eq}, 
    \thickspace k = 0, 1, 2, ..., n.
\end{equation}

Substituindo \ref{eq:acao-de-controle-em-torno-do-ponto-de-equilibrio} em
\ref{eq:estimativa-do-vetor-de-estados}, já que $u[k] := u(kT_a)$ e $x[k] :=
x(kT_a)$, o novo polinômio característico é dado por
det($z\mathbf{I}-\mathbf{\tilde{A}}-\mathbf{\tilde{B}}\mathbf{\tilde{K}}$) e
portanto os polos do sistema discreto podem ser modificados para um valor
desejável. Entretanto, isto só é válido se o sistema for controlável, isto é, a
matriz de controlabilidade $\mathbf{\tilde{U}}$ dado por
\ref{eq:matriz-de-controlabilidade-de-tempo-discreto} tem todas as linhas
linearmente independentes.

\begin{equation}
    \label{eq:matriz-de-controlabilidade-de-tempo-discreto}
    \mathbf{\tilde{U}} = [
    \mathbf{\tilde{B}}
    \thickspace
    \mathbf{\tilde{A}\tilde{B}} 
    \thickspace
    \mathbf{\tilde{A}}\negthinspace^{2}\mathbf{\tilde{B}}
    \thickspace
    \mathbf{\tilde{A}}\negthinspace^{3}\mathbf{\tilde{B}}]
\end{equation}

A aplicação das matrizes \ref{eq:matriz-til-a} e \ref{eq:matriz-til-b} em
\ref{eq:matriz-de-controlabilidade-de-tempo-discreto} resulta em uma matriz
$\mathbf{\tilde{U}}$ cujo posto é igual 4, que é a mesma ordem do sistema e,
portanto, todas as linhas são linearmente independentes. Assim, todos os
autovalores podem ser definidos escolhendo um $\mathbf{\tilde{K}}$ adequado.

A definição do valor de $\mathbf{\tilde{K}}$ foi realizada através de um método
de alocação de autovalores em espaço de estados. Existem vários métodos na
literatura em que usam uma transformação linear de similaridade para facilitar o
cálculo do ganho $\mathbf{\tilde{K}}$. Alguns destes métodos já são
disponibilizados computacionalmente em funções, como é o caso da função
\textit{place} do Octave/Matlab. Dado então o requisito de que os autovalores de
malha fechada de tempo discreto, ou seja os autovalores da matriz
$\mathbf{\tilde{A} - \tilde{B}\tilde{K}}$ devem ser $\lambda_i =
e^{\frac{-T_a}{4}}$ para $i = 1, 2, 3$ e $4$, em que dado $T_a = 0,75$s,
$\lambda_i = 0,8290$. Utilizando portanto a função \textit{place} tendo como
argumento as matrizes $\mathbf{\tilde{A}}$ e $\mathbf{\tilde{B}}$ e o vetor
$\mathbf{\lambda}$ contendo os autovalores desejáveis $\lambda_i$, foi obtida a
matriz de ganhos

\begin{equation}
    \label{eq:matriz-de-ganhos-discreto}
    \mathbf{\tilde{K}} = 
    \begin{bmatrix}
        1,6632 & -0,4644 & 1,0590 & -0,0451    \\
        0,5947 & 1,4181 &  0,5227 & 1,1671
    \end{bmatrix}.
\end{equation}

O resultado da simulação da realização em espaço de estados dado o
$\mathbf{\tilde{K}}$ encontrado é mostrado na Figura
\ref{fig:resultado-do-regulador-via-realimentacao-de-estados}. Percebe-se nesta
figura que o sistema atinge seu valor de equilíbrio dado por $\mathbf{u}_{eq} =
[0,8730 \thickspace 0,3175]^\top$ e $\mathbf{y}_{eq} = [2 \thickspace 1]^\top$.
Repara-se também a presença de um sobressinal. Como o sistema em questão trata
de dois tanques interconectados, a interação entre suas dinâmicas faz surgir
zeros que alteram de forma significativa a saída do sistema, mesmo com
autovalores reais, o que explica o sobressinal nas respostas $y_1(t)$ e $y_1(t)$
obtidas.

\begin{figure}[!htp]
    \caption{Saídas e sinais de controle do sistema MIMO de tanques
    interconectados utilizando regulação via realimentação de estados.}
    \vspace{-10pt}
    \hspace{-30pt}
    \label{fig:resultado-do-regulador-via-realimentacao-de-estados}
    \begin{minipage}{\linewidth}
        % Title: gl2ps_renderer figure
% Creator: GL2PS 1.4.0, (C) 1999-2017 C. Geuzaine
% For: Octave
% CreationDate: Thu Nov 25 18:14:01 2021
\setlength{\unitlength}{1pt}
\begin{picture}(0,0)
\includegraphics{chapters/challenge6/images/resultado-questao-7-inc}
\end{picture}%
\begin{picture}(400,250)(0,0)
\fontsize{6}{0}
\selectfont\put(52,141.404){\makebox(0,0)[t]{\textcolor[rgb]{0.15,0.15,0.15}{{0}}}}
\fontsize{6}{0}
\selectfont\put(70.4365,141.404){\makebox(0,0)[t]{\textcolor[rgb]{0.15,0.15,0.15}{{20}}}}
\fontsize{6}{0}
\selectfont\put(88.873,141.404){\makebox(0,0)[t]{\textcolor[rgb]{0.15,0.15,0.15}{{40}}}}
\fontsize{6}{0}
\selectfont\put(107.309,141.404){\makebox(0,0)[t]{\textcolor[rgb]{0.15,0.15,0.15}{{60}}}}
\fontsize{6}{0}
\selectfont\put(125.746,141.404){\makebox(0,0)[t]{\textcolor[rgb]{0.15,0.15,0.15}{{80}}}}
\fontsize{6}{0}
\selectfont\put(144.182,141.404){\makebox(0,0)[t]{\textcolor[rgb]{0.15,0.15,0.15}{{100}}}}
\fontsize{6}{0}
\selectfont\put(162.619,141.404){\makebox(0,0)[t]{\textcolor[rgb]{0.15,0.15,0.15}{{120}}}}
\fontsize{6}{0}
\selectfont\put(181.055,141.404){\makebox(0,0)[t]{\textcolor[rgb]{0.15,0.15,0.15}{{140}}}}
\fontsize{6}{0}
\selectfont\put(48.5205,153.193){\makebox(0,0)[r]{\textcolor[rgb]{0.15,0.15,0.15}{{0}}}}
\fontsize{6}{0}
\selectfont\put(48.5205,169.582){\makebox(0,0)[r]{\textcolor[rgb]{0.15,0.15,0.15}{{0.5}}}}
\fontsize{6}{0}
\selectfont\put(48.5205,185.97){\makebox(0,0)[r]{\textcolor[rgb]{0.15,0.15,0.15}{{1}}}}
\fontsize{6}{0}
\selectfont\put(48.5205,202.359){\makebox(0,0)[r]{\textcolor[rgb]{0.15,0.15,0.15}{{1.5}}}}
\fontsize{6}{0}
\selectfont\put(48.5205,218.748){\makebox(0,0)[r]{\textcolor[rgb]{0.15,0.15,0.15}{{2}}}}
\fontsize{7}{0}
\selectfont\put(33.5205,186.325){\rotatebox{90}{\makebox(0,0)[b]{\textcolor[rgb]{0.15,0.15,0.15}{{Saida $y_1(t)$}}}}}
\fontsize{7}{0}
\selectfont\put(116.643,130.404){\makebox(0,0)[t]{\textcolor[rgb]{0.15,0.15,0.15}{{Tempo (s)}}}}
\fontsize{6}{0}
\selectfont\put(232.715,141.404){\makebox(0,0)[t]{\textcolor[rgb]{0.15,0.15,0.15}{{0}}}}
\fontsize{6}{0}
\selectfont\put(251.151,141.404){\makebox(0,0)[t]{\textcolor[rgb]{0.15,0.15,0.15}{{20}}}}
\fontsize{6}{0}
\selectfont\put(269.587,141.404){\makebox(0,0)[t]{\textcolor[rgb]{0.15,0.15,0.15}{{40}}}}
\fontsize{6}{0}
\selectfont\put(288.024,141.404){\makebox(0,0)[t]{\textcolor[rgb]{0.15,0.15,0.15}{{60}}}}
\fontsize{6}{0}
\selectfont\put(306.46,141.404){\makebox(0,0)[t]{\textcolor[rgb]{0.15,0.15,0.15}{{80}}}}
\fontsize{6}{0}
\selectfont\put(324.896,141.404){\makebox(0,0)[t]{\textcolor[rgb]{0.15,0.15,0.15}{{100}}}}
\fontsize{6}{0}
\selectfont\put(343.333,141.404){\makebox(0,0)[t]{\textcolor[rgb]{0.15,0.15,0.15}{{120}}}}
\fontsize{6}{0}
\selectfont\put(361.77,141.404){\makebox(0,0)[t]{\textcolor[rgb]{0.15,0.15,0.15}{{140}}}}
\fontsize{6}{0}
\selectfont\put(229.235,146.637){\makebox(0,0)[r]{\textcolor[rgb]{0.15,0.15,0.15}{{-0.2}}}}
\fontsize{6}{0}
\selectfont\put(229.235,157.638){\makebox(0,0)[r]{\textcolor[rgb]{0.15,0.15,0.15}{{0}}}}
\fontsize{6}{0}
\selectfont\put(229.235,168.638){\makebox(0,0)[r]{\textcolor[rgb]{0.15,0.15,0.15}{{0.2}}}}
\fontsize{6}{0}
\selectfont\put(229.235,179.638){\makebox(0,0)[r]{\textcolor[rgb]{0.15,0.15,0.15}{{0.4}}}}
\fontsize{6}{0}
\selectfont\put(229.235,190.638){\makebox(0,0)[r]{\textcolor[rgb]{0.15,0.15,0.15}{{0.6}}}}
\fontsize{6}{0}
\selectfont\put(229.235,201.639){\makebox(0,0)[r]{\textcolor[rgb]{0.15,0.15,0.15}{{0.8}}}}
\fontsize{6}{0}
\selectfont\put(229.235,212.639){\makebox(0,0)[r]{\textcolor[rgb]{0.15,0.15,0.15}{{1}}}}
\fontsize{6}{0}
\selectfont\put(229.235,223.639){\makebox(0,0)[r]{\textcolor[rgb]{0.15,0.15,0.15}{{1.2}}}}
\fontsize{7}{0}
\selectfont\put(212.235,186.325){\rotatebox{90}{\makebox(0,0)[b]{\textcolor[rgb]{0.15,0.15,0.15}{{Saida $y_2(t)$}}}}}
\fontsize{7}{0}
\selectfont\put(297.357,130.404){\makebox(0,0)[t]{\textcolor[rgb]{0.15,0.15,0.15}{{Tempo (s)}}}}
\fontsize{6}{0}
\selectfont\put(52,22.2671){\makebox(0,0)[t]{\textcolor[rgb]{0.15,0.15,0.15}{{0}}}}
\fontsize{6}{0}
\selectfont\put(70.4365,22.2671){\makebox(0,0)[t]{\textcolor[rgb]{0.15,0.15,0.15}{{20}}}}
\fontsize{6}{0}
\selectfont\put(88.873,22.2671){\makebox(0,0)[t]{\textcolor[rgb]{0.15,0.15,0.15}{{40}}}}
\fontsize{6}{0}
\selectfont\put(107.309,22.2671){\makebox(0,0)[t]{\textcolor[rgb]{0.15,0.15,0.15}{{60}}}}
\fontsize{6}{0}
\selectfont\put(125.746,22.2671){\makebox(0,0)[t]{\textcolor[rgb]{0.15,0.15,0.15}{{80}}}}
\fontsize{6}{0}
\selectfont\put(144.182,22.2671){\makebox(0,0)[t]{\textcolor[rgb]{0.15,0.15,0.15}{{100}}}}
\fontsize{6}{0}
\selectfont\put(162.619,22.2671){\makebox(0,0)[t]{\textcolor[rgb]{0.15,0.15,0.15}{{120}}}}
\fontsize{6}{0}
\selectfont\put(181.055,22.2671){\makebox(0,0)[t]{\textcolor[rgb]{0.15,0.15,0.15}{{140}}}}
\fontsize{6}{0}
\selectfont\put(48.5205,36.0054){\makebox(0,0)[r]{\textcolor[rgb]{0.15,0.15,0.15}{{0}}}}
\fontsize{6}{0}
\selectfont\put(48.5205,57.269){\makebox(0,0)[r]{\textcolor[rgb]{0.15,0.15,0.15}{{0.5}}}}
\fontsize{6}{0}
\selectfont\put(48.5205,78.5322){\makebox(0,0)[r]{\textcolor[rgb]{0.15,0.15,0.15}{{1}}}}
\fontsize{6}{0}
\selectfont\put(48.5205,99.7959){\makebox(0,0)[r]{\textcolor[rgb]{0.15,0.15,0.15}{{1.5}}}}
\fontsize{7}{0}
\selectfont\put(33.5205,67.1875){\rotatebox{90}{\makebox(0,0)[b]{\textcolor[rgb]{0.15,0.15,0.15}{{Sinal de Controle $u_1(t)$}}}}}
\fontsize{7}{0}
\selectfont\put(116.643,11.2671){\makebox(0,0)[t]{\textcolor[rgb]{0.15,0.15,0.15}{{Tempo (s)}}}}
\fontsize{6}{0}
\selectfont\put(232.715,22.2671){\makebox(0,0)[t]{\textcolor[rgb]{0.15,0.15,0.15}{{0}}}}
\fontsize{6}{0}
\selectfont\put(251.151,22.2671){\makebox(0,0)[t]{\textcolor[rgb]{0.15,0.15,0.15}{{20}}}}
\fontsize{6}{0}
\selectfont\put(269.587,22.2671){\makebox(0,0)[t]{\textcolor[rgb]{0.15,0.15,0.15}{{40}}}}
\fontsize{6}{0}
\selectfont\put(288.024,22.2671){\makebox(0,0)[t]{\textcolor[rgb]{0.15,0.15,0.15}{{60}}}}
\fontsize{6}{0}
\selectfont\put(306.46,22.2671){\makebox(0,0)[t]{\textcolor[rgb]{0.15,0.15,0.15}{{80}}}}
\fontsize{6}{0}
\selectfont\put(324.896,22.2671){\makebox(0,0)[t]{\textcolor[rgb]{0.15,0.15,0.15}{{100}}}}
\fontsize{6}{0}
\selectfont\put(343.333,22.2671){\makebox(0,0)[t]{\textcolor[rgb]{0.15,0.15,0.15}{{120}}}}
\fontsize{6}{0}
\selectfont\put(361.77,22.2671){\makebox(0,0)[t]{\textcolor[rgb]{0.15,0.15,0.15}{{140}}}}
\fontsize{6}{0}
\selectfont\put(229.235,27.5){\makebox(0,0)[r]{\textcolor[rgb]{0.15,0.15,0.15}{{-0.2}}}}
\fontsize{6}{0}
\selectfont\put(229.235,42.8911){\makebox(0,0)[r]{\textcolor[rgb]{0.15,0.15,0.15}{{0}}}}
\fontsize{6}{0}
\selectfont\put(229.235,58.2817){\makebox(0,0)[r]{\textcolor[rgb]{0.15,0.15,0.15}{{0.2}}}}
\fontsize{6}{0}
\selectfont\put(229.235,73.6729){\makebox(0,0)[r]{\textcolor[rgb]{0.15,0.15,0.15}{{0.4}}}}
\fontsize{6}{0}
\selectfont\put(229.235,89.064){\makebox(0,0)[r]{\textcolor[rgb]{0.15,0.15,0.15}{{0.6}}}}
\fontsize{6}{0}
\selectfont\put(229.235,104.455){\makebox(0,0)[r]{\textcolor[rgb]{0.15,0.15,0.15}{{0.8}}}}
\fontsize{7}{0}
\selectfont\put(212.235,67.1875){\rotatebox{90}{\makebox(0,0)[b]{\textcolor[rgb]{0.15,0.15,0.15}{{Sinal de Controle $u_2(t)$}}}}}
\fontsize{7}{0}
\selectfont\put(297.357,11.2671){\makebox(0,0)[t]{\textcolor[rgb]{0.15,0.15,0.15}{{Tempo (s)}}}}
\fontsize{6}{0}
\selectfont\put(299.842,238.48){\makebox(0,0)[l]{\textcolor[rgb]{0,0,0}{{equilibrio}}}}
\end{picture}

    \end{minipage}
\end{figure}

\subsubsection{Rejeição de perturbações e seguimento de referência}
\label{subsub:rejeicao-de-perturbacoes-e-seguimento-de-referencia}

Na subseção anterior foi abordado a regulação do sistema MIMO de tanques
interconectados para um ponto de equilíbrio utilizando realimentação de estados.
Embora esta abordagem tenha funcionado de forma teórica, na prática ela pode
levar a comportamentos indesejáveis devidos a erros de modelagem, perturbações
externas e ruídos. Uma das soluções para estes possíveis problemas é a
utilização da ação integral no controle da realimentação de estados, tópico
desta seção.

A ação de integração é alcançada fazendo aparecer no espaço de estados um
dinâmica interna, tal que a realização em espaços de estados de tempo discreto
\ref{eq:espaco-de-estados-discreto} torna-se a versão aumentada

\begin{equation}
    \label{eq:espaco-de-estados-discreto-aumentado}
    \begin{bmatrix}
        \Delta x[k+1]\\ 
        y[k+1]
    \end{bmatrix}
    =
    \begin{bmatrix}
        \mathbf{\tilde{A}} & \mathbf{\bar{0}}\\ 
        \mathbf{\tilde{C}}\mathbf{\tilde{A}} & \mathbf{I}
    \end{bmatrix}
    \begin{bmatrix}
        \Delta x[k]\\ 
        y[k]
    \end{bmatrix}
    +
    \begin{bmatrix}
        \mathbf{\tilde{B}}\\ 
        \mathbf{\tilde{C}}\mathbf{\tilde{A}}
    \end{bmatrix}
    \Delta \mathbf{u}[k]
\end{equation} e então as matrizes aumentadas são reescritas como

\begin{subequations}
    \label{eq:matriz-a-e-b-aumentada}
    \begin{align}
        \mathbf{\bar{A}} &=
        \begin{bmatrix}
            \label{eq:matriz-aumentada-a}
            \mathbf{\tilde{A}} & \mathbf{\bar{0}}\\ 
            \mathbf{\tilde{C}}\mathbf{\tilde{A}} & \mathbf{I}
        \end{bmatrix}   \\
        \mathbf{\bar{B}} &=
        \begin{bmatrix}
            \label{eq:matriz-aumentada-b}
            \mathbf{\tilde{B}}\\ 
            \mathbf{\tilde{C}}\mathbf{\tilde{B}}
        \end{bmatrix}
    \end{align}
\end{subequations} em que para a realização de estados para o problema em
questão a ordem de $\mathbf{\bar{A}}$ é $6 \times 6$ e de $\mathbf{\bar{B}}$ é
$6 \times 2$. Portanto, substituindo \ref{eq:matriz-til-a},
\ref{eq:matriz-til-b} e \ref{eq:matriz-c} em \ref{eq:matriz-aumentada-a} e
\ref{eq:matriz-aumentada-b}, chegou-se as matrizes:

\begin{subequations}
        \begin{equation}
            \label{eq:matriz-aumentanda-a-com-valores}
            \mathbf{\bar{A}} =
            \begin{bmatrix}
                1,0179  &  0       &  0,0700  &  0       & 0       & 0      \\
                0       &  1,0179  &  0       &  0,0700  & 0       & 0      \\
                -0,4261 &  0       &  0,5971  &  0       & 0       & 0      \\
                0       &  -0,4261 &  0       &  0,5971  & 0       & 0      \\
                0       &  -0,2125 &  0       &  0,2978  & 1       & 0      \\
                -0,2125 &  0       &  0,2978  &  0       & 0       & 1
            \end{bmatrix}
        \end{equation}    
        \begin{equation}
            \label{eq:matriz-aumentanda-b-com-valores}
            \mathbf{\bar{B}} =
            \begin{bmatrix}
                -0,0644 & -0,1686 \\
                -0,2248 & -0,0859 \\
                0,0146 &  0,2173  \\
                0,2897 &  0,0195  \\
                0,1445 &  0,0097  \\
                0,0073 &  0,1084
            \end{bmatrix}
        \end{equation}
\end{subequations}

Sabida a realização aumentada em espaço de estados acima, foi
utilizada a ação de controle

\begin{equation}
    \label{eq:acao-de-controle-com-integrador}
    \mathbf{u}[k] = \mathbf{u}[k-1] - \mathbf{K}_x(\mathbf{x}[k]-\mathbf{x}[k-1])+\mathbf{K}_i(\mathbf{y}_r[k]-\mathbf{y}[k])
\end{equation} em que a representação $\mathbf{v}[k] := \mathbf{v}(kT_a)$ e
$\mathbf{v}[k-1] := \mathbf{v}((k-1)T_a)$.

Assim é possível novamente alocar os autovalores da matriz
$\mathbf{\bar{A}}-\mathbf{\bar{B}}\mathbf{\bar{K}}$ sendo $\mathbf{\bar{K}} =
[\mathbf{K}_x \thickspace \mathbf{K}_i]$. Vale salientar, que a ordem do sistema
foi aumentada para $n=6$ e, desta forma, são alocados agora 6 autovalores.
Portanto, utilizando a função \textit{place} do Octave e considerando os 6
autovalores repetidos $\lambda_i = 0,8290$, foram obtidas as matrizes de ganho

\begin{subequations}
    \label{eq:matrizes-de-ganho}
    \begin{equation}
        \label{eq:matriz-de-ganhos-kx}
        \mathbf{K}_x =
        \begin{bmatrix}
            1.073066 & -0.786854 &  0.552769 & -0.022183   \\
            -1.049138 &  0.804800 & -0.029577 &  0.414577
        \end{bmatrix}
    \end{equation}
    \text{e}
    \begin{equation}
        \label{eq:matriz-de-ganhos-ki}
        \mathbf{K}_i =
        \begin{bmatrix}
            0.131655 & -0.070216 \\
            -0.052662 &  0.175540
        \end{bmatrix}\text{.}
    \end{equation}
\end{subequations}

A substituição das matrizes de ganhos \ref{eq:matrizes-de-ganho} na ação de
controle \ref{eq:acao-de-controle-com-integrador} resultou no comportamento do
sistema ilustrado na Figura \ref{fig:resultado-do-regulador-com-integrador}.
Nesta figura foi inserida também o comportamento do sistema sem ação de
integração a título de comparação. Observa-se que o tempo de acomodação para o
controlador com ação integral é significativamente superior a regulação com
ganho estático. Entretanto, para perturbações na saída $\mathbf{q}_{y}(t) =
0,2\mathbf{\mathds{1}}(t-50)$ e na entrada $\mathbf{q}_{u}(t) =
0,2\mathbf{\mathds{1}}(t-100)$, fica evidente a rejeição de perturbação para o
sistema com integrador, o que está de acordo com o Princípio do Modelo Interno.
Por outro lado, nota-se que o regulador com ganho estático não rejeitou ambas
perturbações. De forma conceitual, as perturbações não alteram os estados já que
elas aparecem diretamente na saída e entrada do sistema e não internamente ao
sistema. Desta forma, como o regulador por realimentação de estados utiliza
apenas o vetor de estados para levar o sistema ao ponto de equilíbrio desejável,
as perturbações não são identificadas e, portanto, não podem ser rejeitadas. Em
contrapartida, o regulador com integrador utiliza tantos os estados quanto a
saída do sistema e assim é possível identificar as perturbações e rejeitá-las.
Estas duas afirmações ficam evidentes analisando as equações
\ref{eq:acao-de-controle-em-torno-do-ponto-de-equilibrio} e
\ref{eq:acao-de-controle-com-integrador}.

\begin{figure}[!htp]
    \caption{Comparação dos sinais de controle e saída do sistema em malha
    fechada para seguimento de referência com e sem ação integral.}
    \vspace{-10pt}
    \hspace{-30pt}
    \label{fig:resultado-do-regulador-com-integrador}
    \begin{minipage}{\linewidth}
        % Title: gl2ps_renderer figure
% Creator: GL2PS 1.4.0, (C) 1999-2017 C. Geuzaine
% For: Octave
% CreationDate: Thu Nov 25 17:46:16 2021
\setlength{\unitlength}{1pt}
\begin{picture}(0,0)
\includegraphics{chapters/challenge6/images/resultado-questao-9-inc}
\end{picture}%
\begin{picture}(400,250)(0,0)
\fontsize{6}{0}
\selectfont\put(52,141.404){\makebox(0,0)[t]{\textcolor[rgb]{0.15,0.15,0.15}{{0}}}}
\fontsize{6}{0}
\selectfont\put(70.4365,141.404){\makebox(0,0)[t]{\textcolor[rgb]{0.15,0.15,0.15}{{20}}}}
\fontsize{6}{0}
\selectfont\put(88.873,141.404){\makebox(0,0)[t]{\textcolor[rgb]{0.15,0.15,0.15}{{40}}}}
\fontsize{6}{0}
\selectfont\put(107.309,141.404){\makebox(0,0)[t]{\textcolor[rgb]{0.15,0.15,0.15}{{60}}}}
\fontsize{6}{0}
\selectfont\put(125.746,141.404){\makebox(0,0)[t]{\textcolor[rgb]{0.15,0.15,0.15}{{80}}}}
\fontsize{6}{0}
\selectfont\put(144.182,141.404){\makebox(0,0)[t]{\textcolor[rgb]{0.15,0.15,0.15}{{100}}}}
\fontsize{6}{0}
\selectfont\put(162.619,141.404){\makebox(0,0)[t]{\textcolor[rgb]{0.15,0.15,0.15}{{120}}}}
\fontsize{6}{0}
\selectfont\put(181.055,141.404){\makebox(0,0)[t]{\textcolor[rgb]{0.15,0.15,0.15}{{140}}}}
\fontsize{6}{0}
\selectfont\put(48.5205,151.661){\makebox(0,0)[r]{\textcolor[rgb]{0.15,0.15,0.15}{{0}}}}
\fontsize{6}{0}
\selectfont\put(48.5205,164.221){\makebox(0,0)[r]{\textcolor[rgb]{0.15,0.15,0.15}{{0.5}}}}
\fontsize{6}{0}
\selectfont\put(48.5205,176.78){\makebox(0,0)[r]{\textcolor[rgb]{0.15,0.15,0.15}{{1}}}}
\fontsize{6}{0}
\selectfont\put(48.5205,189.34){\makebox(0,0)[r]{\textcolor[rgb]{0.15,0.15,0.15}{{1.5}}}}
\fontsize{6}{0}
\selectfont\put(48.5205,201.899){\makebox(0,0)[r]{\textcolor[rgb]{0.15,0.15,0.15}{{2}}}}
\fontsize{6}{0}
\selectfont\put(48.5205,214.459){\makebox(0,0)[r]{\textcolor[rgb]{0.15,0.15,0.15}{{2.5}}}}
\fontsize{7}{0}
\selectfont\put(33.5205,186.325){\rotatebox{90}{\makebox(0,0)[b]{\textcolor[rgb]{0.15,0.15,0.15}{{Saida $y_1(t)$}}}}}
\fontsize{7}{0}
\selectfont\put(116.643,130.404){\makebox(0,0)[t]{\textcolor[rgb]{0.15,0.15,0.15}{{Tempo (s)}}}}
\fontsize{6}{0}
\selectfont\put(232.715,141.404){\makebox(0,0)[t]{\textcolor[rgb]{0.15,0.15,0.15}{{0}}}}
\fontsize{6}{0}
\selectfont\put(251.151,141.404){\makebox(0,0)[t]{\textcolor[rgb]{0.15,0.15,0.15}{{20}}}}
\fontsize{6}{0}
\selectfont\put(269.587,141.404){\makebox(0,0)[t]{\textcolor[rgb]{0.15,0.15,0.15}{{40}}}}
\fontsize{6}{0}
\selectfont\put(288.024,141.404){\makebox(0,0)[t]{\textcolor[rgb]{0.15,0.15,0.15}{{60}}}}
\fontsize{6}{0}
\selectfont\put(306.46,141.404){\makebox(0,0)[t]{\textcolor[rgb]{0.15,0.15,0.15}{{80}}}}
\fontsize{6}{0}
\selectfont\put(324.896,141.404){\makebox(0,0)[t]{\textcolor[rgb]{0.15,0.15,0.15}{{100}}}}
\fontsize{6}{0}
\selectfont\put(343.333,141.404){\makebox(0,0)[t]{\textcolor[rgb]{0.15,0.15,0.15}{{120}}}}
\fontsize{6}{0}
\selectfont\put(361.77,141.404){\makebox(0,0)[t]{\textcolor[rgb]{0.15,0.15,0.15}{{140}}}}
\fontsize{6}{0}
\selectfont\put(229.235,154.496){\makebox(0,0)[r]{\textcolor[rgb]{0.15,0.15,0.15}{{0}}}}
\fontsize{6}{0}
\selectfont\put(229.235,174.144){\makebox(0,0)[r]{\textcolor[rgb]{0.15,0.15,0.15}{{0.5}}}}
\fontsize{6}{0}
\selectfont\put(229.235,193.792){\makebox(0,0)[r]{\textcolor[rgb]{0.15,0.15,0.15}{{1}}}}
\fontsize{6}{0}
\selectfont\put(229.235,213.44){\makebox(0,0)[r]{\textcolor[rgb]{0.15,0.15,0.15}{{1.5}}}}
\fontsize{7}{0}
\selectfont\put(214.235,186.325){\rotatebox{90}{\makebox(0,0)[b]{\textcolor[rgb]{0.15,0.15,0.15}{{Saida $y_2(t)$}}}}}
\fontsize{7}{0}
\selectfont\put(297.357,130.404){\makebox(0,0)[t]{\textcolor[rgb]{0.15,0.15,0.15}{{Tempo (s)}}}}
\fontsize{6}{0}
\selectfont\put(52,22.2671){\makebox(0,0)[t]{\textcolor[rgb]{0.15,0.15,0.15}{{0}}}}
\fontsize{6}{0}
\selectfont\put(70.4365,22.2671){\makebox(0,0)[t]{\textcolor[rgb]{0.15,0.15,0.15}{{20}}}}
\fontsize{6}{0}
\selectfont\put(88.873,22.2671){\makebox(0,0)[t]{\textcolor[rgb]{0.15,0.15,0.15}{{40}}}}
\fontsize{6}{0}
\selectfont\put(107.309,22.2671){\makebox(0,0)[t]{\textcolor[rgb]{0.15,0.15,0.15}{{60}}}}
\fontsize{6}{0}
\selectfont\put(125.746,22.2671){\makebox(0,0)[t]{\textcolor[rgb]{0.15,0.15,0.15}{{80}}}}
\fontsize{6}{0}
\selectfont\put(144.182,22.2671){\makebox(0,0)[t]{\textcolor[rgb]{0.15,0.15,0.15}{{100}}}}
\fontsize{6}{0}
\selectfont\put(162.619,22.2671){\makebox(0,0)[t]{\textcolor[rgb]{0.15,0.15,0.15}{{120}}}}
\fontsize{6}{0}
\selectfont\put(181.055,22.2671){\makebox(0,0)[t]{\textcolor[rgb]{0.15,0.15,0.15}{{140}}}}
\fontsize{6}{0}
\selectfont\put(48.5205,36.0054){\makebox(0,0)[r]{\textcolor[rgb]{0.15,0.15,0.15}{{0}}}}
\fontsize{6}{0}
\selectfont\put(48.5205,57.269){\makebox(0,0)[r]{\textcolor[rgb]{0.15,0.15,0.15}{{0.5}}}}
\fontsize{6}{0}
\selectfont\put(48.5205,78.5322){\makebox(0,0)[r]{\textcolor[rgb]{0.15,0.15,0.15}{{1}}}}
\fontsize{6}{0}
\selectfont\put(48.5205,99.7959){\makebox(0,0)[r]{\textcolor[rgb]{0.15,0.15,0.15}{{1.5}}}}
\fontsize{7}{0}
\selectfont\put(33.5205,67.1875){\rotatebox{90}{\makebox(0,0)[b]{\textcolor[rgb]{0.15,0.15,0.15}{{Sinal de Controle $u_1(t)$}}}}}
\fontsize{7}{0}
\selectfont\put(116.643,11.2671){\makebox(0,0)[t]{\textcolor[rgb]{0.15,0.15,0.15}{{Tempo (s)}}}}
\fontsize{6}{0}
\selectfont\put(232.715,22.2671){\makebox(0,0)[t]{\textcolor[rgb]{0.15,0.15,0.15}{{0}}}}
\fontsize{6}{0}
\selectfont\put(251.151,22.2671){\makebox(0,0)[t]{\textcolor[rgb]{0.15,0.15,0.15}{{20}}}}
\fontsize{6}{0}
\selectfont\put(269.587,22.2671){\makebox(0,0)[t]{\textcolor[rgb]{0.15,0.15,0.15}{{40}}}}
\fontsize{6}{0}
\selectfont\put(288.024,22.2671){\makebox(0,0)[t]{\textcolor[rgb]{0.15,0.15,0.15}{{60}}}}
\fontsize{6}{0}
\selectfont\put(306.46,22.2671){\makebox(0,0)[t]{\textcolor[rgb]{0.15,0.15,0.15}{{80}}}}
\fontsize{6}{0}
\selectfont\put(324.896,22.2671){\makebox(0,0)[t]{\textcolor[rgb]{0.15,0.15,0.15}{{100}}}}
\fontsize{6}{0}
\selectfont\put(343.333,22.2671){\makebox(0,0)[t]{\textcolor[rgb]{0.15,0.15,0.15}{{120}}}}
\fontsize{6}{0}
\selectfont\put(361.77,22.2671){\makebox(0,0)[t]{\textcolor[rgb]{0.15,0.15,0.15}{{140}}}}
\fontsize{6}{0}
\selectfont\put(229.235,28.3999){\makebox(0,0)[r]{\textcolor[rgb]{0.15,0.15,0.15}{{-0.2}}}}
\fontsize{6}{0}
\selectfont\put(229.235,43.6162){\makebox(0,0)[r]{\textcolor[rgb]{0.15,0.15,0.15}{{0}}}}
\fontsize{6}{0}
\selectfont\put(229.235,58.833){\makebox(0,0)[r]{\textcolor[rgb]{0.15,0.15,0.15}{{0.2}}}}
\fontsize{6}{0}
\selectfont\put(229.235,74.0493){\makebox(0,0)[r]{\textcolor[rgb]{0.15,0.15,0.15}{{0.4}}}}
\fontsize{6}{0}
\selectfont\put(229.235,89.2661){\makebox(0,0)[r]{\textcolor[rgb]{0.15,0.15,0.15}{{0.6}}}}
\fontsize{6}{0}
\selectfont\put(229.235,104.482){\makebox(0,0)[r]{\textcolor[rgb]{0.15,0.15,0.15}{{0.8}}}}
\fontsize{7}{0}
\selectfont\put(212.235,67.1875){\rotatebox{90}{\makebox(0,0)[b]{\textcolor[rgb]{0.15,0.15,0.15}{{Sinal de Controle $u_2(t)$}}}}}
\fontsize{7}{0}
\selectfont\put(297.357,11.2671){\makebox(0,0)[t]{\textcolor[rgb]{0.15,0.15,0.15}{{Tempo (s)}}}}
\fontsize{6}{0}
\selectfont\put(128.001,242.023){\makebox(0,0)[l]{\textcolor[rgb]{0,0,0}{{referencia}}}}
\fontsize{6}{0}
\selectfont\put(185.003,242.023){\makebox(0,0)[l]{\textcolor[rgb]{0,0,0}{{sem acao integral}}}}
\fontsize{6}{0}
\selectfont\put(262.005,242.023){\makebox(0,0)[l]{\textcolor[rgb]{0,0,0}{{com acao integral}}}}
\end{picture}

    \end{minipage}
\end{figure}

\subsubsection{Rejeição de perturbações e seguimento de referência com observador}
\label{subsub:controlador-utilizando-observador-de-estados}

Nos tópicos anteriores foi abordado o projeto de controladores em espaço de
estados considerando que todos os estados são mensurados. Entretanto, em
situações práticas de engenharia, ter todos os estados mensuráveis significa
maior custo com a aquisição de sensores, que a depender da aplicação é algo
indesejável. Além disso, em alguns sistemas, como os aeroespaciais, os estados
não são nem sequer mensuráveis diretamente por sensores. Desta forma, torna-se
necessário estimar os estados do sistema utilizando observadores de estados.
Portanto, foi realizado também simulações para o sistema \ref{eq:matriz-de-fts}
utilizando o observador de Luenberger.

Para o presente projeto foi considerado que a estimação dos estados em tempo
discreto é dada por

\begin{equation}
    \label{eq:predicao-do-estado-tempo-discreto}
    \mathbf{\hat{x}}[k+1] = \mathbf{\tilde{A}}\mathbf{\hat{x}}[k]
                            + \mathbf{\tilde{B}}\mathbf{u}[k]
                            + \mathbf{\tilde{L}}(\mathbf{y}[k] - \mathbf{\hat{y}[k]})
\end{equation} e então o sinal de controle é calculado com base no
vetor estimado, ou seja

\begin{equation}
    \label{eq:acao-de-conrole-com-estimador}
    \mathbf{u}[k] = -\mathbf{\tilde{K}}\mathbf{\hat{x}}[k]\text{.}
\end{equation} Considerando que existe um erro de estimação
$\mathbf{e}[k] = \mathbf{x}[k] - \mathbf{\hat{x}}[k]$ que após substituições das
equações \ref{eq:estimativa-do-vetor-de-estados},
\ref{eq:predicao-do-estado-tempo-discreto} e
\ref{eq:acao-de-conrole-com-estimador} resulta na equação a diferenças homogênea

\begin{equation}
    \label{eq:erro-de-estimacao}
    \mathbf{e}[k+1] = (\mathbf{\tilde{A}}-\mathbf{\tilde{L}}\mathbf{\tilde{C}})\mathbf{e}[k]
\end{equation} em que os autovalores da matriz
$\mathbf{\tilde{A}}-\mathbf{\tilde{L}}\mathbf{\tilde{C}}$ definem a velocidade
de convergência do erro de estimação, que neste caso dado a condição inicial
$\mathbf{e}[0] \neq \mathbf{0}$, a convergência é para erro nulo. Esta dinâmica
de conversão pode ser modificada escolhendo adequadamente a matriz
$\mathbf{\tilde{L}}$.

Portanto, para o controle do sistema em estudo foi dado como requisito que os
autovalores da matriz $\mathbf{\tilde{A}}-\mathbf{\tilde{L}}\mathbf{\tilde{C}}$
sejam um $\sigma_i = e^{\frac{-T_a}{2}}$ para $i = 1, 2, 3$ e $4$ com base no
período de amostragem $T_a = 0,75$s resulta em 4 autovalores $\sigma = 0,6873$.
Utilizando a propriedade que
$(\mathbf{\tilde{A}}-\mathbf{\tilde{L}}\mathbf{\tilde{C}})^\top =
\mathbf{\tilde{A}}^\top - \mathbf{\tilde{C}}^\top\mathbf{\tilde{L}}^\top$,
chega-se no mesmo problema da definição do ganho $\mathbf{\tilde{K}}$ exposto na
seção \ref{subsub:regulacao-em-torno-do-ponto-de-equilibrio}. Assim, utilizando
novamente a função \textit{place} do Octave sendo agora os argumentos
$\mathbf{\tilde{A}}^\top$, $\mathbf{\tilde{C}}^\top$ e $\mathbf{\sigma}$ foi
obtido a matriz de ganhos

\begin{equation}
    \label{eq:matriz-de-ganhos-do-estimador}
    \mathbf{\tilde{L}} =
    \begin{bmatrix}
        0.0000  & -0.3741   \\
        -0.3741 & 0.0000    \\
        -0.0000 & 0.4821    \\
        0.4821  & -0.0000
    \end{bmatrix}
\end{equation}

Aplicando a ação de controle \ref{eq:acao-de-conrole-com-estimador} considerando
também conforme \ref{eq:acao-de-controle-em-torno-do-ponto-de-equilibrio} que se
deseja levar os estados para uma condição de equilíbrio fora da origem, foi
obtido o resultado exibido na Figura
\ref{fig:erro-de-estimacao-de-estados-do-regulador-com-observador}.

\begin{figure}[!ht]
    \caption{Comparação da saída do sistema em malha fechada com ação de
    regulação para ponto de equilíbrio com e sem observador.}
    \hspace{-30pt}
    \label{fig:erro-de-estimacao-de-estados-do-regulador-com-observador}
    \begin{minipage}{\linewidth}
        % Title: gl2ps_renderer figure
% Creator: GL2PS 1.4.0, (C) 1999-2017 C. Geuzaine
% For: Octave
% CreationDate: Fri Nov 26 17:38:50 2021
\setlength{\unitlength}{1pt}
\begin{picture}(0,0)
\includegraphics{chapters/challenge6/images/resultado-1-questao-10-inc}
\end{picture}%
\begin{picture}(400,250)(0,0)
\fontsize{6}{0}
\selectfont\put(52,141.404){\makebox(0,0)[t]{\textcolor[rgb]{0.15,0.15,0.15}{{0}}}}
\fontsize{6}{0}
\selectfont\put(70.4365,141.404){\makebox(0,0)[t]{\textcolor[rgb]{0.15,0.15,0.15}{{20}}}}
\fontsize{6}{0}
\selectfont\put(88.873,141.404){\makebox(0,0)[t]{\textcolor[rgb]{0.15,0.15,0.15}{{40}}}}
\fontsize{6}{0}
\selectfont\put(107.309,141.404){\makebox(0,0)[t]{\textcolor[rgb]{0.15,0.15,0.15}{{60}}}}
\fontsize{6}{0}
\selectfont\put(125.746,141.404){\makebox(0,0)[t]{\textcolor[rgb]{0.15,0.15,0.15}{{80}}}}
\fontsize{6}{0}
\selectfont\put(144.182,141.404){\makebox(0,0)[t]{\textcolor[rgb]{0.15,0.15,0.15}{{100}}}}
\fontsize{6}{0}
\selectfont\put(162.619,141.404){\makebox(0,0)[t]{\textcolor[rgb]{0.15,0.15,0.15}{{120}}}}
\fontsize{6}{0}
\selectfont\put(181.055,141.404){\makebox(0,0)[t]{\textcolor[rgb]{0.15,0.15,0.15}{{140}}}}
\fontsize{6}{0}
\selectfont\put(48.5205,153.191){\makebox(0,0)[r]{\textcolor[rgb]{0.15,0.15,0.15}{{0}}}}
\fontsize{6}{0}
\selectfont\put(48.5205,169.575){\makebox(0,0)[r]{\textcolor[rgb]{0.15,0.15,0.15}{{0.5}}}}
\fontsize{6}{0}
\selectfont\put(48.5205,185.96){\makebox(0,0)[r]{\textcolor[rgb]{0.15,0.15,0.15}{{1}}}}
\fontsize{6}{0}
\selectfont\put(48.5205,202.344){\makebox(0,0)[r]{\textcolor[rgb]{0.15,0.15,0.15}{{1.5}}}}
\fontsize{6}{0}
\selectfont\put(48.5205,218.729){\makebox(0,0)[r]{\textcolor[rgb]{0.15,0.15,0.15}{{2}}}}
\fontsize{7}{0}
\selectfont\put(33.5205,186.325){\rotatebox{90}{\makebox(0,0)[b]{\textcolor[rgb]{0.15,0.15,0.15}{{Saida $y_1(t)$}}}}}
\fontsize{7}{0}
\selectfont\put(116.643,130.404){\makebox(0,0)[t]{\textcolor[rgb]{0.15,0.15,0.15}{{Tempo (s)}}}}
\fontsize{6}{0}
\selectfont\put(232.715,141.404){\makebox(0,0)[t]{\textcolor[rgb]{0.15,0.15,0.15}{{0}}}}
\fontsize{6}{0}
\selectfont\put(251.151,141.404){\makebox(0,0)[t]{\textcolor[rgb]{0.15,0.15,0.15}{{20}}}}
\fontsize{6}{0}
\selectfont\put(269.587,141.404){\makebox(0,0)[t]{\textcolor[rgb]{0.15,0.15,0.15}{{40}}}}
\fontsize{6}{0}
\selectfont\put(288.024,141.404){\makebox(0,0)[t]{\textcolor[rgb]{0.15,0.15,0.15}{{60}}}}
\fontsize{6}{0}
\selectfont\put(306.46,141.404){\makebox(0,0)[t]{\textcolor[rgb]{0.15,0.15,0.15}{{80}}}}
\fontsize{6}{0}
\selectfont\put(324.896,141.404){\makebox(0,0)[t]{\textcolor[rgb]{0.15,0.15,0.15}{{100}}}}
\fontsize{6}{0}
\selectfont\put(343.333,141.404){\makebox(0,0)[t]{\textcolor[rgb]{0.15,0.15,0.15}{{120}}}}
\fontsize{6}{0}
\selectfont\put(361.77,141.404){\makebox(0,0)[t]{\textcolor[rgb]{0.15,0.15,0.15}{{140}}}}
\fontsize{6}{0}
\selectfont\put(229.235,146.637){\makebox(0,0)[r]{\textcolor[rgb]{0.15,0.15,0.15}{{-0.2}}}}
\fontsize{6}{0}
\selectfont\put(229.235,157.632){\makebox(0,0)[r]{\textcolor[rgb]{0.15,0.15,0.15}{{0}}}}
\fontsize{6}{0}
\selectfont\put(229.235,168.627){\makebox(0,0)[r]{\textcolor[rgb]{0.15,0.15,0.15}{{0.2}}}}
\fontsize{6}{0}
\selectfont\put(229.235,179.623){\makebox(0,0)[r]{\textcolor[rgb]{0.15,0.15,0.15}{{0.4}}}}
\fontsize{6}{0}
\selectfont\put(229.235,190.618){\makebox(0,0)[r]{\textcolor[rgb]{0.15,0.15,0.15}{{0.6}}}}
\fontsize{6}{0}
\selectfont\put(229.235,201.613){\makebox(0,0)[r]{\textcolor[rgb]{0.15,0.15,0.15}{{0.8}}}}
\fontsize{6}{0}
\selectfont\put(229.235,212.608){\makebox(0,0)[r]{\textcolor[rgb]{0.15,0.15,0.15}{{1}}}}
\fontsize{6}{0}
\selectfont\put(229.235,223.604){\makebox(0,0)[r]{\textcolor[rgb]{0.15,0.15,0.15}{{1.2}}}}
\fontsize{7}{0}
\selectfont\put(212.235,186.325){\rotatebox{90}{\makebox(0,0)[b]{\textcolor[rgb]{0.15,0.15,0.15}{{Saida $y_2(t)$}}}}}
\fontsize{7}{0}
\selectfont\put(297.357,130.404){\makebox(0,0)[t]{\textcolor[rgb]{0.15,0.15,0.15}{{Tempo (s)}}}}
\fontsize{6}{0}
\selectfont\put(52,22.2671){\makebox(0,0)[t]{\textcolor[rgb]{0.15,0.15,0.15}{{0}}}}
\fontsize{6}{0}
\selectfont\put(70.4365,22.2671){\makebox(0,0)[t]{\textcolor[rgb]{0.15,0.15,0.15}{{20}}}}
\fontsize{6}{0}
\selectfont\put(88.873,22.2671){\makebox(0,0)[t]{\textcolor[rgb]{0.15,0.15,0.15}{{40}}}}
\fontsize{6}{0}
\selectfont\put(107.309,22.2671){\makebox(0,0)[t]{\textcolor[rgb]{0.15,0.15,0.15}{{60}}}}
\fontsize{6}{0}
\selectfont\put(125.746,22.2671){\makebox(0,0)[t]{\textcolor[rgb]{0.15,0.15,0.15}{{80}}}}
\fontsize{6}{0}
\selectfont\put(144.182,22.2671){\makebox(0,0)[t]{\textcolor[rgb]{0.15,0.15,0.15}{{100}}}}
\fontsize{6}{0}
\selectfont\put(162.619,22.2671){\makebox(0,0)[t]{\textcolor[rgb]{0.15,0.15,0.15}{{120}}}}
\fontsize{6}{0}
\selectfont\put(181.055,22.2671){\makebox(0,0)[t]{\textcolor[rgb]{0.15,0.15,0.15}{{140}}}}
\fontsize{6}{0}
\selectfont\put(48.5205,36.0044){\makebox(0,0)[r]{\textcolor[rgb]{0.15,0.15,0.15}{{0}}}}
\fontsize{6}{0}
\selectfont\put(48.5205,57.2651){\makebox(0,0)[r]{\textcolor[rgb]{0.15,0.15,0.15}{{0.5}}}}
\fontsize{6}{0}
\selectfont\put(48.5205,78.5259){\makebox(0,0)[r]{\textcolor[rgb]{0.15,0.15,0.15}{{1}}}}
\fontsize{6}{0}
\selectfont\put(48.5205,99.7866){\makebox(0,0)[r]{\textcolor[rgb]{0.15,0.15,0.15}{{1.5}}}}
\fontsize{7}{0}
\selectfont\put(33.5205,67.1875){\rotatebox{90}{\makebox(0,0)[b]{\textcolor[rgb]{0.15,0.15,0.15}{{Sinal de Controle $u_1(t)$}}}}}
\fontsize{7}{0}
\selectfont\put(116.643,11.2671){\makebox(0,0)[t]{\textcolor[rgb]{0.15,0.15,0.15}{{Tempo (s)}}}}
\fontsize{6}{0}
\selectfont\put(232.715,22.2671){\makebox(0,0)[t]{\textcolor[rgb]{0.15,0.15,0.15}{{0}}}}
\fontsize{6}{0}
\selectfont\put(251.151,22.2671){\makebox(0,0)[t]{\textcolor[rgb]{0.15,0.15,0.15}{{20}}}}
\fontsize{6}{0}
\selectfont\put(269.587,22.2671){\makebox(0,0)[t]{\textcolor[rgb]{0.15,0.15,0.15}{{40}}}}
\fontsize{6}{0}
\selectfont\put(288.024,22.2671){\makebox(0,0)[t]{\textcolor[rgb]{0.15,0.15,0.15}{{60}}}}
\fontsize{6}{0}
\selectfont\put(306.46,22.2671){\makebox(0,0)[t]{\textcolor[rgb]{0.15,0.15,0.15}{{80}}}}
\fontsize{6}{0}
\selectfont\put(324.896,22.2671){\makebox(0,0)[t]{\textcolor[rgb]{0.15,0.15,0.15}{{100}}}}
\fontsize{6}{0}
\selectfont\put(343.333,22.2671){\makebox(0,0)[t]{\textcolor[rgb]{0.15,0.15,0.15}{{120}}}}
\fontsize{6}{0}
\selectfont\put(361.77,22.2671){\makebox(0,0)[t]{\textcolor[rgb]{0.15,0.15,0.15}{{140}}}}
\fontsize{6}{0}
\selectfont\put(229.235,27.5){\makebox(0,0)[r]{\textcolor[rgb]{0.15,0.15,0.15}{{-0.2}}}}
\fontsize{6}{0}
\selectfont\put(229.235,42.8901){\makebox(0,0)[r]{\textcolor[rgb]{0.15,0.15,0.15}{{0}}}}
\fontsize{6}{0}
\selectfont\put(229.235,58.2803){\makebox(0,0)[r]{\textcolor[rgb]{0.15,0.15,0.15}{{0.2}}}}
\fontsize{6}{0}
\selectfont\put(229.235,73.6704){\makebox(0,0)[r]{\textcolor[rgb]{0.15,0.15,0.15}{{0.4}}}}
\fontsize{6}{0}
\selectfont\put(229.235,89.061){\makebox(0,0)[r]{\textcolor[rgb]{0.15,0.15,0.15}{{0.6}}}}
\fontsize{6}{0}
\selectfont\put(229.235,104.451){\makebox(0,0)[r]{\textcolor[rgb]{0.15,0.15,0.15}{{0.8}}}}
\fontsize{7}{0}
\selectfont\put(212.235,67.1875){\rotatebox{90}{\makebox(0,0)[b]{\textcolor[rgb]{0.15,0.15,0.15}{{Sinal de Controle $u_2(t)$}}}}}
\fontsize{7}{0}
\selectfont\put(297.357,11.2671){\makebox(0,0)[t]{\textcolor[rgb]{0.15,0.15,0.15}{{Tempo (s)}}}}
\fontsize{6}{0}
\selectfont\put(128.001,238.75){\makebox(0,0)[l]{\textcolor[rgb]{0,0,0}{{referencia}}}}
\fontsize{6}{0}
\selectfont\put(185.003,238.75){\makebox(0,0)[l]{\textcolor[rgb]{0,0,0}{{sem observador}}}}
\fontsize{6}{0}
\selectfont\put(258.005,238.75){\makebox(0,0)[l]{\textcolor[rgb]{0,0,0}{{com observador}}}}
\end{picture}

    \end{minipage}
\end{figure}

Na Figura \ref{fig:erro-de-estimacao-de-estados-do-regulador-com-observador} é
imperceptível a diferença das duas saídas do sistema fazendo a regulação com e
sem o observador de Luenberger. Entretanto, considerando que a saída estimada é
dada por $\mathbf{\hat{y}}(t) = \mathbf{C}\mathbf{\hat{x}}(t)$ foi possível
gerar o erro entre a saída real e a saída estimada, que está demonstrado na
Figura \ref{fig:erro-de-das-saidas-estimadas-utilizando-observador}. Percebe-se
então através desta figura que há uma pequena variação da saída estimada para a
saída real, porém, após aproximadamente 30s as saídas se igualam. Como trata-se
de uma simulação, embora considerado que os estados não são mensuráveis, a
título didático foi gerado também o erro da estimação dos estados, conforme
Figura \ref{fig:erro-de-estimacao-de-estados-com-observador}. De acordo com o
esperado, todos os erros convergem para zero após um período transitório que é
ditado pelos autovalores da matriz
$\mathbf{\tilde{A}}-\mathbf{\tilde{L}}\mathbf{\tilde{C}}$.

\begin{figure}[!ht]
    \caption{Erro de estimação da saída do sistema utilizando regulador com
    observador.}
    \vspace{-10pt}
    \hspace{-30pt}
    \label{fig:erro-de-das-saidas-estimadas-utilizando-observador}
    \begin{minipage}{\linewidth}
        % Title: gl2ps_renderer figure
% Creator: GL2PS 1.4.0, (C) 1999-2017 C. Geuzaine
% For: Octave
% CreationDate: Fri Nov 26 18:28:36 2021
\setlength{\unitlength}{1pt}
\begin{picture}(0,0)
\includegraphics{chapters/challenge6/images/resultado-3-questao-10-inc}
\end{picture}%
\begin{picture}(400,200)(0,0)
\fontsize{6}{0}
\selectfont\put(52,116.3){\makebox(0,0)[t]{\textcolor[rgb]{0.15,0.15,0.15}{{0}}}}
\fontsize{6}{0}
\selectfont\put(96.2065,116.3){\makebox(0,0)[t]{\textcolor[rgb]{0.15,0.15,0.15}{{20}}}}
\fontsize{6}{0}
\selectfont\put(140.414,116.3){\makebox(0,0)[t]{\textcolor[rgb]{0.15,0.15,0.15}{{40}}}}
\fontsize{6}{0}
\selectfont\put(184.62,116.3){\makebox(0,0)[t]{\textcolor[rgb]{0.15,0.15,0.15}{{60}}}}
\fontsize{6}{0}
\selectfont\put(228.827,116.3){\makebox(0,0)[t]{\textcolor[rgb]{0.15,0.15,0.15}{{80}}}}
\fontsize{6}{0}
\selectfont\put(273.034,116.3){\makebox(0,0)[t]{\textcolor[rgb]{0.15,0.15,0.15}{{100}}}}
\fontsize{6}{0}
\selectfont\put(317.241,116.3){\makebox(0,0)[t]{\textcolor[rgb]{0.15,0.15,0.15}{{120}}}}
\fontsize{6}{0}
\selectfont\put(361.447,116.3){\makebox(0,0)[t]{\textcolor[rgb]{0.15,0.15,0.15}{{140}}}}
\fontsize{6}{0}
\selectfont\put(48.5278,123.564){\makebox(0,0)[r]{\textcolor[rgb]{0.15,0.15,0.15}{{-0.0001}}}}
\fontsize{6}{0}
\selectfont\put(48.5278,136.554){\makebox(0,0)[r]{\textcolor[rgb]{0.15,0.15,0.15}{{0}}}}
\fontsize{6}{0}
\selectfont\put(48.5278,149.544){\makebox(0,0)[r]{\textcolor[rgb]{0.15,0.15,0.15}{{0.0001}}}}
\fontsize{6}{0}
\selectfont\put(48.5278,162.535){\makebox(0,0)[r]{\textcolor[rgb]{0.15,0.15,0.15}{{0.0002}}}}
\fontsize{6}{0}
\selectfont\put(48.5278,175.525){\makebox(0,0)[r]{\textcolor[rgb]{0.15,0.15,0.15}{{0.0003}}}}
\fontsize{7}{0}
\selectfont\put(19.5278,150.666){\rotatebox{90}{\makebox(0,0)[b]{\textcolor[rgb]{0.15,0.15,0.15}{{Saida $y_1(t) - \hat{y}_1(t)$}}}}}
\fontsize{7}{0}
\selectfont\put(207,105.3){\makebox(0,0)[t]{\textcolor[rgb]{0.15,0.15,0.15}{{Tempo (s)}}}}
\fontsize{6}{0}
\selectfont\put(52,20){\makebox(0,0)[t]{\textcolor[rgb]{0.15,0.15,0.15}{{0}}}}
\fontsize{6}{0}
\selectfont\put(96.2065,20){\makebox(0,0)[t]{\textcolor[rgb]{0.15,0.15,0.15}{{20}}}}
\fontsize{6}{0}
\selectfont\put(140.414,20){\makebox(0,0)[t]{\textcolor[rgb]{0.15,0.15,0.15}{{40}}}}
\fontsize{6}{0}
\selectfont\put(184.62,20){\makebox(0,0)[t]{\textcolor[rgb]{0.15,0.15,0.15}{{60}}}}
\fontsize{6}{0}
\selectfont\put(228.827,20){\makebox(0,0)[t]{\textcolor[rgb]{0.15,0.15,0.15}{{80}}}}
\fontsize{6}{0}
\selectfont\put(273.034,20){\makebox(0,0)[t]{\textcolor[rgb]{0.15,0.15,0.15}{{100}}}}
\fontsize{6}{0}
\selectfont\put(317.241,20){\makebox(0,0)[t]{\textcolor[rgb]{0.15,0.15,0.15}{{120}}}}
\fontsize{6}{0}
\selectfont\put(361.447,20){\makebox(0,0)[t]{\textcolor[rgb]{0.15,0.15,0.15}{{140}}}}
\fontsize{6}{0}
\selectfont\put(48.5278,34.3911){\makebox(0,0)[r]{\textcolor[rgb]{0.15,0.15,0.15}{{-0.0001}}}}
\fontsize{6}{0}
\selectfont\put(48.5278,45.4194){\makebox(0,0)[r]{\textcolor[rgb]{0.15,0.15,0.15}{{-5e-05}}}}
\fontsize{6}{0}
\selectfont\put(48.5278,56.4482){\makebox(0,0)[r]{\textcolor[rgb]{0.15,0.15,0.15}{{0}}}}
\fontsize{6}{0}
\selectfont\put(48.5278,67.4766){\makebox(0,0)[r]{\textcolor[rgb]{0.15,0.15,0.15}{{5e-05}}}}
\fontsize{6}{0}
\selectfont\put(48.5278,78.5049){\makebox(0,0)[r]{\textcolor[rgb]{0.15,0.15,0.15}{{0.0001}}}}
\fontsize{7}{0}
\selectfont\put(19.5278,54.3662){\rotatebox{90}{\makebox(0,0)[b]{\textcolor[rgb]{0.15,0.15,0.15}{{Saida $y_2(t) - \hat{y}_2(t)$}}}}}
\fontsize{7}{0}
\selectfont\put(207,9){\makebox(0,0)[t]{\textcolor[rgb]{0.15,0.15,0.15}{{Tempo (s)}}}}
\end{picture}

    \end{minipage}
\end{figure}

Foi realizada também a simulação considerando perturbações na entrada e saída da
planta. Como verificado na seção anterior, as perturbações só são rejeitadas com
a adição da ação integral no projeto do controlador em espaço de estados, o que
é condizente com o Princípio do Modelo Interno. Em virtude disso, para análise
do observador na presença de perturbações, foram realizadas apenas simulações
com projeto incluindo ação integral.

\begin{figure}[!ht]
    \caption{Erro de estimação dos estados do sistema em tempo discreto
    utilizando regulador com observador.}
    \vspace{-10pt}
    \hspace{-30pt}
    \label{fig:erro-de-estimacao-de-estados-com-observador}
    \begin{minipage}{\linewidth}
        % Title: gl2ps_renderer figure
% Creator: GL2PS 1.4.0, (C) 1999-2017 C. Geuzaine
% For: Octave
% CreationDate: Fri Nov 26 17:38:39 2021
\setlength{\unitlength}{1pt}
\begin{picture}(0,0)
\includegraphics{chapters/challenge6/images/resultado-2-questao-10-inc}
\end{picture}%
\begin{picture}(400,250)(0,0)
\fontsize{6}{0}
\selectfont\put(52,142.642){\makebox(0,0)[t]{\textcolor[rgb]{0.15,0.15,0.15}{{0}}}}
\fontsize{6}{0}
\selectfont\put(86.4922,142.642){\makebox(0,0)[t]{\textcolor[rgb]{0.15,0.15,0.15}{{50}}}}
\fontsize{6}{0}
\selectfont\put(120.984,142.642){\makebox(0,0)[t]{\textcolor[rgb]{0.15,0.15,0.15}{{100}}}}
\fontsize{6}{0}
\selectfont\put(155.476,142.642){\makebox(0,0)[t]{\textcolor[rgb]{0.15,0.15,0.15}{{150}}}}
\fontsize{6}{0}
\selectfont\put(48.5278,159.216){\makebox(0,0)[r]{\textcolor[rgb]{0.15,0.15,0.15}{{0}}}}
\fontsize{6}{0}
\selectfont\put(48.5278,178.559){\makebox(0,0)[r]{\textcolor[rgb]{0.15,0.15,0.15}{{0.002}}}}
\fontsize{6}{0}
\selectfont\put(48.5278,197.902){\makebox(0,0)[r]{\textcolor[rgb]{0.15,0.15,0.15}{{0.004}}}}
\fontsize{6}{0}
\selectfont\put(48.5278,217.246){\makebox(0,0)[r]{\textcolor[rgb]{0.15,0.15,0.15}{{0.006}}}}
\fontsize{7}{0}
\selectfont\put(25.5278,187.562){\rotatebox{90}{\makebox(0,0)[b]{\textcolor[rgb]{0.15,0.15,0.15}{{Erro $e_1[k]$}}}}}
\fontsize{7}{0}
\selectfont\put(116.5,131.642){\makebox(0,0)[t]{\textcolor[rgb]{0.15,0.15,0.15}{{Amostra [k]}}}}
\fontsize{6}{0}
\selectfont\put(233,142.642){\makebox(0,0)[t]{\textcolor[rgb]{0.15,0.15,0.15}{{0}}}}
\fontsize{6}{0}
\selectfont\put(267.492,142.642){\makebox(0,0)[t]{\textcolor[rgb]{0.15,0.15,0.15}{{50}}}}
\fontsize{6}{0}
\selectfont\put(301.984,142.642){\makebox(0,0)[t]{\textcolor[rgb]{0.15,0.15,0.15}{{100}}}}
\fontsize{6}{0}
\selectfont\put(336.476,142.642){\makebox(0,0)[t]{\textcolor[rgb]{0.15,0.15,0.15}{{150}}}}
\fontsize{6}{0}
\selectfont\put(229.528,156.911){\makebox(0,0)[r]{\textcolor[rgb]{0.15,0.15,0.15}{{0}}}}
\fontsize{6}{0}
\selectfont\put(229.528,166.968){\makebox(0,0)[r]{\textcolor[rgb]{0.15,0.15,0.15}{{0.002}}}}
\fontsize{6}{0}
\selectfont\put(229.528,177.024){\makebox(0,0)[r]{\textcolor[rgb]{0.15,0.15,0.15}{{0.004}}}}
\fontsize{6}{0}
\selectfont\put(229.528,187.081){\makebox(0,0)[r]{\textcolor[rgb]{0.15,0.15,0.15}{{0.006}}}}
\fontsize{6}{0}
\selectfont\put(229.528,197.137){\makebox(0,0)[r]{\textcolor[rgb]{0.15,0.15,0.15}{{0.008}}}}
\fontsize{6}{0}
\selectfont\put(229.528,207.194){\makebox(0,0)[r]{\textcolor[rgb]{0.15,0.15,0.15}{{0.01}}}}
\fontsize{6}{0}
\selectfont\put(229.528,217.25){\makebox(0,0)[r]{\textcolor[rgb]{0.15,0.15,0.15}{{0.012}}}}
\fontsize{7}{0}
\selectfont\put(206.528,187.562){\rotatebox{90}{\makebox(0,0)[b]{\textcolor[rgb]{0.15,0.15,0.15}{{Erro $e_2[k]$}}}}}
\fontsize{7}{0}
\selectfont\put(297.5,131.642){\makebox(0,0)[t]{\textcolor[rgb]{0.15,0.15,0.15}{{Amostra [k]}}}}
\fontsize{6}{0}
\selectfont\put(52,22.2671){\makebox(0,0)[t]{\textcolor[rgb]{0.15,0.15,0.15}{{0}}}}
\fontsize{6}{0}
\selectfont\put(86.4922,22.2671){\makebox(0,0)[t]{\textcolor[rgb]{0.15,0.15,0.15}{{50}}}}
\fontsize{6}{0}
\selectfont\put(120.984,22.2671){\makebox(0,0)[t]{\textcolor[rgb]{0.15,0.15,0.15}{{100}}}}
\fontsize{6}{0}
\selectfont\put(155.476,22.2671){\makebox(0,0)[t]{\textcolor[rgb]{0.15,0.15,0.15}{{150}}}}
\fontsize{6}{0}
\selectfont\put(48.5278,31.2358){\makebox(0,0)[r]{\textcolor[rgb]{0.15,0.15,0.15}{{-0.006}}}}
\fontsize{6}{0}
\selectfont\put(48.5278,51.8379){\makebox(0,0)[r]{\textcolor[rgb]{0.15,0.15,0.15}{{-0.004}}}}
\fontsize{6}{0}
\selectfont\put(48.5278,72.4399){\makebox(0,0)[r]{\textcolor[rgb]{0.15,0.15,0.15}{{-0.002}}}}
\fontsize{6}{0}
\selectfont\put(48.5278,93.0425){\makebox(0,0)[r]{\textcolor[rgb]{0.15,0.15,0.15}{{0}}}}
\fontsize{7}{0}
\selectfont\put(23.5278,67.1875){\rotatebox{90}{\makebox(0,0)[b]{\textcolor[rgb]{0.15,0.15,0.15}{{Erro $e_3[k]$}}}}}
\fontsize{7}{0}
\selectfont\put(116.5,11.2671){\makebox(0,0)[t]{\textcolor[rgb]{0.15,0.15,0.15}{{Amostra [k]}}}}
\fontsize{6}{0}
\selectfont\put(233,22.2671){\makebox(0,0)[t]{\textcolor[rgb]{0.15,0.15,0.15}{{0}}}}
\fontsize{6}{0}
\selectfont\put(267.492,22.2671){\makebox(0,0)[t]{\textcolor[rgb]{0.15,0.15,0.15}{{50}}}}
\fontsize{6}{0}
\selectfont\put(301.984,22.2671){\makebox(0,0)[t]{\textcolor[rgb]{0.15,0.15,0.15}{{100}}}}
\fontsize{6}{0}
\selectfont\put(336.476,22.2671){\makebox(0,0)[t]{\textcolor[rgb]{0.15,0.15,0.15}{{150}}}}
\fontsize{6}{0}
\selectfont\put(229.528,34.8433){\makebox(0,0)[r]{\textcolor[rgb]{0.15,0.15,0.15}{{-0.015}}}}
\fontsize{6}{0}
\selectfont\put(229.528,55.8301){\makebox(0,0)[r]{\textcolor[rgb]{0.15,0.15,0.15}{{-0.01}}}}
\fontsize{6}{0}
\selectfont\put(229.528,76.8169){\makebox(0,0)[r]{\textcolor[rgb]{0.15,0.15,0.15}{{-0.005}}}}
\fontsize{6}{0}
\selectfont\put(229.528,97.8037){\makebox(0,0)[r]{\textcolor[rgb]{0.15,0.15,0.15}{{0}}}}
\fontsize{7}{0}
\selectfont\put(204.528,67.1875){\rotatebox{90}{\makebox(0,0)[b]{\textcolor[rgb]{0.15,0.15,0.15}{{Erro $e_4[k]$}}}}}
\fontsize{7}{0}
\selectfont\put(297.5,11.2671){\makebox(0,0)[t]{\textcolor[rgb]{0.15,0.15,0.15}{{Amostra [k]}}}}
\end{picture}

    \end{minipage}
\end{figure}

Assim como na seção
\ref{subsub:rejeicao-de-perturbacoes-e-seguimento-de-referencia} é necessário
fazer a realização em espaço de estados de forma aumentada para a realizar a
estimação de estados pelo observador na presença da ação integral. Desta forma a
estimação de estados fica

\begin{equation}
    \label{eq:estimacao-de-estados-com-observador-e-acao-integral}
    \begin{bmatrix}
        \Delta \mathbf{\hat{x}}[k+1] \\
        \mathbf{\hat{y}}[k+1]
    \end{bmatrix}
    =
    \mathbf{\bar{A}}
    \begin{bmatrix}
        \Delta \mathbf{\hat{x}}[k] \\
        \mathbf{\hat{y}}[k]
    \end{bmatrix}
    +
    \mathbf{\bar{B}}
    \Delta \mathbf{u}[k]
    +
    \mathbf{\bar{L}}
    \begin{bmatrix}
        \Delta \mathbf{y}[k] - \mathbf{\tilde{C}} \Delta \mathbf{\hat{x}}[k] \\
        \mathbf{y}[k] - \mathbf{\hat{y}}[k]
    \end{bmatrix}
\end{equation} sendo que $\mathbf{\bar{A}}$ e $\mathbf{\bar{B}}$ são as mesmas
matrizes aumentadas definidas em \ref{eq:matriz-a-e-b-aumentada}.
Adicionalmente, dessa realização surge também a matriz aumentada
$\mathbf{\bar{C}}_{4 \times 6}$ dada por

\begin{equation}
    \label{eq:matriz-aumentada-c}
    \mathbf{\bar{C}}
    =
    \begin{bmatrix}
        \mathbf{\tilde{C}} & \mathbf{\bar{0}} \\
        \mathbf{\bar{0}}   & \mathbf{I}
    \end{bmatrix}\text{.}
\end{equation} Com essa nova realização do observador, a ação de controle se
torna

\begin{equation}
    \label{eq:acao-de-controle-com-obervador-e-integrador}
    \Delta \mathbf{u}[k] =
    - \mathbf{K_x} \Delta \mathbf{\hat{x}}[k]
    + \mathbf{K_i} \Delta \mathbf{\hat{z}}[k]
    \Rightarrow 
    \mathbf{u}[k] = \Delta \mathbf{u}[k] + \mathbf{u}[k-1]
    \text{.}
\end{equation}

A realização aumentada proporciona novamente que a convergência do erro de
estimação seja governada pelos autovalores da matriz $\mathbf{\bar{A}} -
\mathbf{\bar{L}}\mathbf{\bar{C}}$ sendo que a dimensão de $\mathbf{\bar{L}}$ é
agora $6 \times 4$ e, portanto, é necessário alocar 6 autovalores. Assim para o
caso do ganho $\mathbf{\bar{K}}$ (seção
\ref{subsub:rejeicao-de-perturbacoes-e-seguimento-de-referencia}) os autovalores
desejáveis permanecem iguais. Portanto, aplicando as matrizes
$\mathbf{\bar{A}}^\top$ e $\mathbf{\bar{C}}^\top$ e os 6 autovalores $\sigma =
0,6873$ foi obtida a matriz de ganhos

\begin{equation}
    \label{eq:matriz-l-aumentada}
    \mathbf{\bar{L}}
    =
    \begin{bmatrix}
        0       & -0,0235 & 0       & -0,0908   \\
        -0,0235 & 0       & -0,0908 & 0         \\
        0       & 0,0303  & 0       & 0,1171    \\
        0,0303  & 0       & 0,1171  & 0         \\
        0,0584  & 0       & 0,5380  & 0         \\
        0       & 0,0584  & 0       & 0,5380
    \end{bmatrix}
    \text{.}
\end{equation}

\begin{figure}[!ht]
    \caption{Comparação dos sinais de controle e saída do sistema em malha
    fechada para seguimento de referência com e sem observador.}
    \hspace{-30pt}
    \label{fig:resultado-do-regulador-com-observador-e-integrador}
    \begin{minipage}{\linewidth}
        % Title: gl2ps_renderer figure
% Creator: GL2PS 1.4.0, (C) 1999-2017 C. Geuzaine
% For: Octave
% CreationDate: Sat Nov 27 14:07:34 2021
\setlength{\unitlength}{1pt}
\begin{picture}(0,0)
\includegraphics{chapters/challenge6/images/resultado-4-questao-10-inc}
\end{picture}%
\begin{picture}(400,250)(0,0)
\fontsize{6}{0}
\selectfont\put(52,141.404){\makebox(0,0)[t]{\textcolor[rgb]{0.15,0.15,0.15}{{0}}}}
\fontsize{6}{0}
\selectfont\put(70.4365,141.404){\makebox(0,0)[t]{\textcolor[rgb]{0.15,0.15,0.15}{{20}}}}
\fontsize{6}{0}
\selectfont\put(88.873,141.404){\makebox(0,0)[t]{\textcolor[rgb]{0.15,0.15,0.15}{{40}}}}
\fontsize{6}{0}
\selectfont\put(107.309,141.404){\makebox(0,0)[t]{\textcolor[rgb]{0.15,0.15,0.15}{{60}}}}
\fontsize{6}{0}
\selectfont\put(125.746,141.404){\makebox(0,0)[t]{\textcolor[rgb]{0.15,0.15,0.15}{{80}}}}
\fontsize{6}{0}
\selectfont\put(144.182,141.404){\makebox(0,0)[t]{\textcolor[rgb]{0.15,0.15,0.15}{{100}}}}
\fontsize{6}{0}
\selectfont\put(162.619,141.404){\makebox(0,0)[t]{\textcolor[rgb]{0.15,0.15,0.15}{{120}}}}
\fontsize{6}{0}
\selectfont\put(181.055,141.404){\makebox(0,0)[t]{\textcolor[rgb]{0.15,0.15,0.15}{{140}}}}
\fontsize{6}{0}
\selectfont\put(48.5205,152.69){\makebox(0,0)[r]{\textcolor[rgb]{0.15,0.15,0.15}{{0}}}}
\fontsize{6}{0}
\selectfont\put(48.5205,167.824){\makebox(0,0)[r]{\textcolor[rgb]{0.15,0.15,0.15}{{0.5}}}}
\fontsize{6}{0}
\selectfont\put(48.5205,182.957){\makebox(0,0)[r]{\textcolor[rgb]{0.15,0.15,0.15}{{1}}}}
\fontsize{6}{0}
\selectfont\put(48.5205,198.09){\makebox(0,0)[r]{\textcolor[rgb]{0.15,0.15,0.15}{{1.5}}}}
\fontsize{6}{0}
\selectfont\put(48.5205,213.223){\makebox(0,0)[r]{\textcolor[rgb]{0.15,0.15,0.15}{{2}}}}
\fontsize{7}{0}
\selectfont\put(33.5205,186.325){\rotatebox{90}{\makebox(0,0)[b]{\textcolor[rgb]{0.15,0.15,0.15}{{Saida $y_1(t)$}}}}}
\fontsize{7}{0}
\selectfont\put(116.643,130.404){\makebox(0,0)[t]{\textcolor[rgb]{0.15,0.15,0.15}{{Tempo (s)}}}}
\fontsize{6}{0}
\selectfont\put(232.715,141.404){\makebox(0,0)[t]{\textcolor[rgb]{0.15,0.15,0.15}{{0}}}}
\fontsize{6}{0}
\selectfont\put(251.151,141.404){\makebox(0,0)[t]{\textcolor[rgb]{0.15,0.15,0.15}{{20}}}}
\fontsize{6}{0}
\selectfont\put(269.587,141.404){\makebox(0,0)[t]{\textcolor[rgb]{0.15,0.15,0.15}{{40}}}}
\fontsize{6}{0}
\selectfont\put(288.024,141.404){\makebox(0,0)[t]{\textcolor[rgb]{0.15,0.15,0.15}{{60}}}}
\fontsize{6}{0}
\selectfont\put(306.46,141.404){\makebox(0,0)[t]{\textcolor[rgb]{0.15,0.15,0.15}{{80}}}}
\fontsize{6}{0}
\selectfont\put(324.896,141.404){\makebox(0,0)[t]{\textcolor[rgb]{0.15,0.15,0.15}{{100}}}}
\fontsize{6}{0}
\selectfont\put(343.333,141.404){\makebox(0,0)[t]{\textcolor[rgb]{0.15,0.15,0.15}{{120}}}}
\fontsize{6}{0}
\selectfont\put(361.77,141.404){\makebox(0,0)[t]{\textcolor[rgb]{0.15,0.15,0.15}{{140}}}}
\fontsize{6}{0}
\selectfont\put(229.235,156.557){\makebox(0,0)[r]{\textcolor[rgb]{0.15,0.15,0.15}{{0}}}}
\fontsize{6}{0}
\selectfont\put(229.235,181.357){\makebox(0,0)[r]{\textcolor[rgb]{0.15,0.15,0.15}{{0.5}}}}
\fontsize{6}{0}
\selectfont\put(229.235,206.158){\makebox(0,0)[r]{\textcolor[rgb]{0.15,0.15,0.15}{{1}}}}
\fontsize{7}{0}
\selectfont\put(214.235,186.325){\rotatebox{90}{\makebox(0,0)[b]{\textcolor[rgb]{0.15,0.15,0.15}{{Saida $y_2(t)$}}}}}
\fontsize{7}{0}
\selectfont\put(297.357,130.404){\makebox(0,0)[t]{\textcolor[rgb]{0.15,0.15,0.15}{{Tempo (s)}}}}
\fontsize{6}{0}
\selectfont\put(52,22.2671){\makebox(0,0)[t]{\textcolor[rgb]{0.15,0.15,0.15}{{0}}}}
\fontsize{6}{0}
\selectfont\put(70.4365,22.2671){\makebox(0,0)[t]{\textcolor[rgb]{0.15,0.15,0.15}{{20}}}}
\fontsize{6}{0}
\selectfont\put(88.873,22.2671){\makebox(0,0)[t]{\textcolor[rgb]{0.15,0.15,0.15}{{40}}}}
\fontsize{6}{0}
\selectfont\put(107.309,22.2671){\makebox(0,0)[t]{\textcolor[rgb]{0.15,0.15,0.15}{{60}}}}
\fontsize{6}{0}
\selectfont\put(125.746,22.2671){\makebox(0,0)[t]{\textcolor[rgb]{0.15,0.15,0.15}{{80}}}}
\fontsize{6}{0}
\selectfont\put(144.182,22.2671){\makebox(0,0)[t]{\textcolor[rgb]{0.15,0.15,0.15}{{100}}}}
\fontsize{6}{0}
\selectfont\put(162.619,22.2671){\makebox(0,0)[t]{\textcolor[rgb]{0.15,0.15,0.15}{{120}}}}
\fontsize{6}{0}
\selectfont\put(181.055,22.2671){\makebox(0,0)[t]{\textcolor[rgb]{0.15,0.15,0.15}{{140}}}}
\fontsize{6}{0}
\selectfont\put(48.5205,27.5){\makebox(0,0)[r]{\textcolor[rgb]{0.15,0.15,0.15}{{-0.2}}}}
\fontsize{6}{0}
\selectfont\put(48.5205,38.2656){\makebox(0,0)[r]{\textcolor[rgb]{0.15,0.15,0.15}{{0}}}}
\fontsize{6}{0}
\selectfont\put(48.5205,49.0308){\makebox(0,0)[r]{\textcolor[rgb]{0.15,0.15,0.15}{{0.2}}}}
\fontsize{6}{0}
\selectfont\put(48.5205,59.7964){\makebox(0,0)[r]{\textcolor[rgb]{0.15,0.15,0.15}{{0.4}}}}
\fontsize{6}{0}
\selectfont\put(48.5205,70.562){\makebox(0,0)[r]{\textcolor[rgb]{0.15,0.15,0.15}{{0.6}}}}
\fontsize{6}{0}
\selectfont\put(48.5205,81.3276){\makebox(0,0)[r]{\textcolor[rgb]{0.15,0.15,0.15}{{0.8}}}}
\fontsize{6}{0}
\selectfont\put(48.5205,92.0928){\makebox(0,0)[r]{\textcolor[rgb]{0.15,0.15,0.15}{{1}}}}
\fontsize{6}{0}
\selectfont\put(48.5205,102.858){\makebox(0,0)[r]{\textcolor[rgb]{0.15,0.15,0.15}{{1.2}}}}
\fontsize{7}{0}
\selectfont\put(31.5205,67.1875){\rotatebox{90}{\makebox(0,0)[b]{\textcolor[rgb]{0.15,0.15,0.15}{{Sinal de Controle $u_1(t)$}}}}}
\fontsize{7}{0}
\selectfont\put(116.643,11.2671){\makebox(0,0)[t]{\textcolor[rgb]{0.15,0.15,0.15}{{Tempo (s)}}}}
\fontsize{6}{0}
\selectfont\put(232.715,22.2671){\makebox(0,0)[t]{\textcolor[rgb]{0.15,0.15,0.15}{{0}}}}
\fontsize{6}{0}
\selectfont\put(251.151,22.2671){\makebox(0,0)[t]{\textcolor[rgb]{0.15,0.15,0.15}{{20}}}}
\fontsize{6}{0}
\selectfont\put(269.587,22.2671){\makebox(0,0)[t]{\textcolor[rgb]{0.15,0.15,0.15}{{40}}}}
\fontsize{6}{0}
\selectfont\put(288.024,22.2671){\makebox(0,0)[t]{\textcolor[rgb]{0.15,0.15,0.15}{{60}}}}
\fontsize{6}{0}
\selectfont\put(306.46,22.2671){\makebox(0,0)[t]{\textcolor[rgb]{0.15,0.15,0.15}{{80}}}}
\fontsize{6}{0}
\selectfont\put(324.896,22.2671){\makebox(0,0)[t]{\textcolor[rgb]{0.15,0.15,0.15}{{100}}}}
\fontsize{6}{0}
\selectfont\put(343.333,22.2671){\makebox(0,0)[t]{\textcolor[rgb]{0.15,0.15,0.15}{{120}}}}
\fontsize{6}{0}
\selectfont\put(361.77,22.2671){\makebox(0,0)[t]{\textcolor[rgb]{0.15,0.15,0.15}{{140}}}}
\fontsize{6}{0}
\selectfont\put(229.235,27.5){\makebox(0,0)[r]{\textcolor[rgb]{0.15,0.15,0.15}{{-0.2}}}}
\fontsize{6}{0}
\selectfont\put(229.235,44.792){\makebox(0,0)[r]{\textcolor[rgb]{0.15,0.15,0.15}{{0}}}}
\fontsize{6}{0}
\selectfont\put(229.235,62.0845){\makebox(0,0)[r]{\textcolor[rgb]{0.15,0.15,0.15}{{0.2}}}}
\fontsize{6}{0}
\selectfont\put(229.235,79.3765){\makebox(0,0)[r]{\textcolor[rgb]{0.15,0.15,0.15}{{0.4}}}}
\fontsize{6}{0}
\selectfont\put(229.235,96.6689){\makebox(0,0)[r]{\textcolor[rgb]{0.15,0.15,0.15}{{0.6}}}}
\fontsize{7}{0}
\selectfont\put(212.235,67.1875){\rotatebox{90}{\makebox(0,0)[b]{\textcolor[rgb]{0.15,0.15,0.15}{{Sinal de Controle $u_2(t)$}}}}}
\fontsize{7}{0}
\selectfont\put(297.357,11.2671){\makebox(0,0)[t]{\textcolor[rgb]{0.15,0.15,0.15}{{Tempo (s)}}}}
\fontsize{6}{0}
\selectfont\put(128.001,238.75){\makebox(0,0)[l]{\textcolor[rgb]{0,0,0}{{referencia}}}}
\fontsize{6}{0}
\selectfont\put(185.003,238.75){\makebox(0,0)[l]{\textcolor[rgb]{0,0,0}{{sem observador}}}}
\fontsize{6}{0}
\selectfont\put(258.005,238.75){\makebox(0,0)[l]{\textcolor[rgb]{0,0,0}{{com observador}}}}
\end{picture}

    \end{minipage}
\end{figure}

Utilizando o observador regido pela realização
\ref{eq:estimacao-de-estados-com-observador-e-acao-integral} com a matriz
$\mathbf{\bar{L}}$ obtida e utilizando a lei de controle dada por
\ref{eq:acao-de-controle-com-obervador-e-integrador}, foi obtido o resultado
mostrado na Figura \ref{fig:resultado-do-regulador-com-observador-e-integrador}.
Observa-se que não há interferência do observador para seguimento de referência
sem a presença de perturbações. Porém, com o surgimento das perturbações
perturbações na saída $\mathbf{q}_{y}(t) = 0,2\mathbf{\mathds{1}}(t-50)$ e na
entrada $\mathbf{q}_{u}(t) = 0,2\mathbf{\mathds{1}}(t-100)$, a dinâmica imposta
pelos autovalores do observador influencia significativamente na saída do
sistema em malha fechada. Esta divergência das saídas é resultado da diferença
da ação de controle aplicada aos estados estimados pelo observador, como nota-se
também nos gráficos do sinal de controle, em que a ação é mais agressiva
utilizando observador.

Assim para o caso anterior, foi avaliado também o erro entre a saída real e a
saída estimada. O resultado é exibido na Figura
\ref{fig:erro-das-saidas-estimadas-utilizando-observador-e-integrador}.
Percebe-se que o transitório da divergência tem um tempo de acomodação
significativo de $\approx40$s.

\begin{figure}[!ht]
    \caption{Erro de estimação da saída do sistema utilizando regulador com
    observador e ação integral.}
    \vspace{-10pt}
    \hspace{-30pt}
    \label{fig:erro-das-saidas-estimadas-utilizando-observador-e-integrador}
    \begin{minipage}{\linewidth}
        % Title: gl2ps_renderer figure
% Creator: GL2PS 1.4.0, (C) 1999-2017 C. Geuzaine
% For: Octave
% CreationDate: Sat Nov 27 16:17:20 2021
\setlength{\unitlength}{1pt}
\begin{picture}(0,0)
\includegraphics{chapters/challenge6/images/resultado-5-questao-10-inc}
\end{picture}%
\begin{picture}(400,200)(0,0)
\fontsize{6}{0}
\selectfont\put(52,116.3){\makebox(0,0)[t]{\textcolor[rgb]{0.15,0.15,0.15}{{0}}}}
\fontsize{6}{0}
\selectfont\put(96.2065,116.3){\makebox(0,0)[t]{\textcolor[rgb]{0.15,0.15,0.15}{{20}}}}
\fontsize{6}{0}
\selectfont\put(140.414,116.3){\makebox(0,0)[t]{\textcolor[rgb]{0.15,0.15,0.15}{{40}}}}
\fontsize{6}{0}
\selectfont\put(184.62,116.3){\makebox(0,0)[t]{\textcolor[rgb]{0.15,0.15,0.15}{{60}}}}
\fontsize{6}{0}
\selectfont\put(228.827,116.3){\makebox(0,0)[t]{\textcolor[rgb]{0.15,0.15,0.15}{{80}}}}
\fontsize{6}{0}
\selectfont\put(273.034,116.3){\makebox(0,0)[t]{\textcolor[rgb]{0.15,0.15,0.15}{{100}}}}
\fontsize{6}{0}
\selectfont\put(317.241,116.3){\makebox(0,0)[t]{\textcolor[rgb]{0.15,0.15,0.15}{{120}}}}
\fontsize{6}{0}
\selectfont\put(361.447,116.3){\makebox(0,0)[t]{\textcolor[rgb]{0.15,0.15,0.15}{{140}}}}
\fontsize{6}{0}
\selectfont\put(48.5278,129.586){\makebox(0,0)[r]{\textcolor[rgb]{0.15,0.15,0.15}{{-0.1}}}}
\fontsize{6}{0}
\selectfont\put(48.5278,144.968){\makebox(0,0)[r]{\textcolor[rgb]{0.15,0.15,0.15}{{-0.05}}}}
\fontsize{6}{0}
\selectfont\put(48.5278,160.35){\makebox(0,0)[r]{\textcolor[rgb]{0.15,0.15,0.15}{{0}}}}
\fontsize{6}{0}
\selectfont\put(48.5278,175.732){\makebox(0,0)[r]{\textcolor[rgb]{0.15,0.15,0.15}{{0.05}}}}
\fontsize{7}{0}
\selectfont\put(27.5278,150.666){\rotatebox{90}{\makebox(0,0)[b]{\textcolor[rgb]{0.15,0.15,0.15}{{Saida $y_1(t) - \hat{y}_1(t)$}}}}}
\fontsize{7}{0}
\selectfont\put(207,105.3){\makebox(0,0)[t]{\textcolor[rgb]{0.15,0.15,0.15}{{Tempo (s)}}}}
\fontsize{6}{0}
\selectfont\put(52,20){\makebox(0,0)[t]{\textcolor[rgb]{0.15,0.15,0.15}{{0}}}}
\fontsize{6}{0}
\selectfont\put(96.2065,20){\makebox(0,0)[t]{\textcolor[rgb]{0.15,0.15,0.15}{{20}}}}
\fontsize{6}{0}
\selectfont\put(140.414,20){\makebox(0,0)[t]{\textcolor[rgb]{0.15,0.15,0.15}{{40}}}}
\fontsize{6}{0}
\selectfont\put(184.62,20){\makebox(0,0)[t]{\textcolor[rgb]{0.15,0.15,0.15}{{60}}}}
\fontsize{6}{0}
\selectfont\put(228.827,20){\makebox(0,0)[t]{\textcolor[rgb]{0.15,0.15,0.15}{{80}}}}
\fontsize{6}{0}
\selectfont\put(273.034,20){\makebox(0,0)[t]{\textcolor[rgb]{0.15,0.15,0.15}{{100}}}}
\fontsize{6}{0}
\selectfont\put(317.241,20){\makebox(0,0)[t]{\textcolor[rgb]{0.15,0.15,0.15}{{120}}}}
\fontsize{6}{0}
\selectfont\put(361.447,20){\makebox(0,0)[t]{\textcolor[rgb]{0.15,0.15,0.15}{{140}}}}
\fontsize{6}{0}
\selectfont\put(48.5278,30.6553){\makebox(0,0)[r]{\textcolor[rgb]{0.15,0.15,0.15}{{-0.08}}}}
\fontsize{6}{0}
\selectfont\put(48.5278,38.064){\makebox(0,0)[r]{\textcolor[rgb]{0.15,0.15,0.15}{{-0.06}}}}
\fontsize{6}{0}
\selectfont\put(48.5278,45.4722){\makebox(0,0)[r]{\textcolor[rgb]{0.15,0.15,0.15}{{-0.04}}}}
\fontsize{6}{0}
\selectfont\put(48.5278,52.8809){\makebox(0,0)[r]{\textcolor[rgb]{0.15,0.15,0.15}{{-0.02}}}}
\fontsize{6}{0}
\selectfont\put(48.5278,60.2891){\makebox(0,0)[r]{\textcolor[rgb]{0.15,0.15,0.15}{{0}}}}
\fontsize{6}{0}
\selectfont\put(48.5278,67.6978){\makebox(0,0)[r]{\textcolor[rgb]{0.15,0.15,0.15}{{0.02}}}}
\fontsize{6}{0}
\selectfont\put(48.5278,75.106){\makebox(0,0)[r]{\textcolor[rgb]{0.15,0.15,0.15}{{0.04}}}}
\fontsize{6}{0}
\selectfont\put(48.5278,82.5146){\makebox(0,0)[r]{\textcolor[rgb]{0.15,0.15,0.15}{{0.06}}}}
\fontsize{7}{0}
\selectfont\put(27.5278,54.3662){\rotatebox{90}{\makebox(0,0)[b]{\textcolor[rgb]{0.15,0.15,0.15}{{Saida $y_2(t) - \hat{y}_2(t)$}}}}}
\fontsize{7}{0}
\selectfont\put(207,9){\makebox(0,0)[t]{\textcolor[rgb]{0.15,0.15,0.15}{{Tempo (s)}}}}
\end{picture}

    \end{minipage}
\end{figure}

Todavia, como mostra a Figura
\ref{fig:erro-de-estimacao-de-estados-com-observador-e-integrador}, o erro de
estimação de estados não teve o mesmo comportamento do regulador por
realimentação de estados sem ação integral. Este comportamento é esperado pois
a subtração de $\Delta \mathbf{x}[k+1]$ em
\ref{eq:espaco-de-estados-discreto-aumentado} pela estimação do observador $\Delta
\mathbf{\hat{x}}[k+1]$ em
\ref{eq:estimacao-de-estados-com-observador-e-acao-integral} resulta no erro de
estimação de estados

\begin{equation}
    \label{eq:erro-da-estimacao-de-estados-com-observador-e-integrador}
    \Delta \mathbf{e}[k+1] =
    (\mathbf{\bar{A}} - \mathbf{\bar{L}}\mathbf{\bar{C}}) \Delta \mathbf{e}[k]
    \text{.}
\end{equation} Como nota-se, não mais $\mathbf{e}[k] \rightarrow \mathbf{0}$
mas sim $\Delta \mathbf{e}[k] \rightarrow \mathbf{0}$, em que novamente a
dinâmica desta convergência é ditada pelos autovalores de $\mathbf{\bar{A}} -
\mathbf{\bar{L}}\mathbf{\bar{C}}$. Portanto dizer $\Delta \mathbf{e}[k]
\rightarrow \mathbf{0}$ é a mesmo coisa de dizer que $\mathbf{e}[k]$ tende a
zero ou para um valor constante, o que acontece na Figura
\ref{fig:erro-de-estimacao-de-estados-com-observador-e-integrador}. Este erro
não nulo na presença de perturbações pode ser explicada pelo conceito de
observabilidade. Como não há controle das perturbações, o observador não
consegue estimar as perturbações o que acarreta no erro de estimação de estados
não nulo. Entretanto, devido ao integrador a saída converge para o valor de
referência.

\begin{figure}[!ht]
    \caption{Erro de estimação dos estados do sistema em tempo discreto
    utilizando regulador com observador e integrador.}
    \vspace{-10pt}
    \hspace{-30pt}
    \label{fig:erro-de-estimacao-de-estados-com-observador-e-integrador}
    \begin{minipage}{\linewidth}
        % Title: gl2ps_renderer figure
% Creator: GL2PS 1.4.0, (C) 1999-2017 C. Geuzaine
% For: Octave
% CreationDate: Sat Nov 27 16:17:20 2021
\setlength{\unitlength}{1pt}
\begin{picture}(0,0)
\includegraphics{chapters/challenge6/images/resultado-5-questao-10-inc}
\end{picture}%
\begin{picture}(400,200)(0,0)
\fontsize{6}{0}
\selectfont\put(52,116.3){\makebox(0,0)[t]{\textcolor[rgb]{0.15,0.15,0.15}{{0}}}}
\fontsize{6}{0}
\selectfont\put(96.2065,116.3){\makebox(0,0)[t]{\textcolor[rgb]{0.15,0.15,0.15}{{20}}}}
\fontsize{6}{0}
\selectfont\put(140.414,116.3){\makebox(0,0)[t]{\textcolor[rgb]{0.15,0.15,0.15}{{40}}}}
\fontsize{6}{0}
\selectfont\put(184.62,116.3){\makebox(0,0)[t]{\textcolor[rgb]{0.15,0.15,0.15}{{60}}}}
\fontsize{6}{0}
\selectfont\put(228.827,116.3){\makebox(0,0)[t]{\textcolor[rgb]{0.15,0.15,0.15}{{80}}}}
\fontsize{6}{0}
\selectfont\put(273.034,116.3){\makebox(0,0)[t]{\textcolor[rgb]{0.15,0.15,0.15}{{100}}}}
\fontsize{6}{0}
\selectfont\put(317.241,116.3){\makebox(0,0)[t]{\textcolor[rgb]{0.15,0.15,0.15}{{120}}}}
\fontsize{6}{0}
\selectfont\put(361.447,116.3){\makebox(0,0)[t]{\textcolor[rgb]{0.15,0.15,0.15}{{140}}}}
\fontsize{6}{0}
\selectfont\put(48.5278,129.586){\makebox(0,0)[r]{\textcolor[rgb]{0.15,0.15,0.15}{{-0.1}}}}
\fontsize{6}{0}
\selectfont\put(48.5278,144.968){\makebox(0,0)[r]{\textcolor[rgb]{0.15,0.15,0.15}{{-0.05}}}}
\fontsize{6}{0}
\selectfont\put(48.5278,160.35){\makebox(0,0)[r]{\textcolor[rgb]{0.15,0.15,0.15}{{0}}}}
\fontsize{6}{0}
\selectfont\put(48.5278,175.732){\makebox(0,0)[r]{\textcolor[rgb]{0.15,0.15,0.15}{{0.05}}}}
\fontsize{7}{0}
\selectfont\put(27.5278,150.666){\rotatebox{90}{\makebox(0,0)[b]{\textcolor[rgb]{0.15,0.15,0.15}{{Saida $y_1(t) - \hat{y}_1(t)$}}}}}
\fontsize{7}{0}
\selectfont\put(207,105.3){\makebox(0,0)[t]{\textcolor[rgb]{0.15,0.15,0.15}{{Tempo (s)}}}}
\fontsize{6}{0}
\selectfont\put(52,20){\makebox(0,0)[t]{\textcolor[rgb]{0.15,0.15,0.15}{{0}}}}
\fontsize{6}{0}
\selectfont\put(96.2065,20){\makebox(0,0)[t]{\textcolor[rgb]{0.15,0.15,0.15}{{20}}}}
\fontsize{6}{0}
\selectfont\put(140.414,20){\makebox(0,0)[t]{\textcolor[rgb]{0.15,0.15,0.15}{{40}}}}
\fontsize{6}{0}
\selectfont\put(184.62,20){\makebox(0,0)[t]{\textcolor[rgb]{0.15,0.15,0.15}{{60}}}}
\fontsize{6}{0}
\selectfont\put(228.827,20){\makebox(0,0)[t]{\textcolor[rgb]{0.15,0.15,0.15}{{80}}}}
\fontsize{6}{0}
\selectfont\put(273.034,20){\makebox(0,0)[t]{\textcolor[rgb]{0.15,0.15,0.15}{{100}}}}
\fontsize{6}{0}
\selectfont\put(317.241,20){\makebox(0,0)[t]{\textcolor[rgb]{0.15,0.15,0.15}{{120}}}}
\fontsize{6}{0}
\selectfont\put(361.447,20){\makebox(0,0)[t]{\textcolor[rgb]{0.15,0.15,0.15}{{140}}}}
\fontsize{6}{0}
\selectfont\put(48.5278,30.6553){\makebox(0,0)[r]{\textcolor[rgb]{0.15,0.15,0.15}{{-0.08}}}}
\fontsize{6}{0}
\selectfont\put(48.5278,38.064){\makebox(0,0)[r]{\textcolor[rgb]{0.15,0.15,0.15}{{-0.06}}}}
\fontsize{6}{0}
\selectfont\put(48.5278,45.4722){\makebox(0,0)[r]{\textcolor[rgb]{0.15,0.15,0.15}{{-0.04}}}}
\fontsize{6}{0}
\selectfont\put(48.5278,52.8809){\makebox(0,0)[r]{\textcolor[rgb]{0.15,0.15,0.15}{{-0.02}}}}
\fontsize{6}{0}
\selectfont\put(48.5278,60.2891){\makebox(0,0)[r]{\textcolor[rgb]{0.15,0.15,0.15}{{0}}}}
\fontsize{6}{0}
\selectfont\put(48.5278,67.6978){\makebox(0,0)[r]{\textcolor[rgb]{0.15,0.15,0.15}{{0.02}}}}
\fontsize{6}{0}
\selectfont\put(48.5278,75.106){\makebox(0,0)[r]{\textcolor[rgb]{0.15,0.15,0.15}{{0.04}}}}
\fontsize{6}{0}
\selectfont\put(48.5278,82.5146){\makebox(0,0)[r]{\textcolor[rgb]{0.15,0.15,0.15}{{0.06}}}}
\fontsize{7}{0}
\selectfont\put(27.5278,54.3662){\rotatebox{90}{\makebox(0,0)[b]{\textcolor[rgb]{0.15,0.15,0.15}{{Saida $y_2(t) - \hat{y}_2(t)$}}}}}
\fontsize{7}{0}
\selectfont\put(207,9){\makebox(0,0)[t]{\textcolor[rgb]{0.15,0.15,0.15}{{Tempo (s)}}}}
\end{picture}

    \end{minipage}
\end{figure}

\subsection{Conclusões}

Os resultados obtidos e apresentados na seção anterior ratificam a importância
do uso da realização em espaço de estados para o controle de processos
multivariáveis. Foi notório a facilidade para projetar um controlador para
seguimento de referência e rejeição de perturbações cujo os sinais de controle
são aplicados simultaneamente a dois processos que além de terem suas dinâmicas
internas, sofrem ainda com o acoplamento da dinâmica devido a interconexão dos
tanques. Observa-se ainda como é possível alterar a dinâmica do processo através
apenas de análises e manipulações algébricas, as quais estão fortemente
consolidadas na teoria e aplicação da ciência exatas. Em suma, os resultados
obtidos comprovaram a importância do conhecimento da realização de espaço de
estados na resolução de problemas complexo de controle, principalmente controle
multivariável que é o exemplo dos tanques interconectados em questão.
