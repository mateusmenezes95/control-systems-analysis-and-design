\section{Desafio I - Controle Multivariável - Espaço de Estados}

\subsection{Motivação}
(Explicar o problema que motiva o desafio, relevância, possíveis aplicações...)

\subsection{Simulações realizadas}

\begin{equation}
    \label{eq:matriz-de-fts}
    \mathbf{G}(s) =
    \begin{bmatrix}
        \frac{2}{(10s+1)}         & \frac{0,8}{(10s+1)(2s+1)} \\
        \frac{0,6}{(10s+1)(2s+1)} & \frac{2}{(10s+1)}
    \end{bmatrix}
\end{equation}
(Explicar quais simulações foram realizadas de maneira descritiva e sequêncial.)

Algumas definições:

\begin{itemize}
    \item O acento \textbf{~} é utilizado para as variáveis em tempo discreto
    \item $u[k] := u(kT_a)$ para $k = 0, 1, 2, ..., n$ em que $T_a$ é o período
          amostragem do sistema amostrado
    \item A simulação foi realizada utilizando a aproximação por euler
\end{itemize}


\subsection{Resultados obtidos}
\label{sub:resultados-obtidos-desafio6}

\subsubsection{Cálculo do ponto de equilíbrio}
\label{subsub:calculo-do-ponto-de-equilibrio}
Dado que o sistema multivariável é caracterizado por \ref{eq:matriz-de-fts}, a
saída do sistema é $\mathbf{y} = [Y_1(s) \thickspace Y_2(s)]^{T}$ dado o vetor de
sinais de controle $\mathbf{u} = [U_1(s) \thickspace U_2(s)]^{T}$, temos então que
$\mathbf{y} = \mathbf{G}\mathbf{u}$ ou

\begin{equation}
    \label{eq:relacao-entrada-saida-de-g}
    \begin{bmatrix}
        Y_1(s) \\
        Y_2(s)
    \end{bmatrix}
    =
    \begin{bmatrix}
        \frac{2}{(10s+1)}         & \frac{0,8}{(10s+1)(2s+1)} \\
        \frac{0,6}{(10s+1)(2s+1)} & \frac{2}{(10s+1)}
    \end{bmatrix}
    \begin{bmatrix}
        U_1(s) \\
        U_2(s)
    \end{bmatrix}.
\end{equation}

Quando aplicado o sinal de controle $\mathbf{u}_{eq} = \lim_{t \rightarrow
        \infty }\mathbf{u}(t)$, a saída do sistema chega ao equilíbrio $\mathbf{y}_{eq}
    = \lim_{t \rightarrow \infty }\mathbf{y}(t)$. Aplicando o teorema do valor
final em \ref{eq:relacao-entrada-saida-de-g} dado uma entrada em degrau para
$\mathbf{u}(t)$, temos que $\mathbf{u}_{eq} = \mathbf{G}(0)^{-1}\mathbf{y}_{eq}$
se $det(\mathbf{G}(0)) \neq 0$. Como,

\begin{equation}
    \label{eq:ganho-estatico-de-g}
    \mathbf{G}(0)
    =
    \begin{bmatrix}
        2,0 & 0,8 \\
        0,6 & 1,5
    \end{bmatrix},
\end{equation}logo, det($\mathbf{G}(0)$) $\neq 0$ e $\mathbf{G}(0)^{-1}$ existe.
Dado a saída de equilíbrio desejada $\mathbf{y}_{eq} = [2 \thickspace 1]^\top$,
o sinal de controle de equilíbrio

\begin{equation}
    \label{eq:vetor-do-controle-de-equilibrio}
    \mathbf{u}_{eq} = [0,8730 \thickspace 0,3175]^\top.
\end{equation}

\subsubsection{Espaço de Estados no domínio de tempo continuo}
\label{subsub:espaco-de-estados-no-dominio-de-tempo-continuo}

A utilização da função \textit{ss} do Octave para transformar o modelo
\ref{eq:matriz-de-fts} dado no domínio de Laplace para realização em espaço de
estados

\begin{subequations}
    \label{eq:espaco-de-estados-continuo}
    \begin{align}
        \mathbf{\dot{x}}(t) & = \mathbf{A}\mathbf{x}(t) + \mathbf{B}\mathbf{u}(t)
        \label{eq:derivada-do-vetor-de-estados}                                   \\
        \mathbf{y}(t)       & = \mathbf{C}\mathbf{x}(t) + \mathbf{D}\mathbf{u}(t)
        \label{eq:saida-do-sistema-em-espaco-de-estados}
    \end{align}
\end{subequations} resulta nas matrizes $\mathbf{A}_{4\times 4}$,
$\mathbf{B}_{4\times 2}$ e $\mathbf{C}_{2\times 4}$, cujo elementos estão
definidas em \ref{eq:matrizes-do-espaco-de-estados}. Ressalta-se entretanto, que
$\mathbf{D} = \mathbf{\bar{0}}$ para o sistema deste desafio já que não há
transmissão direta do sinal de controle para a saída. Observa-se das dimensões
destas matrizes, que para o sistema dado existem 4 variáveis de estados.

\begin{subequations}
    \label{eq:matrizes-do-espaco-de-estados}
    \begin{align}
        \mathbf{A} & =
        \begin{bmatrix}
            \label{eq:matriz-a}
            0,0500 & 0       & 0,1164  & 0       \\
            0      & 0,0500  & 0       & 0,1164  \\
            0,7089 & 0       & -0,6500 & 0       \\
            0      & -0,7089 & 0       & -0,6500
        \end{bmatrix},  \\
        \mathbf{B} & =
        \begin{bmatrix}
            \label{eq:matriz-b}
            -0,0849 & -0,2333 \\
            -0,3111 & -0,1131 \\
            0       & 0,3007  \\
            0,4010  & 0
        \end{bmatrix}, \\
        \mathbf{C} & =
        \begin{bmatrix}
            \label{eq:matriz-c}
            0 & 0 & 0      & 0,4988 \\
            0 & 0 & 0,4988 & 0
        \end{bmatrix} 
    \end{align}
\end{subequations}

Com estas matrizes é possível calcular quais são os estados do sistema quando a
saída atinge seu valor de equilíbrio. Quando isto acontece, $\mathbf{\dot{x}}(t)
= \mathbf{0}$ e, portanto, a partir de \ref{eq:derivada-do-vetor-de-estados},

\begin{equation}
    \label{eq:calculo-do-vetor-de-estados-de-equilibrio}
    \mathbf{x}_{eq} = -\mathbf{A}^{-1}\mathbf{B}\mathbf{u}_{eq}.
\end{equation} Vale reforçar que isto é verdadeiro se $\mathbf{A}$ é invertível
ou det($\mathbf{A}$) $\neq 0$. De \ref{eq:matriz-a}, verifica-se que
det($\mathbf{A}$) $\neq 0$. Assim, substituindo
\ref{eq:vetor-do-controle-de-equilibrio}, \ref{eq:matriz-a} e \ref{eq:matriz-b}
em \ref{eq:calculo-do-vetor-de-estados-de-equilibrio}, tem se que

\begin{equation}
    \label{eq:estados-de-equilibrio}
    \mathbf{x}_{eq} =
    \begin{bmatrix}
        x_{eq_{1}} \\
        x_{eq_{2}} \\
        x_{eq_{3}} \\
        x_{eq_{4}}
    \end{bmatrix}
    =
    \begin{bmatrix}
        -1,7038 \\
        -3,1831 \\
        2,0050  \\
        4,0100
    \end{bmatrix}\text{.}
\end{equation}

\subsubsection{Espaço de Estado no domínio de tempo discreto}
\label{subsub:espaco-de-estados-no-dominio-de-tempo-discreto}

A versão em tempo discreto para a representação em espaço de estados definido em
\ref{eq:espaco-de-estados-continuo} é dado da seguinte forma

\begin{subequations}
    \label{eq:espaco-de-estados-discreto}
    \begin{align}
        \mathbf{x}[k+1] & = \mathbf{\tilde{A}}\mathbf{x}[k] + \mathbf{\tilde{B}}\mathbf{u}[k]
        \label{eq:estimativa-do-vetor-de-estados}                                             \\
        \mathbf{y}[k]   & = \mathbf{\tilde{C}}\mathbf{x}[t] + \mathbf{\tilde{D}}\mathbf{u}[k]
        \label{eq:saida-do-sistema-em-espaco-de-estados-discreto}
    \end{align}
\end{subequations} que segundo \citeonline{Chen2006}, a relação entre as
matrizes de tempo continuo e discreto é dado como

\begin{equation*}
    \mathbf{\tilde{A}} = e^{\mathbf{A}T},
    \thickspace
    \mathbf{\tilde{B}} = \int_{0}^{T}e^{\mathbf{A}\tau}d\tau \mathbf{B},
    \thickspace
    \mathbf{\tilde{C}} = \mathbf{C}
    \thickspace \text{e} \thickspace
    \mathbf{\tilde{D}} = \mathbf{D},
\end{equation*} em que $T$ é o período de amostragem do sistema de controle
amostrado. Observa-se portanto que é necessário definir um período de amostragem
adequado para ter a realização do modelo \ref{eq:matriz-de-fts} em espaço de
estados de tempo discreto.

O critério para definição do tempo de amostragem foi igual ao utilizado no
desafio 4 de projeto de controladores no tempo discreto (seção
\ref{sub:simulacoes-realizadas-desafio4}). Entretanto, o modelo em questão é
multivariável. Considerando então, que $\mathbf{W}$ é uma matriz cujo elementos
$w_{ij}$ é a frequência de corte de cada função de transferência $g_{ij}$ de
$\mathbf{G}(s)$, optou-se por aplicar o critério na máxima frequência de corte,
ou seja $w_a = 30\times\max(\mathbf{W})$, em que $w_a$ é a frequência de
amostragem em rad/s. Esta abordagem é conservadora, pois os sistemas cuja a
frequência de corte $w_c < \max(\mathbf{W})$ por si só irão atenuar sinais de
alta frequências, e então estes sinais não contribuirão significativamente em
suas respostas. Ou de forma matemática, se $w_a = 30\times\max(\mathbf{W})$,
logo o critério de $30\times$ será válido para as demais plantas. Assim,
obteve-se a matriz $\mathbf{W}$ conforme
\ref{eq:matriz-das-frequencias-de-corte}. Percebe-se que $w_{21} = 0$ pois
$|g_{21}(0)| < -3$dB e esta atribuição não compromete o critério da
definição do período de amostragem.

\begin{equation}
    \label{eq:matriz-das-frequencias-de-corte}
    \mathbf{W} =
    \begin{bmatrix}
        0,2646 & 0,0516 \\
        0      & 0,1871
    \end{bmatrix}
\end{equation}

Portanto, $\max(\mathbf{W}) = 0,2646$ rad/s, o que resultou, após
arrendondamento, num período de amostragem $T_a = 0,75$s.

Novamente com o auxílio da função \textit{ss} do Octave, mas agora convertendo
as matrizes para realização em espaço de estados de tempo discreto utilizando a
função \textit{c2d} passando como argumento período de amostragem definido,
chegou-se nas seguintes matrizes:

\begin{subequations}
    \label{eq:matrizes-do-espaco-de-estados-discretizadas}
    \begin{align}
        \mathbf{\tilde{A}} & =
        \begin{bmatrix}
            \label{eq:matriz-til-a}
            1,0179 & 0       & 0,069961 & 0      \\
            0       & 1,0179 & 0      & 0,069961 \\
            -0,42613 & 0       & 0,59712 & 0      \\
            0       & -0,42613  & 0     & 0,59712
        \end{bmatrix}\text{,} \\
        \mathbf{\tilde{B}} & =
        \begin{bmatrix}
            \label{eq:matriz-til-b}
            -0,064403   & -0,16861  \\
            -0,22481    & -0,08587  \\
            0,014608    &  0,21731  \\
            0,28975     & 0,019477
        \end{bmatrix}\text{,} \\
        \mathbf{\tilde{C}} &= \mathbf{C} \\
        \mathbf{\tilde{D}} &= \mathbf{D}
    \end{align}
\end{subequations}

Com a definição destas matrizes é possível definir os estados de equilíbrio dado
o vetor do sinal de controle em equilíbrio
\ref{eq:vetor-do-controle-de-equilibrio}. Salienta-se entretanto que para o
tempo discreto, o estado de equilíbrio é alcançado quando o estado atual é igual
ao estado anterior, ou de forma algébrica, a partir de
\ref{eq:estimativa-do-vetor-de-estados},

\begin{equation}
    \label{eq:calculo-dos-estados-de-equilibrio-em-tempo-discreto}
    \mathbf{\tilde{x}}_{eq} = \mathbf{\tilde{A}}\mathbf{\tilde{x}}_{eq} + \mathbf{\tilde{B}}\mathbf{u}_{eq}
    \Rightarrow
    \mathbf{\tilde{x}}_{eq} = (\mathbf{I}-\mathbf{\tilde{A}})^{-1}\mathbf{\tilde{B}}\mathbf{u}_{eq}.
\end{equation} Portanto, substituindo \ref{eq:vetor-do-controle-de-equilibrio},
\ref{eq:matriz-til-a} e \ref{eq:matriz-til-b} em
\ref{eq:calculo-dos-estados-de-equilibrio-em-tempo-discreto}, chegou-se ao mesmo
resultado de tempo contínuo conforme \ref{eq:estados-de-equilibrio}. O mesmo
aconteceu para a saída $\mathbf{\tilde{y}}_{eq} =
\mathbf{\tilde{C}}\mathbf{\tilde{x}}_{eq}$, o que é coerente já que
$\mathbf{\tilde{x}}_{eq} = \mathbf{x}_{eq}$ e $\mathbf{\tilde{C}} = \mathbf{C}$.

\subsubsection{Regulação em torno do ponto de equilíbrio}
\label{subsub:regulacao-em-torno-do-ponto-de-equilibrio}

As seções anteriores abordaram a análise do estado de equilíbrio assumindo que o
sistema já estava na sua condição de equilíbrio. Entretanto, para levar o sistema
ao equilíbrio é preciso aplicar uma ação de controle de regulação em torno do
ponto de equilíbrio, dada por

\begin{equation}
    \label{eq:acao-de-controle-em-torno-do-ponto-de-equilibrio}
    \mathbf{u}(kT_a) = -\mathbf{\tilde{K}}[\mathbf{x}(kT_a) - \mathbf{x_{eq}}]+\mathbf{u}_{eq}, 
    \thickspace k = 0, 1, 2, ..., n.
\end{equation}

Substituindo \ref{eq:acao-de-controle-em-torno-do-ponto-de-equilibrio} em
\ref{eq:estimativa-do-vetor-de-estados}, já que $u[k] := u(kT_a)$ e $x[k] :=
x(kT_a)$, o novo polinômio característico é dado por
det($z\mathbf{I}-\mathbf{\tilde{A}}-\mathbf{\tilde{B}}\mathbf{\tilde{K}}$) e
portanto os polos do sistema discreto podem ser modificados para um valor
desejável. Entretanto, isto só é válido se o sistema for controlável, isto é, a
matriz de controlabilidade $\mathbf{\tilde{U}}$ dado por
\ref{eq:matriz-de-controlabilidade-de-tempo-discreto} tem todas as linhas
linearmente independentes.

\begin{equation}
    \label{eq:matriz-de-controlabilidade-de-tempo-discreto}
    \mathbf{\tilde{U}} = [
    \mathbf{\tilde{B}}
    \thickspace
    \mathbf{\tilde{A}\tilde{B}} 
    \thickspace
    \mathbf{\tilde{A}}\negthinspace^{2}\mathbf{\tilde{B}}
    \thickspace
    \mathbf{\tilde{A}}\negthinspace^{3}\mathbf{\tilde{B}}]
\end{equation}

A aplicação das matrizes \ref{eq:matriz-til-a} e \ref{eq:matriz-til-b} em
\ref{eq:matriz-de-controlabilidade-de-tempo-discreto} resulta em uma matriz
$\mathbf{\tilde{U}}$ cujo posto é igual 4, que é a mesma ordem do sistema e,
portanto, todas as linhas são linearmente independentes. Assim, todos os
autovalores podem ser definidos escolhendo um $\mathbf{\tilde{K}}$ adequado.

A definição do valor de $\mathbf{\tilde{K}}$ foi realizada através de um método
de alocação de autovalores em espaço de estados. Existem vários métodos na
literatura em que usam uma transformação linear de similaridade para facilitar o
cálculo do ganho $\mathbf{\tilde{K}}$. Alguns destes métodos já são
disponibilizados computacionalmente em funções, como é o caso da função
\textit{place} do Octave/Matlab. Dado então o requisito de que os autovalores de
malha fechada de tempo discreto, ou seja os autovalores da matriz
$\mathbf{\tilde{A} - \tilde{B}\tilde{K}}$ devem ser $\lambda_i =
e^{\frac{-T_a}{4}}$ para $i = 1, 2, 3$ e $4$, em que dado $T_a = 0,75$s,
$\lambda_i = 0,8290$. Utilizando portanto a função \textit{place} tendo como
argumento as matrizes $\mathbf{\tilde{A}}$ e $\mathbf{\tilde{B}}$ e o vetor
$\mathbf{\lambda}$ contendo os autovalores desejáveis $\lambda_i$, foi obtida a
matriz de ganhos

\begin{equation}
    \label{eq:matriz-de-ganhos-discreto}
    \mathbf{\tilde{K}} = 
    \begin{bmatrix}
        1,663191 & -0,464388 &  1,058965 & -0,045131    \\
        0,594665 &  1,418127 &  0,522660 &  1,167112
    \end{bmatrix}.
\end{equation}

O resultado da simulação da realização em espaço de estados dado o
$\mathbf{\tilde{K}}$ encontrado é mostrado na Figura
\ref{fig:resultado-do-regulador-via-realimentacao-de-estados}. Percebe-se nesta
figura que o sistema atinge seu valor de equilíbrio dado por $\mathbf{u}_{eq} =
[0,8730 \thickspace 0,3175]^\top$ e $\mathbf{y}_{eq} = [2 \thickspace 1]^\top$.
Repara-se também a presença de um sobressinal. Como o sistema em questão trata
de dois tanques interconectados, a interação entre suas dinâmicas faz surgir
zeros que alteram de forma significativa a saída do sistema, mesmo com
autovalores reais, o que explica o sobressinal nas respostas $y_1(t)$ e $y_1(t)$
obtidas.

\begin{figure}[!htp]
    \caption{Saídas e sinais de controle do sistema MIMO de tanques
    interconectados utilizando regulação via realimentação de estados.}
    \vspace{-10pt}
    \hspace{-30pt}
    \label{fig:resultado-do-regulador-via-realimentacao-de-estados}
    \begin{minipage}{\linewidth}
        % Title: gl2ps_renderer figure
% Creator: GL2PS 1.4.0, (C) 1999-2017 C. Geuzaine
% For: Octave
% CreationDate: Thu Nov 25 18:14:01 2021
\setlength{\unitlength}{1pt}
\begin{picture}(0,0)
\includegraphics{chapters/challenge6/images/resultado-questao-7-inc}
\end{picture}%
\begin{picture}(400,250)(0,0)
\fontsize{6}{0}
\selectfont\put(52,141.404){\makebox(0,0)[t]{\textcolor[rgb]{0.15,0.15,0.15}{{0}}}}
\fontsize{6}{0}
\selectfont\put(70.4365,141.404){\makebox(0,0)[t]{\textcolor[rgb]{0.15,0.15,0.15}{{20}}}}
\fontsize{6}{0}
\selectfont\put(88.873,141.404){\makebox(0,0)[t]{\textcolor[rgb]{0.15,0.15,0.15}{{40}}}}
\fontsize{6}{0}
\selectfont\put(107.309,141.404){\makebox(0,0)[t]{\textcolor[rgb]{0.15,0.15,0.15}{{60}}}}
\fontsize{6}{0}
\selectfont\put(125.746,141.404){\makebox(0,0)[t]{\textcolor[rgb]{0.15,0.15,0.15}{{80}}}}
\fontsize{6}{0}
\selectfont\put(144.182,141.404){\makebox(0,0)[t]{\textcolor[rgb]{0.15,0.15,0.15}{{100}}}}
\fontsize{6}{0}
\selectfont\put(162.619,141.404){\makebox(0,0)[t]{\textcolor[rgb]{0.15,0.15,0.15}{{120}}}}
\fontsize{6}{0}
\selectfont\put(181.055,141.404){\makebox(0,0)[t]{\textcolor[rgb]{0.15,0.15,0.15}{{140}}}}
\fontsize{6}{0}
\selectfont\put(48.5205,153.193){\makebox(0,0)[r]{\textcolor[rgb]{0.15,0.15,0.15}{{0}}}}
\fontsize{6}{0}
\selectfont\put(48.5205,169.582){\makebox(0,0)[r]{\textcolor[rgb]{0.15,0.15,0.15}{{0.5}}}}
\fontsize{6}{0}
\selectfont\put(48.5205,185.97){\makebox(0,0)[r]{\textcolor[rgb]{0.15,0.15,0.15}{{1}}}}
\fontsize{6}{0}
\selectfont\put(48.5205,202.359){\makebox(0,0)[r]{\textcolor[rgb]{0.15,0.15,0.15}{{1.5}}}}
\fontsize{6}{0}
\selectfont\put(48.5205,218.748){\makebox(0,0)[r]{\textcolor[rgb]{0.15,0.15,0.15}{{2}}}}
\fontsize{7}{0}
\selectfont\put(33.5205,186.325){\rotatebox{90}{\makebox(0,0)[b]{\textcolor[rgb]{0.15,0.15,0.15}{{Saida $y_1(t)$}}}}}
\fontsize{7}{0}
\selectfont\put(116.643,130.404){\makebox(0,0)[t]{\textcolor[rgb]{0.15,0.15,0.15}{{Tempo (s)}}}}
\fontsize{6}{0}
\selectfont\put(232.715,141.404){\makebox(0,0)[t]{\textcolor[rgb]{0.15,0.15,0.15}{{0}}}}
\fontsize{6}{0}
\selectfont\put(251.151,141.404){\makebox(0,0)[t]{\textcolor[rgb]{0.15,0.15,0.15}{{20}}}}
\fontsize{6}{0}
\selectfont\put(269.587,141.404){\makebox(0,0)[t]{\textcolor[rgb]{0.15,0.15,0.15}{{40}}}}
\fontsize{6}{0}
\selectfont\put(288.024,141.404){\makebox(0,0)[t]{\textcolor[rgb]{0.15,0.15,0.15}{{60}}}}
\fontsize{6}{0}
\selectfont\put(306.46,141.404){\makebox(0,0)[t]{\textcolor[rgb]{0.15,0.15,0.15}{{80}}}}
\fontsize{6}{0}
\selectfont\put(324.896,141.404){\makebox(0,0)[t]{\textcolor[rgb]{0.15,0.15,0.15}{{100}}}}
\fontsize{6}{0}
\selectfont\put(343.333,141.404){\makebox(0,0)[t]{\textcolor[rgb]{0.15,0.15,0.15}{{120}}}}
\fontsize{6}{0}
\selectfont\put(361.77,141.404){\makebox(0,0)[t]{\textcolor[rgb]{0.15,0.15,0.15}{{140}}}}
\fontsize{6}{0}
\selectfont\put(229.235,146.637){\makebox(0,0)[r]{\textcolor[rgb]{0.15,0.15,0.15}{{-0.2}}}}
\fontsize{6}{0}
\selectfont\put(229.235,157.638){\makebox(0,0)[r]{\textcolor[rgb]{0.15,0.15,0.15}{{0}}}}
\fontsize{6}{0}
\selectfont\put(229.235,168.638){\makebox(0,0)[r]{\textcolor[rgb]{0.15,0.15,0.15}{{0.2}}}}
\fontsize{6}{0}
\selectfont\put(229.235,179.638){\makebox(0,0)[r]{\textcolor[rgb]{0.15,0.15,0.15}{{0.4}}}}
\fontsize{6}{0}
\selectfont\put(229.235,190.638){\makebox(0,0)[r]{\textcolor[rgb]{0.15,0.15,0.15}{{0.6}}}}
\fontsize{6}{0}
\selectfont\put(229.235,201.639){\makebox(0,0)[r]{\textcolor[rgb]{0.15,0.15,0.15}{{0.8}}}}
\fontsize{6}{0}
\selectfont\put(229.235,212.639){\makebox(0,0)[r]{\textcolor[rgb]{0.15,0.15,0.15}{{1}}}}
\fontsize{6}{0}
\selectfont\put(229.235,223.639){\makebox(0,0)[r]{\textcolor[rgb]{0.15,0.15,0.15}{{1.2}}}}
\fontsize{7}{0}
\selectfont\put(212.235,186.325){\rotatebox{90}{\makebox(0,0)[b]{\textcolor[rgb]{0.15,0.15,0.15}{{Saida $y_2(t)$}}}}}
\fontsize{7}{0}
\selectfont\put(297.357,130.404){\makebox(0,0)[t]{\textcolor[rgb]{0.15,0.15,0.15}{{Tempo (s)}}}}
\fontsize{6}{0}
\selectfont\put(52,22.2671){\makebox(0,0)[t]{\textcolor[rgb]{0.15,0.15,0.15}{{0}}}}
\fontsize{6}{0}
\selectfont\put(70.4365,22.2671){\makebox(0,0)[t]{\textcolor[rgb]{0.15,0.15,0.15}{{20}}}}
\fontsize{6}{0}
\selectfont\put(88.873,22.2671){\makebox(0,0)[t]{\textcolor[rgb]{0.15,0.15,0.15}{{40}}}}
\fontsize{6}{0}
\selectfont\put(107.309,22.2671){\makebox(0,0)[t]{\textcolor[rgb]{0.15,0.15,0.15}{{60}}}}
\fontsize{6}{0}
\selectfont\put(125.746,22.2671){\makebox(0,0)[t]{\textcolor[rgb]{0.15,0.15,0.15}{{80}}}}
\fontsize{6}{0}
\selectfont\put(144.182,22.2671){\makebox(0,0)[t]{\textcolor[rgb]{0.15,0.15,0.15}{{100}}}}
\fontsize{6}{0}
\selectfont\put(162.619,22.2671){\makebox(0,0)[t]{\textcolor[rgb]{0.15,0.15,0.15}{{120}}}}
\fontsize{6}{0}
\selectfont\put(181.055,22.2671){\makebox(0,0)[t]{\textcolor[rgb]{0.15,0.15,0.15}{{140}}}}
\fontsize{6}{0}
\selectfont\put(48.5205,36.0054){\makebox(0,0)[r]{\textcolor[rgb]{0.15,0.15,0.15}{{0}}}}
\fontsize{6}{0}
\selectfont\put(48.5205,57.269){\makebox(0,0)[r]{\textcolor[rgb]{0.15,0.15,0.15}{{0.5}}}}
\fontsize{6}{0}
\selectfont\put(48.5205,78.5322){\makebox(0,0)[r]{\textcolor[rgb]{0.15,0.15,0.15}{{1}}}}
\fontsize{6}{0}
\selectfont\put(48.5205,99.7959){\makebox(0,0)[r]{\textcolor[rgb]{0.15,0.15,0.15}{{1.5}}}}
\fontsize{7}{0}
\selectfont\put(33.5205,67.1875){\rotatebox{90}{\makebox(0,0)[b]{\textcolor[rgb]{0.15,0.15,0.15}{{Sinal de Controle $u_1(t)$}}}}}
\fontsize{7}{0}
\selectfont\put(116.643,11.2671){\makebox(0,0)[t]{\textcolor[rgb]{0.15,0.15,0.15}{{Tempo (s)}}}}
\fontsize{6}{0}
\selectfont\put(232.715,22.2671){\makebox(0,0)[t]{\textcolor[rgb]{0.15,0.15,0.15}{{0}}}}
\fontsize{6}{0}
\selectfont\put(251.151,22.2671){\makebox(0,0)[t]{\textcolor[rgb]{0.15,0.15,0.15}{{20}}}}
\fontsize{6}{0}
\selectfont\put(269.587,22.2671){\makebox(0,0)[t]{\textcolor[rgb]{0.15,0.15,0.15}{{40}}}}
\fontsize{6}{0}
\selectfont\put(288.024,22.2671){\makebox(0,0)[t]{\textcolor[rgb]{0.15,0.15,0.15}{{60}}}}
\fontsize{6}{0}
\selectfont\put(306.46,22.2671){\makebox(0,0)[t]{\textcolor[rgb]{0.15,0.15,0.15}{{80}}}}
\fontsize{6}{0}
\selectfont\put(324.896,22.2671){\makebox(0,0)[t]{\textcolor[rgb]{0.15,0.15,0.15}{{100}}}}
\fontsize{6}{0}
\selectfont\put(343.333,22.2671){\makebox(0,0)[t]{\textcolor[rgb]{0.15,0.15,0.15}{{120}}}}
\fontsize{6}{0}
\selectfont\put(361.77,22.2671){\makebox(0,0)[t]{\textcolor[rgb]{0.15,0.15,0.15}{{140}}}}
\fontsize{6}{0}
\selectfont\put(229.235,27.5){\makebox(0,0)[r]{\textcolor[rgb]{0.15,0.15,0.15}{{-0.2}}}}
\fontsize{6}{0}
\selectfont\put(229.235,42.8911){\makebox(0,0)[r]{\textcolor[rgb]{0.15,0.15,0.15}{{0}}}}
\fontsize{6}{0}
\selectfont\put(229.235,58.2817){\makebox(0,0)[r]{\textcolor[rgb]{0.15,0.15,0.15}{{0.2}}}}
\fontsize{6}{0}
\selectfont\put(229.235,73.6729){\makebox(0,0)[r]{\textcolor[rgb]{0.15,0.15,0.15}{{0.4}}}}
\fontsize{6}{0}
\selectfont\put(229.235,89.064){\makebox(0,0)[r]{\textcolor[rgb]{0.15,0.15,0.15}{{0.6}}}}
\fontsize{6}{0}
\selectfont\put(229.235,104.455){\makebox(0,0)[r]{\textcolor[rgb]{0.15,0.15,0.15}{{0.8}}}}
\fontsize{7}{0}
\selectfont\put(212.235,67.1875){\rotatebox{90}{\makebox(0,0)[b]{\textcolor[rgb]{0.15,0.15,0.15}{{Sinal de Controle $u_2(t)$}}}}}
\fontsize{7}{0}
\selectfont\put(297.357,11.2671){\makebox(0,0)[t]{\textcolor[rgb]{0.15,0.15,0.15}{{Tempo (s)}}}}
\fontsize{6}{0}
\selectfont\put(299.842,238.48){\makebox(0,0)[l]{\textcolor[rgb]{0,0,0}{{equilibrio}}}}
\end{picture}

    \end{minipage}
\end{figure}

\subsubsection{Rejeição de perturbações e seguimento de referência}
\label{subsub:rejeicao-de-perturbacoes-e-seguimento-de-referencia}

Na subseção anterior foi abordado a regulação do sistema MIMO de tanques
interconectados para um ponto de equilíbrio utilizando realimentação de estados.
Embora esta abordagem tenha funcionado de forma teórica, na prática ela pode
levar a comportamentos indesejáveis devidos a erros de modelagem, perturbações
externas e ruídos. Uma das soluções para estes possíveis problemas é a
utilização da ação integral no controle da realimentação de estados, tópico
desta seção.

A ação de integração é alcançada fazendo aparecer no espaço de estados um
dinâmica interna, tal que a realização em espaços de estados de tempo discreto
\ref{eq:espaco-de-estados-discreto} torna-se a versão aumentada

\begin{equation}
    \label{eq:espaco-de-estados-discreto-aumentado}
    \begin{bmatrix}
        \Delta x[k]\\ 
        y[k+1]
    \end{bmatrix}
    =
    \begin{bmatrix}
        \mathbf{\tilde{A}} & \mathbf{\bar{0}}\\ 
        \mathbf{\tilde{C}}\mathbf{\tilde{A}} & \mathbf{I}
    \end{bmatrix}
    \begin{bmatrix}
        \Delta x[k]\\ 
        y[k]
    \end{bmatrix}
    +
    \begin{bmatrix}
        \mathbf{\tilde{B}}\\ 
        \mathbf{\tilde{C}}\mathbf{\tilde{A}}
    \end{bmatrix}
    \Delta \mathbf{u}[k]
\end{equation} e então as matrizes aumentadas são reescritas como

\begin{subequations}
    \begin{align}
        \mathbf{\bar{A}} &=
        \begin{bmatrix}
            \label{eq:matriz-aumentada-a}
            \mathbf{\tilde{A}} & \mathbf{\bar{0}}\\ 
            \mathbf{\tilde{C}}\mathbf{\tilde{A}} & \mathbf{I}
        \end{bmatrix}   \\
        \mathbf{\bar{B}} &=
        \begin{bmatrix}
            \label{eq:matriz-aumentada-b}
            \mathbf{\tilde{B}}\\ 
            \mathbf{\tilde{C}}\mathbf{\tilde{B}}
        \end{bmatrix}
    \end{align}
\end{subequations} em que para a realização de estados para o problema em
questão a ordem de $\mathbf{\bar{A}}$ é $6 \times 6$ e de $\mathbf{\bar{B}}$ é
$6 \times 2$. Portanto, substituindo \ref{eq:matriz-til-a},
\ref{eq:matriz-til-b} e \ref{eq:matriz-c} em \ref{eq:matriz-aumentada-a} e
\ref{eq:matriz-aumentada-b}, chegou-se as matrizes:

\begin{subequations}
        \begin{equation}
            \label{eq:matriz-aumentanda-a-com-valores}
            \mathbf{\bar{A}} =
            \begin{bmatrix}
                1,0179  &  0       &  0,0700  &  0       & 0       & 0      \\
                0       &  1,0179  &  0       &  0,0700  & 0       & 0      \\
                -0,4261 &  0       &  0,5971  &  0       & 0       & 0      \\
                0       &  -0,4261 &  0       &  0,5971  & 0       & 0      \\
                0       &  -0,2125 &  0       &  0,2978  & 1       & 0      \\
                -0,2125 &  0       &  0,2978  &  0       & 0       & 1
            \end{bmatrix}
        \end{equation}    
        \begin{equation}
            \label{eq:matriz-aumentanda-b-com-valores}
            \mathbf{\bar{B}} =
            \begin{bmatrix}
                -6.4403e^{-02} & -1.6861\times10^{-01} \\
                -2.2481e^{-01} & -8.5870\times10^{-02} \\
                1.4608e^{-02} &  2.1731\times10^{-01}  \\
                2.8975e^{-01} &  1.9477\times10^{-02}  \\
                1.4451e^{-01} &  9.7144\times10^{-03}  \\
                7.2858e^{-03} &  1.0838\times10^{-01}
            \end{bmatrix}
        \end{equation}
\end{subequations}

\begin{equation}
    \mathbf{\bar{A}} =
    \begin{bmatrix}
        1,0179 &  0,0000 & 0,0700 &  0,0000 &  0,0000 &   0,0000 \\
        0,0000 &  1,0179 & 0,0000 &  0,0700 &  0,0000 &   0,0000 \\
        -0,4261 &  0,0000 & 0,5971 & 0,0000 &  0,0000 &   0,0000 \\
        0,0000 & -0,4261 & 0,0000 &  0,5971 &  0,0000 &   0,0000 \\
        0,0000 & -0,2125 & 0,0000 &  0,2978 &  1,0000 &   0,0000 \\
        -0,2125 &  0,0000 & 0,2978 &  0,0000 & 0,0000 &  1,0000
    \end{bmatrix}
\end{equation}

Sabida a realização aumentada em espaço de estados acima, foi
utilizada a ação de controle

\begin{equation}
    \label{eq:acao-de-controle-com-integrador}
    \mathbf{u}[k] = \mathbf{u}[k-1] - \mathbf{K}_x(\mathbf{x}[k]-\mathbf{x}[k-1])+\mathbf{K}_i(\mathbf{y}_r[k]-\mathbf{y}[k])
\end{equation} em que a representação $\mathbf{v}[k] := \mathbf{v}(kT_a)$ e
$\mathbf{v}[k-1] := \mathbf{v}((k-1)T_a)$.

Assim é possível novamente alocar os autovalores da matriz
$\mathbf{\bar{A}}-\mathbf{\bar{B}}\mathbf{\bar{K}}$ sendo $\mathbf{\bar{K}} =
[\mathbf{K}_x \thickspace \mathbf{K}_i]$. Vale salientar, que a ordem do sistema
foi aumentada para $n=6$ e, desta forma, são alocados agora 6 autovalores.
Portanto, utilizando a função \textit{place} do Octave e considerando os 6
autovalores repetidos $\lambda_i = 0,8290$, foram obtidas as matrizes de ganho

\begin{subequations}
    \label{eq:matrizes-de-ganho}
    \begin{equation}
        \label{eq:matriz-de-ganhos-kx}
        \mathbf{K}_x =
        \begin{bmatrix}
            1.073066 & -0.786854 &  0.552769 & -0.022183   \\
            -1.049138 &  0.804800 & -0.029577 &  0.414577
        \end{bmatrix}
    \end{equation}
    \text{e}
    \begin{equation}
        \label{eq:matriz-de-ganhos-ki}
        \mathbf{K}_i =
        \begin{bmatrix}
            0.131655 & -0.070216 \\
            -0.052662 &  0.175540
        \end{bmatrix}\text{.}
    \end{equation}
\end{subequations}

\begin{figure}[!htp]
    \caption{Saídas e sinais de controle do sistema MIMO de tanques
    interconectados utilizando regulação via realimentação de estados.}
    \vspace{-10pt}
    \hspace{-30pt}
    \label{fig:resultado-do-regulador-com-integrador}
    \begin{minipage}{\linewidth}
        % Title: gl2ps_renderer figure
% Creator: GL2PS 1.4.0, (C) 1999-2017 C. Geuzaine
% For: Octave
% CreationDate: Thu Nov 25 17:46:16 2021
\setlength{\unitlength}{1pt}
\begin{picture}(0,0)
\includegraphics{chapters/challenge6/images/resultado-questao-9-inc}
\end{picture}%
\begin{picture}(400,250)(0,0)
\fontsize{6}{0}
\selectfont\put(52,141.404){\makebox(0,0)[t]{\textcolor[rgb]{0.15,0.15,0.15}{{0}}}}
\fontsize{6}{0}
\selectfont\put(70.4365,141.404){\makebox(0,0)[t]{\textcolor[rgb]{0.15,0.15,0.15}{{20}}}}
\fontsize{6}{0}
\selectfont\put(88.873,141.404){\makebox(0,0)[t]{\textcolor[rgb]{0.15,0.15,0.15}{{40}}}}
\fontsize{6}{0}
\selectfont\put(107.309,141.404){\makebox(0,0)[t]{\textcolor[rgb]{0.15,0.15,0.15}{{60}}}}
\fontsize{6}{0}
\selectfont\put(125.746,141.404){\makebox(0,0)[t]{\textcolor[rgb]{0.15,0.15,0.15}{{80}}}}
\fontsize{6}{0}
\selectfont\put(144.182,141.404){\makebox(0,0)[t]{\textcolor[rgb]{0.15,0.15,0.15}{{100}}}}
\fontsize{6}{0}
\selectfont\put(162.619,141.404){\makebox(0,0)[t]{\textcolor[rgb]{0.15,0.15,0.15}{{120}}}}
\fontsize{6}{0}
\selectfont\put(181.055,141.404){\makebox(0,0)[t]{\textcolor[rgb]{0.15,0.15,0.15}{{140}}}}
\fontsize{6}{0}
\selectfont\put(48.5205,151.661){\makebox(0,0)[r]{\textcolor[rgb]{0.15,0.15,0.15}{{0}}}}
\fontsize{6}{0}
\selectfont\put(48.5205,164.221){\makebox(0,0)[r]{\textcolor[rgb]{0.15,0.15,0.15}{{0.5}}}}
\fontsize{6}{0}
\selectfont\put(48.5205,176.78){\makebox(0,0)[r]{\textcolor[rgb]{0.15,0.15,0.15}{{1}}}}
\fontsize{6}{0}
\selectfont\put(48.5205,189.34){\makebox(0,0)[r]{\textcolor[rgb]{0.15,0.15,0.15}{{1.5}}}}
\fontsize{6}{0}
\selectfont\put(48.5205,201.899){\makebox(0,0)[r]{\textcolor[rgb]{0.15,0.15,0.15}{{2}}}}
\fontsize{6}{0}
\selectfont\put(48.5205,214.459){\makebox(0,0)[r]{\textcolor[rgb]{0.15,0.15,0.15}{{2.5}}}}
\fontsize{7}{0}
\selectfont\put(33.5205,186.325){\rotatebox{90}{\makebox(0,0)[b]{\textcolor[rgb]{0.15,0.15,0.15}{{Saida $y_1(t)$}}}}}
\fontsize{7}{0}
\selectfont\put(116.643,130.404){\makebox(0,0)[t]{\textcolor[rgb]{0.15,0.15,0.15}{{Tempo (s)}}}}
\fontsize{6}{0}
\selectfont\put(232.715,141.404){\makebox(0,0)[t]{\textcolor[rgb]{0.15,0.15,0.15}{{0}}}}
\fontsize{6}{0}
\selectfont\put(251.151,141.404){\makebox(0,0)[t]{\textcolor[rgb]{0.15,0.15,0.15}{{20}}}}
\fontsize{6}{0}
\selectfont\put(269.587,141.404){\makebox(0,0)[t]{\textcolor[rgb]{0.15,0.15,0.15}{{40}}}}
\fontsize{6}{0}
\selectfont\put(288.024,141.404){\makebox(0,0)[t]{\textcolor[rgb]{0.15,0.15,0.15}{{60}}}}
\fontsize{6}{0}
\selectfont\put(306.46,141.404){\makebox(0,0)[t]{\textcolor[rgb]{0.15,0.15,0.15}{{80}}}}
\fontsize{6}{0}
\selectfont\put(324.896,141.404){\makebox(0,0)[t]{\textcolor[rgb]{0.15,0.15,0.15}{{100}}}}
\fontsize{6}{0}
\selectfont\put(343.333,141.404){\makebox(0,0)[t]{\textcolor[rgb]{0.15,0.15,0.15}{{120}}}}
\fontsize{6}{0}
\selectfont\put(361.77,141.404){\makebox(0,0)[t]{\textcolor[rgb]{0.15,0.15,0.15}{{140}}}}
\fontsize{6}{0}
\selectfont\put(229.235,154.496){\makebox(0,0)[r]{\textcolor[rgb]{0.15,0.15,0.15}{{0}}}}
\fontsize{6}{0}
\selectfont\put(229.235,174.144){\makebox(0,0)[r]{\textcolor[rgb]{0.15,0.15,0.15}{{0.5}}}}
\fontsize{6}{0}
\selectfont\put(229.235,193.792){\makebox(0,0)[r]{\textcolor[rgb]{0.15,0.15,0.15}{{1}}}}
\fontsize{6}{0}
\selectfont\put(229.235,213.44){\makebox(0,0)[r]{\textcolor[rgb]{0.15,0.15,0.15}{{1.5}}}}
\fontsize{7}{0}
\selectfont\put(214.235,186.325){\rotatebox{90}{\makebox(0,0)[b]{\textcolor[rgb]{0.15,0.15,0.15}{{Saida $y_2(t)$}}}}}
\fontsize{7}{0}
\selectfont\put(297.357,130.404){\makebox(0,0)[t]{\textcolor[rgb]{0.15,0.15,0.15}{{Tempo (s)}}}}
\fontsize{6}{0}
\selectfont\put(52,22.2671){\makebox(0,0)[t]{\textcolor[rgb]{0.15,0.15,0.15}{{0}}}}
\fontsize{6}{0}
\selectfont\put(70.4365,22.2671){\makebox(0,0)[t]{\textcolor[rgb]{0.15,0.15,0.15}{{20}}}}
\fontsize{6}{0}
\selectfont\put(88.873,22.2671){\makebox(0,0)[t]{\textcolor[rgb]{0.15,0.15,0.15}{{40}}}}
\fontsize{6}{0}
\selectfont\put(107.309,22.2671){\makebox(0,0)[t]{\textcolor[rgb]{0.15,0.15,0.15}{{60}}}}
\fontsize{6}{0}
\selectfont\put(125.746,22.2671){\makebox(0,0)[t]{\textcolor[rgb]{0.15,0.15,0.15}{{80}}}}
\fontsize{6}{0}
\selectfont\put(144.182,22.2671){\makebox(0,0)[t]{\textcolor[rgb]{0.15,0.15,0.15}{{100}}}}
\fontsize{6}{0}
\selectfont\put(162.619,22.2671){\makebox(0,0)[t]{\textcolor[rgb]{0.15,0.15,0.15}{{120}}}}
\fontsize{6}{0}
\selectfont\put(181.055,22.2671){\makebox(0,0)[t]{\textcolor[rgb]{0.15,0.15,0.15}{{140}}}}
\fontsize{6}{0}
\selectfont\put(48.5205,36.0054){\makebox(0,0)[r]{\textcolor[rgb]{0.15,0.15,0.15}{{0}}}}
\fontsize{6}{0}
\selectfont\put(48.5205,57.269){\makebox(0,0)[r]{\textcolor[rgb]{0.15,0.15,0.15}{{0.5}}}}
\fontsize{6}{0}
\selectfont\put(48.5205,78.5322){\makebox(0,0)[r]{\textcolor[rgb]{0.15,0.15,0.15}{{1}}}}
\fontsize{6}{0}
\selectfont\put(48.5205,99.7959){\makebox(0,0)[r]{\textcolor[rgb]{0.15,0.15,0.15}{{1.5}}}}
\fontsize{7}{0}
\selectfont\put(33.5205,67.1875){\rotatebox{90}{\makebox(0,0)[b]{\textcolor[rgb]{0.15,0.15,0.15}{{Sinal de Controle $u_1(t)$}}}}}
\fontsize{7}{0}
\selectfont\put(116.643,11.2671){\makebox(0,0)[t]{\textcolor[rgb]{0.15,0.15,0.15}{{Tempo (s)}}}}
\fontsize{6}{0}
\selectfont\put(232.715,22.2671){\makebox(0,0)[t]{\textcolor[rgb]{0.15,0.15,0.15}{{0}}}}
\fontsize{6}{0}
\selectfont\put(251.151,22.2671){\makebox(0,0)[t]{\textcolor[rgb]{0.15,0.15,0.15}{{20}}}}
\fontsize{6}{0}
\selectfont\put(269.587,22.2671){\makebox(0,0)[t]{\textcolor[rgb]{0.15,0.15,0.15}{{40}}}}
\fontsize{6}{0}
\selectfont\put(288.024,22.2671){\makebox(0,0)[t]{\textcolor[rgb]{0.15,0.15,0.15}{{60}}}}
\fontsize{6}{0}
\selectfont\put(306.46,22.2671){\makebox(0,0)[t]{\textcolor[rgb]{0.15,0.15,0.15}{{80}}}}
\fontsize{6}{0}
\selectfont\put(324.896,22.2671){\makebox(0,0)[t]{\textcolor[rgb]{0.15,0.15,0.15}{{100}}}}
\fontsize{6}{0}
\selectfont\put(343.333,22.2671){\makebox(0,0)[t]{\textcolor[rgb]{0.15,0.15,0.15}{{120}}}}
\fontsize{6}{0}
\selectfont\put(361.77,22.2671){\makebox(0,0)[t]{\textcolor[rgb]{0.15,0.15,0.15}{{140}}}}
\fontsize{6}{0}
\selectfont\put(229.235,28.3999){\makebox(0,0)[r]{\textcolor[rgb]{0.15,0.15,0.15}{{-0.2}}}}
\fontsize{6}{0}
\selectfont\put(229.235,43.6162){\makebox(0,0)[r]{\textcolor[rgb]{0.15,0.15,0.15}{{0}}}}
\fontsize{6}{0}
\selectfont\put(229.235,58.833){\makebox(0,0)[r]{\textcolor[rgb]{0.15,0.15,0.15}{{0.2}}}}
\fontsize{6}{0}
\selectfont\put(229.235,74.0493){\makebox(0,0)[r]{\textcolor[rgb]{0.15,0.15,0.15}{{0.4}}}}
\fontsize{6}{0}
\selectfont\put(229.235,89.2661){\makebox(0,0)[r]{\textcolor[rgb]{0.15,0.15,0.15}{{0.6}}}}
\fontsize{6}{0}
\selectfont\put(229.235,104.482){\makebox(0,0)[r]{\textcolor[rgb]{0.15,0.15,0.15}{{0.8}}}}
\fontsize{7}{0}
\selectfont\put(212.235,67.1875){\rotatebox{90}{\makebox(0,0)[b]{\textcolor[rgb]{0.15,0.15,0.15}{{Sinal de Controle $u_2(t)$}}}}}
\fontsize{7}{0}
\selectfont\put(297.357,11.2671){\makebox(0,0)[t]{\textcolor[rgb]{0.15,0.15,0.15}{{Tempo (s)}}}}
\fontsize{6}{0}
\selectfont\put(128.001,242.023){\makebox(0,0)[l]{\textcolor[rgb]{0,0,0}{{referencia}}}}
\fontsize{6}{0}
\selectfont\put(185.003,242.023){\makebox(0,0)[l]{\textcolor[rgb]{0,0,0}{{sem acao integral}}}}
\fontsize{6}{0}
\selectfont\put(262.005,242.023){\makebox(0,0)[l]{\textcolor[rgb]{0,0,0}{{com acao integral}}}}
\end{picture}

    \end{minipage}
\end{figure}

\subsection{Conclusões}
(Concluir em que medida os resultados apresentam relação com a motivação.
Permitem ilustrar ou concluir algo sobre a motivação? )
