\newcommand{\mat}[1]{\MakeUppercase{\mathbf{#1}}}
\newcommand{\ssvec}[1]{\MakeLowercase{\mathbf{#1}}}
\newcommand{\ssveceq}[1]{\MakeLowercase{\mathbf{\bar{#1}}}}
\newcommand{\peqone}{\ensuremath{[1 \thickspace 0]^\top}}
\newcommand{\peqtwo}{\ensuremath{[4 \thickspace 0]^\top}}
\newcommand{\inicond}{\ensuremath{\ssvec{x}(0) = [-5 \thickspace -5]}}

\section{Desafio VII - Controle de Sistema Não-Linear} 

\subsection{Motivação}
Embora muitos processos e fenômenos físicos reais podem ser modelados por
modelos lineares com uma boa proximidade da representação real, há problemas no
universo da engenharia que aproximações lineares não fornecem bons resultados,
ou até mesmo problemas em que é necessário utilizar o modelo não linear. Tanques
de fluídos é um ótimo exemplo de um sistema com não linearidades, mas que a
aproximação por um modelo linear através de uma linearização em torno de um
ponto de equilíbrios trás bons resultados. Entretanto, em situações como o uso
de drones operando em alta velocidade em ambientes desconhecidos, o que decorre
em manobras complexas, torna-se necessário usar o controle com o modelo não
linear para que ele possa desviar de obstáculos desconhecidos de forma rápida e
eficiente.

Por isso, o conhecimento de sistemas e controle não linear é mais um elemento
que deve fazer parte do "canivete" de engenheiros e engenheiras de controle.
Desta forma, é fundamental que tais profissionais tenham um conhecimento teórico
inicial sobre mais este novo mundo dentro da engenharia de controle.

\subsection{Simulações realizadas}

Para o conhecimento dos fenômenos e ferramentas para o controle de sistema não
lineares, foi realizada simulações para analisar o comportamento do oscilador de
Van Der Pol descrito pelo modelo em espaço de estados
\ref{eq:modelo-do-oscilador-de-van-der-pol}. A partir deste modelo foi possível
realizar:

\begin{itemize}
    \item a linearização em torno de dois pontos de equilíbrio;
    \item o projeto de controladores LQR utilizando os modelos linearizados; e,
    por fim
    \item a análise do comportamento do sistema através de diagramas de fases
    obtido com o auxílio do Pplane e respostas temporais a partir da aproximação
    adiante de Euler.
\end{itemize}

\begin{subequations}
    \label{eq:modelo-do-oscilador-de-van-der-pol}
    \begin{equation}
        \begin{bmatrix}
            \dot{x}_1(t) \\
            \dot{x}_2(t)
        \end{bmatrix}
        =
        \begin{bmatrix}
            x_2(t) \\
            -x_1(t) + 0,3(1 - x_1(t)^2)x_2(t) + u(t)
        \end{bmatrix}
    \end{equation}
    \begin{equation}
        y(t) = x_1(t)
    \end{equation}
\end{subequations}

Por simplicidade na representação do Modelo de Van der Pol, o argumento de tempo
será omitido ao longo desta seção, salve exceções em que há necessidade de
explicitar o argumento de tempo para melhor compreensão do leitor.

\subsection{Resultados obtidos}

Como o modelo do oscilador de Van der Pol dado em espaço de estados pela Equação
\ref{eq:modelo-do-oscilador-de-van-der-pol} é não linear, torna-se necessário
fazer a linearização em torno de um ponto de equilíbrio. Considerando $\ssvec{z}
= \ssvec{x} - \ssveceq{x}$, $v = u - \bar{u}$ e $w = y - h(\ssveceq{x},
\bar{u})$ como novas variáveis do espaço de estados e reescrevendo
\ref{eq:modelo-do-oscilador-de-van-der-pol} como 

\begin{subequations}
    \label{eq:modelo-reescrito-do-oscilador-de-van-der-pol}
    \begin{equation}
        \label{eq:mapa-vetorial-dos-estados}
        \dot{\ssvec{x}} = f(\ssvec{x}, u)
        =
        \begin{bmatrix}
            \dot{x}_1 \\
            \dot{x}_2
        \end{bmatrix}
        =
        \begin{bmatrix}
            f_1(x_1, x_2, u) \\
            f_2(x_1, x_2, u) \\
        \end{bmatrix}
        =
        \begin{bmatrix}
            x_2 \\
            -x_1 + 0,3(1 - x_1^2)x_2 + u
        \end{bmatrix}
    \end{equation}
    \begin{equation}
        y = h(\ssvec{x}, u) = x_1
    \end{equation}
\end{subequations} o modelo não linear do Oscilador de Van der Pol pode ser
convertido para o modelo linear em torno do ponto de equilíbrio através de uma
linearização jacobiana \cite{Astrom2008} em que

\begin{subequations}
    \label{eq:modelo-linearizado-em-torno-do-equilibrio}
    \begin{equation}
        \label{eq:derivada-dos-estados-linearizados}
        \dot{\ssvec{z}} = \mat{a}\ssvec{z} + \mathbf{B}v
    \end{equation}
    \begin{equation}
        w = \mat{C}\ssvec{z} + \mat{D}v
    \end{equation}
\end{subequations} sendo as matrizes definidas por

\begin{subequations}
    \label{eq:derivadas-parcias-do-vetor-estados}
    \begin{equation}
        \label{eq:jacobiana-de-a}  
        \mat{A} =
        \left.
            \begin{matrix}
                \frac{\partial f}{\partial x}
            \end{matrix}
            \right|_{(\bar{x}, \thinspace \bar{u})}
        =
        \begin{bmatrix}
            \left.
                \begin{matrix}
                    \frac{\partial f_1}{\partial x_1}
                \end{matrix}
            \right|_{(\bar{x}, \thinspace \bar{u})}
            &
            \left.
                \begin{matrix}
                    \frac{\partial f_1}{\partial x_2}
                \end{matrix}
            \right|_{(\bar{x}, \thinspace \bar{u})}
            \\
            \left.
                \begin{matrix}
                    \frac{\partial f_2}{\partial x_1}
                \end{matrix}
            \right|_{(\bar{x}, \thinspace \bar{u})}
            &
            \left.
                \begin{matrix}
                    \frac{\partial f_2}{\partial x_2}
                \end{matrix}
            \right|_{(\bar{x}, \thinspace \bar{u})}
        \end{bmatrix}
        =
        \begin{bmatrix}
            0 & 1 \\
            -1-0,6x_1x_2 & 0,3(1-\bar{x}_1^2)
        \end{bmatrix}
    \end{equation}

    \begin{equation}
        \label{eq:jacobiano-de-b}
        \mat{B} =
        \left.
            \begin{matrix}
                \frac{\partial f}{\partial u}
            \end{matrix}
        \right|_{(\bar{x}, \thinspace \bar{u})}
        =
        \begin{bmatrix}
            \left.
                \begin{matrix}
                    \frac{\partial f_1}{\partial u}
                \end{matrix}
            \right|_{(\bar{x}, \thinspace \bar{u})}
            \\
            \left.
                \begin{matrix}
                    \frac{\partial f_2}{\partial u}
                \end{matrix}
            \right|_{(\bar{x}, \thinspace \bar{u})}
        \end{bmatrix}
        =
        \begin{bmatrix}
            0 \\
            1
        \end{bmatrix}
    \end{equation}

    \begin{equation}
        \label{eq:jacoabiana-de-c}
        \mat{C} =
        \left.
            \begin{matrix}
                \frac{\partial h}{\partial x}
            \end{matrix}
        \right|_{(\bar{x}, \thinspace \bar{u})}
        =
        \begin{bmatrix}
            \left.
                \begin{matrix}
                    \frac{\partial h}{\partial x_1}
                \end{matrix}
            \right|_{(\bar{x}, \thinspace \bar{u})}
            &
            \left.
                \begin{matrix}
                    \frac{\partial h}{\partial x_2}
                \end{matrix}
            \right|_{(\bar{x}, \thinspace \bar{u})}
        \end{bmatrix}
        =
        \begin{bmatrix}
            1 & 0
        \end{bmatrix}
    \end{equation}

    \begin{equation}
        \mat{D} =
        \left.
            \begin{matrix}
                \frac{\partial h}{\partial u}
            \end{matrix}
        \right|_{(\bar{x}, \thinspace \bar{u})}
        = 0
    \end{equation}
\end{subequations}

Como o ponto de equilíbrio é definido por $[\dot{x}_1 \thickspace
\dot{x}_2]^\top = [0 \thickspace 0]^\top$, de
\ref{eq:modelo-do-oscilador-de-van-der-pol} tem-se que $\bar{x}_2 = 0$ e
$\bar{x_1} = \bar{u}$. Assim, considerando dois pontos de equilíbrio $\bar{x}_1
= 1$ e $\bar{x}_1 = 4$, as matrizes de dinâmica do sistema $\mat{A}_1$ e
$\mat{A}_2$, respectivamente, são definidas por

\begin{subequations}
    \label{eq:matrizes-a-linearizadas}
    \begin{equation}
        \label{eq:matriz-a1}
        \mat{A}_1
        =
        \begin{bmatrix}
            0   & 1 \\
            -1  & 0
        \end{bmatrix}
    \end{equation}

    \begin{equation}
        \label{eq:matriz-a2}
        \mat{A}_2
        =
        \begin{bmatrix}
            0   & 1 \\
            -1  & -4,5
        \end{bmatrix}
    \end{equation}
\end{subequations} cujo os autovalores são respectivamente $\lambda_1 = [i
\thickspace -i]^\top$ e $\lambda_2 = [-0,2344 \thickspace -4,2656]^\top$.
Nota-se portanto que para o ponto de equilíbrio $[1 \thickspace 0]^\top$ o
modelo linearizado possui um par conjugado de autovalores. Isto permite concluir
que partindo de um ponto nas proximidades deste ponto de equilíbrio ou com
pequenas variações do sinal de controle em torno de $\bar{u} = 1$, fará tanto os
estados quanto a saída oscilarem. Esta oscilação no diagrama de fases é
representada por uma trajetória circular em torno do ponto de equilíbrio,
chamada de círculo limite, como pode ser observado no diagrama de fases da
Figura \ref{fig:diagrama-de-fases-para-circulo-limite} obtido no software
Pplane\footnote{https://www.cs.unm.edu/~joel/dfield/}. Por outro lado, para o
ponto de equilíbrio $[4 \thickspace 0]^\top$ nas duas situações, o sistema
converge para o ponto de equilíbrio.

\begin{figure}[]
	\centering
	\caption{Diagrama de fases com trajetória circular (círculo limite) nas
    mediações do ponto de equilíbrio $[1 \thickspace 0]^\top$ para $\bar{u}=1$.}
	\label{fig:diagrama-de-fases-para-circulo-limite}
	\includegraphics[width=\textwidth]{chapters/challenge7/images/diagrama-de-fases-com-circulo-limite.png}
\end{figure}

Com os modelos linearizados foi obtido a matriz de ganhos $\mat{K}$ tal que
a ação de controle $v(t) = \mat{k}\ssvec{z}(t)$ minimiza o custo do controlador
LQR $J$ dado por

\begin{equation}
    \label{eq:custo-do-controlador-lqr}
    J = \int_{t=0}^{\infty }\ssvec{z}(t)^\top \mat{Q} \ssvec{z}(t) + v(t)^\top R v(t)
\end{equation} com $\mat{Q} = \mat{I}$ e $R = 0,1$. Com isso, através da função
\textit{lqr} do Octave foram obtidos os ganhos $\mat{K}_1$ e $\mat{K}_2$, conforme
Equação \ref{eq:ganhos-do-controlador-lqr}, relacionados respectivamente as
matrizes $\mat{A}_1$ e $\mat{A}_2$ (Equação \ref{eq:matrizes-a-linearizadas}).

\begin{subequations}
    \label{eq:ganhos-do-controlador-lqr}
    \begin{equation}
        \mat{K}_1 = [2,3166 \thickspace 3,8253]
    \end{equation}
    \begin{equation}
        \mat{K}_2 = [2,3166 \thickspace 1,4062]
    \end{equation}
\end{subequations}

Com isso, definindo $\mat{k} = [k_1 \thickspace k_2]$ e retornando a ação de
controle para as variáveis de estados originais, isto é,

\begin{equation}
    \label{eq:acao-de-controle-do-sistema-linear}
    v = -\mat{k}\ssvec{z}
    \Rightarrow
    u = -\mat{k}(\ssvec{x} - \ssveceq{x}) + \bar{u}
    = -k_1(x_1 - \bar{x}_1) - k_2(x_2 - \bar{x}_2) + \bar{u}
\end{equation} o modelo do Oscilador de Van Der Pol pode ser reescrito como

\begin{equation}
    \label{eq:modelo-reescrito2-do-oscilador-de-van-der-pol}
    \begin{bmatrix}
        \dot{x}_1 \\
        \dot{x}_2
    \end{bmatrix}
    =
    \begin{bmatrix}
        x_2 \\
        -x_1 + 0,3(1 - x_1^2)x_2 - k_1(x_1 - \bar{x}_1) - k_2(x_2 - \bar{x}_2) + \bar{u}
    \end{bmatrix}.
\end{equation}

Para análise do comportamento do sistema com o modelo acima, foram gerados, com
auxílio do software Pplane com o parâmetro \textit{Solution Direction} igual
\textit{Forward}, diagramas de fase considerando três cenários:

\begin{itemize}
    \item i) Modelo do oscilador com $u(t) = 0$;
    \item ii) Modelo do oscilador utilizando o ganho LQR $\mat{K}_1$ obtido com
    a linearização em torno do ponto de equilíbrio $[1 \thickspace 0]^\top$; e
    \item iii) Modelo do oscilador utilizando o ganho LQR $\mat{K}_2$ obtido com
    a linearização em torno do ponto de equilíbrio $[4 \thickspace 0]^\top$.
\end{itemize}

O diagrama de fases do cenário (i) é exibido na Figura
\ref{fig:diagrama-de-fases-sem-controle}. Percebe-se que definindo a condição
inicial aproximadamente iguais aos pontos de equilíbrios \peqone and \peqtwo, os
estados convergem para oscilação, criando uma órbita em formato de losango com
arestas arredondadas. O mesmo acontece quando são escolhidas condições iniciais
nas vizinhanças do ponto de equilíbrio, como pode ser visto na Figura
\ref{fig:diagrama-de-fases-sem-controle-com-zoom}, que devido ao foco no entorno
dos pontos de equilíbrio, a órbita não é exibida.

\begin{figure}[H]
    \caption{Visão expandida do diagrama de fases do modelo do oscilador com $u(t) = 0$.}
    \label{fig:diagrama-de-fases-sem-controle}
    \centering
    \includegraphics[width=0.8\textwidth]{chapters/challenge7/images/diagrama-de-fases-sem-controle.png}
\end{figure}

\begin{figure}[H]
    \caption{Visão delimitada do diagrama de fases do modelo do oscilador com $u(t) = 0$.}
    \label{fig:diagrama-de-fases-sem-controle-com-zoom}
    \centering
    \begin{subfigure}[t]{0.48\textwidth}
        \centering
	    \includegraphics[width=\textwidth]{chapters/challenge7/images/diagrama-de-fases-sem-controle-zoom1.png}
        \caption{Partindo de condições iniciais na vizinhança do ponto de equilíbrio 1.}
        \label{fig:diagrama-de-fases-sem-controle-ponto-de-equilibrio-1}
    \end{subfigure}
    \hfill
    \begin{subfigure}[t]{0.48\textwidth}
        \centering
	    \includegraphics[width=\textwidth]{chapters/challenge7/images/diagrama-de-fases-sem-controle-zoom2.png}
        \caption{Partindo de condições iniciais na vizinhança do ponto de equilíbrio 2.}
        \label{fig:diagrama-de-fases-sem-controle-ponto-de-equilibrio-2}
    \end{subfigure}
\end{figure}

Para o cenário (ii), foi obtido o diagrama de fases conforme Figura
\ref{fig:diagrama-de-fases-com-controle-na-vizinhanca-do-peq-1}. Percebe-se que
considerando condições iniciais na vizinhança do ponto de equilíbrio \peqone, os
estados do sistema convergem para o equilíbrio. O mesmo se observa para o ponto
de equilíbrio \peqtwo (cenário (iii)), como demonstra o diagrama de fases na Figura
\ref{fig:diagrama-de-fases-com-controle-na-vizinhanca-do-peq-2}

\begin{figure}[H]
    \caption{Diagrama de fases na vizinhança do ponto de equilíbrio \peqone com
    controle LQR.}
    \label{fig:diagrama-de-fases-com-controle-na-vizinhanca-do-peq-1}
    \centering
    \includegraphics[width=0.7\textwidth]{chapters/challenge7/images/diagrama-de-fases-com-controle-peq1.png}
\end{figure}

\begin{figure}[H]
    \caption{Diagrama de fases na vizinhança do ponto de equilíbrio \peqtwo com
    controle LQR.}
    \label{fig:diagrama-de-fases-com-controle-na-vizinhanca-do-peq-2}
    \centering
    \includegraphics[width=0.7\textwidth]{chapters/challenge7/images/diagrama-de-fases-com-controle-peq2.png}
\end{figure}

Foi realizada também a simulação para analisar o comportamento do oscilador
fazendo o controle do sistema não linear com o controlador obtido através da
linearização. Na Figura \ref{fig:diagrama-de-fases-com-controle} é mostrado o
diagrama de fases considerando a condição inicial $\ssvec{x}(0) = [-5
\thickspace -5]$. Nota-se que os estados convergiram para os pontos de
equilíbrio. Entretanto, é notório o comportamento não linear dos estados na
trajetória que os leva até o estado de equilíbrio, comportamento não observado
nas Figuras \ref{fig:diagrama-de-fases-com-controle-na-vizinhanca-do-peq-1} e
\ref{fig:diagrama-de-fases-com-controle-na-vizinhanca-do-peq-2}, por terem
condições iniciais definidas nas vizinhança do ponto de equilíbrio. 

\begin{figure}[H]
    \caption{Diagrama de fases considerando a condição inicial $[-5 \thickspace -5]^\top$ e
    com o controlador LQR.}
    \label{fig:diagrama-de-fases-com-controle}
    \centering
    \begin{subfigure}[t]{0.49\textwidth}
        \centering
	    \includegraphics[width=\textwidth]{chapters/challenge7/images/diagrama-de-fases-resultado-2-questao-5.png}
        \caption{Sistema controlado com controlador projetado para o ponto de equilíbrio \peqone.}
        \label{fig:diagrama-de-fases-com-controle-ponto-de-equilibrio-1}
    \end{subfigure}
    \hfill
    \begin{subfigure}[t]{0.49\textwidth}
        \centering
	    \includegraphics[width=\textwidth]{chapters/challenge7/images/diagrama-de-fases-resultado-3-questao-5.png}
        \caption{Sistema controlado com controlador projetado para o ponto de equilíbrio \peqtwo.}
        \label{fig:diagrama-de-fases-com-controle-ponto-de-equilibrio-2}
    \end{subfigure}
\end{figure}

Com a mesma condição inicial \inicond, foram geradas os sinais do oscilador
considerando os três cenários descritos anteriormente. Para o cenário (i) com
$u(t) = 0$, Figura \ref{fig:sinais-do-scilador-sem-controle}, observa-se o
comportamento oscilatório dos estados e consequentemente saída, o que está de
acordo com a trajetória exibida na Figura
\ref{fig:diagrama-de-fases-sem-controle}. Para o cenário (ii), a resposta obtida
é ilustrada na Figura \ref{fig:sinais-do-scilador-com-controle-e-peq1}.
Percebe-se na figura que o estado $x_2(t)$ tem uma alta variação de amplitude
nos instantes iniciais da trajetória enquanto o estado $x_1(t)$ varia
lentamente. Isto está condizente com o diagrama de fases (Figura
\ref{fig:diagrama-de-fases-com-controle-ponto-de-equilibrio-1}) em que
constata-se que o gradiente de velocidade está aproximadamente perpendicular ao
eixo do estado $x_1(t)$ no início da trajetória. A partir do estado $[-4
\thickspace 2]$ nota-se uma suavidade na trajetória dos estados até o ponto de
equilíbrio, o que é refletido nas respostas temporal considerando $t > 1,7s$.

\begin{figure}[H]
    \caption{Sinais do oscilador com condição inicial $\ssvec{x}(0) = [-5
    \thickspace -5]$ e sinal de controle $u(t)=0$.}
    \vspace{-10pt}
    \hspace{-30pt}
    \label{fig:sinais-do-scilador-sem-controle}
    \begin{minipage}{\linewidth}
        % Title: gl2ps_renderer figure
% Creator: GL2PS 1.4.0, (C) 1999-2017 C. Geuzaine
% For: Octave
% CreationDate: Mon Nov 29 15:58:46 2021
\setlength{\unitlength}{1pt}
\begin{picture}(0,0)
\includegraphics{chapters/challenge7/images/resultado-3-questao-5-inc}
\end{picture}%
\begin{picture}(400,200)(0,0)
\fontsize{6}{0}
\selectfont\put(52,116.3){\makebox(0,0)[t]{\textcolor[rgb]{0.15,0.15,0.15}{{0}}}}
\fontsize{6}{0}
\selectfont\put(77.7998,116.3){\makebox(0,0)[t]{\textcolor[rgb]{0.15,0.15,0.15}{{10}}}}
\fontsize{6}{0}
\selectfont\put(103.6,116.3){\makebox(0,0)[t]{\textcolor[rgb]{0.15,0.15,0.15}{{20}}}}
\fontsize{6}{0}
\selectfont\put(129.4,116.3){\makebox(0,0)[t]{\textcolor[rgb]{0.15,0.15,0.15}{{30}}}}
\fontsize{6}{0}
\selectfont\put(155.2,116.3){\makebox(0,0)[t]{\textcolor[rgb]{0.15,0.15,0.15}{{40}}}}
\fontsize{6}{0}
\selectfont\put(181,116.3){\makebox(0,0)[t]{\textcolor[rgb]{0.15,0.15,0.15}{{50}}}}
\fontsize{6}{0}
\selectfont\put(48.5278,121.532){\makebox(0,0)[r]{\textcolor[rgb]{0.15,0.15,0.15}{{-6}}}}
\fontsize{6}{0}
\selectfont\put(48.5278,131.244){\makebox(0,0)[r]{\textcolor[rgb]{0.15,0.15,0.15}{{-4}}}}
\fontsize{6}{0}
\selectfont\put(48.5278,140.955){\makebox(0,0)[r]{\textcolor[rgb]{0.15,0.15,0.15}{{-2}}}}
\fontsize{6}{0}
\selectfont\put(48.5278,150.666){\makebox(0,0)[r]{\textcolor[rgb]{0.15,0.15,0.15}{{0}}}}
\fontsize{6}{0}
\selectfont\put(48.5278,160.377){\makebox(0,0)[r]{\textcolor[rgb]{0.15,0.15,0.15}{{2}}}}
\fontsize{6}{0}
\selectfont\put(48.5278,170.089){\makebox(0,0)[r]{\textcolor[rgb]{0.15,0.15,0.15}{{4}}}}
\fontsize{6}{0}
\selectfont\put(48.5278,179.8){\makebox(0,0)[r]{\textcolor[rgb]{0.15,0.15,0.15}{{6}}}}
\fontsize{7}{0}
\selectfont\put(37.5278,150.666){\rotatebox{90}{\makebox(0,0)[b]{\textcolor[rgb]{0.15,0.15,0.15}{{$Estado x_1(t)$}}}}}
\fontsize{7}{0}
\selectfont\put(116.5,105.3){\makebox(0,0)[t]{\textcolor[rgb]{0.15,0.15,0.15}{{Tempo [s]}}}}
\fontsize{6}{0}
\selectfont\put(52,20){\makebox(0,0)[t]{\textcolor[rgb]{0.15,0.15,0.15}{{0}}}}
\fontsize{6}{0}
\selectfont\put(77.7998,20){\makebox(0,0)[t]{\textcolor[rgb]{0.15,0.15,0.15}{{10}}}}
\fontsize{6}{0}
\selectfont\put(103.6,20){\makebox(0,0)[t]{\textcolor[rgb]{0.15,0.15,0.15}{{20}}}}
\fontsize{6}{0}
\selectfont\put(129.4,20){\makebox(0,0)[t]{\textcolor[rgb]{0.15,0.15,0.15}{{30}}}}
\fontsize{6}{0}
\selectfont\put(155.2,20){\makebox(0,0)[t]{\textcolor[rgb]{0.15,0.15,0.15}{{40}}}}
\fontsize{6}{0}
\selectfont\put(181,20){\makebox(0,0)[t]{\textcolor[rgb]{0.15,0.15,0.15}{{50}}}}
\fontsize{6}{0}
\selectfont\put(48.5278,25.2319){\makebox(0,0)[r]{\textcolor[rgb]{0.15,0.15,0.15}{{-6}}}}
\fontsize{6}{0}
\selectfont\put(48.5278,34.9434){\makebox(0,0)[r]{\textcolor[rgb]{0.15,0.15,0.15}{{-4}}}}
\fontsize{6}{0}
\selectfont\put(48.5278,44.6548){\makebox(0,0)[r]{\textcolor[rgb]{0.15,0.15,0.15}{{-2}}}}
\fontsize{6}{0}
\selectfont\put(48.5278,54.3662){\makebox(0,0)[r]{\textcolor[rgb]{0.15,0.15,0.15}{{0}}}}
\fontsize{6}{0}
\selectfont\put(48.5278,64.0771){\makebox(0,0)[r]{\textcolor[rgb]{0.15,0.15,0.15}{{2}}}}
\fontsize{6}{0}
\selectfont\put(48.5278,73.7886){\makebox(0,0)[r]{\textcolor[rgb]{0.15,0.15,0.15}{{4}}}}
\fontsize{6}{0}
\selectfont\put(48.5278,83.5){\makebox(0,0)[r]{\textcolor[rgb]{0.15,0.15,0.15}{{6}}}}
\fontsize{7}{0}
\selectfont\put(37.5278,54.3662){\rotatebox{90}{\makebox(0,0)[b]{\textcolor[rgb]{0.15,0.15,0.15}{{Estado $x_2(t)$}}}}}
\fontsize{7}{0}
\selectfont\put(116.5,9){\makebox(0,0)[t]{\textcolor[rgb]{0.15,0.15,0.15}{{Tempo [s]}}}}
\fontsize{6}{0}
\selectfont\put(233,116.3){\makebox(0,0)[t]{\textcolor[rgb]{0.15,0.15,0.15}{{0}}}}
\fontsize{6}{0}
\selectfont\put(258.8,116.3){\makebox(0,0)[t]{\textcolor[rgb]{0.15,0.15,0.15}{{10}}}}
\fontsize{6}{0}
\selectfont\put(284.6,116.3){\makebox(0,0)[t]{\textcolor[rgb]{0.15,0.15,0.15}{{20}}}}
\fontsize{6}{0}
\selectfont\put(310.4,116.3){\makebox(0,0)[t]{\textcolor[rgb]{0.15,0.15,0.15}{{30}}}}
\fontsize{6}{0}
\selectfont\put(336.2,116.3){\makebox(0,0)[t]{\textcolor[rgb]{0.15,0.15,0.15}{{40}}}}
\fontsize{6}{0}
\selectfont\put(362,116.3){\makebox(0,0)[t]{\textcolor[rgb]{0.15,0.15,0.15}{{50}}}}
\fontsize{6}{0}
\selectfont\put(229.528,121.532){\makebox(0,0)[r]{\textcolor[rgb]{0.15,0.15,0.15}{{-6}}}}
\fontsize{6}{0}
\selectfont\put(229.528,131.244){\makebox(0,0)[r]{\textcolor[rgb]{0.15,0.15,0.15}{{-4}}}}
\fontsize{6}{0}
\selectfont\put(229.528,140.955){\makebox(0,0)[r]{\textcolor[rgb]{0.15,0.15,0.15}{{-2}}}}
\fontsize{6}{0}
\selectfont\put(229.528,150.666){\makebox(0,0)[r]{\textcolor[rgb]{0.15,0.15,0.15}{{0}}}}
\fontsize{6}{0}
\selectfont\put(229.528,160.377){\makebox(0,0)[r]{\textcolor[rgb]{0.15,0.15,0.15}{{2}}}}
\fontsize{6}{0}
\selectfont\put(229.528,170.089){\makebox(0,0)[r]{\textcolor[rgb]{0.15,0.15,0.15}{{4}}}}
\fontsize{6}{0}
\selectfont\put(229.528,179.8){\makebox(0,0)[r]{\textcolor[rgb]{0.15,0.15,0.15}{{6}}}}
\fontsize{7}{0}
\selectfont\put(218.528,150.666){\rotatebox{90}{\makebox(0,0)[b]{\textcolor[rgb]{0.15,0.15,0.15}{{Saida y(t)}}}}}
\fontsize{7}{0}
\selectfont\put(297.5,105.3){\makebox(0,0)[t]{\textcolor[rgb]{0.15,0.15,0.15}{{Tempo [s]}}}}
\fontsize{6}{0}
\selectfont\put(233,20){\makebox(0,0)[t]{\textcolor[rgb]{0.15,0.15,0.15}{{0}}}}
\fontsize{6}{0}
\selectfont\put(258.8,20){\makebox(0,0)[t]{\textcolor[rgb]{0.15,0.15,0.15}{{10}}}}
\fontsize{6}{0}
\selectfont\put(284.6,20){\makebox(0,0)[t]{\textcolor[rgb]{0.15,0.15,0.15}{{20}}}}
\fontsize{6}{0}
\selectfont\put(310.4,20){\makebox(0,0)[t]{\textcolor[rgb]{0.15,0.15,0.15}{{30}}}}
\fontsize{6}{0}
\selectfont\put(336.2,20){\makebox(0,0)[t]{\textcolor[rgb]{0.15,0.15,0.15}{{40}}}}
\fontsize{6}{0}
\selectfont\put(362,20){\makebox(0,0)[t]{\textcolor[rgb]{0.15,0.15,0.15}{{50}}}}
\fontsize{6}{0}
\selectfont\put(229.528,25.2319){\makebox(0,0)[r]{\textcolor[rgb]{0.15,0.15,0.15}{{-1}}}}
\fontsize{6}{0}
\selectfont\put(229.528,39.7993){\makebox(0,0)[r]{\textcolor[rgb]{0.15,0.15,0.15}{{-0.5}}}}
\fontsize{6}{0}
\selectfont\put(229.528,54.3662){\makebox(0,0)[r]{\textcolor[rgb]{0.15,0.15,0.15}{{0}}}}
\fontsize{6}{0}
\selectfont\put(229.528,68.9331){\makebox(0,0)[r]{\textcolor[rgb]{0.15,0.15,0.15}{{0.5}}}}
\fontsize{6}{0}
\selectfont\put(229.528,83.5){\makebox(0,0)[r]{\textcolor[rgb]{0.15,0.15,0.15}{{1}}}}
\fontsize{7}{0}
\selectfont\put(212.528,54.3662){\rotatebox{90}{\makebox(0,0)[b]{\textcolor[rgb]{0.15,0.15,0.15}{{Sinal de controle $u(t)$}}}}}
\fontsize{7}{0}
\selectfont\put(297.5,9){\makebox(0,0)[t]{\textcolor[rgb]{0.15,0.15,0.15}{{Tempo [s]}}}}
\end{picture}

    \end{minipage}
    \vspace{-10pt}
\end{figure}

\begin{figure}[H]
    \caption{Sinais do oscilador com condição inicial $\ssvec{x}(0) = [-5
    \thickspace -5]$ e com controlador LQR projetado com o ponto de equilíbrio \peqone.}
    \vspace{-10pt}
    \hspace{-30pt}
    \label{fig:sinais-do-scilador-com-controle-e-peq1}
    \begin{minipage}{\linewidth}
        \input{chapters/challenge7/images/resultado-1-questao-5.tex}
    \end{minipage}
    \vspace{-10pt}
\end{figure}

Analisando agora o cenário (iii) através da Figura
\ref{fig:sinais-do-scilador-com-controle-e-peq2}, percebe-se que o comportamento
dos estados no início da trajetória é similar ao do cenário (ii). Todavia, como
o ponto de equilíbrio \peqtwo está mais distante do ponto de partida \inicond,
comportamentos adicionais surgem na trajetória dos estados. Repara-se que após a
trajetória ser predominantemente influenciada pelo estado $x_2$, há um
comportamento que se assemelha a uma reta, o que também é observado nas resposta
em função do tempo analisando o período $0,4s \leq t \leq 1,35s$. Após isto há
uma concavidade com $\dot{x}_1$ aproximadamente constante e logo após os
estados convergem para o equilíbrio \peqtwo com uma mudança abrupta de $x_2$
e pequena variação de $x_1$, o que também se observa na evolução temporal.
Vale salientar o alto valor do sinal de controle no início da trajetória. Isto
acontece pois ao analisar a Equação \ref{eq:acao-de-controle-do-sistema-linear},
a parcela $k_1(x_1 - \bar{x}_1)$ terá um alto valor devido a grande distância de
$x_1(0)$ para $\bar{x}_1(0)$. Considerando uma situação real, isto trás luz ao
fato que este controlador pode não ser adequado para o controle do oscilador
partindo da condição inicial em questão, pois devido ao seu alto valor na
partida, há uma grande possibilidade de saturação do sinal, podendo levar o
sistema a um comportamento não previsto devido a inserção de mais uma não
linearidade.

\begin{figure}[H]
    \caption{Sinais do oscilador com condição inicial $\ssvec{x}(0) = [-5
    \thickspace -5]$ e com controlador LQR projetado com o ponto de equilíbrio \peqtwo.}
    \vspace{-10pt}
    \hspace{-30pt}
    \label{fig:sinais-do-scilador-com-controle-e-peq2}
    \begin{minipage}{\linewidth}
        % Title: gl2ps_renderer figure
% Creator: GL2PS 1.4.0, (C) 1999-2017 C. Geuzaine
% For: Octave
% CreationDate: Mon Nov 29 16:39:33 2021
\setlength{\unitlength}{1pt}
\begin{picture}(0,0)
\includegraphics{chapters/challenge7/images/resultado-2-questao-5-inc}
\end{picture}%
\begin{picture}(400,200)(0,0)
\fontsize{6}{0}
\selectfont\put(52,116.3){\makebox(0,0)[t]{\textcolor[rgb]{0.15,0.15,0.15}{{0}}}}
\fontsize{6}{0}
\selectfont\put(77.7998,116.3){\makebox(0,0)[t]{\textcolor[rgb]{0.15,0.15,0.15}{{10}}}}
\fontsize{6}{0}
\selectfont\put(103.6,116.3){\makebox(0,0)[t]{\textcolor[rgb]{0.15,0.15,0.15}{{20}}}}
\fontsize{6}{0}
\selectfont\put(129.4,116.3){\makebox(0,0)[t]{\textcolor[rgb]{0.15,0.15,0.15}{{30}}}}
\fontsize{6}{0}
\selectfont\put(155.2,116.3){\makebox(0,0)[t]{\textcolor[rgb]{0.15,0.15,0.15}{{40}}}}
\fontsize{6}{0}
\selectfont\put(181,116.3){\makebox(0,0)[t]{\textcolor[rgb]{0.15,0.15,0.15}{{50}}}}
\fontsize{6}{0}
\selectfont\put(48.5278,121.532){\makebox(0,0)[r]{\textcolor[rgb]{0.15,0.15,0.15}{{-6}}}}
\fontsize{6}{0}
\selectfont\put(48.5278,131.244){\makebox(0,0)[r]{\textcolor[rgb]{0.15,0.15,0.15}{{-4}}}}
\fontsize{6}{0}
\selectfont\put(48.5278,140.955){\makebox(0,0)[r]{\textcolor[rgb]{0.15,0.15,0.15}{{-2}}}}
\fontsize{6}{0}
\selectfont\put(48.5278,150.666){\makebox(0,0)[r]{\textcolor[rgb]{0.15,0.15,0.15}{{0}}}}
\fontsize{6}{0}
\selectfont\put(48.5278,160.377){\makebox(0,0)[r]{\textcolor[rgb]{0.15,0.15,0.15}{{2}}}}
\fontsize{6}{0}
\selectfont\put(48.5278,170.089){\makebox(0,0)[r]{\textcolor[rgb]{0.15,0.15,0.15}{{4}}}}
\fontsize{6}{0}
\selectfont\put(48.5278,179.8){\makebox(0,0)[r]{\textcolor[rgb]{0.15,0.15,0.15}{{6}}}}
\fontsize{7}{0}
\selectfont\put(37.5278,150.666){\rotatebox{90}{\makebox(0,0)[b]{\textcolor[rgb]{0.15,0.15,0.15}{{$Estado x_1(t)$}}}}}
\fontsize{7}{0}
\selectfont\put(116.5,105.3){\makebox(0,0)[t]{\textcolor[rgb]{0.15,0.15,0.15}{{Tempo [s]}}}}
\fontsize{6}{0}
\selectfont\put(52,20){\makebox(0,0)[t]{\textcolor[rgb]{0.15,0.15,0.15}{{0}}}}
\fontsize{6}{0}
\selectfont\put(77.7998,20){\makebox(0,0)[t]{\textcolor[rgb]{0.15,0.15,0.15}{{10}}}}
\fontsize{6}{0}
\selectfont\put(103.6,20){\makebox(0,0)[t]{\textcolor[rgb]{0.15,0.15,0.15}{{20}}}}
\fontsize{6}{0}
\selectfont\put(129.4,20){\makebox(0,0)[t]{\textcolor[rgb]{0.15,0.15,0.15}{{30}}}}
\fontsize{6}{0}
\selectfont\put(155.2,20){\makebox(0,0)[t]{\textcolor[rgb]{0.15,0.15,0.15}{{40}}}}
\fontsize{6}{0}
\selectfont\put(181,20){\makebox(0,0)[t]{\textcolor[rgb]{0.15,0.15,0.15}{{50}}}}
\fontsize{6}{0}
\selectfont\put(48.5278,25.2319){\makebox(0,0)[r]{\textcolor[rgb]{0.15,0.15,0.15}{{-6}}}}
\fontsize{6}{0}
\selectfont\put(48.5278,33.001){\makebox(0,0)[r]{\textcolor[rgb]{0.15,0.15,0.15}{{-4}}}}
\fontsize{6}{0}
\selectfont\put(48.5278,40.77){\makebox(0,0)[r]{\textcolor[rgb]{0.15,0.15,0.15}{{-2}}}}
\fontsize{6}{0}
\selectfont\put(48.5278,48.5391){\makebox(0,0)[r]{\textcolor[rgb]{0.15,0.15,0.15}{{0}}}}
\fontsize{6}{0}
\selectfont\put(48.5278,56.3081){\makebox(0,0)[r]{\textcolor[rgb]{0.15,0.15,0.15}{{2}}}}
\fontsize{6}{0}
\selectfont\put(48.5278,64.0771){\makebox(0,0)[r]{\textcolor[rgb]{0.15,0.15,0.15}{{4}}}}
\fontsize{6}{0}
\selectfont\put(48.5278,71.8462){\makebox(0,0)[r]{\textcolor[rgb]{0.15,0.15,0.15}{{6}}}}
\fontsize{6}{0}
\selectfont\put(48.5278,79.6157){\makebox(0,0)[r]{\textcolor[rgb]{0.15,0.15,0.15}{{8}}}}
\fontsize{7}{0}
\selectfont\put(37.5278,54.3662){\rotatebox{90}{\makebox(0,0)[b]{\textcolor[rgb]{0.15,0.15,0.15}{{Estado $x_2(t)$}}}}}
\fontsize{7}{0}
\selectfont\put(116.5,9){\makebox(0,0)[t]{\textcolor[rgb]{0.15,0.15,0.15}{{Tempo [s]}}}}
\fontsize{6}{0}
\selectfont\put(233,116.3){\makebox(0,0)[t]{\textcolor[rgb]{0.15,0.15,0.15}{{0}}}}
\fontsize{6}{0}
\selectfont\put(258.8,116.3){\makebox(0,0)[t]{\textcolor[rgb]{0.15,0.15,0.15}{{10}}}}
\fontsize{6}{0}
\selectfont\put(284.6,116.3){\makebox(0,0)[t]{\textcolor[rgb]{0.15,0.15,0.15}{{20}}}}
\fontsize{6}{0}
\selectfont\put(310.4,116.3){\makebox(0,0)[t]{\textcolor[rgb]{0.15,0.15,0.15}{{30}}}}
\fontsize{6}{0}
\selectfont\put(336.2,116.3){\makebox(0,0)[t]{\textcolor[rgb]{0.15,0.15,0.15}{{40}}}}
\fontsize{6}{0}
\selectfont\put(362,116.3){\makebox(0,0)[t]{\textcolor[rgb]{0.15,0.15,0.15}{{50}}}}
\fontsize{6}{0}
\selectfont\put(229.528,121.532){\makebox(0,0)[r]{\textcolor[rgb]{0.15,0.15,0.15}{{-6}}}}
\fontsize{6}{0}
\selectfont\put(229.528,131.244){\makebox(0,0)[r]{\textcolor[rgb]{0.15,0.15,0.15}{{-4}}}}
\fontsize{6}{0}
\selectfont\put(229.528,140.955){\makebox(0,0)[r]{\textcolor[rgb]{0.15,0.15,0.15}{{-2}}}}
\fontsize{6}{0}
\selectfont\put(229.528,150.666){\makebox(0,0)[r]{\textcolor[rgb]{0.15,0.15,0.15}{{0}}}}
\fontsize{6}{0}
\selectfont\put(229.528,160.377){\makebox(0,0)[r]{\textcolor[rgb]{0.15,0.15,0.15}{{2}}}}
\fontsize{6}{0}
\selectfont\put(229.528,170.089){\makebox(0,0)[r]{\textcolor[rgb]{0.15,0.15,0.15}{{4}}}}
\fontsize{6}{0}
\selectfont\put(229.528,179.8){\makebox(0,0)[r]{\textcolor[rgb]{0.15,0.15,0.15}{{6}}}}
\fontsize{7}{0}
\selectfont\put(218.528,150.666){\rotatebox{90}{\makebox(0,0)[b]{\textcolor[rgb]{0.15,0.15,0.15}{{Saida y(t)}}}}}
\fontsize{7}{0}
\selectfont\put(297.5,105.3){\makebox(0,0)[t]{\textcolor[rgb]{0.15,0.15,0.15}{{Tempo [s]}}}}
\fontsize{6}{0}
\selectfont\put(233,20){\makebox(0,0)[t]{\textcolor[rgb]{0.15,0.15,0.15}{{0}}}}
\fontsize{6}{0}
\selectfont\put(258.8,20){\makebox(0,0)[t]{\textcolor[rgb]{0.15,0.15,0.15}{{10}}}}
\fontsize{6}{0}
\selectfont\put(284.6,20){\makebox(0,0)[t]{\textcolor[rgb]{0.15,0.15,0.15}{{20}}}}
\fontsize{6}{0}
\selectfont\put(310.4,20){\makebox(0,0)[t]{\textcolor[rgb]{0.15,0.15,0.15}{{30}}}}
\fontsize{6}{0}
\selectfont\put(336.2,20){\makebox(0,0)[t]{\textcolor[rgb]{0.15,0.15,0.15}{{40}}}}
\fontsize{6}{0}
\selectfont\put(362,20){\makebox(0,0)[t]{\textcolor[rgb]{0.15,0.15,0.15}{{50}}}}
\fontsize{6}{0}
\selectfont\put(229.528,25.2319){\makebox(0,0)[r]{\textcolor[rgb]{0.15,0.15,0.15}{{-10}}}}
\fontsize{6}{0}
\selectfont\put(229.528,36.8857){\makebox(0,0)[r]{\textcolor[rgb]{0.15,0.15,0.15}{{0}}}}
\fontsize{6}{0}
\selectfont\put(229.528,48.5391){\makebox(0,0)[r]{\textcolor[rgb]{0.15,0.15,0.15}{{10}}}}
\fontsize{6}{0}
\selectfont\put(229.528,60.1929){\makebox(0,0)[r]{\textcolor[rgb]{0.15,0.15,0.15}{{20}}}}
\fontsize{6}{0}
\selectfont\put(229.528,71.8462){\makebox(0,0)[r]{\textcolor[rgb]{0.15,0.15,0.15}{{30}}}}
\fontsize{6}{0}
\selectfont\put(229.528,83.5){\makebox(0,0)[r]{\textcolor[rgb]{0.15,0.15,0.15}{{40}}}}
\fontsize{7}{0}
\selectfont\put(214.528,54.3662){\rotatebox{90}{\makebox(0,0)[b]{\textcolor[rgb]{0.15,0.15,0.15}{{Sinal de controle $u(t)$}}}}}
\fontsize{7}{0}
\selectfont\put(297.5,9){\makebox(0,0)[t]{\textcolor[rgb]{0.15,0.15,0.15}{{Tempo [s]}}}}
\end{picture}

    \end{minipage}
    \vspace{-10pt}
\end{figure}

\subsection{Conclusões}
Em suma, os resultados alcançados conseguiram mostrar como modelos não lineares
se comportam e como eles podem ser modificados através de uma linearização em
torno do ponto de equilíbrio para que então técnicas de controle de sistemas não
lineares possam ser utilizadas. Mostraram também a importância de mais uma
ferramenta que usada para analisar sistemas modelados em espaços de estados, que
foi o uso do diagrama de fases para ter \textit{insights} do comportamento do
sistema dado uma condição inicial conhecida.
