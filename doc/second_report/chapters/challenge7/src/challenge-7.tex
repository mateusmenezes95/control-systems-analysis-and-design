\newcommand{\mat}[1]{\MakeUppercase{\mathbf{#1}}}
\newcommand{\ssvec}[1]{\MakeLowercase{\mathbf{#1}}}
\newcommand{\ssveceq}[1]{\MakeLowercase{\mathbf{\bar{#1}}}}

\section{Desafio VII - Controle de Sistema Não-Linear} 

\subsection{Motivação}
(Explicar o problema que motiva o desafio, relevância, possíveis aplicações...) 

\subsection{Simulações realizadas}

\begin{subequations}
    \label{eq:modelo-do-oscilador-de-van-der-pol}
    \begin{equation}
        \begin{bmatrix}
            \dot{x}_1(t) \\
            \dot{x}_2(t)
        \end{bmatrix}
        =
        \begin{bmatrix}
            x_2(t) \\
            -x_1(t) + 0,3(1 - x_1(t)^2)x_2(t) + u(t)
        \end{bmatrix}
    \end{equation}
    \begin{equation}
        y(t) = x_1(t)
    \end{equation}
\end{subequations}

Por simplicidade na representação do Modelo de Van der Pol, o argumento de tempo
será omitido ao longo desta seção, salve necessidade de explicitar o argumento
de tempo para melhor compreensão.

(Explicar quais simulações foram realizadas de maneira descritiva e sequêncial.) 

\subsection{Resultados obtidos}

Como o modelo do oscilador de Van der Pol dado em espaço de estados pela Equação
\ref{eq:modelo-do-oscilador-de-van-der-pol} é não linear, torna-se necessário
fazer a linearização em torno de um ponto de equilíbrio. Considerando $\ssvec{z}
= \ssvec{x} - \ssveceq{x}$, $v = u - \bar{u}$ e $w = y - h(\ssveceq{x},
\bar{u})$ como novas variáveis do espaço de e reescrevendo
\ref{eq:modelo-do-oscilador-de-van-der-pol} como 

\begin{subequations}
    \label{eq:modelo-reescrito-do-oscilador-de-van-der-pol}
    \begin{equation}
        \label{eq:mapa-vetorial-dos-estados}
        \dot{\ssvec{x}} = f(\ssvec{x}, u)
        =
        \begin{bmatrix}
            \dot{x}_1 \\
            \dot{x}_2
        \end{bmatrix}
        =
        \begin{bmatrix}
            f_1(\ssvec{x}, u) \\
            f_2(\ssvec{x}, u) \\
        \end{bmatrix}
        =
        \begin{bmatrix}
            x_2 \\
            -x_1 + 0,3(1 - x_1^2)x_2 + u
        \end{bmatrix}
    \end{equation}
    \begin{equation}
        y = h(\ssvec{x}, u) = x_1
    \end{equation}
\end{subequations} o modelo não linear do Oscilador de Van der Pol pode ser
convertido para o modelo linear em torno do ponto de equilíbrio através de uma
linearização jacobiana em que

\begin{subequations}
    \label{eq:modelo-linearizado-em-torno-do-equilibrio}
    \begin{equation}
        \label{eq:derivada-dos-estados-linearizados}
        \dot{\ssvec{z}} = \mat{a}\ssvec{z} + \mathbf{B}v
    \end{equation}
    \begin{equation}
        w = \mat{C}\ssvec{z} + \mat{D}v
    \end{equation}
\end{subequations} sendo as matrizes definidas por

\begin{subequations}
    \label{eq:derivadas-parcias-do-vetor-estados}
    \begin{equation}
        \label{eq:jacobiana-de-a}  
        \mat{A} =
        \left.
            \begin{matrix}
                \frac{\partial f}{\partial x}
            \end{matrix}
            \right|_{(\bar{x}, \thinspace \bar{u})}
        =
        \begin{bmatrix}
            \left.
                \begin{matrix}
                    \frac{\partial f_1}{\partial x_1}
                \end{matrix}
            \right|_{(\bar{x}, \thinspace \bar{u})}
            &
            \left.
                \begin{matrix}
                    \frac{\partial f_1}{\partial x_2}
                \end{matrix}
            \right|_{(\bar{x}, \thinspace \bar{u})}
            \\
            \left.
                \begin{matrix}
                    \frac{\partial f_2}{\partial x_1}
                \end{matrix}
            \right|_{(\bar{x}, \thinspace \bar{u})}
            &
            \left.
                \begin{matrix}
                    \frac{\partial f_2}{\partial x_2}
                \end{matrix}
            \right|_{(\bar{x}, \thinspace \bar{u})}
        \end{bmatrix}
        =
        \begin{bmatrix}
            0 & 1 \\
            -1 & 0,3(1-\bar{x}_1^2)
        \end{bmatrix}
    \end{equation}

    \begin{equation}
        \label{eq:jacobiano-de-b}
        \mat{B} =
        \left.
            \begin{matrix}
                \frac{\partial f}{\partial u}
            \end{matrix}
        \right|_{(\bar{x}, \thinspace \bar{u})}
        =
        \begin{bmatrix}
            \left.
                \begin{matrix}
                    \frac{\partial f_1}{\partial u}
                \end{matrix}
            \right|_{(\bar{x}, \thinspace \bar{u})}
            \\
            \left.
                \begin{matrix}
                    \frac{\partial f_2}{\partial u}
                \end{matrix}
            \right|_{(\bar{x}, \thinspace \bar{u})}
        \end{bmatrix}
        =
        \begin{bmatrix}
            0 \\
            1
        \end{bmatrix}
    \end{equation}

    \begin{equation}
        \label{eq:jacoabiana-de-c}
        \mat{C} =
        \left.
            \begin{matrix}
                \frac{\partial h}{\partial x}
            \end{matrix}
        \right|_{(\bar{x}, \thinspace \bar{u})}
        =
        \begin{bmatrix}
            \left.
                \begin{matrix}
                    \frac{\partial h}{\partial x_1}
                \end{matrix}
            \right|_{(\bar{x}, \thinspace \bar{u})}
            &
            \left.
                \begin{matrix}
                    \frac{\partial h}{\partial x_2}
                \end{matrix}
            \right|_{(\bar{x}, \thinspace \bar{u})}
        \end{bmatrix}
        =
        \begin{bmatrix}
            1 & 0
        \end{bmatrix}
    \end{equation}

    \begin{equation}
        \mat{D} =
        \left.
            \begin{matrix}
                \frac{\partial h}{\partial u}
            \end{matrix}
        \right|_{(\bar{x}, \thinspace \bar{u})}
        = 0
    \end{equation}
\end{subequations}

Como o ponto de equilíbrio é definido por $[\dot{x}_1 \thickspace
\dot{x}_2]^\top = [0 \thickspace 0]^\top$, de
\ref{eq:modelo-do-oscilador-de-van-der-pol} tem-se que $\bar{x}_2 = 0$ e
$\bar{x_1} = \bar{u}$. Assim, considerando dois pontos de equilíbrio $\bar{x}_1
= 1$ e $\bar{x}_1 = 4$, as matrizes de dinâmica do sistema $\mat{A}_1$ e
$\mat{A}_2$, respectivamente, são definidas por

\begin{subequations}
    \label{eq:matrizes-a-linearizadas}
    \begin{equation}
        \label{eq:matriz-a1}
        \mat{A}_1
        =
        \begin{bmatrix}
            0   & 1 \\
            -1  & 0
        \end{bmatrix}
    \end{equation}

    \begin{equation}
        \label{eq:matriz-a2}
        \mat{A}_2
        =
        \begin{bmatrix}
            0   & 1 \\
            -1  & -4,5
        \end{bmatrix}
    \end{equation}
\end{subequations} cujo os autovalores são respectivamente $\lambda_1 = [i
\thickspace -i]^\top$ e $\lambda_2 = [-0,2344 \thickspace -4,2656]^\top$.
Nota-se portanto que para o ponto de equilíbrio $[1 \thickspace 0]^\top$ o
modelo linearizado possui um par conjugado de autovalores. Isto permite concluir
que partindo de um ponto nas proximidades deste ponto de equilíbrio ou com
pequenas variações do sinal de controle em torno de $\bar{u} = 1$, fará tanto os
estados quanto a saída oscilarem. Esta oscilação no diagrama de fases é
representada por uma trajetória circular em torno do ponto de equilíbrio,
chamada de círculo limite, como pode ser observado no diagrama de fases (Figura
\ref{fig:diagrama-de-fases-para-circulo-limite}) obtido no software Pplane
\textbf{CITAR!}. Por outro lado, para o ponto de equilíbrio $[4 \thickspace
0]^\top$ nas duas situações, o sistema converge para o ponto de equilíbrio.

\begin{figure}[h]
	\centering
	\caption{Diagrama de fases com trajetória circular (círculo limite) nas
    mediações do ponto de equilíbrio $[1 \thickspace 0]^\top$.}
	\label{fig:diagrama-de-fases-para-circulo-limite}
	\includegraphics[width=\textwidth]{chapters/challenge7/images/diagrama-de-fases-com-circulo-limite.png}
\end{figure}

Com os modelos linearizados foi obtido o vetor linha de ganhos $\mat{K}$ tal que
a ação de controle $v = \mat{k}\ssvec{z}(t)$ minimiza o custo do controlador
LQR $J$ dado por

\begin{equation}
    \label{eq:custo-do-controlador-lqr}
    J = \int_{t=0}^{\infty }\ssvec{z}(t)^\top \mat{Q} \ssvec{z}(t) + v(t)^\top R v(t)
\end{equation} com $\mat{Q} = \mat{I}$ e $R = 0,1$. Com isso, através da função
\textit{lqr} do Octave resultou ganhos $\mat{K}_1$ e $\mat{K}_2$, conforme
Equação \ref{eq:ganhos-do-controlador-lqr}, relacionados respectivamente as
matrizes $\mat{A}_1$ e $\mat{A}_2$ (Equação \ref{eq:matrizes-a-linearizadas}).

\begin{subequations}
    \label{eq:ganhos-do-controlador-lqr}
    \begin{equation}
        \mat{K}_1 = [2,3166 \thickspace 3,8253]
    \end{equation}
    \begin{equation}
        \mat{K}_2 = [2,3166 \thickspace 1,4062]
    \end{equation}
\end{subequations}

Com isso, definindo $\mat{k} = [k_1 \thickspace k_2]$ e retornando a ação de
controle para as variáveis de estados originais, isto é

\begin{equation}
    \label{eq:acao-de-controle-do-sistema-linear}
    v = -\mat{k}\ssvec{z}
    \Rightarrow
    u = -\mat{k}(\ssvec{x} - \ssveceq{x}) + \bar{u}
    = -k_1(x_1 - \bar{x}_1) - k_2(x_2 - \bar{x}_2) + \bar{u}
\end{equation} o modelo do Oscilador de Van Der Pol pode ser reescrito como

\begin{equation}
    \label{eq:modelo-reescrito2-do-oscilador-de-van-der-pol}
    \begin{bmatrix}
        \dot{x}_1 \\
        \dot{x}_2
    \end{bmatrix}
    =
    \begin{bmatrix}
        x_2 \\
        -x_1 + 0,3(1 - x_1^2)x_2 - k_1(x_1 - \bar{x}_1) - k_2(x_2 - \bar{x}_2) + \bar{u}
    \end{bmatrix}.
\end{equation}

Para análise do comportamento do sistema, foram gerados, com auxílio do software
Pplane com o parâmetro \textit{Solution Direction} igual \textit{Forward},
diagramas de fase com o considerando três cenários:

\begin{itemize}
    \item i) Modelo do oscilador com $u(t) = 0$;
    \item ii) Modelo do oscilador utilizando o ganho LQR $\mat{K}_1$ obtido com
    a linearização em torno do ponto de equilíbrio $[1 \thickspace 0]^\top$; e
    \item iii) Modelo do oscilador utilizando o ganho LQR $\mat{K}_2$ obtido com
    a linearização em torno do ponto de equilíbrio $[4 \thickspace 0]^\top$.
\end{itemize}

O diagrama de fases do cenário (i) é exibido na Figura
\ref{fig:diagrama-de-fases-sem-controle}. Percebe-se na Figura
\ref{fig:diagrama-de-fases-sem-controle-ponto-de-equilibrio-1-e-2} que definindo
a condição inicial aproximadamente os pontos de equilíbrios \peq1

\begin{figure}[ht]
    \caption{Diagrama de fases do modelo do oscilador com $u(t) = 0$.}
    \label{fig:diagrama-de-fases-sem-controle}
    \centering
    \begin{subfigure}[t]{0.32\textwidth}
        \centering
	    \includegraphics[width=\textwidth]{chapters/challenge7/images/diagrama-de-fases-sem-controle.png}
        \caption{Partindo de condições iniciais aproximadamente iguais aos pontos de equilíbrio.}
        \label{fig:diagrama-de-fases-sem-controle-ponto-de-equilibrio-1-e-2}
        
    \end{subfigure}
    \hfill
    \begin{subfigure}[t]{0.32\textwidth}
        \centering
	    \includegraphics[width=\textwidth]{chapters/challenge7/images/diagrama-de-fases-sem-controle-zoom1.png}
        \caption{Partindo de condições iniciais na vizinhança do ponto de equilíbrio 1.}
        \label{fig:diagrama-de-fases-sem-controle-ponto-de-equilibrio-1}
    \end{subfigure}
    \hfill
    \begin{subfigure}[t]{0.32\textwidth}
        \centering
	    \includegraphics[width=\textwidth]{chapters/challenge7/images/diagrama-de-fases-sem-controle-zoom2.png}
        \caption{Partindo de condições iniciais na vizinhança do ponto de equilíbrio 2.}
        \label{fig:diagrama-de-fases-sem-controle-ponto-de-equilibrio-2}
    \end{subfigure}
\end{figure}

\subsection{Conclusões}
(Concluir em que medida os resultados apresentam relação com a motivação.
Permitem ilustratar ou concluir algo sobre a motivação? )
