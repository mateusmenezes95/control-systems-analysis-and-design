\subsubsection{Alocação de Polos x IMC}

A Figura \ref{fig:comparacao-entre-imc-e-alocacao-de-polos} mostra o comparativo
das respostas do sistema de controle amostrado da
\ref{fig:diagrama-de-blocos-malha-fechada} utilizando os dois controladores
projetados. Para esta simulação a perturbação na entrada $q_u(t) =
0,2\mathds{1}(t - 7)$ e na saída $q_y(t) = 0,2\mathds{1}(t - 12)$. Observa-se
que ambos os controladores atendem o requisito de tempo de acomodação e
seguimento de referência. Entretanto, apesar dos dois controladores também façam
com que perturbações constantes do tipo degrau sejam rejeitadas, nota-se que o
projeto via alocação de polos fornecem uma resposta mais rápida a perturbações
se comparada com o projeto via IMC.

\begin{figure}[!ht]
    \caption{Comparação das repostas do sistema em malha fechada com
    controladores projetados via Alocação de Polos e IMC.}
    \vspace{-10pt}
    \hspace{-30pt}
    \label{fig:comparacao-entre-imc-e-alocacao-de-polos}
    \begin{minipage}{\linewidth}
        % Title: gl2ps_renderer figure
% Creator: GL2PS 1.4.0, (C) 1999-2017 C. Geuzaine
% For: Octave
% CreationDate: Sun Sep 26 17:47:58 2021
\setlength{\unitlength}{1pt}
\begin{picture}(0,0)
\includegraphics{images/challenge2/resultado-questao-5-inc}
\end{picture}%
\begin{picture}(400,250)(0,0)
\fontsize{6}{0}
\selectfont\put(52,141.404){\makebox(0,0)[t]{\textcolor[rgb]{0.15,0.15,0.15}{{0}}}}
\fontsize{6}{0}
\selectfont\put(103.675,141.404){\makebox(0,0)[t]{\textcolor[rgb]{0.15,0.15,0.15}{{10}}}}
\fontsize{6}{0}
\selectfont\put(155.351,141.404){\makebox(0,0)[t]{\textcolor[rgb]{0.15,0.15,0.15}{{20}}}}
\fontsize{6}{0}
\selectfont\put(207.026,141.404){\makebox(0,0)[t]{\textcolor[rgb]{0.15,0.15,0.15}{{30}}}}
\fontsize{6}{0}
\selectfont\put(258.701,141.404){\makebox(0,0)[t]{\textcolor[rgb]{0.15,0.15,0.15}{{40}}}}
\fontsize{6}{0}
\selectfont\put(310.376,141.404){\makebox(0,0)[t]{\textcolor[rgb]{0.15,0.15,0.15}{{50}}}}
\fontsize{6}{0}
\selectfont\put(48.5278,155.215){\makebox(0,0)[r]{\textcolor[rgb]{0.15,0.15,0.15}{{0}}}}
\fontsize{6}{0}
\selectfont\put(48.5278,176.659){\makebox(0,0)[r]{\textcolor[rgb]{0.15,0.15,0.15}{{0.5}}}}
\fontsize{6}{0}
\selectfont\put(48.5278,198.104){\makebox(0,0)[r]{\textcolor[rgb]{0.15,0.15,0.15}{{1}}}}
\fontsize{6}{0}
\selectfont\put(48.5278,219.548){\makebox(0,0)[r]{\textcolor[rgb]{0.15,0.15,0.15}{{1.5}}}}
\fontsize{7}{0}
\selectfont\put(33.5278,186.325){\rotatebox{90}{\makebox(0,0)[b]{\textcolor[rgb]{0.15,0.15,0.15}{{Saída $y_{i}(t)$}}}}}
\fontsize{7}{0}
\selectfont\put(207,130.404){\makebox(0,0)[t]{\textcolor[rgb]{0.15,0.15,0.15}{{Tempo (s)}}}}
\fontsize{6}{0}
\selectfont\put(332,210.139){\makebox(0,0)[l]{\textcolor[rgb]{0,0,0}{{$r(t)$}}}}
\fontsize{6}{0}
\selectfont\put(332,198.639){\makebox(0,0)[l]{\textcolor[rgb]{0,0,0}{{$y_{1}(t)$}}}}
\fontsize{6}{0}
\selectfont\put(332,186.638){\makebox(0,0)[l]{\textcolor[rgb]{0,0,0}{{$y_{2}(t)$}}}}
\fontsize{6}{0}
\selectfont\put(332,174.638){\makebox(0,0)[l]{\textcolor[rgb]{0,0,0}{{$y_{3}(t)$}}}}
\fontsize{6}{0}
\selectfont\put(332,162.638){\makebox(0,0)[l]{\textcolor[rgb]{0,0,0}{{$y_{4}(t)$}}}}
\fontsize{6}{0}
\selectfont\put(52,22.2671){\makebox(0,0)[t]{\textcolor[rgb]{0.15,0.15,0.15}{{0}}}}
\fontsize{6}{0}
\selectfont\put(103.675,22.2671){\makebox(0,0)[t]{\textcolor[rgb]{0.15,0.15,0.15}{{10}}}}
\fontsize{6}{0}
\selectfont\put(155.351,22.2671){\makebox(0,0)[t]{\textcolor[rgb]{0.15,0.15,0.15}{{20}}}}
\fontsize{6}{0}
\selectfont\put(207.026,22.2671){\makebox(0,0)[t]{\textcolor[rgb]{0.15,0.15,0.15}{{30}}}}
\fontsize{6}{0}
\selectfont\put(258.701,22.2671){\makebox(0,0)[t]{\textcolor[rgb]{0.15,0.15,0.15}{{40}}}}
\fontsize{6}{0}
\selectfont\put(310.376,22.2671){\makebox(0,0)[t]{\textcolor[rgb]{0.15,0.15,0.15}{{50}}}}
\fontsize{6}{0}
\selectfont\put(48.5278,34.7158){\makebox(0,0)[r]{\textcolor[rgb]{0.15,0.15,0.15}{{0}}}}
\fontsize{6}{0}
\selectfont\put(48.5278,52.7559){\makebox(0,0)[r]{\textcolor[rgb]{0.15,0.15,0.15}{{0.5}}}}
\fontsize{6}{0}
\selectfont\put(48.5278,70.7954){\makebox(0,0)[r]{\textcolor[rgb]{0.15,0.15,0.15}{{1}}}}
\fontsize{6}{0}
\selectfont\put(48.5278,88.8354){\makebox(0,0)[r]{\textcolor[rgb]{0.15,0.15,0.15}{{1.5}}}}
\fontsize{6}{0}
\selectfont\put(48.5278,106.875){\makebox(0,0)[r]{\textcolor[rgb]{0.15,0.15,0.15}{{2}}}}
\fontsize{7}{0}
\selectfont\put(33.5278,67.1875){\rotatebox{90}{\makebox(0,0)[b]{\textcolor[rgb]{0.15,0.15,0.15}{{Sinal $u_{i}(t)$}}}}}
\fontsize{7}{0}
\selectfont\put(207,11.2671){\makebox(0,0)[t]{\textcolor[rgb]{0.15,0.15,0.15}{{Tempo (s)}}}}
\fontsize{6}{0}
\selectfont\put(332,76.0015){\makebox(0,0)[l]{\textcolor[rgb]{0,0,0}{{$u_{1}(t)$}}}}
\fontsize{6}{0}
\selectfont\put(332,65.001){\makebox(0,0)[l]{\textcolor[rgb]{0,0,0}{{$u_{2}(t)$}}}}
\fontsize{6}{0}
\selectfont\put(332,54.001){\makebox(0,0)[l]{\textcolor[rgb]{0,0,0}{{$u_{3}(t)$}}}}
\fontsize{6}{0}
\selectfont\put(332,43.0005){\makebox(0,0)[l]{\textcolor[rgb]{0,0,0}{{$u_{4}(t)$}}}}
\end{picture}

    \end{minipage}
\end{figure}

Para analisar o comportamento da resposta as perturbações dos dois diferentes
projetos, é necessário a inspeção dos pares entrada/saída. Percebe-se nas
Equações \ref{eq:ft-da-saida-para-perturbacao-na-entrada} e
\ref{eq:ft-da-saida-para-perturbacao-na-saida} que os polos das funções de
transferência da entrada para saída foram exatamente os polos alocados, sendo os
polos observáveis 10 vezes mais rápidos que os polos controláveis, que já são
rápidos devido a constante de tempo $\tau = 0,5$. Em contrapartida, devido a
propriedade intrínseca de cancelamento do projeto via IMC, alguns polos lentos
podem aparecer nas funções de transferência das perturbações para saída do
sistema em malha fechada.

A partir da Equação \ref{eq:filtro-de-robustez} e do diagrama de blocos
\ref{fig:diagrama-de-blocos-malha-fechada}, as seguintes funções de
transferência são deduzidas:

\begin{equation}
    \label{eq:ft-perturbacao-entrada-imc}
    \frac{Y(z)}{Q_u(z)} = P_n(z)(1-F_r(z)) = P_n(z)S(z)
\end{equation} em que $S(z)$ é a complementar de sensibilidade e

\begin{equation}
    \label{eq:ft-perturbacao-saida-imc}
    \frac{Y(z)}{Q_y(z)} = 1-F_r(z) = S(z).
\end{equation}

Percebe-se então que para um perturbação na entrada da planta $q_u(t)$, aparecem
os polos da própria planta, que neste caso são polos lentos se comparados com os
da malha fechada via projeto por alocação de polos. Por outro lado, para
perturbações na saída $q_y(t)$ os polos são os mesmos que os desejáveis, que são
rápidos e, portanto, fornecem uma resposta a perturbação mais rápida. 

É notório também a influência de zeros na rejeição a perturbação na saída da
planta. Observa-se que para o projeto via alocação de polos, a saída da planta
$y(t)$ cai abaixo do valor de \textit{set point}, devido a um zero negativo fora
do círculo unitário (-1.3319). Já em \ref{eq:ft-perturbacao-saida-imc} não há
zeros negativo fora do circulo unitário e, consequentemente, o mesmo
comportamento não é observado.

Vale salientar também a importância de não realizar o cancelamento do zero de
fase não mínimas nos projetos alocando os polos do controlador no mesmo local.
Caso este polo fosse alocado nos controladores projetados, ele não iria aparecer
nas funções de transferência listadas neste tópico. Entretanto, ao analisar a
Gangue dos Seis definidas por \cite{Astrom2008}, caso alocado no controlador, o
polo instável aparece em dois pares de entrada saída conforme a Equação abaixo,
em que o argumento foi omitido por simplicidade e o par entrada/saida é definido
pelo subescrito.

\begin{subequations}
    \begin{equation}
        G_{ur} = \frac{CF}{1+PC}
    \end{equation}
    \begin{equation}
        -G_{uw} = \frac{C}{1+PC}
    \end{equation}
\end{subequations} em que $u$, $r$ e $w$ significam sinal de controle,
referência e ruído respectivamente e $P$ e $C$ significam respectivamente função
de transferência da planta e controlador.
