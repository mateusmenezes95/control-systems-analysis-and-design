\subsubsection{Projeto via IMC}
\label{subsub:projeto-via-imc}

O controlador IMC foi projetado com base no diagrama de blocos da Figura
\ref{fig:diagrama-de-blocos-imc} e sua associação com o diagrama da Figura
\ref{fig:diagrama-de-blocos-malha-fechada}. Dessa forma, o filtro de referência
e o controlador é dado da seguinte forma:

\begin{equation}
    \label{eq:filtro-de-referencia-imc}
    F(z) = \frac{H_d(z)}{F_r(z)}
\end{equation} e

\begin{equation}
    \label{eq:controlador-imc}
    C(z) = \frac{F_r(z)P_{n}^{-1}(z)}{1-F_r(z)}
\end{equation} em que $P_n(z)$ neste caso é igual o modelo da planta dado por
\ref{eq:modelo-em-z-da-planta}. As restrições impostas é que para manter
causalidade, $g_r\left [ F_r(z) \right ] \geq g_r\left [ P_n(z) \right ]$ e
$g_r\left [ H_d(z) \right ] \geq g_r\left [ F_r(z) \right ]$, em que $g_r$
significa grau relativo ou excedentes de polos em relação a zeros.

\begin{figure}[ht!]
	\centering
	\caption{Diagrama de blocos de um controlador IMC.}
	\label{fig:diagrama-de-blocos-imc}
	\includegraphics[width=\textwidth]{chapters/challenge4/images/diagrama-de-blocos-imc.png}
    \caption*{Fonte: \cite{Tito2021}}
\end{figure}

Através de álgebra partindo da Equação \ref{fig:diagrama-de-blocos-imc},
observa-se que o filtro de robustez $F_r(z)$ é igual a complementar de
sensibilidade $\boldsymbol{C}(z)$. Assim, temos que

\begin{equation}
    \label{eq:associacao-do-filtro-de-robustez}
    F_r(z) = \boldsymbol{C}(z) = \frac{C(z)G(z)}{1+C(z)G(z)} = \frac{Y(z)}{Y_{r}(z)}.
\end{equation}

Portanto, observa-se que o filtro de robustez $F_r(z)$ é nada mais que e a
função de transferência em $z$ da entrada $y_r(t)$ para a saída $y(t)$. Desta
maneira definiu-se $F_r(z)$ para ter os polos desejáveis $z^* = 0,7408$ que
resulta no atendimento dos requisitos de regime transitório. Já para os zeros,
foram mantidos os zeros de fase não mínima em $F_r(z)$ para evitar cancelamentos
indesejáveis que leve a instabilidade dado uma perturbação na entrada ou saída
da planta. Por fim, para atender o requisito de rejeição de perturbação, o
controlador precisa possuir um integrador. Isso é assegurado fazendo $F_r(1) =
1$ pois $F_r(1)P_{n}^{-1}(1) \neq 0$ e, portanto, $C(1) \rightarrow \infty$, o
que garante que o controlador possui ação integral. Observa-se depois destas
definições que $F_r(z)$ torna-se igual a função de transferência
\ref{eq:ft-da-saida-para-referencia}, que é então

\begin{equation}
    \label{eq:filtro-de-robustez}
    F_r(z^{-1}) = \frac{-0,00943z^{-1}+0,07661z^{-2}}{1-1,482z^{-1}+0,5488z^{-2}}
\end{equation}

Substituindo \ref{eq:modelo-em-z-da-planta} e \ref{eq:filtro-de-robustez} em
\ref{eq:controlador-imc}, tem-se que

\begin{equation}
    \label{eq:controlador-imc-definido}
    C(z^{-1}) = \frac{1,731-2,98z^{-1}+1,282z^{-2}}{1-1,472z^{-1}+0,4722z^{-2}}.
\end{equation}

Como não é necessário um outro grau de liberdade no projeto do controlador,
faz-se $F_r(z) = H_d(z)$ e, portanto, $F(z) = 1$. Por fim, observa-se que a
causalidade do sistema foi mantida, pois $g_r\left [ F_r(z) \right ] = g_r\left
[ P_n(z) \right ]$. 