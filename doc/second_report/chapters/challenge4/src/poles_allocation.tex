\subsubsection{Projeto via Alocação de Polos}
\label{subsub:projeto-via-alocacao-de-polos}
O projeto via alocação de polos começa reescrevendo as funções de transferências
dos elementos que compões a malha fechada (Figura
\ref{fig:diagrama-de-blocos-malha-fechada}) em frações de polinômios em
$z^{-1}$. Dessa forma, tem-se que

\begin{equation}
    \label{eq:modelo-da-planta-com-fracoes-de-polinomios}
    G(z^{-1}) = \frac{B(z^{-1})}{A(z^{-1})},
\end{equation}

\begin{equation}
    \label{eq:filtro-de-referencia-com-fracoes-de-polinomios}
    F(z^{-1}) = \frac{T(z^{-1})}{R(z^{-1})},
\end{equation}

\begin{equation}
    \label{eq:controlador-com-fracoes-de-polinomios}
    C(z^{-1}) = \frac{S(z^{-1})}{R(z^{-1})}.
\end{equation}

A partir das Equações \ref{eq:modelo-da-planta-com-fracoes-de-polinomios},
\ref{eq:filtro-de-referencia-com-fracoes-de-polinomios} e
\ref{eq:controlador-com-fracoes-de-polinomios}, define-se $\eta_{p}$ como o grau
do polinômio $P(z^{-1})$. Assim, para que o sistema seja causal,
consequentemente implementável, as seguintes restrições devem ser seguidas:
$\eta_{r} \geq \eta_{s}$ e $\eta_{s} \geq \eta_{t}$. Das mesmas equações
as funções de transferências para malha fechada da Figura
\ref{fig:diagrama-de-blocos-malha-fechada} podem ser reescritas como

\begin{equation}
    \label{eq:ft-da-saida-para-referencia}
    \frac{Y(z)}{Y_{r}(z)} = \frac{T(z^{-1})B(z^{-1})}
                                 {R(z^{-1})A(z^{-1})+S(z^{-1})B(z^{-1})}
                          = \frac{t_{0}B(z^{-1})}{\lambda_{c}(z^{-1})},
\end{equation}

\begin{equation}
    \label{eq:ft-da-saida-para-perturbacao-na-entrada}
    \frac{Y(z)}{Q_{u}(z)} = \frac{R(z^{-1})B(z^{-1})}
                                 {R(z^{-1})A(z^{-1})+S(z^{-1})B(z^{-1})}
                          = \frac{R(z^{-1})B(z^{-1})}
                                 {\lambda_{c}(z^{-1})\lambda_{o}(z^{-1})},
\end{equation}

\begin{equation}
    \label{eq:ft-da-saida-para-perturbacao-na-saída}
    \frac{Y(z)}{Q_{y}(z)} = \frac{R(z^{-1})A(z^{-1})}
                                 {R(z^{-1})A(z^{-1})+S(z^{-1})B(z^{-1})}
                          = \frac{R(z^{-1})A(z^{-1})}
                                 {\lambda_{c}(z^{-1})\lambda_{o}(z^{-1})}.
\end{equation} em que $\lambda_{c}(z^{-1})$ e $\lambda_{o}(z^{-1})$ são polinômios
contendo os polos controláveis e observáveis do sistema em malha fechada,
respectivamente, e 

\begin{equation}
    \label{eq:numerador-do-filtro-de-referencia}
    T(z^{-1}) = t_{o}\lambda_{o}(z^{-1}).
\end{equation}

Dessa forma, tem-se que a \textbf{equação característica desejada} é dada por
\begin{equation}
    \label{eq:equacao-caracteristica-sem-coeficientes}
    \begin{split}
        \lambda(z^{-1}) &= R(z^{-1})A(z^{-1})+S(z^{-1})B(z^{-1}) \\
                        &= \lambda_{c}(z^{-1})\lambda_{o}(z^{-1}).
    \end{split}
\end{equation}

Como comentado anteriormente, é necessário que o controlador possua ação
integradora para rejeição de perturbação e seguimento de referência. Assim, deve
se empregar o polinômio $V(z^{-1}) = 1 - z^{-1}$ na função de transferência do
controlador. Desse modo, a Equação
\ref{eq:controlador-com-fracoes-de-polinomios} torna-se

\begin{equation}
    \label{eq:controlador-com-fracoes-de-polinomios-com-integrador}
    C(z^{-1}) = \frac{S(z^{-1})}{V(z^{-1})R'(z^{-1})}.
\end{equation} em que $R(z^{-1}) = V(z^{-1})R'(z^{-1})$ e
$\eta_r = \eta_v + \eta_{r'}$. Como agora apenas os coeficientes de $R'$ são
livres, para-se ter uma solução única e determinada para o sistema de equações
que define os coeficientes de $C(z^{-1})$ a igualdade $\eta_s = \eta_v + \eta_a
- 1 = 2$ tem que ser satisfeita. E para que o controlador seja causal, faz-se
$\eta_r = \eta_s$. Portanto, a função de transferência do controlador é então

\begin{equation}
    C(z^{-1}) = \frac{s_0+s_1z^{-1}+s_2z^{-2}}{(1-z^{-1})(1-rz^{-1})} 
              = \frac{s_0+s_1z^{-1}+s_2z^{-2}}{1+(-r-1)z^{-1}+rz^{-2}}.
\end{equation}

Com a imposição da ação integradora, faz-se necessário arbitrar dois polos
adicionais na equação característica desejada para que seja satisfeita a
igualdade $\eta_\lambda = \eta_a + \eta_r = 4$. Optou-se então por mais dois
polos duplos iguais afastados 10 vezes dos polos dominantes. Este polos são
justamente os polos observáveis que originam o polinômio
$\lambda_o(z^{-1})$, enquanto os polos dominantes (ou controláveis)
originam o polinômio $\lambda_c(z^{-1})$. Assim os polos controláveis $z_{c}^{*}
= 0.7408$ e os polos observáveis $z_{o}^{*} = 0.049787$.

Com estas definições, a Equação \ref{eq:equacao-caracteristica-sem-coeficientes}
torna-se
\begin{equation}
    \label{eq:equacao-caracteristica-com-coeficientes}
    \begin{split}
        R(z^{-1})A(z^{-1})+S(z^{-1})B(z^{-1})
        = \lambda_0+\lambda_1z^{-1}+\lambda_2z^{-2}+\lambda_3z^{-3}+\lambda_4z^{-4} \\
        = 1-1,581z^{-1}+0,6988z^{-2}-0,05832z^{-3}+0,00136z^{-4}
    \end{split}
\end{equation}

A partir da Equação \ref{eq:equacao-caracteristica-com-coeficientes}, chega-se
ao conjunto de equações lineares

\begin{equation}
    \label{eq:sistema-de-equacoes-lineares}
    \begin{split}
        s_0b_0 - r &= \lambda_1-a_1+1\\ 
        s_0b_1 + s_1b_0 + r(1-a_1) &= \lambda_2+a_1-a_2 \\ 
        s_1b_1 + s_2b_0 + r(a_1-a_2) &= \lambda_3+a_2 \\  
        s_2b_1 + ra_2 &= \lambda_4 
    \end{split}
\end{equation} que reescrita de forma matricial se torna

\begin{equation}
    \label{eq:equacoes-lineares-em-forma-matricial}
    \begin{bmatrix}
        b_0 & 0   & 0   & -1     \\ 
        b_1 & b_0 & 0   & 1-a_1  \\ 
        0   & b_1 & b_0 & a_1-a2 \\ 
        0   & 0   & b_1 & a_2
        \end{bmatrix}
        \begin{bmatrix}
        s_0 \\ 
        s_1 \\ 
        s_2 \\ 
        r
        \end{bmatrix}
        =
        \begin{bmatrix}
        \lambda_1\\ 
        \lambda_2\\ 
        \lambda_3\\ 
        \lambda_4
        \end{bmatrix}
        +
        \begin{bmatrix}
        -a_1+1\\ 
        a_1-a_2\\ 
        a_2\\ 
        0
        \end{bmatrix}
\end{equation} sendo a solução dada por 

\begin{equation}
    \label{eq:solucao-da-equacao-linear}
    \begin{bmatrix}
        s_0 \\ 
        s_1 \\ 
        s_2 \\ 
        r
        \end{bmatrix}
        =
        \begin{bmatrix}
        b_0 & 0   & 0   & -1     \\ 
        b_1 & b_0 & 0   & 1-a_1  \\ 
        0   & b_1 & b_0 & a_1-a2 \\ 
        0   & 0   & b_1 & a_2
        \end{bmatrix}^{-1}
        \begin{bmatrix}
        \lambda_1-a_1+1\\ 
        \lambda_2+a_1-a_2\\ 
        \lambda_3+a_2\\ 
        \lambda_4
        \end{bmatrix}.
\end{equation}

Substituindo \ref{eq:modelo-em-z-da-planta} e
\ref{eq:equacao-caracteristica-com-coeficientes} em
\ref{eq:solucao-da-equacao-linear}, tem-se então que

\begin{equation}
    \label{eq:coeficientes-do-controlador}
    \begin{bmatrix}
        s_0 \\ 
        s_1 \\ 
        s_2 \\ 
        r
    \end{bmatrix}
        =
    \begin{bmatrix}
        35,168 \\ 
        -55,931 \\ 
        22,326 \\ 
        -1,3319
    \end{bmatrix}
\end{equation} e então

\begin{equation}
    \label{eq:controlador-discreto}
    C(z^{-1}) = \frac{35,17-55,93z^{-1}+22,33z^{-2}}{1+0,3319z^{-1}-1,332z^{-2}}.
\end{equation}

Já a definição do filtro de referência $F(z^{-1})$ (Equação
\ref{eq:filtro-de-referencia-com-fracoes-de-polinomios}) considerou a Equação
\ref{eq:numerador-do-filtro-de-referencia} sendo

\begin{equation}
    \label{eq:definicao-dos-polos-0bservaveis}
    \lambda_o(z^{-1}) = 1 - 0.09957 z^{-1} + 0.002479 z^{-2}
\end{equation} e para garantir seguimento de referência, a partir da Equação
\ref{eq:ft-da-saida-para-referencia}, tem-se que

\begin{equation}
    \label{eq:ganho-do-filtro-de-referencia}
    t_0 = \frac{\lambda_c(1)}{B(1)} = 1,7311
\end{equation} e, portanto,

\begin{equation}
    F(z^{-1}) = \frac{0,04922-0,004901z^{-1}+0,000122z^{-2}}{1-1,59z^{-1}+0,6348z^{-2}}.
\end{equation}

Assim o projeto por alocação de polos está finalizado e os polinômios
$R(z^{-1})$, $S(z^{-1})$ e $T(z^{-1})$ são

\begin{equation}
    \label{eq:polinomios-rst}
    \begin{split}
        R(z^{-1}) &= 1+0,3319z^{-1}-1,332z^{-2}, \\
        S(z^{-1}) &= 35,17-55,93z^{-1}+22,33z^{-2}, \\
        T(z^{-1}) &= 1,731-0,1724z^{-1}+0,004291z^{-2}.
    \end{split}
\end{equation}