\section{Desafio IV - Sistemas de Controle Amostrados} 

\subsection{Motivação}
Este desafio visa demonstrar duas técnicas de projeto de controladores de
sistemas de controle amostrados diretamente no domínio de tempo discreto. O
aprendizado dessas técnicas é de extrema importância na aplicação prática de
sistemas de controle, visto que nas últimas décadas sistemas implementados por
computadores analógicos vem sendo substituídos por computadores digitais. Isto
se observa na predominância de sistemas de controle amostrados em problemas
complexos de controle, como aplicações em diversos sistemas de espaçonaves,
piloto automático de aeronaves, carros elétricos autônomos, geração de energia
elétrica através de usinas nucleares etc, até problemas simples como o controle
de velocidade das rodas de um robô diferencial. Além do mais, controladores
modernos, como o caso de controladores adaptativos, são desenvolvidos partindo
dos conceitos e ferramentas oferecidos por sistemas de controle amostrados.
Desta forma, torna-se imprescindível para profissionais e/ou pesquisadores da
área de controle possuir conhecimento dos sistemas de controle amostrados.

\subsection{Simulações realizadas}
\label{sub:simulacoes-realizadas-desafio4}
As simulações realizadas neste desafio visaram demonstrar e comparar as
respostas de um sistema de controle realimentado utilizando duas técnicas de
projeto de controladores no domínio do tempo discreto: alocação de polos e
controlador por modelo interno (IMC). O sistema em malha fechada utilizado como
base para as simulações está representado na Figura
\ref{fig:diagrama-de-blocos-malha-fechada}, em que:

\begin{itemize}
    \item $R(z)$ é a transformada Z do sinal de referência $r(k)$;
    \item $E(z)$ é a transformada Z do erro $e(k)$;
    \item $C(z)$ é a função de transferência do controlador no domínio Z;
    \item $SoZ$ é a função de transferência do sustentador de ordem zero;
    \item $U(s)$ é a transformada de Laplace do sinal de controle $u(t)$;
    \item $Q_{u}(s)$ é a transformada de Laplace da perturbação de entrada
    $q_{u}(t)$;
    \item $G(s)$ é a função de transferência da planta;
    \item $Q_{y}(s)$ é a transformada de Laplace da perturbação na saída
    $q_{y}(t)$; e, por fim
    \item $Y(s)$ é a transformada de Laplace da saída do sistema $y(t)$.
\end{itemize}

\begin{figure}[ht!]
	\centering
	\caption{Sistema em malha fechada de um sistema de controle amostrado e com
    perturbações persistentes.}
	\label{fig:diagrama-de-blocos-malha-fechada}
	\includegraphics[width=\textwidth]{chapters/challenge4/images/diagrama-de-blocos-sistema-de-controle-amostrado.png}
\end{figure}

Os controladores foram projetados a partir da planta cujo o modelo está definido
na Equação abaixo.

\begin{equation}
    \label{eq:modelo-em-s-da-planta}
    G(s) = \frac{0,2(10 - s)}{(s+1)^{2}}
\end{equation}

Com base no modelo definido na Equação \ref{eq:modelo-em-s-da-planta}, os dois
controladores foram projetados para que o sistema em malha fechada atendesse os
seguintes requisitos:

\begin{enumerate}
    \item Tempo de acomodação de 2\% da resposta da saída para variação de
    referência do tipo degrau seja aproximadamente metade do tempo de acomodação
    de 2\% da resposta em malha aberta;
    \item Seguimento de referência do tipo degrau; e
    \item Rejeição de perturbações constantes.
\end{enumerate}

Como parâmetro do projeto, foi utilizado o período de amostragem de $T_{a} =
0,15 s$, baseado no critério definido por \cite[p. 61]{Franklin1997} em que
"\textit{Generally, sample rates should be faster than 30 times the bandwidth in
order to assure that the digital controller can be made to closely match the
performance of the continuous controller.}". Já a discretização do modelo da
planta foi realizada através da função c2d do Octave utilizando o método de
conversão $zoh$. Assim, o modelo da planta discretizada utilizando os parâmetros
acima é dado por

\begin{equation}
    \label{eq:modelo-em-z-da-planta}
    G(z) = \frac{-0,00545z^{-1} + 0,04425z^{-2}}{1 - 1.721z^{-1} + 0.7408z^{-2}}.
\end{equation}

Na próxima seção são apresentados os resultados para seguimento de referência e
rejeição a perturbações do tipo degrau para o sistema de controle amostrado
contendo os dois controladores projetados.

\subsection{Resultados obtidos}
As duas técnicas de projeto controle visam definir o controlador a partir de um
modelo de planta desejado. Sendo assim, para atender o requisito de regime
transitório, foram escolhidos os mesmos polos desejados para os dois
controladores. Para isso, optou-se por dois polos iguais para se valer da
aproximação de sistema com polos repetidos no domínio do tempo em que o tempo de
acomodação de 2\% ($t_{s_{2\%}}$) é aproximado por $6\tau$. Como os dois polos
de $G(s)$ (Equação \ref{eq:modelo-em-s-da-planta}) fornecem um tempo de
acomodação de $t_{s_{2\%}}\approx 6s$ ($\tau = 1$), definiu-se $\tau = 0,5$ que
consequentemente $t_{s_{2\%}}\approx 3s$. Assim, o objetivo dos
controladores é proporcionar dois polos $s^*=-2$, que no tempo discreto
utilizando a conversão $e^{-T_{a}s}$ e $T_{a} = 0.15s$ chega-se a dois polos
$z^*=0,7408$.

Para atender os requisitos de regime estacionário seguimento de referência e
rejeição a perturbações constantes do tipo degrau, os controladores devem
possuir ao menos um integrador. Esta premissa parte do Princípio do Modelo
Interno, pois como nesse caso não há um integrador na planta, tal ação deve ser
proporcionada pelo controlador.

Como os projetos em si se materializam de forma diferente, apesar das mesmas
premissas, a subseções seguintes aborda de forma detalhada cada uma das
técnicas. Vale salientar, que embora as duas técnicas de projeto possam ser
aplicadas também em tempo contínuo, os projetos foram realizados em sua
completude no domínio do tempo discreto.

\subsubsection{Projeto via Alocação de Polos}
\label{subsub:projeto-via-alocacao-de-polos}
O projeto via alocação de polos começa reescrevendo as funções de transferências
dos elementos que compões a malha fechada (Figura
\ref{fig:diagrama-de-blocos-malha-fechada}) em frações de polinômios em
$z^{-1}$. Dessa forma, tem-se que

\begin{equation}
    \label{eq:modelo-da-planta-com-fracoes-de-polinomios}
    G(z^{-1}) = \frac{B(z^{-1})}{A(z^{-1})},
\end{equation}

\begin{equation}
    \label{eq:filtro-de-referencia-com-fracoes-de-polinomios}
    F(z^{-1}) = \frac{T(z^{-1})}{R(z^{-1})},
\end{equation}

\begin{equation}
    \label{eq:controlador-com-fracoes-de-polinomios}
    C(z^{-1}) = \frac{S(z^{-1})}{R(z^{-1})}.
\end{equation}

A partir das Equações \ref{eq:modelo-da-planta-com-fracoes-de-polinomios},
\ref{eq:filtro-de-referencia-com-fracoes-de-polinomios} e
\ref{eq:controlador-com-fracoes-de-polinomios}, define-se $\eta_{p}$ como o grau
do polinômio $P(z^{-1})$. Assim, para que o sistema seja causal,
consequentemente implementável, as seguintes restrições devem ser seguidas:
$\eta_{r} \geq \eta_{s}$ e $\eta_{s} \geq \eta_{t}$. Das mesmas equações
as funções de transferências para malha fechada da Figura
\ref{fig:diagrama-de-blocos-malha-fechada} podem ser reescritas como

\begin{equation}
    \label{eq:ft-da-saida-para-referencia}
    \frac{Y(z)}{Y_{r}(z)} = \frac{T(z^{-1})B(z^{-1})}
                                 {R(z^{-1})A(z^{-1})+S(z^{-1})B(z^{-1})}
                          = \frac{t_{0}B(z^{-1})}{\lambda_{c}(z^{-1})},
\end{equation}

\begin{equation}
    \label{eq:ft-da-saida-para-perturbacao-na-entrada}
    \frac{Y(z)}{Q_{u}(z)} = \frac{R(z^{-1})B(z^{-1})}
                                 {R(z^{-1})A(z^{-1})+S(z^{-1})B(z^{-1})}
                          = \frac{R(z^{-1})B(z^{-1})}
                                 {\lambda_{c}(z^{-1})\lambda_{o}(z^{-1})},
\end{equation}

\begin{equation}
    \label{eq:ft-da-saida-para-perturbacao-na-saida}
    \frac{Y(z)}{Q_{y}(z)} = \frac{R(z^{-1})A(z^{-1})}
                                 {R(z^{-1})A(z^{-1})+S(z^{-1})B(z^{-1})}
                          = \frac{R(z^{-1})A(z^{-1})}
                                 {\lambda_{c}(z^{-1})\lambda_{o}(z^{-1})}.
\end{equation} em que $\lambda_{c}(z^{-1})$ e $\lambda_{o}(z^{-1})$ são polinômios
contendo os polos controláveis e observáveis do sistema em malha fechada,
respectivamente, e 

\begin{equation}
    \label{eq:numerador-do-filtro-de-referencia}
    T(z^{-1}) = t_{o}\lambda_{o}(z^{-1}).
\end{equation}

Dessa forma, tem-se que a \textbf{equação característica desejada} é dada por
\begin{equation}
    \label{eq:equacao-caracteristica-sem-coeficientes}
    \begin{split}
        \lambda(z^{-1}) &= R(z^{-1})A(z^{-1})+S(z^{-1})B(z^{-1}) \\
                        &= \lambda_{c}(z^{-1})\lambda_{o}(z^{-1}).
    \end{split}
\end{equation}

Como comentado anteriormente, é necessário que o controlador possua ação
integradora para rejeição de perturbação e seguimento de referência. Assim, deve
se empregar o polinômio $V(z^{-1}) = 1 - z^{-1}$ na função de transferência do
controlador. Desse modo, a Equação
\ref{eq:controlador-com-fracoes-de-polinomios} torna-se

\begin{equation}
    \label{eq:controlador-com-fracoes-de-polinomios-com-integrador}
    C(z^{-1}) = \frac{S(z^{-1})}{V(z^{-1})R'(z^{-1})}.
\end{equation} em que $R(z^{-1}) = V(z^{-1})R'(z^{-1})$ e
$\eta_r = \eta_v + \eta_{r'}$. Como agora apenas os coeficientes de $R'$ são
livres, para-se ter uma solução única e determinada para o sistema de equações
que define os coeficientes de $C(z^{-1})$ a igualdade $\eta_s = \eta_v + \eta_a
- 1 = 2$ tem que ser satisfeita. E para que o controlador seja causal, faz-se
$\eta_r = \eta_s$. Portanto, a função de transferência do controlador é então

\begin{equation}
    C(z^{-1}) = \frac{s_0+s_1z^{-1}+s_2z^{-2}}{(1-z^{-1})(1-rz^{-1})} 
              = \frac{s_0+s_1z^{-1}+s_2z^{-2}}{1+(-r-1)z^{-1}+rz^{-2}}.
\end{equation}

Com a imposição da ação integradora, faz-se necessário arbitrar dois polos
adicionais na equação característica desejada para que seja satisfeita a
igualdade $\eta_\lambda = \eta_a + \eta_r = 4$. Optou-se então por mais dois
polos duplos iguais afastados 10 vezes dos polos dominantes. Este polos são
justamente os polos observáveis que originam o polinômio
$\lambda_o(z^{-1})$, enquanto os polos dominantes (ou controláveis)
originam o polinômio $\lambda_c(z^{-1})$. Assim os polos controláveis $z_{c}^{*}
= 0.7408$ e os polos observáveis $z_{o}^{*} = 0.049787$.

Com estas definições, a Equação \ref{eq:equacao-caracteristica-sem-coeficientes}
torna-se
\begin{equation}
    \label{eq:equacao-caracteristica-com-coeficientes}
    \begin{split}
        R(z^{-1})A(z^{-1})+S(z^{-1})B(z^{-1})
        = \lambda_0+\lambda_1z^{-1}+\lambda_2z^{-2}+\lambda_3z^{-3}+\lambda_4z^{-4} \\
        = 1-1,581z^{-1}+0,6988z^{-2}-0,05832z^{-3}+0,00136z^{-4}
    \end{split}
\end{equation}

A partir da Equação \ref{eq:equacao-caracteristica-com-coeficientes}, chega-se
ao conjunto de equações lineares

\begin{equation}
    \label{eq:sistema-de-equacoes-lineares}
    \begin{split}
        s_0b_0 - r &= \lambda_1-a_1+1\\ 
        s_0b_1 + s_1b_0 + r(1-a_1) &= \lambda_2+a_1-a_2 \\ 
        s_1b_1 + s_2b_0 + r(a_1-a_2) &= \lambda_3+a_2 \\  
        s_2b_1 + ra_2 &= \lambda_4 
    \end{split}
\end{equation} que reescrita de forma matricial se torna

\begin{equation}
    \label{eq:equacoes-lineares-em-forma-matricial}
    \begin{bmatrix}
        b_0 & 0   & 0   & -1     \\ 
        b_1 & b_0 & 0   & 1-a_1  \\ 
        0   & b_1 & b_0 & a_1-a2 \\ 
        0   & 0   & b_1 & a_2
        \end{bmatrix}
        \begin{bmatrix}
        s_0 \\ 
        s_1 \\ 
        s_2 \\ 
        r
        \end{bmatrix}
        =
        \begin{bmatrix}
        \lambda_1\\ 
        \lambda_2\\ 
        \lambda_3\\ 
        \lambda_4
        \end{bmatrix}
        +
        \begin{bmatrix}
        -a_1+1\\ 
        a_1-a_2\\ 
        a_2\\ 
        0
        \end{bmatrix}
\end{equation} sendo a solução dada por 

\begin{equation}
    \label{eq:solucao-da-equacao-linear}
    \begin{bmatrix}
        s_0 \\ 
        s_1 \\ 
        s_2 \\ 
        r
        \end{bmatrix}
        =
        \begin{bmatrix}
        b_0 & 0   & 0   & -1     \\ 
        b_1 & b_0 & 0   & 1-a_1  \\ 
        0   & b_1 & b_0 & a_1-a2 \\ 
        0   & 0   & b_1 & a_2
        \end{bmatrix}^{-1}
        \begin{bmatrix}
        \lambda_1-a_1+1\\ 
        \lambda_2+a_1-a_2\\ 
        \lambda_3+a_2\\ 
        \lambda_4
        \end{bmatrix}.
\end{equation}

Substituindo \ref{eq:modelo-em-z-da-planta} e
\ref{eq:equacao-caracteristica-com-coeficientes} em
\ref{eq:solucao-da-equacao-linear}, tem-se então que

\begin{equation}
    \label{eq:coeficientes-do-controlador}
    \begin{bmatrix}
        s_0 \\ 
        s_1 \\ 
        s_2 \\ 
        r
    \end{bmatrix}
        =
    \begin{bmatrix}
        35,168 \\ 
        -55,931 \\ 
        22,326 \\ 
        -1,3319
    \end{bmatrix}
\end{equation} e então

\begin{equation}
    \label{eq:controlador-discreto}
    C(z^{-1}) = \frac{35,17-55,93z^{-1}+22,33z^{-2}}{1+0,3319z^{-1}-1,332z^{-2}}.
\end{equation}

Já a definição do filtro de referência $F(z^{-1})$ (Equação
\ref{eq:filtro-de-referencia-com-fracoes-de-polinomios}) considerou a Equação
\ref{eq:numerador-do-filtro-de-referencia} sendo

\begin{equation}
    \label{eq:definicao-dos-polos-0bservaveis}
    \lambda_o(z^{-1}) = 1 - 0.09957 z^{-1} + 0.002479 z^{-2}
\end{equation} e para garantir seguimento de referência, a partir da Equação
\ref{eq:ft-da-saida-para-referencia}, tem-se que

\begin{equation}
    \label{eq:ganho-do-filtro-de-referencia}
    t_0 = \frac{\lambda_c(1)}{B(1)} = 1,7311
\end{equation} e, portanto,

\begin{equation}
    F(z^{-1}) = \frac{0,04922-0,004901z^{-1}+0,000122z^{-2}}{1-1,59z^{-1}+0,6348z^{-2}}.
\end{equation}

Assim o projeto por alocação de polos está finalizado e os polinômios
$R(z^{-1})$, $S(z^{-1})$ e $T(z^{-1})$ são

\begin{equation}
    \label{eq:polinomios-rst}
    \begin{split}
        R(z^{-1}) &= 1+0,3319z^{-1}-1,332z^{-2}, \\
        S(z^{-1}) &= 35,17-55,93z^{-1}+22,33z^{-2}, \\
        T(z^{-1}) &= 1,731-0,1724z^{-1}+0,004291z^{-2}.
    \end{split}
\end{equation}
\subsubsection{Projeto via IMC}
\label{subsub:projeto-via-imc}

O controlador IMC foi projetado com base no diagrama de blocos da Figura
\ref{fig:diagrama-de-blocos-imc} e sua associação com o diagrama da Figura
\ref{fig:diagrama-de-blocos-malha-fechada}. Dessa forma, o filtro de referência
e o controlador é dado da seguinte forma:

\begin{equation}
    \label{eq:filtro-de-referencia-imc}
    F(z) = \frac{H_d(z)}{F_r(z)}
\end{equation} e

\begin{equation}
    \label{eq:controlador-imc}
    C(z) = \frac{F_r(z)P_{n}^{-1}(z)}{1-F_r(z)}
\end{equation} em que $P_n(z)$ neste caso é igual o modelo da planta dado por
\ref{eq:modelo-em-z-da-planta}. As restrições impostas é que para manter
causalidade, $g_r\left [ F_r(z) \right ] \geq g_r\left [ P_n(z) \right ]$ e
$g_r\left [ H_d(z) \right ] \geq g_r\left [ F_r(z) \right ]$, em que $g_r$
significa grau relativo ou excedentes de polos em relação a zeros.

\begin{figure}[ht!]
	\centering
	\caption{Diagrama de blocos de um controlador IMC.}
	\label{fig:diagrama-de-blocos-imc}
	\includegraphics[width=\textwidth]{chapters/challenge4/images/diagrama-de-blocos-imc.png}
    \caption*{Fonte: \cite{Tito2021}}
\end{figure}

Através de álgebra partindo da Equação \ref{fig:diagrama-de-blocos-imc},
observa-se que o filtro de robustez $F_r(z)$ é igual a complementar de
sensibilidade $\boldsymbol{C}(z)$. Assim, temos que

\begin{equation}
    \label{eq:associacao-do-filtro-de-robustez}
    F_r(z) = \boldsymbol{C}(z) = \frac{C(z)G(z)}{1+C(z)G(z)} = \frac{Y(z)}{Y_{r}(z)}.
\end{equation}

Portanto, observa-se que o filtro de robustez $F_r(z)$ é nada mais que e a
função de transferência em $z$ da entrada $y_r(t)$ para a saída $y(t)$. Desta
maneira definiu-se $F_r(z)$ para ter os polos desejáveis $z^* = 0,7408$ que
resulta no atendimento dos requisitos de regime transitório. Já para os zeros,
foram mantidos os zeros de fase não mínima em $F_r(z)$ para evitar cancelamentos
indesejáveis que leve a instabilidade dado uma perturbação na entrada ou saída
da planta. Por fim, para atender o requisito de rejeição de perturbação, o
controlador precisa possuir um integrador. Isso é assegurado fazendo $F_r(1) =
1$ pois $F_r(1)P_{n}^{-1}(1) \neq 0$ e, portanto, $C(1) \rightarrow \infty$, o
que garante que o controlador possui ação integral. Observa-se depois destas
definições que $F_r(z)$ torna-se igual a função de transferência
\ref{eq:ft-da-saida-para-referencia}, que é então

\begin{equation}
    \label{eq:filtro-de-robustez}
    F_r(z^{-1}) = \frac{-0,00943z^{-1}+0,07661z^{-2}}{1-1,482z^{-1}+0,5488z^{-2}}
\end{equation}

Substituindo \ref{eq:modelo-em-z-da-planta} e \ref{eq:filtro-de-robustez} em
\ref{eq:controlador-imc}, tem-se que

\begin{equation}
    \label{eq:controlador-imc-definido}
    C(z^{-1}) = \frac{1,731-2,98z^{-1}+1,282z^{-2}}{1-1,472z^{-1}+0,4722z^{-2}}.
\end{equation}

Como não é necessário um outro grau de liberdade no projeto do controlador,
faz-se $F_r(z) = H_d(z)$ e, portanto, $F(z) = 1$. Por fim, observa-se que a
causalidade do sistema foi mantida, pois $g_r\left [ F_r(z) \right ] = g_r\left
[ P_n(z) \right ]$. 
\subsubsection{Alocação de Polos x IMC}

A Figura \ref{fig:comparacao-entre-imc-e-alocacao-de-polos} mostra o comparativo
das respostas do sistema de controle amostrado da
\ref{fig:diagrama-de-blocos-malha-fechada} utilizando os dois controladores
projetados. Para esta simulação a perturbação na entrada $q_u(t) =
0,2\mathds{1}(t - 7)$ e na saída $q_y(t) = 0,2\mathds{1}(t - 12)$. Observa-se
que ambos os controladores atendem o requisito de tempo de acomodação e
seguimento de referência. Entretanto, apesar dos dois controladores também façam
com que perturbações constantes do tipo degrau sejam rejeitadas, nota-se que o
projeto via alocação de polos fornecem uma resposta mais rápida a perturbações
se comparada com o projeto via IMC.

\begin{figure}[!ht]
    \caption{Comparação das repostas do sistema em malha fechada com
    controladores projetados via Alocação de Polos e IMC.}
    \vspace{-10pt}
    \hspace{-30pt}
    \label{fig:comparacao-entre-imc-e-alocacao-de-polos}
    \begin{minipage}{\linewidth}
        % Title: gl2ps_renderer figure
% Creator: GL2PS 1.4.0, (C) 1999-2017 C. Geuzaine
% For: Octave
% CreationDate: Sun Sep 26 17:47:58 2021
\setlength{\unitlength}{1pt}
\begin{picture}(0,0)
\includegraphics{images/challenge2/resultado-questao-5-inc}
\end{picture}%
\begin{picture}(400,250)(0,0)
\fontsize{6}{0}
\selectfont\put(52,141.404){\makebox(0,0)[t]{\textcolor[rgb]{0.15,0.15,0.15}{{0}}}}
\fontsize{6}{0}
\selectfont\put(103.675,141.404){\makebox(0,0)[t]{\textcolor[rgb]{0.15,0.15,0.15}{{10}}}}
\fontsize{6}{0}
\selectfont\put(155.351,141.404){\makebox(0,0)[t]{\textcolor[rgb]{0.15,0.15,0.15}{{20}}}}
\fontsize{6}{0}
\selectfont\put(207.026,141.404){\makebox(0,0)[t]{\textcolor[rgb]{0.15,0.15,0.15}{{30}}}}
\fontsize{6}{0}
\selectfont\put(258.701,141.404){\makebox(0,0)[t]{\textcolor[rgb]{0.15,0.15,0.15}{{40}}}}
\fontsize{6}{0}
\selectfont\put(310.376,141.404){\makebox(0,0)[t]{\textcolor[rgb]{0.15,0.15,0.15}{{50}}}}
\fontsize{6}{0}
\selectfont\put(48.5278,155.215){\makebox(0,0)[r]{\textcolor[rgb]{0.15,0.15,0.15}{{0}}}}
\fontsize{6}{0}
\selectfont\put(48.5278,176.659){\makebox(0,0)[r]{\textcolor[rgb]{0.15,0.15,0.15}{{0.5}}}}
\fontsize{6}{0}
\selectfont\put(48.5278,198.104){\makebox(0,0)[r]{\textcolor[rgb]{0.15,0.15,0.15}{{1}}}}
\fontsize{6}{0}
\selectfont\put(48.5278,219.548){\makebox(0,0)[r]{\textcolor[rgb]{0.15,0.15,0.15}{{1.5}}}}
\fontsize{7}{0}
\selectfont\put(33.5278,186.325){\rotatebox{90}{\makebox(0,0)[b]{\textcolor[rgb]{0.15,0.15,0.15}{{Saída $y_{i}(t)$}}}}}
\fontsize{7}{0}
\selectfont\put(207,130.404){\makebox(0,0)[t]{\textcolor[rgb]{0.15,0.15,0.15}{{Tempo (s)}}}}
\fontsize{6}{0}
\selectfont\put(332,210.139){\makebox(0,0)[l]{\textcolor[rgb]{0,0,0}{{$r(t)$}}}}
\fontsize{6}{0}
\selectfont\put(332,198.639){\makebox(0,0)[l]{\textcolor[rgb]{0,0,0}{{$y_{1}(t)$}}}}
\fontsize{6}{0}
\selectfont\put(332,186.638){\makebox(0,0)[l]{\textcolor[rgb]{0,0,0}{{$y_{2}(t)$}}}}
\fontsize{6}{0}
\selectfont\put(332,174.638){\makebox(0,0)[l]{\textcolor[rgb]{0,0,0}{{$y_{3}(t)$}}}}
\fontsize{6}{0}
\selectfont\put(332,162.638){\makebox(0,0)[l]{\textcolor[rgb]{0,0,0}{{$y_{4}(t)$}}}}
\fontsize{6}{0}
\selectfont\put(52,22.2671){\makebox(0,0)[t]{\textcolor[rgb]{0.15,0.15,0.15}{{0}}}}
\fontsize{6}{0}
\selectfont\put(103.675,22.2671){\makebox(0,0)[t]{\textcolor[rgb]{0.15,0.15,0.15}{{10}}}}
\fontsize{6}{0}
\selectfont\put(155.351,22.2671){\makebox(0,0)[t]{\textcolor[rgb]{0.15,0.15,0.15}{{20}}}}
\fontsize{6}{0}
\selectfont\put(207.026,22.2671){\makebox(0,0)[t]{\textcolor[rgb]{0.15,0.15,0.15}{{30}}}}
\fontsize{6}{0}
\selectfont\put(258.701,22.2671){\makebox(0,0)[t]{\textcolor[rgb]{0.15,0.15,0.15}{{40}}}}
\fontsize{6}{0}
\selectfont\put(310.376,22.2671){\makebox(0,0)[t]{\textcolor[rgb]{0.15,0.15,0.15}{{50}}}}
\fontsize{6}{0}
\selectfont\put(48.5278,34.7158){\makebox(0,0)[r]{\textcolor[rgb]{0.15,0.15,0.15}{{0}}}}
\fontsize{6}{0}
\selectfont\put(48.5278,52.7559){\makebox(0,0)[r]{\textcolor[rgb]{0.15,0.15,0.15}{{0.5}}}}
\fontsize{6}{0}
\selectfont\put(48.5278,70.7954){\makebox(0,0)[r]{\textcolor[rgb]{0.15,0.15,0.15}{{1}}}}
\fontsize{6}{0}
\selectfont\put(48.5278,88.8354){\makebox(0,0)[r]{\textcolor[rgb]{0.15,0.15,0.15}{{1.5}}}}
\fontsize{6}{0}
\selectfont\put(48.5278,106.875){\makebox(0,0)[r]{\textcolor[rgb]{0.15,0.15,0.15}{{2}}}}
\fontsize{7}{0}
\selectfont\put(33.5278,67.1875){\rotatebox{90}{\makebox(0,0)[b]{\textcolor[rgb]{0.15,0.15,0.15}{{Sinal $u_{i}(t)$}}}}}
\fontsize{7}{0}
\selectfont\put(207,11.2671){\makebox(0,0)[t]{\textcolor[rgb]{0.15,0.15,0.15}{{Tempo (s)}}}}
\fontsize{6}{0}
\selectfont\put(332,76.0015){\makebox(0,0)[l]{\textcolor[rgb]{0,0,0}{{$u_{1}(t)$}}}}
\fontsize{6}{0}
\selectfont\put(332,65.001){\makebox(0,0)[l]{\textcolor[rgb]{0,0,0}{{$u_{2}(t)$}}}}
\fontsize{6}{0}
\selectfont\put(332,54.001){\makebox(0,0)[l]{\textcolor[rgb]{0,0,0}{{$u_{3}(t)$}}}}
\fontsize{6}{0}
\selectfont\put(332,43.0005){\makebox(0,0)[l]{\textcolor[rgb]{0,0,0}{{$u_{4}(t)$}}}}
\end{picture}

    \end{minipage}
\end{figure}

Para analisar o comportamento da resposta as perturbações dos dois diferentes
projetos, é necessário a inspeção dos pares entrada/saída. Percebe-se nas
Equações \ref{eq:ft-da-saida-para-perturbacao-na-entrada} e
\ref{eq:ft-da-saida-para-perturbacao-na-saida} que os polos das funções de
transferência da entrada para saída foram exatamente os polos alocados, sendo os
polos observáveis 10 vezes mais rápidos que os polos controláveis, que já são
rápidos devido a constante de tempo $\tau = 0,5$. Em contrapartida, devido a
propriedade intrínseca de cancelamento do projeto via IMC, alguns polos lentos
podem aparecer nas funções de transferência das perturbações para saída do
sistema em malha fechada.

A partir da Equação \ref{eq:filtro-de-robustez} e do diagrama de blocos
\ref{fig:diagrama-de-blocos-malha-fechada}, as seguintes funções de
transferência são deduzidas:

\begin{equation}
    \label{eq:ft-perturbacao-entrada-imc}
    \frac{Y(z)}{Q_u(z)} = P_n(z)(1-F_r(z)) = P_n(z)S(z)
\end{equation} em que $S(z)$ é a complementar de sensibilidade e

\begin{equation}
    \label{eq:ft-perturbacao-saida-imc}
    \frac{Y(z)}{Q_y(z)} = 1-F_r(z) = S(z).
\end{equation}

Percebe-se então que para um perturbação na entrada da planta $q_u(t)$, aparecem
os polos da própria planta, que neste caso são polos lentos se comparados com os
da malha fechada via projeto por alocação de polos. Por outro lado, para
perturbações na saída $q_y(t)$ os polos são os mesmos que os desejáveis, que são
rápidos e, portanto, fornecem uma resposta a perturbação mais rápida. 

É notório também a influência de zeros na rejeição a perturbação na saída da
planta. Observa-se que para o projeto via alocação de polos, a saída da planta
$y(t)$ cai abaixo do valor de \textit{set point}, devido a um zero negativo fora
do círculo unitário (-1.3319). Já em \ref{eq:ft-perturbacao-saida-imc} não há
zeros negativo fora do circulo unitário e, consequentemente, o mesmo
comportamento não é observado.

Vale salientar também a importância de não realizar o cancelamento do zero de
fase não mínimas nos projetos alocando os polos do controlador no mesmo local.
Caso este polo fosse alocado nos controladores projetados, ele não iria aparecer
nas funções de transferência listadas neste tópico. Entretanto, ao analisar a
Gangue dos Seis definidas por \cite{Astrom2008}, caso alocado no controlador, o
polo instável aparece em dois pares de entrada saída conforme a Equação abaixo,
em que o argumento foi omitido por simplicidade e o par entrada/saida é definido
pelo subescrito.

\begin{subequations}
    \begin{equation}
        G_{ur} = \frac{CF}{1+PC}
    \end{equation}
    \begin{equation}
        -G_{uw} = \frac{C}{1+PC}
    \end{equation}
\end{subequations} em que $u$, $r$ e $w$ significam sinal de controle,
referência e ruído respectivamente e $P$ e $C$ significam respectivamente função
de transferência da planta e controlador.


\subsection{Conclusões}
Os resultados alcançados expressam a utilidade para o projetista do projeto de
controladores diretamente no domínio de tempo discreto. Eles demonstram também
como é possível alcançar os requisitos desejáveis através de abordagem de
síntese direta em que o compensador é projetado com a planta desejável já
previamente definida, em comparação com métodos que não se tem conhecimento
prévio de como a função de transferência de malha fechada irá ficar.
