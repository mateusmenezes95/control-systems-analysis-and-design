\documentclass[a4paper,10pt]{article}

\usepackage[brazilian]{babel}
\usepackage[utf8]{inputenc}
\usepackage[T1]{fontenc}
\usepackage{psfrag}
\usepackage{times}
\usepackage{indentfirst}
\usepackage{amsmath,amsfonts,amssymb}
\usepackage{graphicx}
\usepackage{caption}
\usepackage{dsfont}
\usepackage{xcolor}
\usepackage{tipa}
\usepackage{tipx}
\usepackage{amsmath}
\usepackage{tabularx}
\usepackage{cancel}
\usepackage{float}
\usepackage{hyperref}
\usepackage[num]{abntex2cite}

% Extracted from:
% https://tex.stackexchange.com/questions/95838/how-to-write-a-perfect-equation-parameters-description
\newenvironment{conditions*}
    {\par\vspace{\abovedisplayskip}\noindent
    \tabularx{\columnwidth}{>{$}l<{$} @{${}$ é ${}$} >{\raggedright\arraybackslash}X}}
    {\endtabularx\par\vspace{\belowdisplayskip}}

\title{\textbf{Análise e Projeto de Sistemas de Controle} \\
\vspace{0.5cm}
\underline{Relatório II} \\
\vspace{4.5cm}
\includegraphics[width=4.0cm]{brasao_ufba.jpg}
\vspace{4.5cm} }
\author{Mateus dos Santos de Meneses}
\date{Novembro, 2021}

\begin{document}
\maketitle
\captionsetup{justification=centering}

\clearpage 

\tableofcontents

\clearpage

\listoffigures

\clearpage

\section{Consideração Inicial}
Os resultados alcançados nos tópicos que se sucedem podem ser obtidos com os
scripts que estão disponíveis em um repositório público do
GitHub\footnote{\url{https://github.com/mateusmenezes95/control-systems-analysis-and-design.git}}.
\section{Desafio I - Controle Multivariável - Espaço de Estados} 

\subsection{Motivação}
(Explicar o problema que motiva o desafio, relevância, possíveis aplicações...) 

\subsection{Simulações realizadas}

\begin{equation}
    \label{eq:matriz-de-fts}
    \mathbf{G}(s) = 
        \begin{bmatrix}
            \frac{2}{(10s+1)} & \frac{0,8}{(10s+1)(2s+1)}\\ 
            \frac{0,6}{(10s+1)(2s+1)} & \frac{2}{(10s+1)}
        \end{bmatrix}
\end{equation}
(Explicar quais simulações foram realizadas de maneira descritiva e sequêncial.) 

\subsection{Resultados obtidos}
\label{sub:resultados-obtidos-desafio6}

\subsubsection{Cálculo do ponto de equilíbrio}
\label{subsub:calculo-do-ponto-de-equilibrio}
Dado que o sistema multivariável é caracterizado por \ref{eq:matriz-de-fts}, a
saída do sistema é $\mathbf{y} = [Y_1(s) \thickspace Y_2(s)]^{T}$ dado o vetor de
sinais de controle $\mathbf{u} = [U_1(s) \thickspace U_2(s)]^{T}$, temos então que
$\mathbf{y} = \mathbf{G}\mathbf{u}$ ou 

\begin{equation}
    \label{eq:relacao-entrada-saida-de-g}
    \begin{bmatrix}
        Y_1(s)\\ 
        Y_2(s)
    \end{bmatrix}
    =
    \begin{bmatrix}
        \frac{2}{(10s+1)} & \frac{0,8}{(10s+1)(2s+1)}\\ 
        \frac{0,6}{(10s+1)(2s+1)} & \frac{2}{(10s+1)}
    \end{bmatrix}
    \begin{bmatrix}
        U_1(s)\\ 
        U_2(s)
    \end{bmatrix}.
\end{equation}

Quando aplicado o sinal de controle $\mathbf{u}_{eq} = \lim_{t \rightarrow
\infty }\mathbf{u}(t)$, a saída do sistema chega ao equilíbrio $\mathbf{y}_{eq}
= \lim_{t \rightarrow \infty }\mathbf{y}(t)$. Aplicando o teorema do valor
final em \ref{eq:relacao-entrada-saida-de-g} dado uma entrada em degrau para
$\mathbf{u}(t)$, temos que $\mathbf{u}_{eq} = \mathbf{G}(0)^{-1}\mathbf{y}_{eq}$
se $det(\mathbf{G}(0)) \neq 0$. Como,

\begin{equation}
    \label{eq:ganho-estatico-de-g}
    \mathbf{G}(0)
    =
    \begin{bmatrix}
        2,0 & 0,8\\ 
        0,6 & 1,5
    \end{bmatrix},
\end{equation}logo, $det(\mathbf{G}(0)) \neq 0$ e $\mathbf{G}(0)^{-1}$ existe.
Dado a saída de equilíbrio desejada $\mathbf{y}_{eq} = [2 \thickspace
1]^\top$, o sinal de controle de equilíbrio 

\begin{equation}
    \label{eq:vetor-do-controle-de-equilibrio}
    \mathbf{u}_{eq} = [0,8730 \thickspace 0,3175]^\top.
\end{equation}

\subsubsection{Espaço de Estados no domínio de tempo continuo}
\label{subsub:espaco-de-estados-no-dominio-de-tempo-continuo}

A utilização da função \textit{ss} do Octave para transformar o modelo
\ref{eq:matriz-de-fts} dado no domínio de Laplace para realização em espaço de
estados

\begin{subequations}
    \label{eq:espaco-de-estados-continuo}
    \begin{align}
        \mathbf{\dot{x}}(t) &= \mathbf{A}\mathbf{x}(t) + \mathbf{B}\mathbf{u}(t)
        \label{eq:derivada-do-vetor-de-estados}\\
        \mathbf{y}(t) &= \mathbf{C}\mathbf{x}(t) + \mathbf{D}\mathbf{u}(t)
        \label{eq:saida-do-sistema-em-espaco-de-estados}
    \end{align}
\end{subequations} resulta nas matrizes $\mathbf{A}_{4\times 4}$,
$\mathbf{B}_{4\times 2}$ e $\mathbf{C}_{2\times 4}$, cujo elementos estão
definidas em \ref{eq:matrizes-do-espaco-de-estados}. Ressalta-se entretanto, que
$\mathbf{D} = \mathbf{\bar{0}}$ para o sistema deste desafio. Observa-se das dimensões
destas matrizes, que para o sistema dado existem 4 variáveis de estados.

\begin{subequations}
    \label{eq:matrizes-do-espaco-de-estados}
    \begin{align}
        \mathbf{A} &= 
        \begin{bmatrix}
            0.0500 &  0      &  0.1164  &  0        \\
            0      &  0.0500 &  0       &  0.1164   \\
            0.7089 &  0      & -0.6500  &  0        \\
            0      & -0.7089 &  0       & -0.6500
        \end{bmatrix},  \label{eq:matriz-a} \\
        \mathbf{B} &=
        \begin{bmatrix}
            -0.0849 & -0.2333  \\
            -0.3111 & -0.1131  \\
            0       &  0.3007  \\
            0.4010  &  0
        \end{bmatrix},  \label{eq:matriz-b} \\
        \mathbf{C} &=
        \begin{bmatrix}
            0    &    0  &      0  & 0.4988  \\
            0    &    0  & 0.4988  & 0
        \end{bmatrix}   \label{eq:matriz-c}
    \end{align}
\end{subequations}

Com estas matrizes é possível calcular qual são os estados do sistema quando a
saída atinge seu valor de equilíbrio. Quando isto acontece, $\mathbf{\dot{x}}(t)
= \mathbf{0}$ e, portanto, a partir de \ref{eq:derivada-do-vetor-de-estados},

\begin{equation}
    \label{eq:calculo-do-vetor-de-estados-de-equilibrio}
    \mathbf{x}_{eq} = -\mathbf{A}^{-1}\mathbf{B}\mathbf{u}_{eq}.
\end{equation} Vale reforçar que isto é verdadeiro se $\mathbf{A}$ é invertível
ou det($\mathbf{A}$) $\neq 0$. De \ref{eq:matrizes-do-espaco-de-estados},
verifica-se que det($\mathbf{A}$) $\neq 0$. Assim, substituindo
\ref{eq:vetor-do-controle-de-equilibrio}, \ref{eq:matriz-a} e \ref{eq:matriz-b}
em \ref{eq:calculo-do-vetor-de-estados-de-equilibrio}, tem se que

\begin{equation}
    \label{eq:estados-de-equilibrio}
    \mathbf{x}_{eq} =
    \begin{bmatrix}
        x_{eq_{1}} \\
        x_{eq_{2}} \\
        x_{eq_{3}}  \\
        x_{eq_{4}}
    \end{bmatrix}
    =
    \begin{bmatrix}
        -1.7038 \\
        -3.1831 \\
        2.0050  \\
        4.0100
    \end{bmatrix}.
\end{equation}

\subsubsection{Espaço de Estado no domínio de tempo discreto}
\label{subsub:espaco-de-estados-no-dominio-de-tempo-discreto}

A versão em tempo discreto para a representação em espaço de estados definido em
\ref{eq:espaco-de-estados-continuo} é dado da seguinte forma

\begin{subequations}
    \label{eq:espaco-de-estados-discreto}
    \begin{align}
        \mathbf{x}[k+1] &= \mathbf{\tilde{A}}\mathbf{x}[k] + \mathbf{\tilde{B}}\mathbf{u}[k]
        \label{eq:estimativa-do-vetor-de-estados}\\
        \mathbf{y}[k] &= \mathbf{\tilde{C}}\mathbf{x}[t] + \mathbf{\tilde{D}}\mathbf{u}[k]
        \label{eq:saida-do-sistema-em-espaco-de-estados-discreto}
    \end{align}
\end{subequations} que segundo \citeonline{Chen2006}, a relação entre as
matrizes de tempo continuo e discreto é dado como

\begin{equation*}
    \mathbf{\tilde{A}} = e^{\mathbf{A}T},
    \thickspace
    \mathbf{\tilde{B}} = \int_{0}^{T}e^{A\tau}d\tau \mathbf{B},
    \thickspace
    \mathbf{\tilde{C}} = \mathbf{C}
    \thickspace e \thickspace
    \mathbf{\tilde{D}} = \mathbf{D},
\end{equation*} em que $T$ é o período de amostragem do sistema de controle
amostrado. Observa-se portanto que é necessário definir um período de amostragem
adequado para ter a realização do modelo \ref{eq:matriz-de-fts} em espaço de
estados de tempo discreto.

O critério para definição do tempo de amostragem foi igual ao utilizado no
desafio 4 de projetos no tempo discreto (seção
\ref{sub:simulacoes-realizadas-desafio4}). Entretanto, o modelo em questão é
multivariável. Considerando então, que $\mathbf{W}$ é uma matriz cujo elementos
$w_{ij}$ é a frequência de corte de cada função de transferência $g_{ij}$ de
$\mathbf{G}(s)$, optou-se por aplicar o critério na máxima frequência de corte,
ou seja $w_a = 30\times\max(\mathbf{W})$, em que $w_a$ é a frequência de
amostragem em $rad/s$. Esta abordagem é conservadora, pois os sistemas cuja a
frequência de corte $w_c < \max(\mathbf{W})$ por si só irão atenuar sinais de
alta frequências, e então estes sinais não contribuirão significativamente em
suas respostas. Ou de forma matemática, se $w_a = 30\times\max(\mathbf{W})$,
logo o critério de $30\times$ será válido para as demais plantas. Assim,
obteve-se a matriz $W$ conforme \ref{eq:matriz-das-frequencias-de-corte}.
Observa-se que $w_{21} = 0$. Foi atribuído o valor zero pois $\mod{g_{21}(0)} <
-3$dB, e esta atribuição não compromete o critério da definição do período de
amostragem.

\begin{equation}
    \label{eq:matriz-das-frequencias-de-corte}
    \mathbf{W} = 
        \begin{bmatrix}
            0,2646 & 0,0516\\ 
            0 & 0,1871
        \end{bmatrix}
\end{equation}

Portanto, $\max(\mathbf{W}) = 0,2646$ rad/s, o que resultou, após
arrendondamento, num período de amostragem $T_a = 0,75$s.

Novamente com o auxílio da função \textit{ss} do Octave, mas agora passando o
período de amostragem como argumento, chegou-se as seguintes matrizes

ALTERAR VALORES DA MATRIZ B E C!!!!

\begin{subequations}
    \label{eq:matrizes-do-espaco-de-estados-discretizadas}
    \begin{align}
        \mathbf{\tilde{A}} &= 
        \begin{bmatrix}
            -0.4986  &   0        &  1.9513  &   -0 \\
            0        &   -0.4986  &  0       &   -1.9513    \\
            -0.8668  &   0        &  2.1136  &   -0 \\
            0        &   0.8668   &  -0      &    2.1136         
        \end{bmatrix},  \label{eq:matriz-til-a} \\
        \mathbf{\tilde{B}} &=
        \begin{bmatrix}
            -0.0849 & -0.2333  \\
            -0.3111 & -0.1131  \\
            0       &  0.3007  \\
            0.4010  &  0
        \end{bmatrix},  \label{eq:matriz-til-b} \\
        \mathbf{\tilde{C}} &=
        \begin{bmatrix}
            0    &    0  &      0  & 0.4988  \\
            0    &    0  & 0.4988  & 0
        \end{bmatrix}   \label{eq:matriz-til-c}
    \end{align}
\end{subequations}

\subsection{Conclusões}
(Concluir em que medida os resultados apresentam relação com a motivação.
Permitem ilustrar ou concluir algo sobre a motivação? )


\addcontentsline{toc}{section}{Referências}
\bibliographystyle{apalike}
\bibliography{references.bib}

\end{document}
