\documentclass[a4paper,10pt]{article}

\usepackage[brazilian]{babel}
\usepackage[utf8]{inputenc}
\usepackage[T1]{fontenc}
\usepackage{psfrag}
\usepackage{times}
\usepackage{indentfirst}
\usepackage{amsmath,amsfonts,amssymb}
\usepackage{graphicx}
\usepackage{caption}
\usepackage{dsfont}
\usepackage{xcolor}
\usepackage{tipa}
\usepackage{tipx}
\usepackage{amsmath}
\usepackage{tabularx}
\usepackage{cancel}
\usepackage{float}
\usepackage{hyperref}
\usepackage{mathtools}
\usepackage{icomma}
\usepackage{subcaption}

% Extracted from:
% https://tex.stackexchange.com/questions/95838/how-to-write-a-perfect-equation-parameters-description
\newenvironment{conditions*}
    {\par\vspace{\abovedisplayskip}\noindent
    \tabularx{\columnwidth}{>{$}l<{$} @{${}$ é ${}$} >{\raggedright\arraybackslash}X}}
    {\endtabularx\par\vspace{\belowdisplayskip}}

\title{\textbf{Análise e Projeto de Sistemas de Controle} \\
\vspace{0.5cm}
\underline{Relatório II} \\
\vspace{4.5cm}
\includegraphics[width=4.0cm]{brasao_ufba.jpg}
\vspace{4.5cm} }
\author{Mateus dos Santos de Meneses}
\date{Novembro, 2021}

\begin{document}
\maketitle
\captionsetup{justification=centering}

\clearpage 

\tableofcontents

\clearpage

\listoffigures

\clearpage

\section{Consideração Inicial}
Os resultados alcançados nos tópicos que se sucedem podem ser obtidos com os
scripts que estão disponíveis em um repositório público do
GitHub\footnote{\url{https://github.com/mateusmenezes95/control-systems-analysis-and-design.git}}.
\section{Desafio IV - Sistemas de Controle Amostrados} 

\subsection{Motivação}
Este desafio visa demonstrar duas técnicas de projeto de controladores de
sistemas de controle amostrados diretamente no domínio de tempo discreto. O
aprendizado dessas técnicas é de extrema importância na aplicação prática de
sistemas de controle, visto que nas últimas décadas sistemas implementados por
computadores analógicos vem sendo substituídos por computadores digitais. Isto
se observa na predominância de sistemas de controle amostrados em problemas
complexos de controle, como aplicações em diversos sistemas de espaçonaves,
piloto automático de aeronaves, carros elétricos autônomos, geração de energia
elétrica através de usinas nucleares etc, até problemas simples como o controle
de velocidade das rodas de um robô diferencial. Além do mais, controladores
modernos, como o caso de controladores adaptativos, são desenvolvidos partindo
dos conceitos e ferramentas oferecidos por sistemas de controle amostrados.
Desta forma, torna-se imprescindível para profissionais e/ou pesquisadores da
área de controle possuir conhecimento dos sistemas de controle amostrados.

\subsection{Simulações realizadas}
\label{sub:simulacoes-realizadas-desafio4}
As simulações realizadas neste desafio visaram demonstrar e comparar as
respostas de um sistema de controle realimentado utilizando duas técnicas de
projeto de controladores no domínio do tempo discreto: alocação de polos e
controlador por modelo interno (IMC). O sistema em malha fechada utilizado como
base para as simulações está representado na Figura
\ref{fig:diagrama-de-blocos-malha-fechada}, em que:

\begin{itemize}
    \item $R(z)$ é a transformada Z do sinal de referência $r(k)$;
    \item $E(z)$ é a transformada Z do erro $e(k)$;
    \item $C(z)$ é a função de transferência do controlador no domínio Z;
    \item $SoZ$ é a função de transferência do sustentador de ordem zero;
    \item $U(s)$ é a transformada de Laplace do sinal de controle $u(t)$;
    \item $Q_{u}(s)$ é a transformada de Laplace da perturbação de entrada
    $q_{u}(t)$;
    \item $G(s)$ é a função de transferência da planta;
    \item $Q_{y}(s)$ é a transformada de Laplace da perturbação na saída
    $q_{y}(t)$; e, por fim
    \item $Y(s)$ é a transformada de Laplace da saída do sistema $y(t)$.
\end{itemize}

\begin{figure}[ht!]
	\centering
	\caption{Sistema em malha fechada de um sistema de controle amostrado e com
    perturbações persistentes.}
	\label{fig:diagrama-de-blocos-malha-fechada}
	\includegraphics[width=\textwidth]{chapters/challenge4/images/diagrama-de-blocos-sistema-de-controle-amostrado.png}
\end{figure}

Os controladores foram projetados a partir da planta cujo o modelo está definido
na Equação abaixo.

\begin{equation}
    \label{eq:modelo-em-s-da-planta}
    G(s) = \frac{0,2(10 - s)}{(s+1)^{2}}
\end{equation}

Com base no modelo definido na Equação \ref{eq:modelo-em-s-da-planta}, os dois
controladores foram projetados para que o sistema em malha fechada atendesse os
seguintes requisitos:

\begin{enumerate}
    \item Tempo de acomodação de 2\% da resposta da saída para variação de
    referência do tipo degrau seja aproximadamente metade do tempo de acomodação
    de 2\% da resposta em malha aberta;
    \item Seguimento de referência do tipo degrau; e
    \item Rejeição de perturbações constantes.
\end{enumerate}

Como parâmetro do projeto, foi utilizado o período de amostragem de $T_{a} =
0,15 s$, baseado no critério definido por \cite[p. 61]{Franklin1997} em que
"\textit{Generally, sample rates should be faster than 30 times the bandwidth in
order to assure that the digital controller can be made to closely match the
performance of the continuous controller.}". Já a discretização do modelo da
planta foi realizada através da função c2d do Octave utilizando o método de
conversão $zoh$. Assim, o modelo da planta discretizada utilizando os parâmetros
acima é dado por

\begin{equation}
    \label{eq:modelo-em-z-da-planta}
    G(z) = \frac{-0,00545z^{-1} + 0,04425z^{-2}}{1 - 1.721z^{-1} + 0.7408z^{-2}}.
\end{equation}

Na próxima seção são apresentados os resultados para seguimento de referência e
rejeição a perturbações do tipo degrau para o sistema de controle amostrado
contendo os dois controladores projetados.

\subsection{Resultados obtidos}
As duas técnicas de projeto controle visam definir o controlador a partir de um
modelo de planta desejado. Sendo assim, para atender o requisito de regime
transitório, foram escolhidos os mesmos polos desejados para os dois
controladores. Para isso, optou-se por dois polos iguais para se valer da
aproximação de sistema com polos repetidos no domínio do tempo em que o tempo de
acomodação de 2\% ($t_{s_{2\%}}$) é aproximado por $6\tau$. Como os dois polos
de $G(s)$ (Equação \ref{eq:modelo-em-s-da-planta}) fornecem um tempo de
acomodação de $t_{s_{2\%}}\approx 6s$ ($\tau = 1$), definiu-se $\tau = 0,5$ que
consequentemente $t_{s_{2\%}}\approx 3s$. Assim, o objetivo dos
controladores é proporcionar dois polos $s^*=-2$, que no tempo discreto
utilizando a conversão $e^{-T_{a}s^*}$ e $T_{a} = 0.15s$ chega-se a dois polos
$z^*=0,7408$.

Para atender os requisitos de regime estacionário seguimento de referência e
rejeição a perturbações constantes do tipo degrau, os controladores devem
possuir ao menos um integrador. Esta premissa parte do Princípio do Modelo
Interno, pois como nesse caso não há um integrador na planta, tal ação deve ser
proporcionada pelo controlador.

Como os projetos em si se materializam de forma diferente, apesar das mesmas
premissas, a subseções seguintes aborda de forma detalhada cada uma das
técnicas. Vale salientar, que embora as duas técnicas de projeto possam ser
aplicadas também em tempo contínuo, os projetos foram realizados em sua
completude no domínio do tempo discreto.

\subsubsection{Projeto via Alocação de Polos}
\label{subsub:projeto-via-alocacao-de-polos}
O projeto via alocação de polos começa reescrevendo as funções de transferências
dos elementos que compões a malha fechada (Figura
\ref{fig:diagrama-de-blocos-malha-fechada}) em frações de polinômios em
$z^{-1}$. Dessa forma, tem-se que

\begin{equation}
    \label{eq:modelo-da-planta-com-fracoes-de-polinomios}
    G(z^{-1}) = \frac{B(z^{-1})}{A(z^{-1})},
\end{equation}

\begin{equation}
    \label{eq:filtro-de-referencia-com-fracoes-de-polinomios}
    F(z^{-1}) = \frac{T(z^{-1})}{R(z^{-1})},
\end{equation}

\begin{equation}
    \label{eq:controlador-com-fracoes-de-polinomios}
    C(z^{-1}) = \frac{S(z^{-1})}{R(z^{-1})}.
\end{equation}

A partir das Equações \ref{eq:modelo-da-planta-com-fracoes-de-polinomios},
\ref{eq:filtro-de-referencia-com-fracoes-de-polinomios} e
\ref{eq:controlador-com-fracoes-de-polinomios}, define-se $\eta_{p}$ como o grau
do polinômio $P(z^{-1})$. Assim, para que o sistema seja causal,
consequentemente implementável, as seguintes restrições devem ser seguidas:
$\eta_{r} \geq \eta_{s}$ e $\eta_{s} \geq \eta_{t}$. Das mesmas equações
as funções de transferências para malha fechada da Figura
\ref{fig:diagrama-de-blocos-malha-fechada} podem ser reescritas como

\begin{equation}
    \label{eq:ft-da-saida-para-referencia}
    \frac{Y(z)}{Y_{r}(z)} = \frac{T(z^{-1})B(z^{-1})}
                                 {R(z^{-1})A(z^{-1})+S(z^{-1})B(z^{-1})}
                          = \frac{t_{0}B(z^{-1})}{\lambda_{c}(z^{-1})},
\end{equation}

\begin{equation}
    \label{eq:ft-da-saida-para-perturbacao-na-entrada}
    \frac{Y(z)}{Q_{u}(z)} = \frac{R(z^{-1})B(z^{-1})}
                                 {R(z^{-1})A(z^{-1})+S(z^{-1})B(z^{-1})}
                          = \frac{R(z^{-1})B(z^{-1})}
                                 {\lambda_{c}(z^{-1})\lambda_{o}(z^{-1})},
\end{equation}

\begin{equation}
    \label{eq:ft-da-saida-para-perturbacao-na-saida}
    \frac{Y(z)}{Q_{y}(z)} = \frac{R(z^{-1})A(z^{-1})}
                                 {R(z^{-1})A(z^{-1})+S(z^{-1})B(z^{-1})}
                          = \frac{R(z^{-1})A(z^{-1})}
                                 {\lambda_{c}(z^{-1})\lambda_{o}(z^{-1})}.
\end{equation} em que $\lambda_{c}(z^{-1})$ e $\lambda_{o}(z^{-1})$ são polinômios
contendo os polos controláveis e observáveis do sistema em malha fechada,
respectivamente, e 

\begin{equation}
    \label{eq:numerador-do-filtro-de-referencia}
    T(z^{-1}) = t_{o}\lambda_{o}(z^{-1}).
\end{equation}

Dessa forma, tem-se que a \textbf{equação característica desejada} é dada por
\begin{equation}
    \label{eq:equacao-caracteristica-sem-coeficientes}
    \begin{split}
        \lambda(z^{-1}) &= R(z^{-1})A(z^{-1})+S(z^{-1})B(z^{-1}) \\
                        &= \lambda_{c}(z^{-1})\lambda_{o}(z^{-1}).
    \end{split}
\end{equation}

Como comentado anteriormente, é necessário que o controlador possua ação
integradora para rejeição de perturbação e seguimento de referência. Assim, deve
se empregar o polinômio $V(z^{-1}) = 1 - z^{-1}$ na função de transferência do
controlador. Desse modo, a Equação
\ref{eq:controlador-com-fracoes-de-polinomios} torna-se

\begin{equation}
    \label{eq:controlador-com-fracoes-de-polinomios-com-integrador}
    C(z^{-1}) = \frac{S(z^{-1})}{V(z^{-1})R'(z^{-1})}.
\end{equation} em que $R(z^{-1}) = V(z^{-1})R'(z^{-1})$ e
$\eta_r = \eta_v + \eta_{r'}$. Como agora apenas os coeficientes de $R'$ são
livres, para-se ter uma solução única e determinada para o sistema de equações
que define os coeficientes de $C(z^{-1})$ a igualdade $\eta_s = \eta_v + \eta_a
- 1 = 2$ tem que ser satisfeita. E para que o controlador seja causal, faz-se
$\eta_r = \eta_s$. Portanto, a função de transferência do controlador é então

\begin{equation}
    C(z^{-1}) = \frac{s_0+s_1z^{-1}+s_2z^{-2}}{(1-z^{-1})(1-rz^{-1})} 
              = \frac{s_0+s_1z^{-1}+s_2z^{-2}}{1+(-r-1)z^{-1}+rz^{-2}}.
\end{equation}

Com a imposição da ação integradora, faz-se necessário arbitrar dois polos
adicionais na equação característica desejada para que seja satisfeita a
igualdade $\eta_\lambda = \eta_a + \eta_r = 4$. Optou-se então por mais dois
polos duplos iguais afastados 10 vezes dos polos dominantes. Este polos são
justamente os polos observáveis que originam o polinômio
$\lambda_o(z^{-1})$, enquanto os polos dominantes (ou controláveis)
originam o polinômio $\lambda_c(z^{-1})$. Assim os polos controláveis $z_{c}^{*}
= 0.7408$ e os polos observáveis $z_{o}^{*} = 0.049787$.

Com estas definições, a Equação \ref{eq:equacao-caracteristica-sem-coeficientes}
torna-se
\begin{equation}
    \label{eq:equacao-caracteristica-com-coeficientes}
    \begin{split}
        R(z^{-1})A(z^{-1})+S(z^{-1})B(z^{-1})
        = \lambda_0+\lambda_1z^{-1}+\lambda_2z^{-2}+\lambda_3z^{-3}+\lambda_4z^{-4} \\
        = 1-1,581z^{-1}+0,6988z^{-2}-0,05832z^{-3}+0,00136z^{-4}
    \end{split}
\end{equation}

A partir da Equação \ref{eq:equacao-caracteristica-com-coeficientes}, chega-se
ao conjunto de equações lineares

\begin{equation}
    \label{eq:sistema-de-equacoes-lineares}
    \begin{split}
        s_0b_0 - r &= \lambda_1-a_1+1\\ 
        s_0b_1 + s_1b_0 + r(1-a_1) &= \lambda_2+a_1-a_2 \\ 
        s_1b_1 + s_2b_0 + r(a_1-a_2) &= \lambda_3+a_2 \\  
        s_2b_1 + ra_2 &= \lambda_4 
    \end{split}
\end{equation} que reescrita de forma matricial se torna

\begin{equation}
    \label{eq:equacoes-lineares-em-forma-matricial}
    \begin{bmatrix}
        b_0 & 0   & 0   & -1     \\ 
        b_1 & b_0 & 0   & 1-a_1  \\ 
        0   & b_1 & b_0 & a_1-a2 \\ 
        0   & 0   & b_1 & a_2
        \end{bmatrix}
        \begin{bmatrix}
        s_0 \\ 
        s_1 \\ 
        s_2 \\ 
        r
        \end{bmatrix}
        =
        \begin{bmatrix}
        \lambda_1\\ 
        \lambda_2\\ 
        \lambda_3\\ 
        \lambda_4
        \end{bmatrix}
        +
        \begin{bmatrix}
        -a_1+1\\ 
        a_1-a_2\\ 
        a_2\\ 
        0
        \end{bmatrix}
\end{equation} sendo a solução dada por 

\begin{equation}
    \label{eq:solucao-da-equacao-linear}
    \begin{bmatrix}
        s_0 \\ 
        s_1 \\ 
        s_2 \\ 
        r
        \end{bmatrix}
        =
        \begin{bmatrix}
        b_0 & 0   & 0   & -1     \\ 
        b_1 & b_0 & 0   & 1-a_1  \\ 
        0   & b_1 & b_0 & a_1-a2 \\ 
        0   & 0   & b_1 & a_2
        \end{bmatrix}^{-1}
        \begin{bmatrix}
        \lambda_1-a_1+1\\ 
        \lambda_2+a_1-a_2\\ 
        \lambda_3+a_2\\ 
        \lambda_4
        \end{bmatrix}.
\end{equation}

Substituindo \ref{eq:modelo-em-z-da-planta} e
\ref{eq:equacao-caracteristica-com-coeficientes} em
\ref{eq:solucao-da-equacao-linear}, tem-se então que

\begin{equation}
    \label{eq:coeficientes-do-controlador}
    \begin{bmatrix}
        s_0 \\ 
        s_1 \\ 
        s_2 \\ 
        r
    \end{bmatrix}
        =
    \begin{bmatrix}
        35,168 \\ 
        -55,931 \\ 
        22,326 \\ 
        -1,3319
    \end{bmatrix}
\end{equation} e então

\begin{equation}
    \label{eq:controlador-discreto}
    C(z^{-1}) = \frac{35,17-55,93z^{-1}+22,33z^{-2}}{1+0,3319z^{-1}-1,332z^{-2}}.
\end{equation}

Já a definição do filtro de referência $F(z^{-1})$ (Equação
\ref{eq:filtro-de-referencia-com-fracoes-de-polinomios}) considerou a Equação
\ref{eq:numerador-do-filtro-de-referencia} sendo

\begin{equation}
    \label{eq:definicao-dos-polos-0bservaveis}
    \lambda_o(z^{-1}) = 1 - 0.09957 z^{-1} + 0.002479 z^{-2}
\end{equation} e para garantir seguimento de referência, a partir da Equação
\ref{eq:ft-da-saida-para-referencia}, tem-se que

\begin{equation}
    \label{eq:ganho-do-filtro-de-referencia}
    t_0 = \frac{\lambda_c(1)}{B(1)} = 1,7311
\end{equation} e, portanto,

\begin{equation}
    F(z^{-1}) = \frac{0,04922-0,004901z^{-1}+0,000122z^{-2}}{1-1,59z^{-1}+0,6348z^{-2}}.
\end{equation}

Assim o projeto por alocação de polos está finalizado e os polinômios
$R(z^{-1})$, $S(z^{-1})$ e $T(z^{-1})$ são

\begin{equation}
    \label{eq:polinomios-rst}
    \begin{split}
        R(z^{-1}) &= 1+0,3319z^{-1}-1,332z^{-2}, \\
        S(z^{-1}) &= 35,17-55,93z^{-1}+22,33z^{-2}, \\
        T(z^{-1}) &= 1,731-0,1724z^{-1}+0,004291z^{-2}.
    \end{split}
\end{equation}
\subsubsection{Projeto via IMC}
\label{subsub:projeto-via-imc}

O controlador IMC foi projetado com base no diagrama de blocos da Figura
\ref{fig:diagrama-de-blocos-imc} e sua associação com o diagrama da Figura
\ref{fig:diagrama-de-blocos-malha-fechada}. Dessa forma, o filtro de referência
e o controlador é dado da seguinte forma:

\begin{equation}
    \label{eq:filtro-de-referencia-imc}
    F(z) = \frac{H_d(z)}{F_r(z)}
\end{equation} e

\begin{equation}
    \label{eq:controlador-imc}
    C(z) = \frac{F_r(z)P_{n}^{-1}(z)}{1-F_r(z)}
\end{equation} em que $P_n(z)$ neste caso é igual o modelo da planta dado por
\ref{eq:modelo-em-z-da-planta}. As restrições impostas é que para manter
causalidade, $g_r\left [ F_r(z) \right ] \geq g_r\left [ P_n(z) \right ]$ e
$g_r\left [ H_d(z) \right ] \geq g_r\left [ F_r(z) \right ]$, em que $g_r$
significa grau relativo ou excedentes de polos em relação a zeros.

\begin{figure}[ht!]
	\centering
	\caption{Diagrama de blocos de um controlador IMC.}
	\label{fig:diagrama-de-blocos-imc}
	\includegraphics[width=\textwidth]{chapters/challenge4/images/diagrama-de-blocos-imc.png}
    \caption*{Fonte: \cite{Tito2021}}
\end{figure}

Através de álgebra partindo da Equação \ref{fig:diagrama-de-blocos-imc},
observa-se que o filtro de robustez $F_r(z)$ é igual a complementar de
sensibilidade $\boldsymbol{C}(z)$. Assim, temos que

\begin{equation}
    \label{eq:associacao-do-filtro-de-robustez}
    F_r(z) = \boldsymbol{C}(z) = \frac{C(z)G(z)}{1+C(z)G(z)} = \frac{Y(z)}{Y_{r}(z)}.
\end{equation}

Portanto, observa-se que o filtro de robustez $F_r(z)$ é nada mais que e a
função de transferência em $z$ da entrada $y_r(t)$ para a saída $y(t)$. Desta
maneira definiu-se $F_r(z)$ para ter os polos desejáveis $z^* = 0,7408$ que
resulta no atendimento dos requisitos de regime transitório. Já para os zeros,
foram mantidos os zeros de fase não mínima em $F_r(z)$ para evitar cancelamentos
indesejáveis que leve a instabilidade dado uma perturbação na entrada ou saída
da planta. Por fim, para atender o requisito de rejeição de perturbação, o
controlador precisa possuir um integrador. Isso é assegurado fazendo $F_r(1) =
1$ pois $F_r(1)P_{n}^{-1}(1) \neq 0$ e, portanto, $C(1) \rightarrow \infty$, o
que garante que o controlador possui ação integral. Observa-se depois destas
definições que $F_r(z)$ torna-se igual a função de transferência
\ref{eq:ft-da-saida-para-referencia}, que é então

\begin{equation}
    \label{eq:filtro-de-robustez}
    F_r(z^{-1}) = \frac{-0,00943z^{-1}+0,07661z^{-2}}{1-1,482z^{-1}+0,5488z^{-2}}
\end{equation}

Substituindo \ref{eq:modelo-em-z-da-planta} e \ref{eq:filtro-de-robustez} em
\ref{eq:controlador-imc}, tem-se que

\begin{equation}
    \label{eq:controlador-imc-definido}
    C(z^{-1}) = \frac{1,731-2,98z^{-1}+1,282z^{-2}}{1-1,472z^{-1}+0,4722z^{-2}}.
\end{equation}

Como não é necessário um outro grau de liberdade no projeto do controlador,
faz-se $F_r(z) = H_d(z)$ e, portanto, $F(z) = 1$. Por fim, observa-se que a
causalidade do sistema foi mantida, pois $g_r\left [ F_r(z) \right ] = g_r\left
[ P_n(z) \right ]$. 
\subsubsection{Alocação de Polos x IMC}

A Figura \ref{fig:comparacao-entre-imc-e-alocacao-de-polos} mostra o comparativo
das respostas do sistema de controle amostrado da
\ref{fig:diagrama-de-blocos-malha-fechada} utilizando os dois controladores
projetados. Para esta simulação a perturbação na entrada $q_u(t) =
0,2\mathds{1}(t - 7)$ e na saída $q_y(t) = 0,2\mathds{1}(t - 12)$. Observa-se
que ambos os controladores atendem o requisito de tempo de acomodação e
seguimento de referência. Entretanto, apesar dos dois controladores também façam
com que perturbações constantes do tipo degrau sejam rejeitadas, nota-se que o
projeto via alocação de polos fornecem uma resposta mais rápida a perturbações
se comparada com o projeto via IMC.

\begin{figure}[!ht]
    \caption{Comparação das repostas do sistema em malha fechada com
    controladores projetados via Alocação de Polos e IMC.}
    \vspace{-10pt}
    \hspace{-30pt}
    \label{fig:comparacao-entre-imc-e-alocacao-de-polos}
    \begin{minipage}{\linewidth}
        % Title: gl2ps_renderer figure
% Creator: GL2PS 1.4.0, (C) 1999-2017 C. Geuzaine
% For: Octave
% CreationDate: Sun Sep 26 17:47:58 2021
\setlength{\unitlength}{1pt}
\begin{picture}(0,0)
\includegraphics{images/challenge2/resultado-questao-5-inc}
\end{picture}%
\begin{picture}(400,250)(0,0)
\fontsize{6}{0}
\selectfont\put(52,141.404){\makebox(0,0)[t]{\textcolor[rgb]{0.15,0.15,0.15}{{0}}}}
\fontsize{6}{0}
\selectfont\put(103.675,141.404){\makebox(0,0)[t]{\textcolor[rgb]{0.15,0.15,0.15}{{10}}}}
\fontsize{6}{0}
\selectfont\put(155.351,141.404){\makebox(0,0)[t]{\textcolor[rgb]{0.15,0.15,0.15}{{20}}}}
\fontsize{6}{0}
\selectfont\put(207.026,141.404){\makebox(0,0)[t]{\textcolor[rgb]{0.15,0.15,0.15}{{30}}}}
\fontsize{6}{0}
\selectfont\put(258.701,141.404){\makebox(0,0)[t]{\textcolor[rgb]{0.15,0.15,0.15}{{40}}}}
\fontsize{6}{0}
\selectfont\put(310.376,141.404){\makebox(0,0)[t]{\textcolor[rgb]{0.15,0.15,0.15}{{50}}}}
\fontsize{6}{0}
\selectfont\put(48.5278,155.215){\makebox(0,0)[r]{\textcolor[rgb]{0.15,0.15,0.15}{{0}}}}
\fontsize{6}{0}
\selectfont\put(48.5278,176.659){\makebox(0,0)[r]{\textcolor[rgb]{0.15,0.15,0.15}{{0.5}}}}
\fontsize{6}{0}
\selectfont\put(48.5278,198.104){\makebox(0,0)[r]{\textcolor[rgb]{0.15,0.15,0.15}{{1}}}}
\fontsize{6}{0}
\selectfont\put(48.5278,219.548){\makebox(0,0)[r]{\textcolor[rgb]{0.15,0.15,0.15}{{1.5}}}}
\fontsize{7}{0}
\selectfont\put(33.5278,186.325){\rotatebox{90}{\makebox(0,0)[b]{\textcolor[rgb]{0.15,0.15,0.15}{{Saída $y_{i}(t)$}}}}}
\fontsize{7}{0}
\selectfont\put(207,130.404){\makebox(0,0)[t]{\textcolor[rgb]{0.15,0.15,0.15}{{Tempo (s)}}}}
\fontsize{6}{0}
\selectfont\put(332,210.139){\makebox(0,0)[l]{\textcolor[rgb]{0,0,0}{{$r(t)$}}}}
\fontsize{6}{0}
\selectfont\put(332,198.639){\makebox(0,0)[l]{\textcolor[rgb]{0,0,0}{{$y_{1}(t)$}}}}
\fontsize{6}{0}
\selectfont\put(332,186.638){\makebox(0,0)[l]{\textcolor[rgb]{0,0,0}{{$y_{2}(t)$}}}}
\fontsize{6}{0}
\selectfont\put(332,174.638){\makebox(0,0)[l]{\textcolor[rgb]{0,0,0}{{$y_{3}(t)$}}}}
\fontsize{6}{0}
\selectfont\put(332,162.638){\makebox(0,0)[l]{\textcolor[rgb]{0,0,0}{{$y_{4}(t)$}}}}
\fontsize{6}{0}
\selectfont\put(52,22.2671){\makebox(0,0)[t]{\textcolor[rgb]{0.15,0.15,0.15}{{0}}}}
\fontsize{6}{0}
\selectfont\put(103.675,22.2671){\makebox(0,0)[t]{\textcolor[rgb]{0.15,0.15,0.15}{{10}}}}
\fontsize{6}{0}
\selectfont\put(155.351,22.2671){\makebox(0,0)[t]{\textcolor[rgb]{0.15,0.15,0.15}{{20}}}}
\fontsize{6}{0}
\selectfont\put(207.026,22.2671){\makebox(0,0)[t]{\textcolor[rgb]{0.15,0.15,0.15}{{30}}}}
\fontsize{6}{0}
\selectfont\put(258.701,22.2671){\makebox(0,0)[t]{\textcolor[rgb]{0.15,0.15,0.15}{{40}}}}
\fontsize{6}{0}
\selectfont\put(310.376,22.2671){\makebox(0,0)[t]{\textcolor[rgb]{0.15,0.15,0.15}{{50}}}}
\fontsize{6}{0}
\selectfont\put(48.5278,34.7158){\makebox(0,0)[r]{\textcolor[rgb]{0.15,0.15,0.15}{{0}}}}
\fontsize{6}{0}
\selectfont\put(48.5278,52.7559){\makebox(0,0)[r]{\textcolor[rgb]{0.15,0.15,0.15}{{0.5}}}}
\fontsize{6}{0}
\selectfont\put(48.5278,70.7954){\makebox(0,0)[r]{\textcolor[rgb]{0.15,0.15,0.15}{{1}}}}
\fontsize{6}{0}
\selectfont\put(48.5278,88.8354){\makebox(0,0)[r]{\textcolor[rgb]{0.15,0.15,0.15}{{1.5}}}}
\fontsize{6}{0}
\selectfont\put(48.5278,106.875){\makebox(0,0)[r]{\textcolor[rgb]{0.15,0.15,0.15}{{2}}}}
\fontsize{7}{0}
\selectfont\put(33.5278,67.1875){\rotatebox{90}{\makebox(0,0)[b]{\textcolor[rgb]{0.15,0.15,0.15}{{Sinal $u_{i}(t)$}}}}}
\fontsize{7}{0}
\selectfont\put(207,11.2671){\makebox(0,0)[t]{\textcolor[rgb]{0.15,0.15,0.15}{{Tempo (s)}}}}
\fontsize{6}{0}
\selectfont\put(332,76.0015){\makebox(0,0)[l]{\textcolor[rgb]{0,0,0}{{$u_{1}(t)$}}}}
\fontsize{6}{0}
\selectfont\put(332,65.001){\makebox(0,0)[l]{\textcolor[rgb]{0,0,0}{{$u_{2}(t)$}}}}
\fontsize{6}{0}
\selectfont\put(332,54.001){\makebox(0,0)[l]{\textcolor[rgb]{0,0,0}{{$u_{3}(t)$}}}}
\fontsize{6}{0}
\selectfont\put(332,43.0005){\makebox(0,0)[l]{\textcolor[rgb]{0,0,0}{{$u_{4}(t)$}}}}
\end{picture}

    \end{minipage}
\end{figure}

Para analisar o comportamento da resposta as perturbações dos dois diferentes
projetos, é necessário a inspeção dos pares entrada/saída. Percebe-se nas
Equações \ref{eq:ft-da-saida-para-perturbacao-na-entrada} e
\ref{eq:ft-da-saida-para-perturbacao-na-saida} que os polos das funções de
transferência da entrada para saída foram exatamente os polos alocados, sendo os
polos observáveis 10 vezes mais rápidos que os polos controláveis, que já são
rápidos devido a constante de tempo $\tau = 0,5$. Em contrapartida, devido a
propriedade intrínseca de cancelamento do projeto via IMC, alguns polos lentos
podem aparecer nas funções de transferência das perturbações para saída do
sistema em malha fechada.

A partir da Equação \ref{eq:filtro-de-robustez} e do diagrama de blocos
\ref{fig:diagrama-de-blocos-malha-fechada}, as seguintes funções de
transferência são deduzidas:

\begin{equation}
    \label{eq:ft-perturbacao-entrada-imc}
    \frac{Y(z)}{Q_u(z)} = P_n(z)(1-F_r(z)) = P_n(z)S(z)
\end{equation} em que $S(z)$ é a complementar de sensibilidade e

\begin{equation}
    \label{eq:ft-perturbacao-saida-imc}
    \frac{Y(z)}{Q_y(z)} = 1-F_r(z) = S(z).
\end{equation}

Percebe-se então que para um perturbação na entrada da planta $q_u(t)$, aparecem
os polos da própria planta, que neste caso são polos lentos se comparados com os
da malha fechada via projeto por alocação de polos. Por outro lado, para
perturbações na saída $q_y(t)$ os polos são os mesmos que os desejáveis, que são
rápidos e, portanto, fornecem uma resposta a perturbação mais rápida. 

É notório também a influência de zeros na rejeição a perturbação na saída da
planta. Observa-se que para o projeto via alocação de polos, a saída da planta
$y(t)$ cai abaixo do valor de \textit{set point}, devido a um zero negativo fora
do círculo unitário (-1.3319). Já em \ref{eq:ft-perturbacao-saida-imc} não há
zeros negativo fora do circulo unitário e, consequentemente, o mesmo
comportamento não é observado.

Vale salientar também a importância de não realizar o cancelamento do zero de
fase não mínimas nos projetos alocando os polos do controlador no mesmo local.
Caso este polo fosse alocado nos controladores projetados, ele não iria aparecer
nas funções de transferência listadas neste tópico. Entretanto, ao analisar a
Gangue dos Seis definidas por \cite{Astrom2008}, caso alocado no controlador, o
polo instável aparece em dois pares de entrada saída conforme a Equação abaixo,
em que o argumento foi omitido por simplicidade e o par entrada/saida é definido
pelo subescrito.

\begin{subequations}
    \begin{equation}
        G_{ur} = \frac{CF}{1+PC}
    \end{equation}
    \begin{equation}
        -G_{uw} = \frac{C}{1+PC}
    \end{equation}
\end{subequations} em que $u$, $r$ e $w$ significam sinal de controle,
referência e ruído respectivamente e $P$ e $C$ significam respectivamente função
de transferência da planta e controlador.


\subsection{Conclusões}
Os resultados alcançados expressam a utilidade para o projetista do projeto de
controladores diretamente no domínio de tempo discreto. Eles demonstram também
como é possível alcançar os requisitos desejáveis através de abordagem de
síntese direta em que o compensador é projetado com a planta desejável já
previamente definida, em comparação com métodos que não se tem conhecimento
prévio de como a função de transferência de malha fechada irá ficar.

\section{Desafio VI - Controle Multivariável - Espaço de Estados}

\subsection{Motivação}

Com o avanço da tecnologia e consequentemente com avanço das aplicações e
técnicas de controle, os Engenheiros e Engenheiras de Controle passaram a ter
uma gama de técnicas para resolução de problemas de controle. Uma delas e que é
de extrema importância no campo de controle moderno, é a modelagem de processos
por uma realização em espaço de estados. A realização em espaço de estados se
aproveita adicionalmente da teoria de Álgebra Linear para solucionar diversos
problemas de controle. Se destaca no conjunto de problemas o uso desta técnica
para lidar com controle de processos multivariáveis, que é tema deste desafio.
Não obstante, a realização em espaço de estados permite modelar observadores de
estados que são de extrema importância em problemas em que a dinâmica interna do
processo não pode ser facilmente medida ou até mesmo podem ser utilizados para
diminuir o custo com a aquisição de sensores em processos que ainda sim podem
ter sua dinâmica interna mensurada. Diante disso, é imprescindível que
profissionais que atuam na resolução de problemas de controle tenham
conhecimento das técnica de controle por realização em espaço de estados.  

\subsection{Simulações realizadas}

Para aplicação das técnicas de controle via realização em espaço de estados, foi
utilizado o modelo multivariável de dois tanques interconectados conforme a
função de transferência $\mathbf{G}(s)$ dado pela Equação
\ref{eq:matriz-de-fts}. A partir da conversão deste modelo para a representação
em espaço de estados em tempo continuo e discreto, as seguintes análises foram
realizadas:

\begin{itemize}
    \item Avaliação das variáveis do espaço de estados considerando o processo
    na sua condição de equilíbrio;
    \item Projeto de um controlador discreto por realimentação de estados para
    regulação dos estados ao valor de equilíbrio desejado;
    \item Projeto de um controlador discreto por realimentação de estados e com
    ação integral para assegurar seguimento de referência e rejeição de
    perturbações constantes do tipo degrau; e
    \item Projeto dos dois controladores anteriores com o uso do observador de
    estados de Luenberger.
\end{itemize}

\begin{equation}
    \label{eq:matriz-de-fts}
    \mathbf{G}(s) =
    \begin{bmatrix}
        \frac{2}{(10s+1)}         & \frac{0,8}{(10s+1)(2s+1)} \\
        \frac{0,6}{(10s+1)(2s+1)} & \frac{2}{(10s+1)}
    \end{bmatrix}
\end{equation}

Como as análise realizadas se basearam em técnicas de controle de sistemas
amostrados, todos os resultados foram obtidos considerando que dinâmica da
planta foi simulada utilizando a aproximação de Euler com um passo de integração
de 0,025s.

Além disso, para melhor compreensão os vetores, linha e coluna foram
representados por letras minúsculas em negrito ($\mathbf{v}$) e as
matrizes por letra maiúscula também em negrito, exemplo $\mathbf{A}$. Para
distinguir as matrizes do tempo continuo das do tempo discreto, optou-se por
usar o til nas matrizes relacionadas ao tempo discreto, como
$\mathbf{\tilde{A}}$ para a matriz de dinâmica em tempo discreto. Já as matrizes
relacionadas a realização aumentada de espaço de estados foram representadas com
uma barra superior, isto é $\mathbf{\bar{A}}$. Uma exceção a essa regra foi a
adoção do símbolo $\mathbf{\bar{0}}$, que significa uma matriz de zeros. Esta
distinção foi para evitar confusão com o vetor nulo definido como $\mathbf{0}$.

\subsection{Resultados obtidos}
\label{sub:resultados-obtidos-desafio6}

\subsubsection{Cálculo do ponto de equilíbrio}
\label{subsub:calculo-do-ponto-de-equilibrio}
Dado que o sistema multivariável é caracterizado por \ref{eq:matriz-de-fts}, a
saída do sistema é $\mathbf{y} = [Y_1(s) \thickspace Y_2(s)]^{T}$ dado o vetor de
sinais de controle $\mathbf{u} = [U_1(s) \thickspace U_2(s)]^{T}$, temos então que
$\mathbf{y} = \mathbf{G}\mathbf{u}$ ou

\begin{equation}
    \label{eq:relacao-entrada-saida-de-g}
    \begin{bmatrix}
        Y_1(s) \\
        Y_2(s)
    \end{bmatrix}
    =
    \begin{bmatrix}
        \frac{2}{(10s+1)}         & \frac{0,8}{(10s+1)(2s+1)} \\
        \frac{0,6}{(10s+1)(2s+1)} & \frac{2}{(10s+1)}
    \end{bmatrix}
    \begin{bmatrix}
        U_1(s) \\
        U_2(s)
    \end{bmatrix}.
\end{equation}

Quando aplicado o sinal de controle $\mathbf{u}_{eq} = \lim_{t \rightarrow
        \infty }\mathbf{u}(t)$, a saída do sistema chega ao equilíbrio $\mathbf{y}_{eq}
    = \lim_{t \rightarrow \infty }\mathbf{y}(t)$. Aplicando o teorema do valor
final em \ref{eq:relacao-entrada-saida-de-g} dado uma entrada em degrau para
$\mathbf{u}(t)$, temos que $\mathbf{u}_{eq} = \mathbf{G}(0)^{-1}\mathbf{y}_{eq}$
se $det(\mathbf{G}(0)) \neq 0$. Como,

\begin{equation}
    \label{eq:ganho-estatico-de-g}
    \mathbf{G}(0)
    =
    \begin{bmatrix}
        2,0 & 0,8 \\
        0,6 & 1,5
    \end{bmatrix},
\end{equation}logo, det($\mathbf{G}(0)$) $\neq 0$ e $\mathbf{G}(0)^{-1}$ existe.
Dado a saída de equilíbrio desejada $\mathbf{y}_{eq} = [2 \thickspace 1]^\top$,
o sinal de controle de equilíbrio

\begin{equation}
    \label{eq:vetor-do-controle-de-equilibrio}
    \mathbf{u}_{eq} =
    \begin{bmatrix}
        0,8730 \\
        0,3175
    \end{bmatrix}
\end{equation}

\subsubsection{Espaço de Estados no domínio de tempo continuo}
\label{subsub:espaco-de-estados-no-dominio-de-tempo-continuo}

A utilização da função \textit{ss} do Octave para transformar o modelo
\ref{eq:matriz-de-fts} dado no domínio de Laplace para realização em espaço de
estados

\begin{subequations}
    \label{eq:espaco-de-estados-continuo}
    \begin{align}
        \mathbf{\dot{x}}(t) & = \mathbf{A}\mathbf{x}(t) + \mathbf{B}\mathbf{u}(t)
        \label{eq:derivada-do-vetor-de-estados}                                   \\
        \mathbf{y}(t)       & = \mathbf{C}\mathbf{x}(t) + \mathbf{D}\mathbf{u}(t)
        \label{eq:saida-do-sistema-em-espaco-de-estados}
    \end{align}
\end{subequations} resulta nas matrizes $\mathbf{A}_{4\times 4}$,
$\mathbf{B}_{4\times 2}$ e $\mathbf{C}_{2\times 4}$, cujo elementos estão
definidas em \ref{eq:matrizes-do-espaco-de-estados}. Ressalta-se entretanto, que
$\mathbf{D} = \mathbf{\bar{0}}$ para o sistema deste desafio já que não há
transmissão direta do sinal de controle para a saída. Observa-se das dimensões
destas matrizes, que para o sistema dado existem 4 variáveis de estados.

\begin{subequations}
    \label{eq:matrizes-do-espaco-de-estados}
    \begin{align}
        \mathbf{A} & =
        \begin{bmatrix}
            \label{eq:matriz-a}
            0,0500 & 0       & 0,1164  & 0       \\
            0      & 0,0500  & 0       & 0,1164  \\
            0,7089 & 0       & -0,6500 & 0       \\
            0      & -0,7089 & 0       & -0,6500
        \end{bmatrix},  \\
        \mathbf{B} & =
        \begin{bmatrix}
            \label{eq:matriz-b}
            -0,0849 & -0,2333 \\
            -0,3111 & -0,1131 \\
            0       & 0,3007  \\
            0,4010  & 0
        \end{bmatrix}, \\
        \mathbf{C} & =
        \begin{bmatrix}
            \label{eq:matriz-c}
            0 & 0 & 0      & 0,4988 \\
            0 & 0 & 0,4988 & 0
        \end{bmatrix} 
    \end{align}
\end{subequations}

Com estas matrizes é possível calcular quais são os estados do sistema quando a
saída atinge seu valor de equilíbrio. Quando isto acontece, $\mathbf{\dot{x}}(t)
= \mathbf{0}$ e, portanto, a partir de \ref{eq:derivada-do-vetor-de-estados},

\begin{equation}
    \label{eq:calculo-do-vetor-de-estados-de-equilibrio}
    \mathbf{x}_{eq} = -\mathbf{A}^{-1}\mathbf{B}\mathbf{u}_{eq}.
\end{equation} Vale reforçar que isto é verdadeiro se $\mathbf{A}$ é invertível
ou det($\mathbf{A}$) $\neq 0$. De \ref{eq:matriz-a}, verifica-se que
det($\mathbf{A}$) $\neq 0$. Assim, substituindo
\ref{eq:vetor-do-controle-de-equilibrio}, \ref{eq:matriz-a} e \ref{eq:matriz-b}
em \ref{eq:calculo-do-vetor-de-estados-de-equilibrio}, tem se que

\begin{equation}
    \label{eq:estados-de-equilibrio}
    \mathbf{x}_{eq} =
    \begin{bmatrix}
        x_{eq_{1}} \\
        x_{eq_{2}} \\
        x_{eq_{3}} \\
        x_{eq_{4}}
    \end{bmatrix}
    =
    \begin{bmatrix}
        -1,7038 \\
        -3,1831 \\
        2,0050  \\
        4,0100
    \end{bmatrix}\text{.}
\end{equation}

\subsubsection{Espaço de Estado no domínio de tempo discreto}
\label{subsub:espaco-de-estados-no-dominio-de-tempo-discreto}

A versão em tempo discreto para a representação em espaço de estados definido em
\ref{eq:espaco-de-estados-continuo} é dado da seguinte forma

\begin{subequations}
    \label{eq:espaco-de-estados-discreto}
    \begin{align}
        \mathbf{x}[k+1] & = \mathbf{\tilde{A}}\mathbf{x}[k] + \mathbf{\tilde{B}}\mathbf{u}[k]
        \label{eq:estimativa-do-vetor-de-estados}                                             \\
        \mathbf{y}[k]   & = \mathbf{\tilde{C}}\mathbf{x}[t] + \mathbf{\tilde{D}}\mathbf{u}[k]
        \label{eq:saida-do-sistema-em-espaco-de-estados-discreto}
    \end{align}
\end{subequations} que segundo \citeonline{Chen2006}, a relação entre as
matrizes de tempo continuo e discreto é dado como

\begin{equation*}
    \mathbf{\tilde{A}} = e^{\mathbf{A}T},
    \thickspace
    \mathbf{\tilde{B}} = \int_{0}^{T}e^{\mathbf{A}\tau}d\tau \mathbf{B},
    \thickspace
    \mathbf{\tilde{C}} = \mathbf{C}
    \thickspace \text{e} \thickspace
    \mathbf{\tilde{D}} = \mathbf{D},
\end{equation*} em que $T$ é o período de amostragem do sistema de controle
amostrado. Observa-se portanto que é necessário definir um período de amostragem
adequado para ter a realização do modelo \ref{eq:matriz-de-fts} em espaço de
estados de tempo discreto.

O critério para definição do tempo de amostragem foi igual ao utilizado no
desafio 4 de projeto de controladores no tempo discreto (seção
\ref{sub:simulacoes-realizadas-desafio4}). Entretanto, o modelo em questão é
multivariável. Considerando então, que $\mathbf{W}$ é uma matriz cujo elementos
$w_{ij}$ é a frequência de corte de cada função de transferência $g_{ij}$ de
$\mathbf{G}(s)$, optou-se por aplicar o critério na máxima frequência de corte,
ou seja $w_a = 30\times\max(\mathbf{W})$, em que $w_a$ é a frequência de
amostragem em rad/s. Esta abordagem é conservadora, pois os sistemas cuja a
frequência de corte $w_c < \max(\mathbf{W})$ por si só irão atenuar sinais de
alta frequências, e então estes sinais não contribuirão significativamente em
suas respostas. Ou de forma matemática, se $w_a = 30\times\max(\mathbf{W})$,
logo o critério de $30\times$ será válido para as demais plantas. Assim,
obteve-se a matriz $\mathbf{W}$ conforme
\ref{eq:matriz-das-frequencias-de-corte}. Percebe-se que $w_{21} = 0$ pois
$|g_{21}(0)| < -3$dB e esta atribuição não compromete o critério da
definição do período de amostragem.

\begin{equation}
    \label{eq:matriz-das-frequencias-de-corte}
    \mathbf{W} =
    \begin{bmatrix}
        0,2646 & 0,0516 \\
        0      & 0,1871
    \end{bmatrix}
\end{equation}

Portanto, $\max(\mathbf{W}) = 0,2646$ rad/s, o que resultou, após
arrendondamento, num período de amostragem $T_a = 0,75$s.

Novamente com o auxílio da função \textit{ss} do Octave, mas agora convertendo
as matrizes para realização em espaço de estados de tempo discreto utilizando a
função \textit{c2d} passando como argumento período de amostragem definido,
chegou-se nas seguintes matrizes:

\begin{subequations}
    \label{eq:matrizes-do-espaco-de-estados-discretizadas}
    \begin{align}
        \mathbf{\tilde{A}} & =
        \begin{bmatrix}
            \label{eq:matriz-til-a}
            1,0179 & 0       & 0,069961 & 0      \\
            0       & 1,0179 & 0      & 0,069961 \\
            -0,42613 & 0       & 0,59712 & 0      \\
            0       & -0,42613  & 0     & 0,59712
        \end{bmatrix}\text{,} \\
        \mathbf{\tilde{B}} & =
        \begin{bmatrix}
            \label{eq:matriz-til-b}
            -0,0644 & -0,1686  \\
            -0,2248 & -0,0859  \\
            0,0146  & 0,2173   \\
            0,2897  & 0,0195
        \end{bmatrix}\text{,} \\
        \mathbf{\tilde{C}} &= \mathbf{C} \\
        \mathbf{\tilde{D}} &= \mathbf{D}
    \end{align}
\end{subequations}

Com a definição destas matrizes é possível definir os estados de equilíbrio dado
o vetor do sinal de controle em equilíbrio
\ref{eq:vetor-do-controle-de-equilibrio}. Salienta-se entretanto que para o
tempo discreto, o estado de equilíbrio é alcançado quando o estado atual é igual
ao estado anterior, ou de forma algébrica, a partir de
\ref{eq:estimativa-do-vetor-de-estados},

\begin{equation}
    \label{eq:calculo-dos-estados-de-equilibrio-em-tempo-discreto}
    \mathbf{\tilde{x}}_{eq} = \mathbf{\tilde{A}}\mathbf{\tilde{x}}_{eq} + \mathbf{\tilde{B}}\mathbf{u}_{eq}
    \Rightarrow
    \mathbf{\tilde{x}}_{eq} = (\mathbf{I}-\mathbf{\tilde{A}})^{-1}\mathbf{\tilde{B}}\mathbf{u}_{eq}.
\end{equation} Portanto, substituindo \ref{eq:vetor-do-controle-de-equilibrio},
\ref{eq:matriz-til-a} e \ref{eq:matriz-til-b} em
\ref{eq:calculo-dos-estados-de-equilibrio-em-tempo-discreto}, chegou-se ao mesmo
resultado de tempo contínuo conforme \ref{eq:estados-de-equilibrio}. O mesmo
aconteceu para a saída $\mathbf{\tilde{y}}_{eq} =
\mathbf{\tilde{C}}\mathbf{\tilde{x}}_{eq}$, o que é coerente já que
$\mathbf{\tilde{x}}_{eq} = \mathbf{x}_{eq}$ e $\mathbf{\tilde{C}} = \mathbf{C}$.

\subsubsection{Regulação em torno do ponto de equilíbrio}
\label{subsub:regulacao-em-torno-do-ponto-de-equilibrio}

As seções anteriores abordaram a análise do estado de equilíbrio assumindo que o
sistema já estava na sua condição de equilíbrio. Entretanto, para levar o sistema
ao equilíbrio é preciso aplicar uma ação de controle de regulação em torno do
ponto de equilíbrio, dada por

\begin{equation}
    \label{eq:acao-de-controle-em-torno-do-ponto-de-equilibrio}
    \mathbf{u}(kT_a) = -\mathbf{\tilde{K}}[\mathbf{x}(kT_a) - \mathbf{x_{eq}}]+\mathbf{u}_{eq}, 
    \thickspace k = 0, 1, 2, ..., n.
\end{equation}

Substituindo \ref{eq:acao-de-controle-em-torno-do-ponto-de-equilibrio} em
\ref{eq:estimativa-do-vetor-de-estados}, já que $u[k] := u(kT_a)$ e $x[k] :=
x(kT_a)$, o novo polinômio característico é dado por
det($z\mathbf{I}-\mathbf{\tilde{A}}-\mathbf{\tilde{B}}\mathbf{\tilde{K}}$) e
portanto os polos do sistema discreto podem ser modificados para um valor
desejável. Entretanto, isto só é válido se o sistema for controlável, isto é, a
matriz de controlabilidade $\mathbf{\tilde{U}}$ dado por
\ref{eq:matriz-de-controlabilidade-de-tempo-discreto} tem todas as linhas
linearmente independentes.

\begin{equation}
    \label{eq:matriz-de-controlabilidade-de-tempo-discreto}
    \mathbf{\tilde{U}} = [
    \mathbf{\tilde{B}}
    \thickspace
    \mathbf{\tilde{A}\tilde{B}} 
    \thickspace
    \mathbf{\tilde{A}}\negthinspace^{2}\mathbf{\tilde{B}}
    \thickspace
    \mathbf{\tilde{A}}\negthinspace^{3}\mathbf{\tilde{B}}]
\end{equation}

A aplicação das matrizes \ref{eq:matriz-til-a} e \ref{eq:matriz-til-b} em
\ref{eq:matriz-de-controlabilidade-de-tempo-discreto} resulta em uma matriz
$\mathbf{\tilde{U}}$ cujo posto é igual 4, que é a mesma ordem do sistema e,
portanto, todas as linhas são linearmente independentes. Assim, todos os
autovalores podem ser definidos escolhendo um $\mathbf{\tilde{K}}$ adequado.

A definição do valor de $\mathbf{\tilde{K}}$ foi realizada através de um método
de alocação de autovalores em espaço de estados. Existem vários métodos na
literatura em que usam uma transformação linear de similaridade para facilitar o
cálculo do ganho $\mathbf{\tilde{K}}$. Alguns destes métodos já são
disponibilizados computacionalmente em funções, como é o caso da função
\textit{place} do Octave/Matlab. Dado então o requisito de que os autovalores de
malha fechada de tempo discreto, ou seja os autovalores da matriz
$\mathbf{\tilde{A} - \tilde{B}\tilde{K}}$ devem ser $\lambda_i =
e^{\frac{-T_a}{4}}$ para $i = 1, 2, 3$ e $4$, em que dado $T_a = 0,75$s,
$\lambda_i = 0,8290$. Utilizando portanto a função \textit{place} tendo como
argumento as matrizes $\mathbf{\tilde{A}}$ e $\mathbf{\tilde{B}}$ e o vetor
$\mathbf{\lambda}$ contendo os autovalores desejáveis $\lambda_i$, foi obtida a
matriz de ganhos

\begin{equation}
    \label{eq:matriz-de-ganhos-discreto}
    \mathbf{\tilde{K}} = 
    \begin{bmatrix}
        1,6632 & -0,4644 & 1,0590 & -0,0451    \\
        0,5947 & 1,4181 &  0,5227 & 1,1671
    \end{bmatrix}.
\end{equation}

O resultado da simulação da realização em espaço de estados dado o
$\mathbf{\tilde{K}}$ encontrado é mostrado na Figura
\ref{fig:resultado-do-regulador-via-realimentacao-de-estados}. Percebe-se nesta
figura que o sistema atinge seu valor de equilíbrio dado por $\mathbf{u}_{eq} =
[0,8730 \thickspace 0,3175]^\top$ e $\mathbf{y}_{eq} = [2 \thickspace 1]^\top$.
Repara-se também a presença de um sobressinal. Como o sistema em questão trata
de dois tanques interconectados, a interação entre suas dinâmicas faz surgir
zeros que alteram de forma significativa a saída do sistema, mesmo com
autovalores reais, o que explica o sobressinal nas respostas $y_1(t)$ e $y_1(t)$
obtidas.

\begin{figure}[!htp]
    \caption{Saídas e sinais de controle do sistema MIMO de tanques
    interconectados utilizando regulação via realimentação de estados.}
    \vspace{-10pt}
    \hspace{-30pt}
    \label{fig:resultado-do-regulador-via-realimentacao-de-estados}
    \begin{minipage}{\linewidth}
        % Title: gl2ps_renderer figure
% Creator: GL2PS 1.4.0, (C) 1999-2017 C. Geuzaine
% For: Octave
% CreationDate: Thu Nov 25 18:14:01 2021
\setlength{\unitlength}{1pt}
\begin{picture}(0,0)
\includegraphics{chapters/challenge6/images/resultado-questao-7-inc}
\end{picture}%
\begin{picture}(400,250)(0,0)
\fontsize{6}{0}
\selectfont\put(52,141.404){\makebox(0,0)[t]{\textcolor[rgb]{0.15,0.15,0.15}{{0}}}}
\fontsize{6}{0}
\selectfont\put(70.4365,141.404){\makebox(0,0)[t]{\textcolor[rgb]{0.15,0.15,0.15}{{20}}}}
\fontsize{6}{0}
\selectfont\put(88.873,141.404){\makebox(0,0)[t]{\textcolor[rgb]{0.15,0.15,0.15}{{40}}}}
\fontsize{6}{0}
\selectfont\put(107.309,141.404){\makebox(0,0)[t]{\textcolor[rgb]{0.15,0.15,0.15}{{60}}}}
\fontsize{6}{0}
\selectfont\put(125.746,141.404){\makebox(0,0)[t]{\textcolor[rgb]{0.15,0.15,0.15}{{80}}}}
\fontsize{6}{0}
\selectfont\put(144.182,141.404){\makebox(0,0)[t]{\textcolor[rgb]{0.15,0.15,0.15}{{100}}}}
\fontsize{6}{0}
\selectfont\put(162.619,141.404){\makebox(0,0)[t]{\textcolor[rgb]{0.15,0.15,0.15}{{120}}}}
\fontsize{6}{0}
\selectfont\put(181.055,141.404){\makebox(0,0)[t]{\textcolor[rgb]{0.15,0.15,0.15}{{140}}}}
\fontsize{6}{0}
\selectfont\put(48.5205,153.193){\makebox(0,0)[r]{\textcolor[rgb]{0.15,0.15,0.15}{{0}}}}
\fontsize{6}{0}
\selectfont\put(48.5205,169.582){\makebox(0,0)[r]{\textcolor[rgb]{0.15,0.15,0.15}{{0.5}}}}
\fontsize{6}{0}
\selectfont\put(48.5205,185.97){\makebox(0,0)[r]{\textcolor[rgb]{0.15,0.15,0.15}{{1}}}}
\fontsize{6}{0}
\selectfont\put(48.5205,202.359){\makebox(0,0)[r]{\textcolor[rgb]{0.15,0.15,0.15}{{1.5}}}}
\fontsize{6}{0}
\selectfont\put(48.5205,218.748){\makebox(0,0)[r]{\textcolor[rgb]{0.15,0.15,0.15}{{2}}}}
\fontsize{7}{0}
\selectfont\put(33.5205,186.325){\rotatebox{90}{\makebox(0,0)[b]{\textcolor[rgb]{0.15,0.15,0.15}{{Saida $y_1(t)$}}}}}
\fontsize{7}{0}
\selectfont\put(116.643,130.404){\makebox(0,0)[t]{\textcolor[rgb]{0.15,0.15,0.15}{{Tempo (s)}}}}
\fontsize{6}{0}
\selectfont\put(232.715,141.404){\makebox(0,0)[t]{\textcolor[rgb]{0.15,0.15,0.15}{{0}}}}
\fontsize{6}{0}
\selectfont\put(251.151,141.404){\makebox(0,0)[t]{\textcolor[rgb]{0.15,0.15,0.15}{{20}}}}
\fontsize{6}{0}
\selectfont\put(269.587,141.404){\makebox(0,0)[t]{\textcolor[rgb]{0.15,0.15,0.15}{{40}}}}
\fontsize{6}{0}
\selectfont\put(288.024,141.404){\makebox(0,0)[t]{\textcolor[rgb]{0.15,0.15,0.15}{{60}}}}
\fontsize{6}{0}
\selectfont\put(306.46,141.404){\makebox(0,0)[t]{\textcolor[rgb]{0.15,0.15,0.15}{{80}}}}
\fontsize{6}{0}
\selectfont\put(324.896,141.404){\makebox(0,0)[t]{\textcolor[rgb]{0.15,0.15,0.15}{{100}}}}
\fontsize{6}{0}
\selectfont\put(343.333,141.404){\makebox(0,0)[t]{\textcolor[rgb]{0.15,0.15,0.15}{{120}}}}
\fontsize{6}{0}
\selectfont\put(361.77,141.404){\makebox(0,0)[t]{\textcolor[rgb]{0.15,0.15,0.15}{{140}}}}
\fontsize{6}{0}
\selectfont\put(229.235,146.637){\makebox(0,0)[r]{\textcolor[rgb]{0.15,0.15,0.15}{{-0.2}}}}
\fontsize{6}{0}
\selectfont\put(229.235,157.638){\makebox(0,0)[r]{\textcolor[rgb]{0.15,0.15,0.15}{{0}}}}
\fontsize{6}{0}
\selectfont\put(229.235,168.638){\makebox(0,0)[r]{\textcolor[rgb]{0.15,0.15,0.15}{{0.2}}}}
\fontsize{6}{0}
\selectfont\put(229.235,179.638){\makebox(0,0)[r]{\textcolor[rgb]{0.15,0.15,0.15}{{0.4}}}}
\fontsize{6}{0}
\selectfont\put(229.235,190.638){\makebox(0,0)[r]{\textcolor[rgb]{0.15,0.15,0.15}{{0.6}}}}
\fontsize{6}{0}
\selectfont\put(229.235,201.639){\makebox(0,0)[r]{\textcolor[rgb]{0.15,0.15,0.15}{{0.8}}}}
\fontsize{6}{0}
\selectfont\put(229.235,212.639){\makebox(0,0)[r]{\textcolor[rgb]{0.15,0.15,0.15}{{1}}}}
\fontsize{6}{0}
\selectfont\put(229.235,223.639){\makebox(0,0)[r]{\textcolor[rgb]{0.15,0.15,0.15}{{1.2}}}}
\fontsize{7}{0}
\selectfont\put(212.235,186.325){\rotatebox{90}{\makebox(0,0)[b]{\textcolor[rgb]{0.15,0.15,0.15}{{Saida $y_2(t)$}}}}}
\fontsize{7}{0}
\selectfont\put(297.357,130.404){\makebox(0,0)[t]{\textcolor[rgb]{0.15,0.15,0.15}{{Tempo (s)}}}}
\fontsize{6}{0}
\selectfont\put(52,22.2671){\makebox(0,0)[t]{\textcolor[rgb]{0.15,0.15,0.15}{{0}}}}
\fontsize{6}{0}
\selectfont\put(70.4365,22.2671){\makebox(0,0)[t]{\textcolor[rgb]{0.15,0.15,0.15}{{20}}}}
\fontsize{6}{0}
\selectfont\put(88.873,22.2671){\makebox(0,0)[t]{\textcolor[rgb]{0.15,0.15,0.15}{{40}}}}
\fontsize{6}{0}
\selectfont\put(107.309,22.2671){\makebox(0,0)[t]{\textcolor[rgb]{0.15,0.15,0.15}{{60}}}}
\fontsize{6}{0}
\selectfont\put(125.746,22.2671){\makebox(0,0)[t]{\textcolor[rgb]{0.15,0.15,0.15}{{80}}}}
\fontsize{6}{0}
\selectfont\put(144.182,22.2671){\makebox(0,0)[t]{\textcolor[rgb]{0.15,0.15,0.15}{{100}}}}
\fontsize{6}{0}
\selectfont\put(162.619,22.2671){\makebox(0,0)[t]{\textcolor[rgb]{0.15,0.15,0.15}{{120}}}}
\fontsize{6}{0}
\selectfont\put(181.055,22.2671){\makebox(0,0)[t]{\textcolor[rgb]{0.15,0.15,0.15}{{140}}}}
\fontsize{6}{0}
\selectfont\put(48.5205,36.0054){\makebox(0,0)[r]{\textcolor[rgb]{0.15,0.15,0.15}{{0}}}}
\fontsize{6}{0}
\selectfont\put(48.5205,57.269){\makebox(0,0)[r]{\textcolor[rgb]{0.15,0.15,0.15}{{0.5}}}}
\fontsize{6}{0}
\selectfont\put(48.5205,78.5322){\makebox(0,0)[r]{\textcolor[rgb]{0.15,0.15,0.15}{{1}}}}
\fontsize{6}{0}
\selectfont\put(48.5205,99.7959){\makebox(0,0)[r]{\textcolor[rgb]{0.15,0.15,0.15}{{1.5}}}}
\fontsize{7}{0}
\selectfont\put(33.5205,67.1875){\rotatebox{90}{\makebox(0,0)[b]{\textcolor[rgb]{0.15,0.15,0.15}{{Sinal de Controle $u_1(t)$}}}}}
\fontsize{7}{0}
\selectfont\put(116.643,11.2671){\makebox(0,0)[t]{\textcolor[rgb]{0.15,0.15,0.15}{{Tempo (s)}}}}
\fontsize{6}{0}
\selectfont\put(232.715,22.2671){\makebox(0,0)[t]{\textcolor[rgb]{0.15,0.15,0.15}{{0}}}}
\fontsize{6}{0}
\selectfont\put(251.151,22.2671){\makebox(0,0)[t]{\textcolor[rgb]{0.15,0.15,0.15}{{20}}}}
\fontsize{6}{0}
\selectfont\put(269.587,22.2671){\makebox(0,0)[t]{\textcolor[rgb]{0.15,0.15,0.15}{{40}}}}
\fontsize{6}{0}
\selectfont\put(288.024,22.2671){\makebox(0,0)[t]{\textcolor[rgb]{0.15,0.15,0.15}{{60}}}}
\fontsize{6}{0}
\selectfont\put(306.46,22.2671){\makebox(0,0)[t]{\textcolor[rgb]{0.15,0.15,0.15}{{80}}}}
\fontsize{6}{0}
\selectfont\put(324.896,22.2671){\makebox(0,0)[t]{\textcolor[rgb]{0.15,0.15,0.15}{{100}}}}
\fontsize{6}{0}
\selectfont\put(343.333,22.2671){\makebox(0,0)[t]{\textcolor[rgb]{0.15,0.15,0.15}{{120}}}}
\fontsize{6}{0}
\selectfont\put(361.77,22.2671){\makebox(0,0)[t]{\textcolor[rgb]{0.15,0.15,0.15}{{140}}}}
\fontsize{6}{0}
\selectfont\put(229.235,27.5){\makebox(0,0)[r]{\textcolor[rgb]{0.15,0.15,0.15}{{-0.2}}}}
\fontsize{6}{0}
\selectfont\put(229.235,42.8911){\makebox(0,0)[r]{\textcolor[rgb]{0.15,0.15,0.15}{{0}}}}
\fontsize{6}{0}
\selectfont\put(229.235,58.2817){\makebox(0,0)[r]{\textcolor[rgb]{0.15,0.15,0.15}{{0.2}}}}
\fontsize{6}{0}
\selectfont\put(229.235,73.6729){\makebox(0,0)[r]{\textcolor[rgb]{0.15,0.15,0.15}{{0.4}}}}
\fontsize{6}{0}
\selectfont\put(229.235,89.064){\makebox(0,0)[r]{\textcolor[rgb]{0.15,0.15,0.15}{{0.6}}}}
\fontsize{6}{0}
\selectfont\put(229.235,104.455){\makebox(0,0)[r]{\textcolor[rgb]{0.15,0.15,0.15}{{0.8}}}}
\fontsize{7}{0}
\selectfont\put(212.235,67.1875){\rotatebox{90}{\makebox(0,0)[b]{\textcolor[rgb]{0.15,0.15,0.15}{{Sinal de Controle $u_2(t)$}}}}}
\fontsize{7}{0}
\selectfont\put(297.357,11.2671){\makebox(0,0)[t]{\textcolor[rgb]{0.15,0.15,0.15}{{Tempo (s)}}}}
\fontsize{6}{0}
\selectfont\put(299.842,238.48){\makebox(0,0)[l]{\textcolor[rgb]{0,0,0}{{equilibrio}}}}
\end{picture}

    \end{minipage}
\end{figure}

\subsubsection{Rejeição de perturbações e seguimento de referência}
\label{subsub:rejeicao-de-perturbacoes-e-seguimento-de-referencia}

Na subseção anterior foi abordado a regulação do sistema MIMO de tanques
interconectados para um ponto de equilíbrio utilizando realimentação de estados.
Embora esta abordagem tenha funcionado de forma teórica, na prática ela pode
levar a comportamentos indesejáveis devidos a erros de modelagem, perturbações
externas e ruídos. Uma das soluções para estes possíveis problemas é a
utilização da ação integral no controle da realimentação de estados, tópico
desta seção.

A ação de integração é alcançada fazendo aparecer no espaço de estados um
dinâmica interna, tal que a realização em espaços de estados de tempo discreto
\ref{eq:espaco-de-estados-discreto} torna-se a versão aumentada

\begin{equation}
    \label{eq:espaco-de-estados-discreto-aumentado}
    \begin{bmatrix}
        \Delta x[k+1]\\ 
        y[k+1]
    \end{bmatrix}
    =
    \begin{bmatrix}
        \mathbf{\tilde{A}} & \mathbf{\bar{0}}\\ 
        \mathbf{\tilde{C}}\mathbf{\tilde{A}} & \mathbf{I}
    \end{bmatrix}
    \begin{bmatrix}
        \Delta x[k]\\ 
        y[k]
    \end{bmatrix}
    +
    \begin{bmatrix}
        \mathbf{\tilde{B}}\\ 
        \mathbf{\tilde{C}}\mathbf{\tilde{A}}
    \end{bmatrix}
    \Delta \mathbf{u}[k]
\end{equation} e então as matrizes aumentadas são reescritas como

\begin{subequations}
    \label{eq:matriz-a-e-b-aumentada}
    \begin{align}
        \mathbf{\bar{A}} &=
        \begin{bmatrix}
            \label{eq:matriz-aumentada-a}
            \mathbf{\tilde{A}} & \mathbf{\bar{0}}\\ 
            \mathbf{\tilde{C}}\mathbf{\tilde{A}} & \mathbf{I}
        \end{bmatrix}   \\
        \mathbf{\bar{B}} &=
        \begin{bmatrix}
            \label{eq:matriz-aumentada-b}
            \mathbf{\tilde{B}}\\ 
            \mathbf{\tilde{C}}\mathbf{\tilde{B}}
        \end{bmatrix}
    \end{align}
\end{subequations} em que para a realização de estados para o problema em
questão a ordem de $\mathbf{\bar{A}}$ é $6 \times 6$ e de $\mathbf{\bar{B}}$ é
$6 \times 2$. Portanto, substituindo \ref{eq:matriz-til-a},
\ref{eq:matriz-til-b} e \ref{eq:matriz-c} em \ref{eq:matriz-aumentada-a} e
\ref{eq:matriz-aumentada-b}, chegou-se as matrizes:

\begin{subequations}
        \begin{equation}
            \label{eq:matriz-aumentanda-a-com-valores}
            \mathbf{\bar{A}} =
            \begin{bmatrix}
                1,0179  &  0       &  0,0700  &  0       & 0       & 0      \\
                0       &  1,0179  &  0       &  0,0700  & 0       & 0      \\
                -0,4261 &  0       &  0,5971  &  0       & 0       & 0      \\
                0       &  -0,4261 &  0       &  0,5971  & 0       & 0      \\
                0       &  -0,2125 &  0       &  0,2978  & 1       & 0      \\
                -0,2125 &  0       &  0,2978  &  0       & 0       & 1
            \end{bmatrix}
        \end{equation}    
        \begin{equation}
            \label{eq:matriz-aumentanda-b-com-valores}
            \mathbf{\bar{B}} =
            \begin{bmatrix}
                -0,0644 & -0,1686 \\
                -0,2248 & -0,0859 \\
                0,0146 &  0,2173  \\
                0,2897 &  0,0195  \\
                0,1445 &  0,0097  \\
                0,0073 &  0,1084
            \end{bmatrix}
        \end{equation}
\end{subequations}

Sabida a realização aumentada em espaço de estados acima, foi
utilizada a ação de controle

\begin{equation}
    \label{eq:acao-de-controle-com-integrador}
    \mathbf{u}[k] = \mathbf{u}[k-1] - \mathbf{K}_x(\mathbf{x}[k]-\mathbf{x}[k-1])+\mathbf{K}_i(\mathbf{y}_r[k]-\mathbf{y}[k])
\end{equation} em que a representação $\mathbf{v}[k] := \mathbf{v}(kT_a)$ e
$\mathbf{v}[k-1] := \mathbf{v}((k-1)T_a)$.

Assim é possível novamente alocar os autovalores da matriz
$\mathbf{\bar{A}}-\mathbf{\bar{B}}\mathbf{\bar{K}}$ sendo $\mathbf{\bar{K}} =
[\mathbf{K}_x \thickspace \mathbf{K}_i]$. Vale salientar, que a ordem do sistema
foi aumentada para $n=6$ e, desta forma, são alocados agora 6 autovalores.
Portanto, utilizando a função \textit{place} do Octave e considerando os 6
autovalores repetidos $\lambda_i = 0,8290$, foram obtidas as matrizes de ganho

\begin{subequations}
    \label{eq:matrizes-de-ganho}
    \begin{equation}
        \label{eq:matriz-de-ganhos-kx}
        \mathbf{K}_x =
        \begin{bmatrix}
            1.073066 & -0.786854 &  0.552769 & -0.022183   \\
            -1.049138 &  0.804800 & -0.029577 &  0.414577
        \end{bmatrix}
    \end{equation}
    \text{e}
    \begin{equation}
        \label{eq:matriz-de-ganhos-ki}
        \mathbf{K}_i =
        \begin{bmatrix}
            0.131655 & -0.070216 \\
            -0.052662 &  0.175540
        \end{bmatrix}\text{.}
    \end{equation}
\end{subequations}

A substituição das matrizes de ganhos \ref{eq:matrizes-de-ganho} na ação de
controle \ref{eq:acao-de-controle-com-integrador} resultou no comportamento do
sistema ilustrado na Figura \ref{fig:resultado-do-regulador-com-integrador}.
Nesta figura foi inserida também o comportamento do sistema sem ação de
integração a título de comparação. Observa-se que o tempo de acomodação para o
controlador com ação integral é significativamente superior a regulação com
ganho estático. Entretanto, para perturbações na saída $\mathbf{q}_{y}(t) =
0,2\mathbf{\mathds{1}}(t-50)$ e na entrada $\mathbf{q}_{u}(t) =
0,2\mathbf{\mathds{1}}(t-100)$, fica evidente a rejeição de perturbação para o
sistema com integrador, o que está de acordo com o Princípio do Modelo Interno.
Por outro lado, nota-se que o regulador com ganho estático não rejeitou ambas
perturbações. De forma conceitual, as perturbações não alteram os estados já que
elas aparecem diretamente na saída e entrada do sistema e não internamente ao
sistema. Desta forma, como o regulador por realimentação de estados utiliza
apenas o vetor de estados para levar o sistema ao ponto de equilíbrio desejável,
as perturbações não são identificadas e, portanto, não podem ser rejeitadas. Em
contrapartida, o regulador com integrador utiliza tantos os estados quanto a
saída do sistema e assim é possível identificar as perturbações e rejeitá-las.
Estas duas afirmações ficam evidentes analisando as equações
\ref{eq:acao-de-controle-em-torno-do-ponto-de-equilibrio} e
\ref{eq:acao-de-controle-com-integrador}.

\begin{figure}[!htp]
    \caption{Comparação dos sinais de controle e saída do sistema em malha
    fechada para seguimento de referência com e sem ação integral.}
    \vspace{-10pt}
    \hspace{-30pt}
    \label{fig:resultado-do-regulador-com-integrador}
    \begin{minipage}{\linewidth}
        % Title: gl2ps_renderer figure
% Creator: GL2PS 1.4.0, (C) 1999-2017 C. Geuzaine
% For: Octave
% CreationDate: Thu Nov 25 17:46:16 2021
\setlength{\unitlength}{1pt}
\begin{picture}(0,0)
\includegraphics{chapters/challenge6/images/resultado-questao-9-inc}
\end{picture}%
\begin{picture}(400,250)(0,0)
\fontsize{6}{0}
\selectfont\put(52,141.404){\makebox(0,0)[t]{\textcolor[rgb]{0.15,0.15,0.15}{{0}}}}
\fontsize{6}{0}
\selectfont\put(70.4365,141.404){\makebox(0,0)[t]{\textcolor[rgb]{0.15,0.15,0.15}{{20}}}}
\fontsize{6}{0}
\selectfont\put(88.873,141.404){\makebox(0,0)[t]{\textcolor[rgb]{0.15,0.15,0.15}{{40}}}}
\fontsize{6}{0}
\selectfont\put(107.309,141.404){\makebox(0,0)[t]{\textcolor[rgb]{0.15,0.15,0.15}{{60}}}}
\fontsize{6}{0}
\selectfont\put(125.746,141.404){\makebox(0,0)[t]{\textcolor[rgb]{0.15,0.15,0.15}{{80}}}}
\fontsize{6}{0}
\selectfont\put(144.182,141.404){\makebox(0,0)[t]{\textcolor[rgb]{0.15,0.15,0.15}{{100}}}}
\fontsize{6}{0}
\selectfont\put(162.619,141.404){\makebox(0,0)[t]{\textcolor[rgb]{0.15,0.15,0.15}{{120}}}}
\fontsize{6}{0}
\selectfont\put(181.055,141.404){\makebox(0,0)[t]{\textcolor[rgb]{0.15,0.15,0.15}{{140}}}}
\fontsize{6}{0}
\selectfont\put(48.5205,151.661){\makebox(0,0)[r]{\textcolor[rgb]{0.15,0.15,0.15}{{0}}}}
\fontsize{6}{0}
\selectfont\put(48.5205,164.221){\makebox(0,0)[r]{\textcolor[rgb]{0.15,0.15,0.15}{{0.5}}}}
\fontsize{6}{0}
\selectfont\put(48.5205,176.78){\makebox(0,0)[r]{\textcolor[rgb]{0.15,0.15,0.15}{{1}}}}
\fontsize{6}{0}
\selectfont\put(48.5205,189.34){\makebox(0,0)[r]{\textcolor[rgb]{0.15,0.15,0.15}{{1.5}}}}
\fontsize{6}{0}
\selectfont\put(48.5205,201.899){\makebox(0,0)[r]{\textcolor[rgb]{0.15,0.15,0.15}{{2}}}}
\fontsize{6}{0}
\selectfont\put(48.5205,214.459){\makebox(0,0)[r]{\textcolor[rgb]{0.15,0.15,0.15}{{2.5}}}}
\fontsize{7}{0}
\selectfont\put(33.5205,186.325){\rotatebox{90}{\makebox(0,0)[b]{\textcolor[rgb]{0.15,0.15,0.15}{{Saida $y_1(t)$}}}}}
\fontsize{7}{0}
\selectfont\put(116.643,130.404){\makebox(0,0)[t]{\textcolor[rgb]{0.15,0.15,0.15}{{Tempo (s)}}}}
\fontsize{6}{0}
\selectfont\put(232.715,141.404){\makebox(0,0)[t]{\textcolor[rgb]{0.15,0.15,0.15}{{0}}}}
\fontsize{6}{0}
\selectfont\put(251.151,141.404){\makebox(0,0)[t]{\textcolor[rgb]{0.15,0.15,0.15}{{20}}}}
\fontsize{6}{0}
\selectfont\put(269.587,141.404){\makebox(0,0)[t]{\textcolor[rgb]{0.15,0.15,0.15}{{40}}}}
\fontsize{6}{0}
\selectfont\put(288.024,141.404){\makebox(0,0)[t]{\textcolor[rgb]{0.15,0.15,0.15}{{60}}}}
\fontsize{6}{0}
\selectfont\put(306.46,141.404){\makebox(0,0)[t]{\textcolor[rgb]{0.15,0.15,0.15}{{80}}}}
\fontsize{6}{0}
\selectfont\put(324.896,141.404){\makebox(0,0)[t]{\textcolor[rgb]{0.15,0.15,0.15}{{100}}}}
\fontsize{6}{0}
\selectfont\put(343.333,141.404){\makebox(0,0)[t]{\textcolor[rgb]{0.15,0.15,0.15}{{120}}}}
\fontsize{6}{0}
\selectfont\put(361.77,141.404){\makebox(0,0)[t]{\textcolor[rgb]{0.15,0.15,0.15}{{140}}}}
\fontsize{6}{0}
\selectfont\put(229.235,154.496){\makebox(0,0)[r]{\textcolor[rgb]{0.15,0.15,0.15}{{0}}}}
\fontsize{6}{0}
\selectfont\put(229.235,174.144){\makebox(0,0)[r]{\textcolor[rgb]{0.15,0.15,0.15}{{0.5}}}}
\fontsize{6}{0}
\selectfont\put(229.235,193.792){\makebox(0,0)[r]{\textcolor[rgb]{0.15,0.15,0.15}{{1}}}}
\fontsize{6}{0}
\selectfont\put(229.235,213.44){\makebox(0,0)[r]{\textcolor[rgb]{0.15,0.15,0.15}{{1.5}}}}
\fontsize{7}{0}
\selectfont\put(214.235,186.325){\rotatebox{90}{\makebox(0,0)[b]{\textcolor[rgb]{0.15,0.15,0.15}{{Saida $y_2(t)$}}}}}
\fontsize{7}{0}
\selectfont\put(297.357,130.404){\makebox(0,0)[t]{\textcolor[rgb]{0.15,0.15,0.15}{{Tempo (s)}}}}
\fontsize{6}{0}
\selectfont\put(52,22.2671){\makebox(0,0)[t]{\textcolor[rgb]{0.15,0.15,0.15}{{0}}}}
\fontsize{6}{0}
\selectfont\put(70.4365,22.2671){\makebox(0,0)[t]{\textcolor[rgb]{0.15,0.15,0.15}{{20}}}}
\fontsize{6}{0}
\selectfont\put(88.873,22.2671){\makebox(0,0)[t]{\textcolor[rgb]{0.15,0.15,0.15}{{40}}}}
\fontsize{6}{0}
\selectfont\put(107.309,22.2671){\makebox(0,0)[t]{\textcolor[rgb]{0.15,0.15,0.15}{{60}}}}
\fontsize{6}{0}
\selectfont\put(125.746,22.2671){\makebox(0,0)[t]{\textcolor[rgb]{0.15,0.15,0.15}{{80}}}}
\fontsize{6}{0}
\selectfont\put(144.182,22.2671){\makebox(0,0)[t]{\textcolor[rgb]{0.15,0.15,0.15}{{100}}}}
\fontsize{6}{0}
\selectfont\put(162.619,22.2671){\makebox(0,0)[t]{\textcolor[rgb]{0.15,0.15,0.15}{{120}}}}
\fontsize{6}{0}
\selectfont\put(181.055,22.2671){\makebox(0,0)[t]{\textcolor[rgb]{0.15,0.15,0.15}{{140}}}}
\fontsize{6}{0}
\selectfont\put(48.5205,36.0054){\makebox(0,0)[r]{\textcolor[rgb]{0.15,0.15,0.15}{{0}}}}
\fontsize{6}{0}
\selectfont\put(48.5205,57.269){\makebox(0,0)[r]{\textcolor[rgb]{0.15,0.15,0.15}{{0.5}}}}
\fontsize{6}{0}
\selectfont\put(48.5205,78.5322){\makebox(0,0)[r]{\textcolor[rgb]{0.15,0.15,0.15}{{1}}}}
\fontsize{6}{0}
\selectfont\put(48.5205,99.7959){\makebox(0,0)[r]{\textcolor[rgb]{0.15,0.15,0.15}{{1.5}}}}
\fontsize{7}{0}
\selectfont\put(33.5205,67.1875){\rotatebox{90}{\makebox(0,0)[b]{\textcolor[rgb]{0.15,0.15,0.15}{{Sinal de Controle $u_1(t)$}}}}}
\fontsize{7}{0}
\selectfont\put(116.643,11.2671){\makebox(0,0)[t]{\textcolor[rgb]{0.15,0.15,0.15}{{Tempo (s)}}}}
\fontsize{6}{0}
\selectfont\put(232.715,22.2671){\makebox(0,0)[t]{\textcolor[rgb]{0.15,0.15,0.15}{{0}}}}
\fontsize{6}{0}
\selectfont\put(251.151,22.2671){\makebox(0,0)[t]{\textcolor[rgb]{0.15,0.15,0.15}{{20}}}}
\fontsize{6}{0}
\selectfont\put(269.587,22.2671){\makebox(0,0)[t]{\textcolor[rgb]{0.15,0.15,0.15}{{40}}}}
\fontsize{6}{0}
\selectfont\put(288.024,22.2671){\makebox(0,0)[t]{\textcolor[rgb]{0.15,0.15,0.15}{{60}}}}
\fontsize{6}{0}
\selectfont\put(306.46,22.2671){\makebox(0,0)[t]{\textcolor[rgb]{0.15,0.15,0.15}{{80}}}}
\fontsize{6}{0}
\selectfont\put(324.896,22.2671){\makebox(0,0)[t]{\textcolor[rgb]{0.15,0.15,0.15}{{100}}}}
\fontsize{6}{0}
\selectfont\put(343.333,22.2671){\makebox(0,0)[t]{\textcolor[rgb]{0.15,0.15,0.15}{{120}}}}
\fontsize{6}{0}
\selectfont\put(361.77,22.2671){\makebox(0,0)[t]{\textcolor[rgb]{0.15,0.15,0.15}{{140}}}}
\fontsize{6}{0}
\selectfont\put(229.235,28.3999){\makebox(0,0)[r]{\textcolor[rgb]{0.15,0.15,0.15}{{-0.2}}}}
\fontsize{6}{0}
\selectfont\put(229.235,43.6162){\makebox(0,0)[r]{\textcolor[rgb]{0.15,0.15,0.15}{{0}}}}
\fontsize{6}{0}
\selectfont\put(229.235,58.833){\makebox(0,0)[r]{\textcolor[rgb]{0.15,0.15,0.15}{{0.2}}}}
\fontsize{6}{0}
\selectfont\put(229.235,74.0493){\makebox(0,0)[r]{\textcolor[rgb]{0.15,0.15,0.15}{{0.4}}}}
\fontsize{6}{0}
\selectfont\put(229.235,89.2661){\makebox(0,0)[r]{\textcolor[rgb]{0.15,0.15,0.15}{{0.6}}}}
\fontsize{6}{0}
\selectfont\put(229.235,104.482){\makebox(0,0)[r]{\textcolor[rgb]{0.15,0.15,0.15}{{0.8}}}}
\fontsize{7}{0}
\selectfont\put(212.235,67.1875){\rotatebox{90}{\makebox(0,0)[b]{\textcolor[rgb]{0.15,0.15,0.15}{{Sinal de Controle $u_2(t)$}}}}}
\fontsize{7}{0}
\selectfont\put(297.357,11.2671){\makebox(0,0)[t]{\textcolor[rgb]{0.15,0.15,0.15}{{Tempo (s)}}}}
\fontsize{6}{0}
\selectfont\put(128.001,242.023){\makebox(0,0)[l]{\textcolor[rgb]{0,0,0}{{referencia}}}}
\fontsize{6}{0}
\selectfont\put(185.003,242.023){\makebox(0,0)[l]{\textcolor[rgb]{0,0,0}{{sem acao integral}}}}
\fontsize{6}{0}
\selectfont\put(262.005,242.023){\makebox(0,0)[l]{\textcolor[rgb]{0,0,0}{{com acao integral}}}}
\end{picture}

    \end{minipage}
\end{figure}

\subsubsection{Rejeição de perturbações e seguimento de referência com observador}
\label{subsub:controlador-utilizando-observador-de-estados}

Nos tópicos anteriores foi abordado o projeto de controladores em espaço de
estados considerando que todos os estados são mensurados. Entretanto, em
situações práticas de engenharia, ter todos os estados mensuráveis significa
maior custo com a aquisição de sensores, que a depender da aplicação é algo
indesejável. Além disso, em alguns sistemas, como os aeroespaciais, os estados
não são nem sequer mensuráveis diretamente por sensores. Desta forma, torna-se
necessário estimar os estados do sistema utilizando observadores de estados.
Portanto, foi realizado também simulações para o sistema \ref{eq:matriz-de-fts}
utilizando o observador de Luenberger.

Para o presente projeto foi considerado que a estimação dos estados em tempo
discreto é dada por

\begin{equation}
    \label{eq:predicao-do-estado-tempo-discreto}
    \mathbf{\hat{x}}[k+1] = \mathbf{\tilde{A}}\mathbf{\hat{x}}[k]
                            + \mathbf{\tilde{B}}\mathbf{u}[k]
                            + \mathbf{\tilde{L}}(\mathbf{y}[k] - \mathbf{\hat{y}[k]})
\end{equation} e então o sinal de controle é calculado com base no
vetor estimado, ou seja

\begin{equation}
    \label{eq:acao-de-conrole-com-estimador}
    \mathbf{u}[k] = -\mathbf{\tilde{K}}\mathbf{\hat{x}}[k]\text{.}
\end{equation} Considerando que existe um erro de estimação
$\mathbf{e}[k] = \mathbf{x}[k] - \mathbf{\hat{x}}[k]$ que após substituições das
equações \ref{eq:estimativa-do-vetor-de-estados},
\ref{eq:predicao-do-estado-tempo-discreto} e
\ref{eq:acao-de-conrole-com-estimador} resulta na equação a diferenças homogênea

\begin{equation}
    \label{eq:erro-de-estimacao}
    \mathbf{e}[k+1] = (\mathbf{\tilde{A}}-\mathbf{\tilde{L}}\mathbf{\tilde{C}})\mathbf{e}[k]
\end{equation} em que os autovalores da matriz
$\mathbf{\tilde{A}}-\mathbf{\tilde{L}}\mathbf{\tilde{C}}$ definem a velocidade
de convergência do erro de estimação, que neste caso dado a condição inicial
$\mathbf{e}[0] \neq \mathbf{0}$, a convergência é para erro nulo. Esta dinâmica
de conversão pode ser modificada escolhendo adequadamente a matriz
$\mathbf{\tilde{L}}$.

Portanto, para o controle do sistema em estudo foi dado como requisito que os
autovalores da matriz $\mathbf{\tilde{A}}-\mathbf{\tilde{L}}\mathbf{\tilde{C}}$
sejam um $\sigma_i = e^{\frac{-T_a}{2}}$ para $i = 1, 2, 3$ e $4$ com base no
período de amostragem $T_a = 0,75$s resulta em 4 autovalores $\sigma = 0,6873$.
Utilizando a propriedade que
$(\mathbf{\tilde{A}}-\mathbf{\tilde{L}}\mathbf{\tilde{C}})^\top =
\mathbf{\tilde{A}}^\top - \mathbf{\tilde{C}}^\top\mathbf{\tilde{L}}^\top$,
chega-se no mesmo problema da definição do ganho $\mathbf{\tilde{K}}$ exposto na
seção \ref{subsub:regulacao-em-torno-do-ponto-de-equilibrio}. Assim, utilizando
novamente a função \textit{place} do Octave sendo agora os argumentos
$\mathbf{\tilde{A}}^\top$, $\mathbf{\tilde{C}}^\top$ e $\mathbf{\sigma}$ foi
obtido a matriz de ganhos

\begin{equation}
    \label{eq:matriz-de-ganhos-do-estimador}
    \mathbf{\tilde{L}} =
    \begin{bmatrix}
        0.0000  & -0.3741   \\
        -0.3741 & 0.0000    \\
        -0.0000 & 0.4821    \\
        0.4821  & -0.0000
    \end{bmatrix}
\end{equation}

Aplicando a ação de controle \ref{eq:acao-de-conrole-com-estimador} considerando
também conforme \ref{eq:acao-de-controle-em-torno-do-ponto-de-equilibrio} que se
deseja levar os estados para uma condição de equilíbrio fora da origem, foi
obtido o resultado exibido na Figura
\ref{fig:erro-de-estimacao-de-estados-do-regulador-com-observador}.

\begin{figure}[!ht]
    \caption{Comparação da saída do sistema em malha fechada com ação de
    regulação para ponto de equilíbrio com e sem observador.}
    \hspace{-30pt}
    \label{fig:erro-de-estimacao-de-estados-do-regulador-com-observador}
    \begin{minipage}{\linewidth}
        % Title: gl2ps_renderer figure
% Creator: GL2PS 1.4.0, (C) 1999-2017 C. Geuzaine
% For: Octave
% CreationDate: Fri Nov 26 17:38:50 2021
\setlength{\unitlength}{1pt}
\begin{picture}(0,0)
\includegraphics{chapters/challenge6/images/resultado-1-questao-10-inc}
\end{picture}%
\begin{picture}(400,250)(0,0)
\fontsize{6}{0}
\selectfont\put(52,141.404){\makebox(0,0)[t]{\textcolor[rgb]{0.15,0.15,0.15}{{0}}}}
\fontsize{6}{0}
\selectfont\put(70.4365,141.404){\makebox(0,0)[t]{\textcolor[rgb]{0.15,0.15,0.15}{{20}}}}
\fontsize{6}{0}
\selectfont\put(88.873,141.404){\makebox(0,0)[t]{\textcolor[rgb]{0.15,0.15,0.15}{{40}}}}
\fontsize{6}{0}
\selectfont\put(107.309,141.404){\makebox(0,0)[t]{\textcolor[rgb]{0.15,0.15,0.15}{{60}}}}
\fontsize{6}{0}
\selectfont\put(125.746,141.404){\makebox(0,0)[t]{\textcolor[rgb]{0.15,0.15,0.15}{{80}}}}
\fontsize{6}{0}
\selectfont\put(144.182,141.404){\makebox(0,0)[t]{\textcolor[rgb]{0.15,0.15,0.15}{{100}}}}
\fontsize{6}{0}
\selectfont\put(162.619,141.404){\makebox(0,0)[t]{\textcolor[rgb]{0.15,0.15,0.15}{{120}}}}
\fontsize{6}{0}
\selectfont\put(181.055,141.404){\makebox(0,0)[t]{\textcolor[rgb]{0.15,0.15,0.15}{{140}}}}
\fontsize{6}{0}
\selectfont\put(48.5205,153.191){\makebox(0,0)[r]{\textcolor[rgb]{0.15,0.15,0.15}{{0}}}}
\fontsize{6}{0}
\selectfont\put(48.5205,169.575){\makebox(0,0)[r]{\textcolor[rgb]{0.15,0.15,0.15}{{0.5}}}}
\fontsize{6}{0}
\selectfont\put(48.5205,185.96){\makebox(0,0)[r]{\textcolor[rgb]{0.15,0.15,0.15}{{1}}}}
\fontsize{6}{0}
\selectfont\put(48.5205,202.344){\makebox(0,0)[r]{\textcolor[rgb]{0.15,0.15,0.15}{{1.5}}}}
\fontsize{6}{0}
\selectfont\put(48.5205,218.729){\makebox(0,0)[r]{\textcolor[rgb]{0.15,0.15,0.15}{{2}}}}
\fontsize{7}{0}
\selectfont\put(33.5205,186.325){\rotatebox{90}{\makebox(0,0)[b]{\textcolor[rgb]{0.15,0.15,0.15}{{Saida $y_1(t)$}}}}}
\fontsize{7}{0}
\selectfont\put(116.643,130.404){\makebox(0,0)[t]{\textcolor[rgb]{0.15,0.15,0.15}{{Tempo (s)}}}}
\fontsize{6}{0}
\selectfont\put(232.715,141.404){\makebox(0,0)[t]{\textcolor[rgb]{0.15,0.15,0.15}{{0}}}}
\fontsize{6}{0}
\selectfont\put(251.151,141.404){\makebox(0,0)[t]{\textcolor[rgb]{0.15,0.15,0.15}{{20}}}}
\fontsize{6}{0}
\selectfont\put(269.587,141.404){\makebox(0,0)[t]{\textcolor[rgb]{0.15,0.15,0.15}{{40}}}}
\fontsize{6}{0}
\selectfont\put(288.024,141.404){\makebox(0,0)[t]{\textcolor[rgb]{0.15,0.15,0.15}{{60}}}}
\fontsize{6}{0}
\selectfont\put(306.46,141.404){\makebox(0,0)[t]{\textcolor[rgb]{0.15,0.15,0.15}{{80}}}}
\fontsize{6}{0}
\selectfont\put(324.896,141.404){\makebox(0,0)[t]{\textcolor[rgb]{0.15,0.15,0.15}{{100}}}}
\fontsize{6}{0}
\selectfont\put(343.333,141.404){\makebox(0,0)[t]{\textcolor[rgb]{0.15,0.15,0.15}{{120}}}}
\fontsize{6}{0}
\selectfont\put(361.77,141.404){\makebox(0,0)[t]{\textcolor[rgb]{0.15,0.15,0.15}{{140}}}}
\fontsize{6}{0}
\selectfont\put(229.235,146.637){\makebox(0,0)[r]{\textcolor[rgb]{0.15,0.15,0.15}{{-0.2}}}}
\fontsize{6}{0}
\selectfont\put(229.235,157.632){\makebox(0,0)[r]{\textcolor[rgb]{0.15,0.15,0.15}{{0}}}}
\fontsize{6}{0}
\selectfont\put(229.235,168.627){\makebox(0,0)[r]{\textcolor[rgb]{0.15,0.15,0.15}{{0.2}}}}
\fontsize{6}{0}
\selectfont\put(229.235,179.623){\makebox(0,0)[r]{\textcolor[rgb]{0.15,0.15,0.15}{{0.4}}}}
\fontsize{6}{0}
\selectfont\put(229.235,190.618){\makebox(0,0)[r]{\textcolor[rgb]{0.15,0.15,0.15}{{0.6}}}}
\fontsize{6}{0}
\selectfont\put(229.235,201.613){\makebox(0,0)[r]{\textcolor[rgb]{0.15,0.15,0.15}{{0.8}}}}
\fontsize{6}{0}
\selectfont\put(229.235,212.608){\makebox(0,0)[r]{\textcolor[rgb]{0.15,0.15,0.15}{{1}}}}
\fontsize{6}{0}
\selectfont\put(229.235,223.604){\makebox(0,0)[r]{\textcolor[rgb]{0.15,0.15,0.15}{{1.2}}}}
\fontsize{7}{0}
\selectfont\put(212.235,186.325){\rotatebox{90}{\makebox(0,0)[b]{\textcolor[rgb]{0.15,0.15,0.15}{{Saida $y_2(t)$}}}}}
\fontsize{7}{0}
\selectfont\put(297.357,130.404){\makebox(0,0)[t]{\textcolor[rgb]{0.15,0.15,0.15}{{Tempo (s)}}}}
\fontsize{6}{0}
\selectfont\put(52,22.2671){\makebox(0,0)[t]{\textcolor[rgb]{0.15,0.15,0.15}{{0}}}}
\fontsize{6}{0}
\selectfont\put(70.4365,22.2671){\makebox(0,0)[t]{\textcolor[rgb]{0.15,0.15,0.15}{{20}}}}
\fontsize{6}{0}
\selectfont\put(88.873,22.2671){\makebox(0,0)[t]{\textcolor[rgb]{0.15,0.15,0.15}{{40}}}}
\fontsize{6}{0}
\selectfont\put(107.309,22.2671){\makebox(0,0)[t]{\textcolor[rgb]{0.15,0.15,0.15}{{60}}}}
\fontsize{6}{0}
\selectfont\put(125.746,22.2671){\makebox(0,0)[t]{\textcolor[rgb]{0.15,0.15,0.15}{{80}}}}
\fontsize{6}{0}
\selectfont\put(144.182,22.2671){\makebox(0,0)[t]{\textcolor[rgb]{0.15,0.15,0.15}{{100}}}}
\fontsize{6}{0}
\selectfont\put(162.619,22.2671){\makebox(0,0)[t]{\textcolor[rgb]{0.15,0.15,0.15}{{120}}}}
\fontsize{6}{0}
\selectfont\put(181.055,22.2671){\makebox(0,0)[t]{\textcolor[rgb]{0.15,0.15,0.15}{{140}}}}
\fontsize{6}{0}
\selectfont\put(48.5205,36.0044){\makebox(0,0)[r]{\textcolor[rgb]{0.15,0.15,0.15}{{0}}}}
\fontsize{6}{0}
\selectfont\put(48.5205,57.2651){\makebox(0,0)[r]{\textcolor[rgb]{0.15,0.15,0.15}{{0.5}}}}
\fontsize{6}{0}
\selectfont\put(48.5205,78.5259){\makebox(0,0)[r]{\textcolor[rgb]{0.15,0.15,0.15}{{1}}}}
\fontsize{6}{0}
\selectfont\put(48.5205,99.7866){\makebox(0,0)[r]{\textcolor[rgb]{0.15,0.15,0.15}{{1.5}}}}
\fontsize{7}{0}
\selectfont\put(33.5205,67.1875){\rotatebox{90}{\makebox(0,0)[b]{\textcolor[rgb]{0.15,0.15,0.15}{{Sinal de Controle $u_1(t)$}}}}}
\fontsize{7}{0}
\selectfont\put(116.643,11.2671){\makebox(0,0)[t]{\textcolor[rgb]{0.15,0.15,0.15}{{Tempo (s)}}}}
\fontsize{6}{0}
\selectfont\put(232.715,22.2671){\makebox(0,0)[t]{\textcolor[rgb]{0.15,0.15,0.15}{{0}}}}
\fontsize{6}{0}
\selectfont\put(251.151,22.2671){\makebox(0,0)[t]{\textcolor[rgb]{0.15,0.15,0.15}{{20}}}}
\fontsize{6}{0}
\selectfont\put(269.587,22.2671){\makebox(0,0)[t]{\textcolor[rgb]{0.15,0.15,0.15}{{40}}}}
\fontsize{6}{0}
\selectfont\put(288.024,22.2671){\makebox(0,0)[t]{\textcolor[rgb]{0.15,0.15,0.15}{{60}}}}
\fontsize{6}{0}
\selectfont\put(306.46,22.2671){\makebox(0,0)[t]{\textcolor[rgb]{0.15,0.15,0.15}{{80}}}}
\fontsize{6}{0}
\selectfont\put(324.896,22.2671){\makebox(0,0)[t]{\textcolor[rgb]{0.15,0.15,0.15}{{100}}}}
\fontsize{6}{0}
\selectfont\put(343.333,22.2671){\makebox(0,0)[t]{\textcolor[rgb]{0.15,0.15,0.15}{{120}}}}
\fontsize{6}{0}
\selectfont\put(361.77,22.2671){\makebox(0,0)[t]{\textcolor[rgb]{0.15,0.15,0.15}{{140}}}}
\fontsize{6}{0}
\selectfont\put(229.235,27.5){\makebox(0,0)[r]{\textcolor[rgb]{0.15,0.15,0.15}{{-0.2}}}}
\fontsize{6}{0}
\selectfont\put(229.235,42.8901){\makebox(0,0)[r]{\textcolor[rgb]{0.15,0.15,0.15}{{0}}}}
\fontsize{6}{0}
\selectfont\put(229.235,58.2803){\makebox(0,0)[r]{\textcolor[rgb]{0.15,0.15,0.15}{{0.2}}}}
\fontsize{6}{0}
\selectfont\put(229.235,73.6704){\makebox(0,0)[r]{\textcolor[rgb]{0.15,0.15,0.15}{{0.4}}}}
\fontsize{6}{0}
\selectfont\put(229.235,89.061){\makebox(0,0)[r]{\textcolor[rgb]{0.15,0.15,0.15}{{0.6}}}}
\fontsize{6}{0}
\selectfont\put(229.235,104.451){\makebox(0,0)[r]{\textcolor[rgb]{0.15,0.15,0.15}{{0.8}}}}
\fontsize{7}{0}
\selectfont\put(212.235,67.1875){\rotatebox{90}{\makebox(0,0)[b]{\textcolor[rgb]{0.15,0.15,0.15}{{Sinal de Controle $u_2(t)$}}}}}
\fontsize{7}{0}
\selectfont\put(297.357,11.2671){\makebox(0,0)[t]{\textcolor[rgb]{0.15,0.15,0.15}{{Tempo (s)}}}}
\fontsize{6}{0}
\selectfont\put(128.001,238.75){\makebox(0,0)[l]{\textcolor[rgb]{0,0,0}{{referencia}}}}
\fontsize{6}{0}
\selectfont\put(185.003,238.75){\makebox(0,0)[l]{\textcolor[rgb]{0,0,0}{{sem observador}}}}
\fontsize{6}{0}
\selectfont\put(258.005,238.75){\makebox(0,0)[l]{\textcolor[rgb]{0,0,0}{{com observador}}}}
\end{picture}

    \end{minipage}
\end{figure}

Na Figura \ref{fig:erro-de-estimacao-de-estados-do-regulador-com-observador} é
imperceptível a diferença das duas saídas do sistema fazendo a regulação com e
sem o observador de Luenberger. Entretanto, considerando que a saída estimada é
dada por $\mathbf{\hat{y}}(t) = \mathbf{C}\mathbf{\hat{x}}(t)$ foi possível
gerar o erro entre a saída real e a saída estimada, que está demonstrado na
Figura \ref{fig:erro-de-das-saidas-estimadas-utilizando-observador}. Percebe-se
então através desta figura que há uma pequena variação da saída estimada para a
saída real, porém, após aproximadamente 30s as saídas se igualam. Como trata-se
de uma simulação, embora considerado que os estados não são mensuráveis, a
título didático foi gerado também o erro da estimação dos estados, conforme
Figura \ref{fig:erro-de-estimacao-de-estados-com-observador}. De acordo com o
esperado, todos os erros convergem para zero após um período transitório que é
ditado pelos autovalores da matriz
$\mathbf{\tilde{A}}-\mathbf{\tilde{L}}\mathbf{\tilde{C}}$.

\begin{figure}[!ht]
    \caption{Erro de estimação da saída do sistema utilizando regulador com
    observador.}
    \vspace{-10pt}
    \hspace{-30pt}
    \label{fig:erro-de-das-saidas-estimadas-utilizando-observador}
    \begin{minipage}{\linewidth}
        % Title: gl2ps_renderer figure
% Creator: GL2PS 1.4.0, (C) 1999-2017 C. Geuzaine
% For: Octave
% CreationDate: Fri Nov 26 18:28:36 2021
\setlength{\unitlength}{1pt}
\begin{picture}(0,0)
\includegraphics{chapters/challenge6/images/resultado-3-questao-10-inc}
\end{picture}%
\begin{picture}(400,200)(0,0)
\fontsize{6}{0}
\selectfont\put(52,116.3){\makebox(0,0)[t]{\textcolor[rgb]{0.15,0.15,0.15}{{0}}}}
\fontsize{6}{0}
\selectfont\put(96.2065,116.3){\makebox(0,0)[t]{\textcolor[rgb]{0.15,0.15,0.15}{{20}}}}
\fontsize{6}{0}
\selectfont\put(140.414,116.3){\makebox(0,0)[t]{\textcolor[rgb]{0.15,0.15,0.15}{{40}}}}
\fontsize{6}{0}
\selectfont\put(184.62,116.3){\makebox(0,0)[t]{\textcolor[rgb]{0.15,0.15,0.15}{{60}}}}
\fontsize{6}{0}
\selectfont\put(228.827,116.3){\makebox(0,0)[t]{\textcolor[rgb]{0.15,0.15,0.15}{{80}}}}
\fontsize{6}{0}
\selectfont\put(273.034,116.3){\makebox(0,0)[t]{\textcolor[rgb]{0.15,0.15,0.15}{{100}}}}
\fontsize{6}{0}
\selectfont\put(317.241,116.3){\makebox(0,0)[t]{\textcolor[rgb]{0.15,0.15,0.15}{{120}}}}
\fontsize{6}{0}
\selectfont\put(361.447,116.3){\makebox(0,0)[t]{\textcolor[rgb]{0.15,0.15,0.15}{{140}}}}
\fontsize{6}{0}
\selectfont\put(48.5278,123.564){\makebox(0,0)[r]{\textcolor[rgb]{0.15,0.15,0.15}{{-0.0001}}}}
\fontsize{6}{0}
\selectfont\put(48.5278,136.554){\makebox(0,0)[r]{\textcolor[rgb]{0.15,0.15,0.15}{{0}}}}
\fontsize{6}{0}
\selectfont\put(48.5278,149.544){\makebox(0,0)[r]{\textcolor[rgb]{0.15,0.15,0.15}{{0.0001}}}}
\fontsize{6}{0}
\selectfont\put(48.5278,162.535){\makebox(0,0)[r]{\textcolor[rgb]{0.15,0.15,0.15}{{0.0002}}}}
\fontsize{6}{0}
\selectfont\put(48.5278,175.525){\makebox(0,0)[r]{\textcolor[rgb]{0.15,0.15,0.15}{{0.0003}}}}
\fontsize{7}{0}
\selectfont\put(19.5278,150.666){\rotatebox{90}{\makebox(0,0)[b]{\textcolor[rgb]{0.15,0.15,0.15}{{Saida $y_1(t) - \hat{y}_1(t)$}}}}}
\fontsize{7}{0}
\selectfont\put(207,105.3){\makebox(0,0)[t]{\textcolor[rgb]{0.15,0.15,0.15}{{Tempo (s)}}}}
\fontsize{6}{0}
\selectfont\put(52,20){\makebox(0,0)[t]{\textcolor[rgb]{0.15,0.15,0.15}{{0}}}}
\fontsize{6}{0}
\selectfont\put(96.2065,20){\makebox(0,0)[t]{\textcolor[rgb]{0.15,0.15,0.15}{{20}}}}
\fontsize{6}{0}
\selectfont\put(140.414,20){\makebox(0,0)[t]{\textcolor[rgb]{0.15,0.15,0.15}{{40}}}}
\fontsize{6}{0}
\selectfont\put(184.62,20){\makebox(0,0)[t]{\textcolor[rgb]{0.15,0.15,0.15}{{60}}}}
\fontsize{6}{0}
\selectfont\put(228.827,20){\makebox(0,0)[t]{\textcolor[rgb]{0.15,0.15,0.15}{{80}}}}
\fontsize{6}{0}
\selectfont\put(273.034,20){\makebox(0,0)[t]{\textcolor[rgb]{0.15,0.15,0.15}{{100}}}}
\fontsize{6}{0}
\selectfont\put(317.241,20){\makebox(0,0)[t]{\textcolor[rgb]{0.15,0.15,0.15}{{120}}}}
\fontsize{6}{0}
\selectfont\put(361.447,20){\makebox(0,0)[t]{\textcolor[rgb]{0.15,0.15,0.15}{{140}}}}
\fontsize{6}{0}
\selectfont\put(48.5278,34.3911){\makebox(0,0)[r]{\textcolor[rgb]{0.15,0.15,0.15}{{-0.0001}}}}
\fontsize{6}{0}
\selectfont\put(48.5278,45.4194){\makebox(0,0)[r]{\textcolor[rgb]{0.15,0.15,0.15}{{-5e-05}}}}
\fontsize{6}{0}
\selectfont\put(48.5278,56.4482){\makebox(0,0)[r]{\textcolor[rgb]{0.15,0.15,0.15}{{0}}}}
\fontsize{6}{0}
\selectfont\put(48.5278,67.4766){\makebox(0,0)[r]{\textcolor[rgb]{0.15,0.15,0.15}{{5e-05}}}}
\fontsize{6}{0}
\selectfont\put(48.5278,78.5049){\makebox(0,0)[r]{\textcolor[rgb]{0.15,0.15,0.15}{{0.0001}}}}
\fontsize{7}{0}
\selectfont\put(19.5278,54.3662){\rotatebox{90}{\makebox(0,0)[b]{\textcolor[rgb]{0.15,0.15,0.15}{{Saida $y_2(t) - \hat{y}_2(t)$}}}}}
\fontsize{7}{0}
\selectfont\put(207,9){\makebox(0,0)[t]{\textcolor[rgb]{0.15,0.15,0.15}{{Tempo (s)}}}}
\end{picture}

    \end{minipage}
\end{figure}

Foi realizada também a simulação considerando perturbações na entrada e saída da
planta. Como verificado na seção anterior, as perturbações só são rejeitadas com
a adição da ação integral no projeto do controlador em espaço de estados, o que
é condizente com o Princípio do Modelo Interno. Em virtude disso, para análise
do observador na presença de perturbações, foram realizadas apenas simulações
com projeto incluindo ação integral.

\begin{figure}[!ht]
    \caption{Erro de estimação dos estados do sistema em tempo discreto
    utilizando regulador com observador.}
    \vspace{-10pt}
    \hspace{-30pt}
    \label{fig:erro-de-estimacao-de-estados-com-observador}
    \begin{minipage}{\linewidth}
        % Title: gl2ps_renderer figure
% Creator: GL2PS 1.4.0, (C) 1999-2017 C. Geuzaine
% For: Octave
% CreationDate: Fri Nov 26 17:38:39 2021
\setlength{\unitlength}{1pt}
\begin{picture}(0,0)
\includegraphics{chapters/challenge6/images/resultado-2-questao-10-inc}
\end{picture}%
\begin{picture}(400,250)(0,0)
\fontsize{6}{0}
\selectfont\put(52,142.642){\makebox(0,0)[t]{\textcolor[rgb]{0.15,0.15,0.15}{{0}}}}
\fontsize{6}{0}
\selectfont\put(86.4922,142.642){\makebox(0,0)[t]{\textcolor[rgb]{0.15,0.15,0.15}{{50}}}}
\fontsize{6}{0}
\selectfont\put(120.984,142.642){\makebox(0,0)[t]{\textcolor[rgb]{0.15,0.15,0.15}{{100}}}}
\fontsize{6}{0}
\selectfont\put(155.476,142.642){\makebox(0,0)[t]{\textcolor[rgb]{0.15,0.15,0.15}{{150}}}}
\fontsize{6}{0}
\selectfont\put(48.5278,159.216){\makebox(0,0)[r]{\textcolor[rgb]{0.15,0.15,0.15}{{0}}}}
\fontsize{6}{0}
\selectfont\put(48.5278,178.559){\makebox(0,0)[r]{\textcolor[rgb]{0.15,0.15,0.15}{{0.002}}}}
\fontsize{6}{0}
\selectfont\put(48.5278,197.902){\makebox(0,0)[r]{\textcolor[rgb]{0.15,0.15,0.15}{{0.004}}}}
\fontsize{6}{0}
\selectfont\put(48.5278,217.246){\makebox(0,0)[r]{\textcolor[rgb]{0.15,0.15,0.15}{{0.006}}}}
\fontsize{7}{0}
\selectfont\put(25.5278,187.562){\rotatebox{90}{\makebox(0,0)[b]{\textcolor[rgb]{0.15,0.15,0.15}{{Erro $e_1[k]$}}}}}
\fontsize{7}{0}
\selectfont\put(116.5,131.642){\makebox(0,0)[t]{\textcolor[rgb]{0.15,0.15,0.15}{{Amostra [k]}}}}
\fontsize{6}{0}
\selectfont\put(233,142.642){\makebox(0,0)[t]{\textcolor[rgb]{0.15,0.15,0.15}{{0}}}}
\fontsize{6}{0}
\selectfont\put(267.492,142.642){\makebox(0,0)[t]{\textcolor[rgb]{0.15,0.15,0.15}{{50}}}}
\fontsize{6}{0}
\selectfont\put(301.984,142.642){\makebox(0,0)[t]{\textcolor[rgb]{0.15,0.15,0.15}{{100}}}}
\fontsize{6}{0}
\selectfont\put(336.476,142.642){\makebox(0,0)[t]{\textcolor[rgb]{0.15,0.15,0.15}{{150}}}}
\fontsize{6}{0}
\selectfont\put(229.528,156.911){\makebox(0,0)[r]{\textcolor[rgb]{0.15,0.15,0.15}{{0}}}}
\fontsize{6}{0}
\selectfont\put(229.528,166.968){\makebox(0,0)[r]{\textcolor[rgb]{0.15,0.15,0.15}{{0.002}}}}
\fontsize{6}{0}
\selectfont\put(229.528,177.024){\makebox(0,0)[r]{\textcolor[rgb]{0.15,0.15,0.15}{{0.004}}}}
\fontsize{6}{0}
\selectfont\put(229.528,187.081){\makebox(0,0)[r]{\textcolor[rgb]{0.15,0.15,0.15}{{0.006}}}}
\fontsize{6}{0}
\selectfont\put(229.528,197.137){\makebox(0,0)[r]{\textcolor[rgb]{0.15,0.15,0.15}{{0.008}}}}
\fontsize{6}{0}
\selectfont\put(229.528,207.194){\makebox(0,0)[r]{\textcolor[rgb]{0.15,0.15,0.15}{{0.01}}}}
\fontsize{6}{0}
\selectfont\put(229.528,217.25){\makebox(0,0)[r]{\textcolor[rgb]{0.15,0.15,0.15}{{0.012}}}}
\fontsize{7}{0}
\selectfont\put(206.528,187.562){\rotatebox{90}{\makebox(0,0)[b]{\textcolor[rgb]{0.15,0.15,0.15}{{Erro $e_2[k]$}}}}}
\fontsize{7}{0}
\selectfont\put(297.5,131.642){\makebox(0,0)[t]{\textcolor[rgb]{0.15,0.15,0.15}{{Amostra [k]}}}}
\fontsize{6}{0}
\selectfont\put(52,22.2671){\makebox(0,0)[t]{\textcolor[rgb]{0.15,0.15,0.15}{{0}}}}
\fontsize{6}{0}
\selectfont\put(86.4922,22.2671){\makebox(0,0)[t]{\textcolor[rgb]{0.15,0.15,0.15}{{50}}}}
\fontsize{6}{0}
\selectfont\put(120.984,22.2671){\makebox(0,0)[t]{\textcolor[rgb]{0.15,0.15,0.15}{{100}}}}
\fontsize{6}{0}
\selectfont\put(155.476,22.2671){\makebox(0,0)[t]{\textcolor[rgb]{0.15,0.15,0.15}{{150}}}}
\fontsize{6}{0}
\selectfont\put(48.5278,31.2358){\makebox(0,0)[r]{\textcolor[rgb]{0.15,0.15,0.15}{{-0.006}}}}
\fontsize{6}{0}
\selectfont\put(48.5278,51.8379){\makebox(0,0)[r]{\textcolor[rgb]{0.15,0.15,0.15}{{-0.004}}}}
\fontsize{6}{0}
\selectfont\put(48.5278,72.4399){\makebox(0,0)[r]{\textcolor[rgb]{0.15,0.15,0.15}{{-0.002}}}}
\fontsize{6}{0}
\selectfont\put(48.5278,93.0425){\makebox(0,0)[r]{\textcolor[rgb]{0.15,0.15,0.15}{{0}}}}
\fontsize{7}{0}
\selectfont\put(23.5278,67.1875){\rotatebox{90}{\makebox(0,0)[b]{\textcolor[rgb]{0.15,0.15,0.15}{{Erro $e_3[k]$}}}}}
\fontsize{7}{0}
\selectfont\put(116.5,11.2671){\makebox(0,0)[t]{\textcolor[rgb]{0.15,0.15,0.15}{{Amostra [k]}}}}
\fontsize{6}{0}
\selectfont\put(233,22.2671){\makebox(0,0)[t]{\textcolor[rgb]{0.15,0.15,0.15}{{0}}}}
\fontsize{6}{0}
\selectfont\put(267.492,22.2671){\makebox(0,0)[t]{\textcolor[rgb]{0.15,0.15,0.15}{{50}}}}
\fontsize{6}{0}
\selectfont\put(301.984,22.2671){\makebox(0,0)[t]{\textcolor[rgb]{0.15,0.15,0.15}{{100}}}}
\fontsize{6}{0}
\selectfont\put(336.476,22.2671){\makebox(0,0)[t]{\textcolor[rgb]{0.15,0.15,0.15}{{150}}}}
\fontsize{6}{0}
\selectfont\put(229.528,34.8433){\makebox(0,0)[r]{\textcolor[rgb]{0.15,0.15,0.15}{{-0.015}}}}
\fontsize{6}{0}
\selectfont\put(229.528,55.8301){\makebox(0,0)[r]{\textcolor[rgb]{0.15,0.15,0.15}{{-0.01}}}}
\fontsize{6}{0}
\selectfont\put(229.528,76.8169){\makebox(0,0)[r]{\textcolor[rgb]{0.15,0.15,0.15}{{-0.005}}}}
\fontsize{6}{0}
\selectfont\put(229.528,97.8037){\makebox(0,0)[r]{\textcolor[rgb]{0.15,0.15,0.15}{{0}}}}
\fontsize{7}{0}
\selectfont\put(204.528,67.1875){\rotatebox{90}{\makebox(0,0)[b]{\textcolor[rgb]{0.15,0.15,0.15}{{Erro $e_4[k]$}}}}}
\fontsize{7}{0}
\selectfont\put(297.5,11.2671){\makebox(0,0)[t]{\textcolor[rgb]{0.15,0.15,0.15}{{Amostra [k]}}}}
\end{picture}

    \end{minipage}
\end{figure}

Assim como na seção
\ref{subsub:rejeicao-de-perturbacoes-e-seguimento-de-referencia} é necessário
fazer a realização em espaço de estados de forma aumentada para a realizar a
estimação de estados pelo observador na presença da ação integral. Desta forma a
estimação de estados fica

\begin{equation}
    \label{eq:estimacao-de-estados-com-observador-e-acao-integral}
    \begin{bmatrix}
        \Delta \mathbf{\hat{x}}[k+1] \\
        \mathbf{\hat{y}}[k+1]
    \end{bmatrix}
    =
    \mathbf{\bar{A}}
    \begin{bmatrix}
        \Delta \mathbf{\hat{x}}[k] \\
        \mathbf{\hat{y}}[k]
    \end{bmatrix}
    +
    \mathbf{\bar{B}}
    \Delta \mathbf{u}[k]
    +
    \mathbf{\bar{L}}
    \begin{bmatrix}
        \Delta \mathbf{y}[k] - \mathbf{\tilde{C}} \Delta \mathbf{\hat{x}}[k] \\
        \mathbf{y}[k] - \mathbf{\hat{y}}[k]
    \end{bmatrix}
\end{equation} sendo que $\mathbf{\bar{A}}$ e $\mathbf{\bar{B}}$ são as mesmas
matrizes aumentadas definidas em \ref{eq:matriz-a-e-b-aumentada}.
Adicionalmente, dessa realização surge também a matriz aumentada
$\mathbf{\bar{C}}_{4 \times 6}$ dada por

\begin{equation}
    \label{eq:matriz-aumentada-c}
    \mathbf{\bar{C}}
    =
    \begin{bmatrix}
        \mathbf{\tilde{C}} & \mathbf{\bar{0}} \\
        \mathbf{\bar{0}}   & \mathbf{I}
    \end{bmatrix}\text{.}
\end{equation} Com essa nova realização do observador, a ação de controle se
torna

\begin{equation}
    \label{eq:acao-de-controle-com-obervador-e-integrador}
    \Delta \mathbf{u}[k] =
    - \mathbf{K_x} \Delta \mathbf{\hat{x}}[k]
    + \mathbf{K_i} \Delta \mathbf{\hat{z}}[k]
    \Rightarrow 
    \mathbf{u}[k] = \Delta \mathbf{u}[k] + \mathbf{u}[k-1]
    \text{.}
\end{equation}

A realização aumentada proporciona novamente que a convergência do erro de
estimação seja governada pelos autovalores da matriz $\mathbf{\bar{A}} -
\mathbf{\bar{L}}\mathbf{\bar{C}}$ sendo que a dimensão de $\mathbf{\bar{L}}$ é
agora $6 \times 4$ e, portanto, é necessário alocar 6 autovalores. Assim para o
caso do ganho $\mathbf{\bar{K}}$ (seção
\ref{subsub:rejeicao-de-perturbacoes-e-seguimento-de-referencia}) os autovalores
desejáveis permanecem iguais. Portanto, aplicando as matrizes
$\mathbf{\bar{A}}^\top$ e $\mathbf{\bar{C}}^\top$ e os 6 autovalores $\sigma =
0,6873$ foi obtida a matriz de ganhos

\begin{equation}
    \label{eq:matriz-l-aumentada}
    \mathbf{\bar{L}}
    =
    \begin{bmatrix}
        0       & -0,0235 & 0       & -0,0908   \\
        -0,0235 & 0       & -0,0908 & 0         \\
        0       & 0,0303  & 0       & 0,1171    \\
        0,0303  & 0       & 0,1171  & 0         \\
        0,0584  & 0       & 0,5380  & 0         \\
        0       & 0,0584  & 0       & 0,5380
    \end{bmatrix}
    \text{.}
\end{equation}

\begin{figure}[!ht]
    \caption{Comparação dos sinais de controle e saída do sistema em malha
    fechada para seguimento de referência com e sem observador.}
    \hspace{-30pt}
    \label{fig:resultado-do-regulador-com-observador-e-integrador}
    \begin{minipage}{\linewidth}
        % Title: gl2ps_renderer figure
% Creator: GL2PS 1.4.0, (C) 1999-2017 C. Geuzaine
% For: Octave
% CreationDate: Sat Nov 27 14:07:34 2021
\setlength{\unitlength}{1pt}
\begin{picture}(0,0)
\includegraphics{chapters/challenge6/images/resultado-4-questao-10-inc}
\end{picture}%
\begin{picture}(400,250)(0,0)
\fontsize{6}{0}
\selectfont\put(52,141.404){\makebox(0,0)[t]{\textcolor[rgb]{0.15,0.15,0.15}{{0}}}}
\fontsize{6}{0}
\selectfont\put(70.4365,141.404){\makebox(0,0)[t]{\textcolor[rgb]{0.15,0.15,0.15}{{20}}}}
\fontsize{6}{0}
\selectfont\put(88.873,141.404){\makebox(0,0)[t]{\textcolor[rgb]{0.15,0.15,0.15}{{40}}}}
\fontsize{6}{0}
\selectfont\put(107.309,141.404){\makebox(0,0)[t]{\textcolor[rgb]{0.15,0.15,0.15}{{60}}}}
\fontsize{6}{0}
\selectfont\put(125.746,141.404){\makebox(0,0)[t]{\textcolor[rgb]{0.15,0.15,0.15}{{80}}}}
\fontsize{6}{0}
\selectfont\put(144.182,141.404){\makebox(0,0)[t]{\textcolor[rgb]{0.15,0.15,0.15}{{100}}}}
\fontsize{6}{0}
\selectfont\put(162.619,141.404){\makebox(0,0)[t]{\textcolor[rgb]{0.15,0.15,0.15}{{120}}}}
\fontsize{6}{0}
\selectfont\put(181.055,141.404){\makebox(0,0)[t]{\textcolor[rgb]{0.15,0.15,0.15}{{140}}}}
\fontsize{6}{0}
\selectfont\put(48.5205,152.69){\makebox(0,0)[r]{\textcolor[rgb]{0.15,0.15,0.15}{{0}}}}
\fontsize{6}{0}
\selectfont\put(48.5205,167.824){\makebox(0,0)[r]{\textcolor[rgb]{0.15,0.15,0.15}{{0.5}}}}
\fontsize{6}{0}
\selectfont\put(48.5205,182.957){\makebox(0,0)[r]{\textcolor[rgb]{0.15,0.15,0.15}{{1}}}}
\fontsize{6}{0}
\selectfont\put(48.5205,198.09){\makebox(0,0)[r]{\textcolor[rgb]{0.15,0.15,0.15}{{1.5}}}}
\fontsize{6}{0}
\selectfont\put(48.5205,213.223){\makebox(0,0)[r]{\textcolor[rgb]{0.15,0.15,0.15}{{2}}}}
\fontsize{7}{0}
\selectfont\put(33.5205,186.325){\rotatebox{90}{\makebox(0,0)[b]{\textcolor[rgb]{0.15,0.15,0.15}{{Saida $y_1(t)$}}}}}
\fontsize{7}{0}
\selectfont\put(116.643,130.404){\makebox(0,0)[t]{\textcolor[rgb]{0.15,0.15,0.15}{{Tempo (s)}}}}
\fontsize{6}{0}
\selectfont\put(232.715,141.404){\makebox(0,0)[t]{\textcolor[rgb]{0.15,0.15,0.15}{{0}}}}
\fontsize{6}{0}
\selectfont\put(251.151,141.404){\makebox(0,0)[t]{\textcolor[rgb]{0.15,0.15,0.15}{{20}}}}
\fontsize{6}{0}
\selectfont\put(269.587,141.404){\makebox(0,0)[t]{\textcolor[rgb]{0.15,0.15,0.15}{{40}}}}
\fontsize{6}{0}
\selectfont\put(288.024,141.404){\makebox(0,0)[t]{\textcolor[rgb]{0.15,0.15,0.15}{{60}}}}
\fontsize{6}{0}
\selectfont\put(306.46,141.404){\makebox(0,0)[t]{\textcolor[rgb]{0.15,0.15,0.15}{{80}}}}
\fontsize{6}{0}
\selectfont\put(324.896,141.404){\makebox(0,0)[t]{\textcolor[rgb]{0.15,0.15,0.15}{{100}}}}
\fontsize{6}{0}
\selectfont\put(343.333,141.404){\makebox(0,0)[t]{\textcolor[rgb]{0.15,0.15,0.15}{{120}}}}
\fontsize{6}{0}
\selectfont\put(361.77,141.404){\makebox(0,0)[t]{\textcolor[rgb]{0.15,0.15,0.15}{{140}}}}
\fontsize{6}{0}
\selectfont\put(229.235,156.557){\makebox(0,0)[r]{\textcolor[rgb]{0.15,0.15,0.15}{{0}}}}
\fontsize{6}{0}
\selectfont\put(229.235,181.357){\makebox(0,0)[r]{\textcolor[rgb]{0.15,0.15,0.15}{{0.5}}}}
\fontsize{6}{0}
\selectfont\put(229.235,206.158){\makebox(0,0)[r]{\textcolor[rgb]{0.15,0.15,0.15}{{1}}}}
\fontsize{7}{0}
\selectfont\put(214.235,186.325){\rotatebox{90}{\makebox(0,0)[b]{\textcolor[rgb]{0.15,0.15,0.15}{{Saida $y_2(t)$}}}}}
\fontsize{7}{0}
\selectfont\put(297.357,130.404){\makebox(0,0)[t]{\textcolor[rgb]{0.15,0.15,0.15}{{Tempo (s)}}}}
\fontsize{6}{0}
\selectfont\put(52,22.2671){\makebox(0,0)[t]{\textcolor[rgb]{0.15,0.15,0.15}{{0}}}}
\fontsize{6}{0}
\selectfont\put(70.4365,22.2671){\makebox(0,0)[t]{\textcolor[rgb]{0.15,0.15,0.15}{{20}}}}
\fontsize{6}{0}
\selectfont\put(88.873,22.2671){\makebox(0,0)[t]{\textcolor[rgb]{0.15,0.15,0.15}{{40}}}}
\fontsize{6}{0}
\selectfont\put(107.309,22.2671){\makebox(0,0)[t]{\textcolor[rgb]{0.15,0.15,0.15}{{60}}}}
\fontsize{6}{0}
\selectfont\put(125.746,22.2671){\makebox(0,0)[t]{\textcolor[rgb]{0.15,0.15,0.15}{{80}}}}
\fontsize{6}{0}
\selectfont\put(144.182,22.2671){\makebox(0,0)[t]{\textcolor[rgb]{0.15,0.15,0.15}{{100}}}}
\fontsize{6}{0}
\selectfont\put(162.619,22.2671){\makebox(0,0)[t]{\textcolor[rgb]{0.15,0.15,0.15}{{120}}}}
\fontsize{6}{0}
\selectfont\put(181.055,22.2671){\makebox(0,0)[t]{\textcolor[rgb]{0.15,0.15,0.15}{{140}}}}
\fontsize{6}{0}
\selectfont\put(48.5205,27.5){\makebox(0,0)[r]{\textcolor[rgb]{0.15,0.15,0.15}{{-0.2}}}}
\fontsize{6}{0}
\selectfont\put(48.5205,38.2656){\makebox(0,0)[r]{\textcolor[rgb]{0.15,0.15,0.15}{{0}}}}
\fontsize{6}{0}
\selectfont\put(48.5205,49.0308){\makebox(0,0)[r]{\textcolor[rgb]{0.15,0.15,0.15}{{0.2}}}}
\fontsize{6}{0}
\selectfont\put(48.5205,59.7964){\makebox(0,0)[r]{\textcolor[rgb]{0.15,0.15,0.15}{{0.4}}}}
\fontsize{6}{0}
\selectfont\put(48.5205,70.562){\makebox(0,0)[r]{\textcolor[rgb]{0.15,0.15,0.15}{{0.6}}}}
\fontsize{6}{0}
\selectfont\put(48.5205,81.3276){\makebox(0,0)[r]{\textcolor[rgb]{0.15,0.15,0.15}{{0.8}}}}
\fontsize{6}{0}
\selectfont\put(48.5205,92.0928){\makebox(0,0)[r]{\textcolor[rgb]{0.15,0.15,0.15}{{1}}}}
\fontsize{6}{0}
\selectfont\put(48.5205,102.858){\makebox(0,0)[r]{\textcolor[rgb]{0.15,0.15,0.15}{{1.2}}}}
\fontsize{7}{0}
\selectfont\put(31.5205,67.1875){\rotatebox{90}{\makebox(0,0)[b]{\textcolor[rgb]{0.15,0.15,0.15}{{Sinal de Controle $u_1(t)$}}}}}
\fontsize{7}{0}
\selectfont\put(116.643,11.2671){\makebox(0,0)[t]{\textcolor[rgb]{0.15,0.15,0.15}{{Tempo (s)}}}}
\fontsize{6}{0}
\selectfont\put(232.715,22.2671){\makebox(0,0)[t]{\textcolor[rgb]{0.15,0.15,0.15}{{0}}}}
\fontsize{6}{0}
\selectfont\put(251.151,22.2671){\makebox(0,0)[t]{\textcolor[rgb]{0.15,0.15,0.15}{{20}}}}
\fontsize{6}{0}
\selectfont\put(269.587,22.2671){\makebox(0,0)[t]{\textcolor[rgb]{0.15,0.15,0.15}{{40}}}}
\fontsize{6}{0}
\selectfont\put(288.024,22.2671){\makebox(0,0)[t]{\textcolor[rgb]{0.15,0.15,0.15}{{60}}}}
\fontsize{6}{0}
\selectfont\put(306.46,22.2671){\makebox(0,0)[t]{\textcolor[rgb]{0.15,0.15,0.15}{{80}}}}
\fontsize{6}{0}
\selectfont\put(324.896,22.2671){\makebox(0,0)[t]{\textcolor[rgb]{0.15,0.15,0.15}{{100}}}}
\fontsize{6}{0}
\selectfont\put(343.333,22.2671){\makebox(0,0)[t]{\textcolor[rgb]{0.15,0.15,0.15}{{120}}}}
\fontsize{6}{0}
\selectfont\put(361.77,22.2671){\makebox(0,0)[t]{\textcolor[rgb]{0.15,0.15,0.15}{{140}}}}
\fontsize{6}{0}
\selectfont\put(229.235,27.5){\makebox(0,0)[r]{\textcolor[rgb]{0.15,0.15,0.15}{{-0.2}}}}
\fontsize{6}{0}
\selectfont\put(229.235,44.792){\makebox(0,0)[r]{\textcolor[rgb]{0.15,0.15,0.15}{{0}}}}
\fontsize{6}{0}
\selectfont\put(229.235,62.0845){\makebox(0,0)[r]{\textcolor[rgb]{0.15,0.15,0.15}{{0.2}}}}
\fontsize{6}{0}
\selectfont\put(229.235,79.3765){\makebox(0,0)[r]{\textcolor[rgb]{0.15,0.15,0.15}{{0.4}}}}
\fontsize{6}{0}
\selectfont\put(229.235,96.6689){\makebox(0,0)[r]{\textcolor[rgb]{0.15,0.15,0.15}{{0.6}}}}
\fontsize{7}{0}
\selectfont\put(212.235,67.1875){\rotatebox{90}{\makebox(0,0)[b]{\textcolor[rgb]{0.15,0.15,0.15}{{Sinal de Controle $u_2(t)$}}}}}
\fontsize{7}{0}
\selectfont\put(297.357,11.2671){\makebox(0,0)[t]{\textcolor[rgb]{0.15,0.15,0.15}{{Tempo (s)}}}}
\fontsize{6}{0}
\selectfont\put(128.001,238.75){\makebox(0,0)[l]{\textcolor[rgb]{0,0,0}{{referencia}}}}
\fontsize{6}{0}
\selectfont\put(185.003,238.75){\makebox(0,0)[l]{\textcolor[rgb]{0,0,0}{{sem observador}}}}
\fontsize{6}{0}
\selectfont\put(258.005,238.75){\makebox(0,0)[l]{\textcolor[rgb]{0,0,0}{{com observador}}}}
\end{picture}

    \end{minipage}
\end{figure}

Utilizando o observador regido pela realização
\ref{eq:estimacao-de-estados-com-observador-e-acao-integral} com a matriz
$\mathbf{\bar{L}}$ obtida e utilizando a lei de controle dada por
\ref{eq:acao-de-controle-com-obervador-e-integrador}, foi obtido o resultado
mostrado na Figura \ref{fig:resultado-do-regulador-com-observador-e-integrador}.
Observa-se que não há interferência do observador para seguimento de referência
sem a presença de perturbações. Porém, com o surgimento das perturbações
perturbações na saída $\mathbf{q}_{y}(t) = 0,2\mathbf{\mathds{1}}(t-50)$ e na
entrada $\mathbf{q}_{u}(t) = 0,2\mathbf{\mathds{1}}(t-100)$, a dinâmica imposta
pelos autovalores do observador influencia significativamente na saída do
sistema em malha fechada. Esta divergência das saídas é resultado da diferença
da ação de controle aplicada aos estados estimados pelo observador, como nota-se
também nos gráficos do sinal de controle, em que a ação é mais agressiva
utilizando observador.

Assim para o caso anterior, foi avaliado também o erro entre a saída real e a
saída estimada. O resultado é exibido na Figura
\ref{fig:erro-das-saidas-estimadas-utilizando-observador-e-integrador}.
Percebe-se que o transitório da divergência tem um tempo de acomodação
significativo de $\approx40$s.

\begin{figure}[!ht]
    \caption{Erro de estimação da saída do sistema utilizando regulador com
    observador e ação integral.}
    \vspace{-10pt}
    \hspace{-30pt}
    \label{fig:erro-das-saidas-estimadas-utilizando-observador-e-integrador}
    \begin{minipage}{\linewidth}
        % Title: gl2ps_renderer figure
% Creator: GL2PS 1.4.0, (C) 1999-2017 C. Geuzaine
% For: Octave
% CreationDate: Sat Nov 27 16:17:20 2021
\setlength{\unitlength}{1pt}
\begin{picture}(0,0)
\includegraphics{chapters/challenge6/images/resultado-5-questao-10-inc}
\end{picture}%
\begin{picture}(400,200)(0,0)
\fontsize{6}{0}
\selectfont\put(52,116.3){\makebox(0,0)[t]{\textcolor[rgb]{0.15,0.15,0.15}{{0}}}}
\fontsize{6}{0}
\selectfont\put(96.2065,116.3){\makebox(0,0)[t]{\textcolor[rgb]{0.15,0.15,0.15}{{20}}}}
\fontsize{6}{0}
\selectfont\put(140.414,116.3){\makebox(0,0)[t]{\textcolor[rgb]{0.15,0.15,0.15}{{40}}}}
\fontsize{6}{0}
\selectfont\put(184.62,116.3){\makebox(0,0)[t]{\textcolor[rgb]{0.15,0.15,0.15}{{60}}}}
\fontsize{6}{0}
\selectfont\put(228.827,116.3){\makebox(0,0)[t]{\textcolor[rgb]{0.15,0.15,0.15}{{80}}}}
\fontsize{6}{0}
\selectfont\put(273.034,116.3){\makebox(0,0)[t]{\textcolor[rgb]{0.15,0.15,0.15}{{100}}}}
\fontsize{6}{0}
\selectfont\put(317.241,116.3){\makebox(0,0)[t]{\textcolor[rgb]{0.15,0.15,0.15}{{120}}}}
\fontsize{6}{0}
\selectfont\put(361.447,116.3){\makebox(0,0)[t]{\textcolor[rgb]{0.15,0.15,0.15}{{140}}}}
\fontsize{6}{0}
\selectfont\put(48.5278,129.586){\makebox(0,0)[r]{\textcolor[rgb]{0.15,0.15,0.15}{{-0.1}}}}
\fontsize{6}{0}
\selectfont\put(48.5278,144.968){\makebox(0,0)[r]{\textcolor[rgb]{0.15,0.15,0.15}{{-0.05}}}}
\fontsize{6}{0}
\selectfont\put(48.5278,160.35){\makebox(0,0)[r]{\textcolor[rgb]{0.15,0.15,0.15}{{0}}}}
\fontsize{6}{0}
\selectfont\put(48.5278,175.732){\makebox(0,0)[r]{\textcolor[rgb]{0.15,0.15,0.15}{{0.05}}}}
\fontsize{7}{0}
\selectfont\put(27.5278,150.666){\rotatebox{90}{\makebox(0,0)[b]{\textcolor[rgb]{0.15,0.15,0.15}{{Saida $y_1(t) - \hat{y}_1(t)$}}}}}
\fontsize{7}{0}
\selectfont\put(207,105.3){\makebox(0,0)[t]{\textcolor[rgb]{0.15,0.15,0.15}{{Tempo (s)}}}}
\fontsize{6}{0}
\selectfont\put(52,20){\makebox(0,0)[t]{\textcolor[rgb]{0.15,0.15,0.15}{{0}}}}
\fontsize{6}{0}
\selectfont\put(96.2065,20){\makebox(0,0)[t]{\textcolor[rgb]{0.15,0.15,0.15}{{20}}}}
\fontsize{6}{0}
\selectfont\put(140.414,20){\makebox(0,0)[t]{\textcolor[rgb]{0.15,0.15,0.15}{{40}}}}
\fontsize{6}{0}
\selectfont\put(184.62,20){\makebox(0,0)[t]{\textcolor[rgb]{0.15,0.15,0.15}{{60}}}}
\fontsize{6}{0}
\selectfont\put(228.827,20){\makebox(0,0)[t]{\textcolor[rgb]{0.15,0.15,0.15}{{80}}}}
\fontsize{6}{0}
\selectfont\put(273.034,20){\makebox(0,0)[t]{\textcolor[rgb]{0.15,0.15,0.15}{{100}}}}
\fontsize{6}{0}
\selectfont\put(317.241,20){\makebox(0,0)[t]{\textcolor[rgb]{0.15,0.15,0.15}{{120}}}}
\fontsize{6}{0}
\selectfont\put(361.447,20){\makebox(0,0)[t]{\textcolor[rgb]{0.15,0.15,0.15}{{140}}}}
\fontsize{6}{0}
\selectfont\put(48.5278,30.6553){\makebox(0,0)[r]{\textcolor[rgb]{0.15,0.15,0.15}{{-0.08}}}}
\fontsize{6}{0}
\selectfont\put(48.5278,38.064){\makebox(0,0)[r]{\textcolor[rgb]{0.15,0.15,0.15}{{-0.06}}}}
\fontsize{6}{0}
\selectfont\put(48.5278,45.4722){\makebox(0,0)[r]{\textcolor[rgb]{0.15,0.15,0.15}{{-0.04}}}}
\fontsize{6}{0}
\selectfont\put(48.5278,52.8809){\makebox(0,0)[r]{\textcolor[rgb]{0.15,0.15,0.15}{{-0.02}}}}
\fontsize{6}{0}
\selectfont\put(48.5278,60.2891){\makebox(0,0)[r]{\textcolor[rgb]{0.15,0.15,0.15}{{0}}}}
\fontsize{6}{0}
\selectfont\put(48.5278,67.6978){\makebox(0,0)[r]{\textcolor[rgb]{0.15,0.15,0.15}{{0.02}}}}
\fontsize{6}{0}
\selectfont\put(48.5278,75.106){\makebox(0,0)[r]{\textcolor[rgb]{0.15,0.15,0.15}{{0.04}}}}
\fontsize{6}{0}
\selectfont\put(48.5278,82.5146){\makebox(0,0)[r]{\textcolor[rgb]{0.15,0.15,0.15}{{0.06}}}}
\fontsize{7}{0}
\selectfont\put(27.5278,54.3662){\rotatebox{90}{\makebox(0,0)[b]{\textcolor[rgb]{0.15,0.15,0.15}{{Saida $y_2(t) - \hat{y}_2(t)$}}}}}
\fontsize{7}{0}
\selectfont\put(207,9){\makebox(0,0)[t]{\textcolor[rgb]{0.15,0.15,0.15}{{Tempo (s)}}}}
\end{picture}

    \end{minipage}
\end{figure}

Todavia, como mostra a Figura
\ref{fig:erro-de-estimacao-de-estados-com-observador-e-integrador}, o erro de
estimação de estados não teve o mesmo comportamento do regulador por
realimentação de estados sem ação integral. Este comportamento é esperado pois
a subtração de $\Delta \mathbf{x}[k+1]$ em
\ref{eq:espaco-de-estados-discreto-aumentado} pela estimação do observador $\Delta
\mathbf{\hat{x}}[k+1]$ em
\ref{eq:estimacao-de-estados-com-observador-e-acao-integral} resulta no erro de
estimação de estados

\begin{equation}
    \label{eq:erro-da-estimacao-de-estados-com-observador-e-integrador}
    \Delta \mathbf{e}[k+1] =
    (\mathbf{\bar{A}} - \mathbf{\bar{L}}\mathbf{\bar{C}}) \Delta \mathbf{e}[k]
    \text{.}
\end{equation} Como nota-se, não mais $\mathbf{e}[k] \rightarrow \mathbf{0}$
mas sim $\Delta \mathbf{e}[k] \rightarrow \mathbf{0}$, em que novamente a
dinâmica desta convergência é ditada pelos autovalores de $\mathbf{\bar{A}} -
\mathbf{\bar{L}}\mathbf{\bar{C}}$. Portanto dizer $\Delta \mathbf{e}[k]
\rightarrow \mathbf{0}$ é a mesmo coisa de dizer que $\mathbf{e}[k]$ tende a
zero ou para um valor constante, o que acontece na Figura
\ref{fig:erro-de-estimacao-de-estados-com-observador-e-integrador}. Este erro
não nulo na presença de perturbações pode ser explicada pelo conceito de
observabilidade. Como não há controle das perturbações, o observador não
consegue estimar as perturbações o que acarreta no erro de estimação de estados
não nulo. Entretanto, devido ao integrador a saída converge para o valor de
referência.

\begin{figure}[!ht]
    \caption{Erro de estimação dos estados do sistema em tempo discreto
    utilizando regulador com observador e integrador.}
    \vspace{-10pt}
    \hspace{-30pt}
    \label{fig:erro-de-estimacao-de-estados-com-observador-e-integrador}
    \begin{minipage}{\linewidth}
        % Title: gl2ps_renderer figure
% Creator: GL2PS 1.4.0, (C) 1999-2017 C. Geuzaine
% For: Octave
% CreationDate: Sat Nov 27 16:17:20 2021
\setlength{\unitlength}{1pt}
\begin{picture}(0,0)
\includegraphics{chapters/challenge6/images/resultado-5-questao-10-inc}
\end{picture}%
\begin{picture}(400,200)(0,0)
\fontsize{6}{0}
\selectfont\put(52,116.3){\makebox(0,0)[t]{\textcolor[rgb]{0.15,0.15,0.15}{{0}}}}
\fontsize{6}{0}
\selectfont\put(96.2065,116.3){\makebox(0,0)[t]{\textcolor[rgb]{0.15,0.15,0.15}{{20}}}}
\fontsize{6}{0}
\selectfont\put(140.414,116.3){\makebox(0,0)[t]{\textcolor[rgb]{0.15,0.15,0.15}{{40}}}}
\fontsize{6}{0}
\selectfont\put(184.62,116.3){\makebox(0,0)[t]{\textcolor[rgb]{0.15,0.15,0.15}{{60}}}}
\fontsize{6}{0}
\selectfont\put(228.827,116.3){\makebox(0,0)[t]{\textcolor[rgb]{0.15,0.15,0.15}{{80}}}}
\fontsize{6}{0}
\selectfont\put(273.034,116.3){\makebox(0,0)[t]{\textcolor[rgb]{0.15,0.15,0.15}{{100}}}}
\fontsize{6}{0}
\selectfont\put(317.241,116.3){\makebox(0,0)[t]{\textcolor[rgb]{0.15,0.15,0.15}{{120}}}}
\fontsize{6}{0}
\selectfont\put(361.447,116.3){\makebox(0,0)[t]{\textcolor[rgb]{0.15,0.15,0.15}{{140}}}}
\fontsize{6}{0}
\selectfont\put(48.5278,129.586){\makebox(0,0)[r]{\textcolor[rgb]{0.15,0.15,0.15}{{-0.1}}}}
\fontsize{6}{0}
\selectfont\put(48.5278,144.968){\makebox(0,0)[r]{\textcolor[rgb]{0.15,0.15,0.15}{{-0.05}}}}
\fontsize{6}{0}
\selectfont\put(48.5278,160.35){\makebox(0,0)[r]{\textcolor[rgb]{0.15,0.15,0.15}{{0}}}}
\fontsize{6}{0}
\selectfont\put(48.5278,175.732){\makebox(0,0)[r]{\textcolor[rgb]{0.15,0.15,0.15}{{0.05}}}}
\fontsize{7}{0}
\selectfont\put(27.5278,150.666){\rotatebox{90}{\makebox(0,0)[b]{\textcolor[rgb]{0.15,0.15,0.15}{{Saida $y_1(t) - \hat{y}_1(t)$}}}}}
\fontsize{7}{0}
\selectfont\put(207,105.3){\makebox(0,0)[t]{\textcolor[rgb]{0.15,0.15,0.15}{{Tempo (s)}}}}
\fontsize{6}{0}
\selectfont\put(52,20){\makebox(0,0)[t]{\textcolor[rgb]{0.15,0.15,0.15}{{0}}}}
\fontsize{6}{0}
\selectfont\put(96.2065,20){\makebox(0,0)[t]{\textcolor[rgb]{0.15,0.15,0.15}{{20}}}}
\fontsize{6}{0}
\selectfont\put(140.414,20){\makebox(0,0)[t]{\textcolor[rgb]{0.15,0.15,0.15}{{40}}}}
\fontsize{6}{0}
\selectfont\put(184.62,20){\makebox(0,0)[t]{\textcolor[rgb]{0.15,0.15,0.15}{{60}}}}
\fontsize{6}{0}
\selectfont\put(228.827,20){\makebox(0,0)[t]{\textcolor[rgb]{0.15,0.15,0.15}{{80}}}}
\fontsize{6}{0}
\selectfont\put(273.034,20){\makebox(0,0)[t]{\textcolor[rgb]{0.15,0.15,0.15}{{100}}}}
\fontsize{6}{0}
\selectfont\put(317.241,20){\makebox(0,0)[t]{\textcolor[rgb]{0.15,0.15,0.15}{{120}}}}
\fontsize{6}{0}
\selectfont\put(361.447,20){\makebox(0,0)[t]{\textcolor[rgb]{0.15,0.15,0.15}{{140}}}}
\fontsize{6}{0}
\selectfont\put(48.5278,30.6553){\makebox(0,0)[r]{\textcolor[rgb]{0.15,0.15,0.15}{{-0.08}}}}
\fontsize{6}{0}
\selectfont\put(48.5278,38.064){\makebox(0,0)[r]{\textcolor[rgb]{0.15,0.15,0.15}{{-0.06}}}}
\fontsize{6}{0}
\selectfont\put(48.5278,45.4722){\makebox(0,0)[r]{\textcolor[rgb]{0.15,0.15,0.15}{{-0.04}}}}
\fontsize{6}{0}
\selectfont\put(48.5278,52.8809){\makebox(0,0)[r]{\textcolor[rgb]{0.15,0.15,0.15}{{-0.02}}}}
\fontsize{6}{0}
\selectfont\put(48.5278,60.2891){\makebox(0,0)[r]{\textcolor[rgb]{0.15,0.15,0.15}{{0}}}}
\fontsize{6}{0}
\selectfont\put(48.5278,67.6978){\makebox(0,0)[r]{\textcolor[rgb]{0.15,0.15,0.15}{{0.02}}}}
\fontsize{6}{0}
\selectfont\put(48.5278,75.106){\makebox(0,0)[r]{\textcolor[rgb]{0.15,0.15,0.15}{{0.04}}}}
\fontsize{6}{0}
\selectfont\put(48.5278,82.5146){\makebox(0,0)[r]{\textcolor[rgb]{0.15,0.15,0.15}{{0.06}}}}
\fontsize{7}{0}
\selectfont\put(27.5278,54.3662){\rotatebox{90}{\makebox(0,0)[b]{\textcolor[rgb]{0.15,0.15,0.15}{{Saida $y_2(t) - \hat{y}_2(t)$}}}}}
\fontsize{7}{0}
\selectfont\put(207,9){\makebox(0,0)[t]{\textcolor[rgb]{0.15,0.15,0.15}{{Tempo (s)}}}}
\end{picture}

    \end{minipage}
    \vspace{-10pt}
\end{figure}

\subsection{Conclusões}

Os resultados obtidos e apresentados na seção anterior ratificam a importância
do uso da realização em espaço de estados para o controle de processos
multivariáveis. Foi notório a facilidade para projetar um controlador para
seguimento de referência e rejeição de perturbações cujo os sinais de controle
são aplicados simultaneamente a dois processos que além de terem suas dinâmicas
internas, sofrem ainda com o acoplamento da dinâmica devido a interconexão dos
tanques. Observa-se ainda como é possível alterar a dinâmica do processo através
apenas de análises e manipulações algébricas, as quais estão fortemente
consolidadas na teoria e aplicação da ciência exatas. Em suma, os resultados
obtidos comprovaram a importância do conhecimento da realização de espaço de
estados na resolução de problemas complexo de controle, principalmente controle
multivariável que é o exemplo dos tanques interconectados em questão.

\newcommand{\mat}[1]{\MakeUppercase{\mathbf{#1}}}
\newcommand{\ssvec}[1]{\MakeLowercase{\mathbf{#1}}}
\newcommand{\ssveceq}[1]{\MakeLowercase{\mathbf{\bar{#1}}}}

\section{Desafio VII - Controle de Sistema Não-Linear} 

\subsection{Motivação}
(Explicar o problema que motiva o desafio, relevância, possíveis aplicações...) 

\subsection{Simulações realizadas}

\begin{subequations}
    \label{eq:modelo-do-oscilador-de-van-der-pol}
    \begin{equation}
        \begin{bmatrix}
            \dot{x}_1(t) \\
            \dot{x}_2(t)
        \end{bmatrix}
        =
        \begin{bmatrix}
            x_2(t) \\
            -x_1(t) + 0,3(1 - x_1(t)^2)x_2(t) + u(t)
        \end{bmatrix}
    \end{equation}
    \begin{equation}
        y(t) = x_1(t)
    \end{equation}
\end{subequations}

Por simplicidade na representação do Modelo de Van der Pol, o argumento de tempo
será omitido ao longo desta seção, salve necessidade de explicitar o argumento
de tempo para melhor compreensão.

(Explicar quais simulações foram realizadas de maneira descritiva e sequêncial.) 

\subsection{Resultados obtidos}

Como o modelo do oscilador de Van der Pol dado em espaço de estados pela Equação
\ref{eq:modelo-do-oscilador-de-van-der-pol} é não linear, torna-se necessário
fazer a linearização em torno de um ponto de equilíbrio. Considerando $\ssvec{z}
= \ssvec{x} - \ssveceq{x}$, $v = u - \bar{u}$ e $w = y - h(\ssveceq{x},
\bar{u})$ como novas variáveis do espaço de e reescrevendo
\ref{eq:modelo-do-oscilador-de-van-der-pol} como 

\begin{subequations}
    \label{eq:modelo-reescrito-do-oscilador-de-van-der-pol}
    \begin{equation}
        \label{eq:mapa-vetorial-dos-estados}
        \dot{\ssvec{x}} = f(\ssvec{x}, u)
        =
        \begin{bmatrix}
            \dot{x}_1 \\
            \dot{x}_2
        \end{bmatrix}
        =
        \begin{bmatrix}
            f_1(\ssvec{x}, u) \\
            f_2(\ssvec{x}, u) \\
        \end{bmatrix}
        =
        \begin{bmatrix}
            x_2 \\
            -x_1 + 0,3(1 - x_1^2)x_2 + u
        \end{bmatrix}
    \end{equation}
    \begin{equation}
        y = h(\ssvec{x}, u) = x_1
    \end{equation}
\end{subequations} o modelo não linear do Oscilador de Van der Pol pode ser
convertido para o modelo linear em torno do ponto de equilíbrio através de uma
linearização jacobiana em que

\begin{subequations}
    \label{eq:modelo-linearizado-em-torno-do-equilibrio}
    \begin{equation}
        \label{eq:derivada-dos-estados-linearizados}
        \dot{\ssvec{z}} = \mat{a}\ssvec{z} + \mathbf{B}v
    \end{equation}
    \begin{equation}
        w = \mat{C}\ssvec{z} + \mat{D}v
    \end{equation}
\end{subequations} sendo as matrizes definidas por

\begin{subequations}
    \label{eq:derivadas-parcias-do-vetor-estados}
    \begin{equation}
        \label{eq:jacobiana-de-a}  
        \mat{A} =
        \left.
            \begin{matrix}
                \frac{\partial f}{\partial x}
            \end{matrix}
            \right|_{(\bar{x}, \thinspace \bar{u})}
        =
        \begin{bmatrix}
            \left.
                \begin{matrix}
                    \frac{\partial f_1}{\partial x_1}
                \end{matrix}
            \right|_{(\bar{x}, \thinspace \bar{u})}
            &
            \left.
                \begin{matrix}
                    \frac{\partial f_1}{\partial x_2}
                \end{matrix}
            \right|_{(\bar{x}, \thinspace \bar{u})}
            \\
            \left.
                \begin{matrix}
                    \frac{\partial f_2}{\partial x_1}
                \end{matrix}
            \right|_{(\bar{x}, \thinspace \bar{u})}
            &
            \left.
                \begin{matrix}
                    \frac{\partial f_2}{\partial x_2}
                \end{matrix}
            \right|_{(\bar{x}, \thinspace \bar{u})}
        \end{bmatrix}
        =
        \begin{bmatrix}
            0 & 1 \\
            -1 & 0,3(1-\bar{x}_1^2)
        \end{bmatrix}
    \end{equation}

    \begin{equation}
        \label{eq:jacobiano-de-b}
        \mat{B} =
        \left.
            \begin{matrix}
                \frac{\partial f}{\partial u}
            \end{matrix}
        \right|_{(\bar{x}, \thinspace \bar{u})}
        =
        \begin{bmatrix}
            \left.
                \begin{matrix}
                    \frac{\partial f_1}{\partial u}
                \end{matrix}
            \right|_{(\bar{x}, \thinspace \bar{u})}
            \\
            \left.
                \begin{matrix}
                    \frac{\partial f_2}{\partial u}
                \end{matrix}
            \right|_{(\bar{x}, \thinspace \bar{u})}
        \end{bmatrix}
        =
        \begin{bmatrix}
            0 \\
            1
        \end{bmatrix}
    \end{equation}

    \begin{equation}
        \label{eq:jacoabiana-de-c}
        \mat{C} =
        \left.
            \begin{matrix}
                \frac{\partial h}{\partial x}
            \end{matrix}
        \right|_{(\bar{x}, \thinspace \bar{u})}
        =
        \begin{bmatrix}
            \left.
                \begin{matrix}
                    \frac{\partial h}{\partial x_1}
                \end{matrix}
            \right|_{(\bar{x}, \thinspace \bar{u})}
            &
            \left.
                \begin{matrix}
                    \frac{\partial h}{\partial x_2}
                \end{matrix}
            \right|_{(\bar{x}, \thinspace \bar{u})}
        \end{bmatrix}
        =
        \begin{bmatrix}
            1 & 0
        \end{bmatrix}
    \end{equation}

    \begin{equation}
        \mat{D} =
        \left.
            \begin{matrix}
                \frac{\partial h}{\partial u}
            \end{matrix}
        \right|_{(\bar{x}, \thinspace \bar{u})}
        = 0
    \end{equation}
\end{subequations}

Como o ponto de equilíbrio é definido por $[\dot{x}_1 \thickspace
\dot{x}_2]^\top = [0 \thickspace 0]^\top$, de
\ref{eq:modelo-do-oscilador-de-van-der-pol} tem-se que $\bar{x}_2 = 0$ e
$\bar{x_1} = \bar{u}$. Assim, considerando dois pontos de equilíbrio $\bar{x}_1
= 1$ e $\bar{x}_1 = 4$, as matrizes de dinâmica do sistema $\mat{A}_1$ e
$\mat{A}_2$, respectivamente, são definidas por

\begin{subequations}
    \label{eq:matrizes-a-linearizadas}
    \begin{equation}
        \label{eq:matriz-a1}
        \mat{A}_1
        =
        \begin{bmatrix}
            0   & 1 \\
            -1  & 0
        \end{bmatrix}
    \end{equation}

    \begin{equation}
        \label{eq:matriz-a2}
        \mat{A}_2
        =
        \begin{bmatrix}
            0   & 1 \\
            -1  & -4,5
        \end{bmatrix}
    \end{equation}
\end{subequations} cujo os autovalores são respectivamente $\lambda_1 = [i
\thickspace -i]^\top$ e $\lambda_2 = [-0,2344 \thickspace -4,2656]^\top$.
Nota-se portanto que para o ponto de equilíbrio $[1 \thickspace 0]^\top$ o
modelo linearizado possui um par conjugado de autovalores. Isto permite concluir
que partindo de um ponto nas proximidades deste ponto de equilíbrio ou com
pequenas variações do sinal de controle em torno de $\bar{u} = 1$, fará tanto os
estados quanto a saída oscilarem. Esta oscilação no diagrama de fases é
representada por uma trajetória circular em torno do ponto de equilíbrio,
chamada de círculo limite, como pode ser observado no diagrama de fases (Figura
\ref{fig:diagrama-de-fases-para-circulo-limite}) obtido no software Pplane
\textbf{CITAR!}. Por outro lado, para o ponto de equilíbrio $[4 \thickspace
0]^\top$ nas duas situações, o sistema converge para o ponto de equilíbrio.

\begin{figure}[h]
	\centering
	\caption{Diagrama de fases com trajetória circular (círculo limite) nas
    mediações do ponto de equilíbrio $[1 \thickspace 0]^\top$.}
	\label{fig:diagrama-de-fases-para-circulo-limite}
	\includegraphics[width=\textwidth]{chapters/challenge7/images/diagrama-de-fases-com-circulo-limite.png}
\end{figure}

Com os modelos linearizados foi obtido o vetor linha de ganhos $\mat{K}$ tal que
a ação de controle $v = \mat{k}\ssvec{z}(t)$ minimiza o custo do controlador
LQR $J$ dado por

\begin{equation}
    \label{eq:custo-do-controlador-lqr}
    J = \int_{t=0}^{\infty }\ssvec{z}(t)^\top \mat{Q} \ssvec{z}(t) + v(t)^\top R v(t)
\end{equation} com $\mat{Q} = \mat{I}$ e $R = 0,1$. Com isso, através da função
\textit{lqr} do Octave resultou ganhos $\mat{K}_1$ e $\mat{K}_2$, conforme
Equação \ref{eq:ganhos-do-controlador-lqr}, relacionados respectivamente as
matrizes $\mat{A}_1$ e $\mat{A}_2$ (Equação \ref{eq:matrizes-a-linearizadas}).

\begin{subequations}
    \label{eq:ganhos-do-controlador-lqr}
    \begin{equation}
        \mat{K}_1 = [2,3166 \thickspace 3,8253]
    \end{equation}
    \begin{equation}
        \mat{K}_2 = [2,3166 \thickspace 1,4062]
    \end{equation}
\end{subequations}

Com isso, definindo $\mat{k} = [k_1 \thickspace k_2]$ e retornando a ação de
controle para as variáveis de estados originais, isto é

\begin{equation}
    \label{eq:acao-de-controle-do-sistema-linear}
    v = -\mat{k}\ssvec{z}
    \Rightarrow
    u = -\mat{k}(\ssvec{x} - \ssveceq{x}) + \bar{u}
    = -k_1(x_1 - \bar{x}_1) - k_2(x_2 - \bar{x}_2) + \bar{u}
\end{equation} o modelo do Oscilador de Van Der Pol pode ser reescrito como

\begin{equation}
    \label{eq:modelo-reescrito2-do-oscilador-de-van-der-pol}
    \begin{bmatrix}
        \dot{x}_1 \\
        \dot{x}_2
    \end{bmatrix}
    =
    \begin{bmatrix}
        x_2 \\
        -x_1 + 0,3(1 - x_1^2)x_2 - k_1(x_1 - \bar{x}_1) - k_2(x_2 - \bar{x}_2) + \bar{u}
    \end{bmatrix}.
\end{equation}

Para análise do comportamento do sistema, foram gerados, com auxílio do software
Pplane com o parâmetro \textit{Solution Direction} igual \textit{Forward},
diagramas de fase com o considerando três cenários:

\begin{itemize}
    \item i) Modelo do oscilador com $u(t) = 0$;
    \item ii) Modelo do oscilador utilizando o ganho LQR $\mat{K}_1$ obtido com
    a linearização em torno do ponto de equilíbrio $[1 \thickspace 0]^\top$; e
    \item iii) Modelo do oscilador utilizando o ganho LQR $\mat{K}_2$ obtido com
    a linearização em torno do ponto de equilíbrio $[4 \thickspace 0]^\top$.
\end{itemize}

O diagrama de fases do cenário (i) é exibido na Figura
\ref{fig:diagrama-de-fases-sem-controle}. Percebe-se na Figura
\ref{fig:diagrama-de-fases-sem-controle-ponto-de-equilibrio-1-e-2} que definindo
a condição inicial aproximadamente os pontos de equilíbrios \peq1

\begin{figure}[ht]
    \caption{Diagrama de fases do modelo do oscilador com $u(t) = 0$.}
    \label{fig:diagrama-de-fases-sem-controle}
    \centering
    \begin{subfigure}[t]{0.32\textwidth}
        \centering
	    \includegraphics[width=\textwidth]{chapters/challenge7/images/diagrama-de-fases-sem-controle.png}
        \caption{Partindo de condições iniciais aproximadamente iguais aos pontos de equilíbrio.}
        \label{fig:diagrama-de-fases-sem-controle-ponto-de-equilibrio-1-e-2}
        
    \end{subfigure}
    \hfill
    \begin{subfigure}[t]{0.32\textwidth}
        \centering
	    \includegraphics[width=\textwidth]{chapters/challenge7/images/diagrama-de-fases-sem-controle-zoom1.png}
        \caption{Partindo de condições iniciais na vizinhança do ponto de equilíbrio 1.}
        \label{fig:diagrama-de-fases-sem-controle-ponto-de-equilibrio-1}
    \end{subfigure}
    \hfill
    \begin{subfigure}[t]{0.32\textwidth}
        \centering
	    \includegraphics[width=\textwidth]{chapters/challenge7/images/diagrama-de-fases-sem-controle-zoom2.png}
        \caption{Partindo de condições iniciais na vizinhança do ponto de equilíbrio 2.}
        \label{fig:diagrama-de-fases-sem-controle-ponto-de-equilibrio-2}
    \end{subfigure}
\end{figure}

\subsection{Conclusões}
(Concluir em que medida os resultados apresentam relação com a motivação.
Permitem ilustratar ou concluir algo sobre a motivação? )


\addcontentsline{toc}{section}{Referências}
\bibliographystyle{ieeetr}
\bibliography{references.bib}

\end{document}
