\documentclass[a4paper,10pt]{article}

\usepackage[brazilian]{babel}
\usepackage[utf8]{inputenc}
\usepackage[T1]{fontenc}
\usepackage{psfrag}
\usepackage{times}
\usepackage{indentfirst}
\usepackage{amsmath,amsfonts,amssymb}
\usepackage{graphicx}
\usepackage{caption}
\usepackage{dsfont}
\usepackage{xcolor}
\usepackage{tipa}
\usepackage{tipx}
\usepackage{amsmath}
\usepackage{tabularx}
\usepackage{cancel}
\usepackage{float}
\usepackage{hyperref}

% Extracted from:
% https://tex.stackexchange.com/questions/95838/how-to-write-a-perfect-equation-parameters-description
\newenvironment{conditions*}
    {\par\vspace{\abovedisplayskip}\noindent
    \tabularx{\columnwidth}{>{$}l<{$} @{${}$ é ${}$} >{\raggedright\arraybackslash}X}}
    {\endtabularx\par\vspace{\belowdisplayskip}}

\title{\textbf{Análise e Projeto de Sistemas de Controle} \\
\vspace{0.5cm}
\underline{Relatório I} \\
\vspace{4.5cm}
\includegraphics[width=4.0cm]{images/brasao_ufba.jpg}
\vspace{4.5cm} }
\author{Mateus dos Santos de Meneses}
\date{Outubro, 2021}

\begin{document}
\maketitle
\captionsetup{justification=centering}

\clearpage 

\tableofcontents

\clearpage

\listoffigures

\clearpage

\section{Desafio I - Princípio do Modelo Interno}

\subsection{Motivação}
O presente desafio visa analisar como a resposta em regime estacionário de um
sistema de controle em malha fechada se comporta a partir de peturbações
persistentes aplicadas na entrada e na saída da planta em questão. A análise foi
realizada utilizando o Princípio do Modelo Interno que é uma proposição
conceitual, a qual permite concluir sobre as condições para seguimento de
referência em sistema de controle realimentados sujeito a perturbações
persistentes (referenciar documento do professor).

A análise de seguimento de referência via Princípio do Modelo Interno permite
concluir sobre o comportamento da planta controlada para uma dada entrada
inspecionando somente as funções de transferência em malha aberta do sistema.
Dessa forma, pode se inferir mais facilmente o comportamento de uma planta sem
preciso memorizar regras como "modelos de tipo x ou y" e até mesmo sem precisar
realizar cálculos algébricos extensivos.

\subsection{Simulações realizadas}
\label{sec:simulacao-realizadas}
As simulações realizadas utiliziram como base o sistema em malha fechada
representado na Figura \ref{fig:diagrama-de-blocos-malha-fechada} em que:
\begin{itemize}
    \item $R(s)$ é a tranformada de Laplace do sinal de referência $r(t)$;
    \item $E(s)$ é a tranformada de Laplace do erro $e(t)$;
    \item $C(s)$ é a função de transferência do controlador;
    \item $U(s)$ é a tranformada de Laplace do sinal de controle $u(t)$;
    \item $Q_{u}(s)$ é a tranformada de Laplace da perturbação de entrada;
    \item $G(s)$ é a função de transferência da planta;
    \item $Q_{y}(s)$ é a tranformada de Laplace da perturbação na saída; e,
    por fim
    \item $Y(s)$ é a transformada de Laplace da saída do sistema $y(t)$.
\end{itemize}

\begin{figure}[htp]
	\centering
    \captionsetup{justification=centering}
	\caption{Sistema em malha fechada com realimentação unitária e perturbações
    persistentes}
	\label{fig:diagrama-de-blocos-malha-fechada}
	\includegraphics[width=\textwidth]{images/diagrama-de-blocos-malha-fechada.png}
\end{figure}

A partir da Figura \ref{fig:diagrama-de-blocos-malha-fechada}, da álgebra de
diagramas de blocos e do princípio da superposição, chegou-se nas funções de
transferência abaixo. Estas funções serão utilizadas ao longo deste desafio.

\begin{equation}
    \label{eq:y2r-tf}
    \frac{Y(s)}{R(s)} = F(s)\frac{C(s)G(s)}{1 + C(s)G(s)}
\end{equation}

\begin{equation}
    \label{eq:y2qy-tf}
    \frac{Y(s)}{Q_{y}(s)} = \frac{1}{1 + C(s)G(s)}
\end{equation}

\begin{equation}
    \label{eq:y2qu-tf}
    \frac{Y(s)}{Q_{u}(s)} = \frac{G(s)}{1 + C(s)G(s)}
\end{equation}

\begin{equation}
    \label{eq:e2t-tf}
    \frac{E(s)}{R(s)} = F(s)\frac{1}{1 + C(s)G(s)}
\end{equation}

\begin{equation}
    \label{eq:u2r-tf}
    \frac{U(s)}{R(s)} = F(s)\frac{C(s)}{1 + C(s)G(s)}
\end{equation}

\begin{equation}
    \label{eq:u2qy-tf}
    \frac{U(s)}{Q_{y}(s)} = \frac{-C(s)G(s)}{1 + C(s)G(s)}
\end{equation}

\begin{equation}
    \label{eq:u2qu-tf}
    \frac{U(s)}{Q_{u}(s)} = \frac{-C(s)}{1 + C(s)G(s)}
\end{equation}

As simulações foram realizadas considerandos 4 cenários:
\begin{enumerate}
    \item Simulação do sistema para $R(s) = \frac{e^{-2s}}{s}$,
    $Q_{y}(s) = -0,2\frac{e^{-15s}}{s}$ e $Q_{u}(s) = -0,2\frac{e^{-25s}}{s}$
    dado $G(s) = \frac{2}{s}$, $C(s) = K$ e $F(s) = 1$ para um $K$ tal que o
    tempo de acomodação de 2\% para mudança de referência ao degrau seja igual
    a 4s;

    \item Simulação do sistema para $R(s) = \frac{e^{-2s}}{s}$,
    $R(s) = \frac{e^{-2s}}{s^{2}}$, $Q_{y}(s) = -0,2\frac{e^{-15s}}{s}$,
    $Q_{y}(s) = -0,2\frac{e^{-15s}}{s^{2}}$, $Q_{u}(s) = -0,2\frac{e^{-25s}}{s}$
    e $Q_{u}(s) = -0,2\frac{e^{-25s}}{s^{2}}$ dado $G(s) = \frac{2}{s}$, 
    $C(s) = K\frac{s + z}{s}$ e $F(s) = \frac{\tau_{n}s + 1}{\tau_{d}s + 1}$ 
    para um $K$ e $z$ tal que $P(s) = 1 + C(s)G(s)$ possua duas raízes dadas 
    por $s^{*} = -1$ e também $\tau_{n}$ e $\tau_{d}$ tal que o tempo de
    acomodação de 2\% para mudança de referência ao degrau seja igual a 6s
    dados os valores $K$ e $z$ obtidos;

    \item Simulação do sistema para $R(s) = \frac{e^{-2s}}{s}$,
    $Q_{y}(s) = -0,2\frac{e^{-15s}}{s^{2}}$ e
    $Q_{u}(s) = -0,2\frac{e^{-25s}}{s^{2}}$ dado $G(s) = \frac{2}{s}$,
    $C(s) = K\frac{s + z}{s}$ e $F(s) = \frac{\tau_{n}s + 1}{\tau_{d}s + 1}$ 
    para um $K$, $z$, $\tau_{n}$ e $\tau_{d}$ tal que o tempo de acomodação de
    2\% para mudança de referência ao degrau seja igual a 3s;

    \item Simulação do sistema para as mesmas entradas $R(s)$ do cenário 1 mas 
    também para $q_{y}(t) = 0,2sen(2t)\mathds{1}(t - 20)$ e 
    $q_{u}(t) = 0,2sen(2t)\mathds{1}(t - 40)$ dado $G(s) = \frac{2}{s + 1,5}$, 
    $C(s) = K\frac{s + z}{s}$, 
    $C(s) = K\left(\frac{s + z}{s}\right)\left(\frac{s^2 + 0,5s + 1.8^2}{s^2 + 2^2}\right)$ 
    e $F(s) = 1$ para um $K$, e $z$ tal que o tempo de acomodação de 2\% para 
    mudança de referência ao degrau seja igual a 2s.
\end{enumerate}

\subsection{Resultados obtidos}
As seguintes subseções irão descrever os resultados obtidos para os cenários
listados no tópico \ref{sec:simulacao-realizadas}.

\subsubsection{Cenário 1}
Para $G(s)$, $C(s)$ e $F(s)$ dados, temos que

\begin{equation}
    \label{eq:y2r-comk-cenario1}
    \frac{Y(s)}{R(s)} = \frac{2K}{s + 2K}
\end{equation}

Como trata-se de um sistema de primeira ordem, temos também que $t_{s_{2\%}} =
4\tau$. Aplicando esta equação em \ref{eq:y2r-comk-cenario1}, obtemos $K = 0.5$.
Assim chegamos as seguintes funções de transferência:

\begin{equation}
    \label{eq:y2r-cenario1}
    \frac{Y(s)}{R(s)} = \frac{1}{s + 1},
\end{equation}

\begin{equation}
    \label{eq:y2qy-cenario1}
    \frac{Y(s)}{Q_{y}(s)} = \frac{s}{s + 1},
\end{equation}

\begin{equation}
    \label{eq:y2qu-cenario1}
    \frac{Y(s)}{Q_{u}(s)} = \frac{2}{s + 1}.
\end{equation}

A partir do princípio da superposição e da propriedade de linearidade pode se
obter a saída total do sistema através das somas das respostas individuais da
referência e das perturbações, isto é,

\begin{equation}
    \label{eq:saida-do-sistema}
    y(t) = y_{r}(t) + y_{q_{y}}(t) + y_{q_{u}}(t).
\end{equation}

Partindo da equação \ref{eq:saida-do-sistema} obteve-se a saída $y(t)$ para as
entradas $r(t) = \mathds{1}(t - 2)$, $q_{y}(t) = -0,2\mathds{1}(t - 15)$
$q_{u}(t) = -0,2\mathds{1}(t - 25)$. O mesmo princípio foi usado para obter o
sinal de controle $u(t)$ dadas as entradas deslocadas no tempo. O resultado
obtido da simulação para o cenário 1 pode ser visualizado na Figura.

\begin{figure}[!ht]
    \label{fig:resultado-desafio1-cenario1}
    % \centering    
    \caption{Saída $y(t)$ e sinal de controle $u(t)$ para entradas $r(t)$,
    $q_{y}(t)$ e $q_{u}(t)$.}
    \vspace{-10pt}
    \hspace{-30pt}
    \begin{minipage}{\linewidth}
        % Title: gl2ps_renderer figure
% Creator: GL2PS 1.4.0, (C) 1999-2017 C. Geuzaine
% For: Octave
% CreationDate: Sun Sep 12 03:32:01 2021
\setlength{\unitlength}{1pt}
\begin{picture}(0,0)
\includegraphics{images/resultado-desafio1-cenario1-inc}
\end{picture}%
\begin{picture}(400,250)(0,0)
\fontsize{6}{0}
\selectfont\put(52,141.404){\makebox(0,0)[t]{\textcolor[rgb]{0.15,0.15,0.15}{{0}}}}
\fontsize{6}{0}
\selectfont\put(77.8569,141.404){\makebox(0,0)[t]{\textcolor[rgb]{0.15,0.15,0.15}{{10}}}}
\fontsize{6}{0}
\selectfont\put(103.714,141.404){\makebox(0,0)[t]{\textcolor[rgb]{0.15,0.15,0.15}{{20}}}}
\fontsize{6}{0}
\selectfont\put(129.571,141.404){\makebox(0,0)[t]{\textcolor[rgb]{0.15,0.15,0.15}{{30}}}}
\fontsize{6}{0}
\selectfont\put(155.428,141.404){\makebox(0,0)[t]{\textcolor[rgb]{0.15,0.15,0.15}{{40}}}}
\fontsize{6}{0}
\selectfont\put(181.285,141.404){\makebox(0,0)[t]{\textcolor[rgb]{0.15,0.15,0.15}{{50}}}}
\fontsize{6}{0}
\selectfont\put(48.5205,146.637){\makebox(0,0)[r]{\textcolor[rgb]{0.15,0.15,0.15}{{-0.2}}}}
\fontsize{6}{0}
\selectfont\put(48.5205,157.977){\makebox(0,0)[r]{\textcolor[rgb]{0.15,0.15,0.15}{{0}}}}
\fontsize{6}{0}
\selectfont\put(48.5205,169.316){\makebox(0,0)[r]{\textcolor[rgb]{0.15,0.15,0.15}{{0.2}}}}
\fontsize{6}{0}
\selectfont\put(48.5205,180.655){\makebox(0,0)[r]{\textcolor[rgb]{0.15,0.15,0.15}{{0.4}}}}
\fontsize{6}{0}
\selectfont\put(48.5205,191.995){\makebox(0,0)[r]{\textcolor[rgb]{0.15,0.15,0.15}{{0.6}}}}
\fontsize{6}{0}
\selectfont\put(48.5205,203.333){\makebox(0,0)[r]{\textcolor[rgb]{0.15,0.15,0.15}{{0.8}}}}
\fontsize{6}{0}
\selectfont\put(48.5205,214.673){\makebox(0,0)[r]{\textcolor[rgb]{0.15,0.15,0.15}{{1}}}}
\fontsize{7}{0}
\selectfont\put(31.5205,186.325){\rotatebox{90}{\makebox(0,0)[b]{\textcolor[rgb]{0.15,0.15,0.15}{{Saída}}}}}
\fontsize{7}{0}
\selectfont\put(116.643,130.404){\makebox(0,0)[t]{\textcolor[rgb]{0.15,0.15,0.15}{{Tempo (s)}}}}
\fontsize{6}{0}
\selectfont\put(160.285,174.138){\makebox(0,0)[l]{\textcolor[rgb]{0,0,0}{{r(t)}}}}
\fontsize{6}{0}
\selectfont\put(160.285,162.638){\makebox(0,0)[l]{\textcolor[rgb]{0,0,0}{{y(t)}}}}
\fontsize{6}{0}
\selectfont\put(52,22.2671){\makebox(0,0)[t]{\textcolor[rgb]{0.15,0.15,0.15}{{0}}}}
\fontsize{6}{0}
\selectfont\put(77.8569,22.2671){\makebox(0,0)[t]{\textcolor[rgb]{0.15,0.15,0.15}{{10}}}}
\fontsize{6}{0}
\selectfont\put(103.714,22.2671){\makebox(0,0)[t]{\textcolor[rgb]{0.15,0.15,0.15}{{20}}}}
\fontsize{6}{0}
\selectfont\put(129.571,22.2671){\makebox(0,0)[t]{\textcolor[rgb]{0.15,0.15,0.15}{{30}}}}
\fontsize{6}{0}
\selectfont\put(155.428,22.2671){\makebox(0,0)[t]{\textcolor[rgb]{0.15,0.15,0.15}{{40}}}}
\fontsize{6}{0}
\selectfont\put(181.285,22.2671){\makebox(0,0)[t]{\textcolor[rgb]{0.15,0.15,0.15}{{50}}}}
\fontsize{6}{0}
\selectfont\put(48.5205,27.5){\makebox(0,0)[r]{\textcolor[rgb]{0.15,0.15,0.15}{{-0.2}}}}
\fontsize{6}{0}
\selectfont\put(48.5205,45.188){\makebox(0,0)[r]{\textcolor[rgb]{0.15,0.15,0.15}{{0}}}}
\fontsize{6}{0}
\selectfont\put(48.5205,62.8755){\makebox(0,0)[r]{\textcolor[rgb]{0.15,0.15,0.15}{{0.2}}}}
\fontsize{6}{0}
\selectfont\put(48.5205,80.5635){\makebox(0,0)[r]{\textcolor[rgb]{0.15,0.15,0.15}{{0.4}}}}
\fontsize{6}{0}
\selectfont\put(48.5205,98.2515){\makebox(0,0)[r]{\textcolor[rgb]{0.15,0.15,0.15}{{0.6}}}}
\fontsize{7}{0}
\selectfont\put(31.5205,67.1875){\rotatebox{90}{\makebox(0,0)[b]{\textcolor[rgb]{0.15,0.15,0.15}{{Sinal de Controle}}}}}
\fontsize{7}{0}
\selectfont\put(116.643,11.2671){\makebox(0,0)[t]{\textcolor[rgb]{0.15,0.15,0.15}{{Tempo (s)}}}}
\fontsize{6}{0}
\selectfont\put(232.715,141.404){\makebox(0,0)[t]{\textcolor[rgb]{0.15,0.15,0.15}{{0}}}}
\fontsize{6}{0}
\selectfont\put(258.572,141.404){\makebox(0,0)[t]{\textcolor[rgb]{0.15,0.15,0.15}{{10}}}}
\fontsize{6}{0}
\selectfont\put(284.429,141.404){\makebox(0,0)[t]{\textcolor[rgb]{0.15,0.15,0.15}{{20}}}}
\fontsize{6}{0}
\selectfont\put(310.286,141.404){\makebox(0,0)[t]{\textcolor[rgb]{0.15,0.15,0.15}{{30}}}}
\fontsize{6}{0}
\selectfont\put(336.143,141.404){\makebox(0,0)[t]{\textcolor[rgb]{0.15,0.15,0.15}{{40}}}}
\fontsize{6}{0}
\selectfont\put(362,141.404){\makebox(0,0)[t]{\textcolor[rgb]{0.15,0.15,0.15}{{50}}}}
\fontsize{6}{0}
\selectfont\put(229.235,146.637){\makebox(0,0)[r]{\textcolor[rgb]{0.15,0.15,0.15}{{-0.4}}}}
\fontsize{6}{0}
\selectfont\put(229.235,159.866){\makebox(0,0)[r]{\textcolor[rgb]{0.15,0.15,0.15}{{-0.3}}}}
\fontsize{6}{0}
\selectfont\put(229.235,173.096){\makebox(0,0)[r]{\textcolor[rgb]{0.15,0.15,0.15}{{-0.2}}}}
\fontsize{6}{0}
\selectfont\put(229.235,186.325){\makebox(0,0)[r]{\textcolor[rgb]{0.15,0.15,0.15}{{-0.1}}}}
\fontsize{6}{0}
\selectfont\put(229.235,199.554){\makebox(0,0)[r]{\textcolor[rgb]{0.15,0.15,0.15}{{0}}}}
\fontsize{6}{0}
\selectfont\put(229.235,212.783){\makebox(0,0)[r]{\textcolor[rgb]{0.15,0.15,0.15}{{0.1}}}}
\fontsize{6}{0}
\selectfont\put(229.235,226.012){\makebox(0,0)[r]{\textcolor[rgb]{0.15,0.15,0.15}{{0.2}}}}
\fontsize{7}{0}
\selectfont\put(212.235,186.325){\rotatebox{90}{\makebox(0,0)[b]{\textcolor[rgb]{0.15,0.15,0.15}{{Perturbação - Saída}}}}}
\fontsize{7}{0}
\selectfont\put(297.357,130.404){\makebox(0,0)[t]{\textcolor[rgb]{0.15,0.15,0.15}{{Tempo (s)}}}}
\fontsize{6}{0}
\selectfont\put(232.715,22.2671){\makebox(0,0)[t]{\textcolor[rgb]{0.15,0.15,0.15}{{0}}}}
\fontsize{6}{0}
\selectfont\put(258.572,22.2671){\makebox(0,0)[t]{\textcolor[rgb]{0.15,0.15,0.15}{{10}}}}
\fontsize{6}{0}
\selectfont\put(284.429,22.2671){\makebox(0,0)[t]{\textcolor[rgb]{0.15,0.15,0.15}{{20}}}}
\fontsize{6}{0}
\selectfont\put(310.286,22.2671){\makebox(0,0)[t]{\textcolor[rgb]{0.15,0.15,0.15}{{30}}}}
\fontsize{6}{0}
\selectfont\put(336.143,22.2671){\makebox(0,0)[t]{\textcolor[rgb]{0.15,0.15,0.15}{{40}}}}
\fontsize{6}{0}
\selectfont\put(362,22.2671){\makebox(0,0)[t]{\textcolor[rgb]{0.15,0.15,0.15}{{50}}}}
\fontsize{6}{0}
\selectfont\put(229.235,27.5){\makebox(0,0)[r]{\textcolor[rgb]{0.15,0.15,0.15}{{-0.4}}}}
\fontsize{6}{0}
\selectfont\put(229.235,40.729){\makebox(0,0)[r]{\textcolor[rgb]{0.15,0.15,0.15}{{-0.3}}}}
\fontsize{6}{0}
\selectfont\put(229.235,53.9585){\makebox(0,0)[r]{\textcolor[rgb]{0.15,0.15,0.15}{{-0.2}}}}
\fontsize{6}{0}
\selectfont\put(229.235,67.1875){\makebox(0,0)[r]{\textcolor[rgb]{0.15,0.15,0.15}{{-0.1}}}}
\fontsize{6}{0}
\selectfont\put(229.235,80.4165){\makebox(0,0)[r]{\textcolor[rgb]{0.15,0.15,0.15}{{0}}}}
\fontsize{6}{0}
\selectfont\put(229.235,93.646){\makebox(0,0)[r]{\textcolor[rgb]{0.15,0.15,0.15}{{0.1}}}}
\fontsize{6}{0}
\selectfont\put(229.235,106.875){\makebox(0,0)[r]{\textcolor[rgb]{0.15,0.15,0.15}{{0.2}}}}
\fontsize{7}{0}
\selectfont\put(212.235,67.1875){\rotatebox{90}{\makebox(0,0)[b]{\textcolor[rgb]{0.15,0.15,0.15}{{Perturbação - Entrada}}}}}
\fontsize{7}{0}
\selectfont\put(297.357,11.2671){\makebox(0,0)[t]{\textcolor[rgb]{0.15,0.15,0.15}{{Tempo (s)}}}}
\end{picture}

    \end{minipage}
\end{figure}

\subsubsection{Cenário 2}
\subsubsection{Cenário 3}
\subsubsection{Cenário 4}

\subsection{Conclusões}
(Concluir em que medida os resultados apresentam relação com a motivação.
Permitem ilustratar ou concluir algo sobre a motivação? )
\section{Desafio II - Teorema do Pequeno Ganho e Anti-Windup} 
 
\subsection{Motivação}
Este dasafio foi dividido em duas partes. A primeira visa a análise de
incertezas na modelagem de sistemas via Teorema do Pequeno Ganho e Teoria de
Controle Robusto. Já a segunda, em série com o resultado da primeira parte,
analisa o comportamento da saída de um modelo de planta, considerando um ponto
de operação específico, e sinal de controle para um sistema em malha fechada na
presença de saturação do sinal de controle.

A análise de incertezas é de suma importância na avaliação de modelagem de plantas
via alguma técnica específica, como resposta ao degrau ou resposta em
frequência. Esta modelagem é realizada principalmente quando não se tem o modelo
teórico. Desta forma é preciso realizar a modelagem com a planta real, obtendo
assim um modelo nominal da planta. Assim, torna-se necessário analisar qual o
efeito das incertezas de modelagem, para garantir a estabilidade do sistema em
malha fechada sendo controlado por um controlador projetado utilizando o modelo
nominal da planta. Esta análise de robustez é obtida via Teorema do Pequeno
Ganho e Teoria da Instabilidade Interna (não contemplada neste desafio).

Além da análise de robustez, é ainda necessário analisar o sistema em malha
fechada obtido através do modelo nominal em um ponto específico de operação na
presença de saturação do sinal de controle. Esta análise também é de extrema
relevância na teoria  e aplicação de controle, pois os sistemas de controle
reais tem limitações, como a energia finita de atuadores, que caso não analisado
corretamente, podem surgir no sistema de controle implementado comportamentos
indesejáveis. Um exemplo comum de saturação é nas malhas de controle de vazão,
em que válvulas na linha do processo tem um limiar de abertura além da presença
de zonas mortas de operação.

\subsection{Simulações realizadas}
O sistema de controle em malha fechada utilizado nas simulações do segundo
desafio é o mesmo do desafio 1 e está descrito no tópico
\ref{sec:desafio-1-simulacoes-realizadas} e ilustrado em diagrama de blocos na
Figura \ref{fig:diagrama-de-blocos-malha-fechada}. O modelo $G(s)$ utilizado nas
simulações é dado por

\begin{equation}
    \label{eq:modelo-g-de-s-utilizado-no-desafio-2}
    G(s) = \frac{K_{i}e^{-sL_{i}}}{T_{i}s + 1}.
\end{equation}

O modelo descrito na Equação \ref{eq:modelo-g-de-s-utilizado-no-desafio-2} foi
levantado a partir de 4 pontos de operação distintos, gerando 4 modelos: M1, M2,
M3 e M4. Os parâmetros obtidos para cada modelo $i$ foram os seguintes: $L_{i} =
[0,9; 0,7; 0,6; 0,4]$, $K_{i} = [1,3; 0,9; 1,2; 0,8]$ e $\tau_{i} = [0,9; 0,7;
0,6; 0,4]$.

Para o projeto do controlador utilizou o modelo nominal de $G(s)$ obtido
através das médias $L_{n}$, $K_{n}$ e $\tau_{n}$ dos parâmetros $L_{i}$, $K_{i}$
e $\tau_{i}$, respectivamente. Assim, obteve-se $G{n}(s)$ dado por

\begin{equation}
    \label{eq:modelo-gn-de-s-utilizado-no-desafio-2}
    G_{n}(s) = \frac{1,05e^{-0,65s}}{s + 1}.
\end{equation}

O controlador utilizado nas simulações foi o PI de Skogestad para um sistema de
primeira ordem com atraso, dado por

\begin{equation}
    \label{eq:modelo-c-de-s-utilizado-no-desafio-2}
    C(s) = K_{c}\frac{sT_{i} + 1}{sT_{i}} = \frac{0,8282(s + 1)}{s}
\end{equation}
em que $K_{c} = \tau_{n}/[K_{n}(\tau_{c} + L_{n})]$, $T_{i} = min(\tau_{n},
4(\tau_{c} + L_{n}))$ e $\tau_{c} = 0,5$. 

A partir destes modelos foram realizadas 3 simulações:
\begin{enumerate}
    \item Geração das incertezas multiplicativas dos 4 modelos e análise de
    robustez via Teorema do Pequeno Ganho;
    \item Resposta em malha fechada dos 4 modelos, dado o controlador definido
    na Equação \ref{eq:modelo-c-de-s-utilizado-no-desafio-2}, com uma entrada de
    referência $r(t)$ em degrau unitário e uma perturbação na entrada $q_{u}(t)$
    também em degrau;
    \item Resposta do modelo M1 em malha fechada com diferentes configurações:
    com filtro de referência $F(s)$, com controlador I+P, com controlador I+P e
    saturação do sinal de controle e com os dois itens anteriores mais a ação de
    anti-windup.
\end{enumerate}

O tópico seguinte irar descrever os resultados obtidos para os 3 itens acima.

\subsection{Resultados obtidos}
\subsubsection{Geração de incertezas multiplicativas}
Para cada modelo $G_{i}(s)$ com $i \in [1,4]$, calculou-se o módulo da incerteza
multiplicativa conforme Equação \ref{eq:desafio-2-calculo-das-incertezas}. O
resultado obtido está ilustrado na Figura
\ref{fig:desafio-2-resultado-questao-1}.

\begin{equation}
    \label{eq:desafio-2-calculo-das-incertezas}
    \Delta_{i}(w) = \left | \frac{G_{n}(jw) - G_{i}(jw)}{G_{n}(jw)} \right |,
    10^{-2} \leq w \leq 10^{4}.
\end{equation}

\begin{figure}[!ht]
    \caption{Módulo das incertezas multiplicativas de cada modelo $i$ em
    função da frequência $w$.}
    \vspace{-10pt}
    \hspace{-30pt}
    \label{fig:desafio-2-resultado-questao-1}
    \begin{minipage}{\linewidth}
        % Title: gl2ps_renderer figure
% Creator: GL2PS 1.4.0, (C) 1999-2017 C. Geuzaine
% For: Octave
% CreationDate: Sun Sep 26 17:17:39 2021
\setlength{\unitlength}{1pt}
\begin{picture}(0,0)
\includegraphics{images/challenge2/resultado-questao-1-inc}
\end{picture}%
\begin{picture}(400,300)(0,0)
\fontsize{6}{0}
\selectfont\put(48.5278,240.483){\makebox(0,0)[r]{\textcolor[rgb]{0.15,0.15,0.15}{{0.5}}}}
\fontsize{6}{0}
\selectfont\put(48.5278,248.666){\makebox(0,0)[r]{\textcolor[rgb]{0.15,0.15,0.15}{{1}}}}
\fontsize{6}{0}
\selectfont\put(48.5278,256.849){\makebox(0,0)[r]{\textcolor[rgb]{0.15,0.15,0.15}{{1.5}}}}
\fontsize{6}{0}
\selectfont\put(48.5278,265.032){\makebox(0,0)[r]{\textcolor[rgb]{0.15,0.15,0.15}{{2}}}}
\fontsize{6}{0}
\selectfont\put(48.5278,273.215){\makebox(0,0)[r]{\textcolor[rgb]{0.15,0.15,0.15}{{2.5}}}}
\fontsize{7}{0}
\selectfont\put(33.5278,254.706){\rotatebox{90}{\makebox(0,0)[b]{\textcolor[rgb]{0.15,0.15,0.15}{{$\Delta_{1}(jw)$}}}}}
\fontsize{6}{0}
\selectfont\put(48.5278,172.75){\makebox(0,0)[r]{\textcolor[rgb]{0.15,0.15,0.15}{{0.5}}}}
\fontsize{6}{0}
\selectfont\put(48.5278,180.934){\makebox(0,0)[r]{\textcolor[rgb]{0.15,0.15,0.15}{{1}}}}
\fontsize{6}{0}
\selectfont\put(48.5278,189.116){\makebox(0,0)[r]{\textcolor[rgb]{0.15,0.15,0.15}{{1.5}}}}
\fontsize{6}{0}
\selectfont\put(48.5278,197.299){\makebox(0,0)[r]{\textcolor[rgb]{0.15,0.15,0.15}{{2}}}}
\fontsize{6}{0}
\selectfont\put(48.5278,205.482){\makebox(0,0)[r]{\textcolor[rgb]{0.15,0.15,0.15}{{2.5}}}}
\fontsize{7}{0}
\selectfont\put(33.5278,186.974){\rotatebox{90}{\makebox(0,0)[b]{\textcolor[rgb]{0.15,0.15,0.15}{{$\Delta_{2}(jw)$}}}}}
\fontsize{6}{0}
\selectfont\put(48.5278,105.018){\makebox(0,0)[r]{\textcolor[rgb]{0.15,0.15,0.15}{{0.5}}}}
\fontsize{6}{0}
\selectfont\put(48.5278,113.201){\makebox(0,0)[r]{\textcolor[rgb]{0.15,0.15,0.15}{{1}}}}
\fontsize{6}{0}
\selectfont\put(48.5278,121.384){\makebox(0,0)[r]{\textcolor[rgb]{0.15,0.15,0.15}{{1.5}}}}
\fontsize{6}{0}
\selectfont\put(48.5278,129.567){\makebox(0,0)[r]{\textcolor[rgb]{0.15,0.15,0.15}{{2}}}}
\fontsize{6}{0}
\selectfont\put(48.5278,137.75){\makebox(0,0)[r]{\textcolor[rgb]{0.15,0.15,0.15}{{2.5}}}}
\fontsize{7}{0}
\selectfont\put(33.5278,119.241){\rotatebox{90}{\makebox(0,0)[b]{\textcolor[rgb]{0.15,0.15,0.15}{{$\Delta_{3}(jw)$}}}}}
\fontsize{6}{0}
\selectfont\put(52,27.7896){\makebox(0,0)[t]{\textcolor[rgb]{0.15,0.15,0.15}{{$10^{-2}$}}}}
\fontsize{6}{0}
\selectfont\put(103.667,27.7896){\makebox(0,0)[t]{\textcolor[rgb]{0.15,0.15,0.15}{{$10^{-1}$}}}}
\fontsize{6}{0}
\selectfont\put(155.333,27.7896){\makebox(0,0)[t]{\textcolor[rgb]{0.15,0.15,0.15}{{$10^{0}$}}}}
\fontsize{6}{0}
\selectfont\put(207,27.7896){\makebox(0,0)[t]{\textcolor[rgb]{0.15,0.15,0.15}{{$10^{1}$}}}}
\fontsize{6}{0}
\selectfont\put(258.667,27.7896){\makebox(0,0)[t]{\textcolor[rgb]{0.15,0.15,0.15}{{$10^{2}$}}}}
\fontsize{6}{0}
\selectfont\put(310.333,27.7896){\makebox(0,0)[t]{\textcolor[rgb]{0.15,0.15,0.15}{{$10^{3}$}}}}
\fontsize{6}{0}
\selectfont\put(362,27.7896){\makebox(0,0)[t]{\textcolor[rgb]{0.15,0.15,0.15}{{$10^{4}$}}}}
\fontsize{6}{0}
\selectfont\put(48.5278,37.2856){\makebox(0,0)[r]{\textcolor[rgb]{0.15,0.15,0.15}{{0.5}}}}
\fontsize{6}{0}
\selectfont\put(48.5278,45.4688){\makebox(0,0)[r]{\textcolor[rgb]{0.15,0.15,0.15}{{1}}}}
\fontsize{6}{0}
\selectfont\put(48.5278,53.6519){\makebox(0,0)[r]{\textcolor[rgb]{0.15,0.15,0.15}{{1.5}}}}
\fontsize{6}{0}
\selectfont\put(48.5278,61.835){\makebox(0,0)[r]{\textcolor[rgb]{0.15,0.15,0.15}{{2}}}}
\fontsize{6}{0}
\selectfont\put(48.5278,70.0176){\makebox(0,0)[r]{\textcolor[rgb]{0.15,0.15,0.15}{{2.5}}}}
\fontsize{7}{0}
\selectfont\put(33.5278,51.5088){\rotatebox{90}{\makebox(0,0)[b]{\textcolor[rgb]{0.15,0.15,0.15}{{$\Delta_{4}(jw)$}}}}}
\fontsize{7}{0}
\selectfont\put(207,13.7896){\makebox(0,0)[t]{\textcolor[rgb]{0.15,0.15,0.15}{{Frequência $w$ (rad/s)}}}}
\end{picture}

    \end{minipage}
\end{figure}

Para verificar se o sistema em malha fechada com o modelo nominal $G_{n}(s)$ é
estável, calculou-se o módulo da complementar de sensibidade, Equação
\ref{eq:desafio-2:modulo-da-complementar-de-sensibilidade}, que foi multiplicado
pelo limitante superior das incertezas multiplicativas definido pela Equação
\ref{eq:desafio-2:limitante-superior-das-incertezas-multiplicativas}.

\begin{equation}
    \label{eq:desafio-2:modulo-da-complementar-de-sensibilidade}
    |\textit{C}(jw)| = \left | \frac{C(jw)G_{n}(jw)}{1 + C(jw)G_{n}(jw)} \right |
\end{equation}

\begin{equation}
    \label{eq:desafio-2:limitante-superior-das-incertezas-multiplicativas}
    \bar{\Delta}(w[k]) = min(\Delta_{1}(w[k]), \Delta_{2}(w[k]), \Delta_{3}(w[k]),
    \Delta_{4}(w[k])). 
\end{equation}
em que $w[k]$ é um vetor de 10000 pontos em escala logarítimica com $10^{-2}
\leq w \leq 10^{4}$ rad/s.

O resultado obtido está ilustrado na Figura
\ref{fig:desafio-2:resultado-questao-3-4}, em que é possível visualizar o
limitante superior das incertezas multiplicativas $\bar{\Delta}(w)$, o módulo da
função complementar de sensibidade $|\textit{C}(jw)|$ e o resultado da
multiplicação dos dois valores.

\begin{figure}[!ht]
    \caption{Resultado da análise das incertezas multiplicativas.}
    \vspace{-10pt}
    \hspace{-30pt}
    \label{fig:desafio-2:resultado-questao-3-4}
    \begin{minipage}{\linewidth}
        % Title: gl2ps_renderer figure
% Creator: GL2PS 1.4.0, (C) 1999-2017 C. Geuzaine
% For: Octave
% CreationDate: Sun Sep 26 17:19:09 2021
\setlength{\unitlength}{1pt}
\begin{picture}(0,0)
\includegraphics{images/challenge2/resultado-questao-3-4-inc}
\end{picture}%
\begin{picture}(400,300)(0,0)
\fontsize{6}{0}
\selectfont\put(48.5278,216.953){\makebox(0,0)[r]{\textcolor[rgb]{0.15,0.15,0.15}{{0}}}}
\fontsize{6}{0}
\selectfont\put(48.5278,228.205){\makebox(0,0)[r]{\textcolor[rgb]{0.15,0.15,0.15}{{0.5}}}}
\fontsize{6}{0}
\selectfont\put(48.5278,239.458){\makebox(0,0)[r]{\textcolor[rgb]{0.15,0.15,0.15}{{1}}}}
\fontsize{6}{0}
\selectfont\put(48.5278,250.71){\makebox(0,0)[r]{\textcolor[rgb]{0.15,0.15,0.15}{{1.5}}}}
\fontsize{6}{0}
\selectfont\put(48.5278,261.962){\makebox(0,0)[r]{\textcolor[rgb]{0.15,0.15,0.15}{{2}}}}
\fontsize{6}{0}
\selectfont\put(48.5278,273.215){\makebox(0,0)[r]{\textcolor[rgb]{0.15,0.15,0.15}{{2.5}}}}
\fontsize{7}{0}
\selectfont\put(33.5278,245.083){\rotatebox{90}{\makebox(0,0)[b]{\textcolor[rgb]{0.15,0.15,0.15}{{$\bar{\Delta}(w)$}}}}}
\fontsize{6}{0}
\selectfont\put(48.5278,124.977){\makebox(0,0)[r]{\textcolor[rgb]{0.15,0.15,0.15}{{0}}}}
\fontsize{6}{0}
\selectfont\put(48.5278,133.014){\makebox(0,0)[r]{\textcolor[rgb]{0.15,0.15,0.15}{{0.2}}}}
\fontsize{6}{0}
\selectfont\put(48.5278,141.051){\makebox(0,0)[r]{\textcolor[rgb]{0.15,0.15,0.15}{{0.4}}}}
\fontsize{6}{0}
\selectfont\put(48.5278,149.088){\makebox(0,0)[r]{\textcolor[rgb]{0.15,0.15,0.15}{{0.6}}}}
\fontsize{6}{0}
\selectfont\put(48.5278,157.126){\makebox(0,0)[r]{\textcolor[rgb]{0.15,0.15,0.15}{{0.8}}}}
\fontsize{6}{0}
\selectfont\put(48.5278,165.164){\makebox(0,0)[r]{\textcolor[rgb]{0.15,0.15,0.15}{{1}}}}
\fontsize{6}{0}
\selectfont\put(48.5278,173.201){\makebox(0,0)[r]{\textcolor[rgb]{0.15,0.15,0.15}{{1.2}}}}
\fontsize{6}{0}
\selectfont\put(48.5278,181.238){\makebox(0,0)[r]{\textcolor[rgb]{0.15,0.15,0.15}{{1.4}}}}
\fontsize{7}{0}
\selectfont\put(33.5278,153.107){\rotatebox{90}{\makebox(0,0)[b]{\textcolor[rgb]{0.15,0.15,0.15}{{$|\textit{C}(jw)|$}}}}}
\fontsize{6}{0}
\selectfont\put(52,27.7676){\makebox(0,0)[t]{\textcolor[rgb]{0.15,0.15,0.15}{{$10^{-2}$}}}}
\fontsize{6}{0}
\selectfont\put(103.667,27.7676){\makebox(0,0)[t]{\textcolor[rgb]{0.15,0.15,0.15}{{$10^{-1}$}}}}
\fontsize{6}{0}
\selectfont\put(155.333,27.7676){\makebox(0,0)[t]{\textcolor[rgb]{0.15,0.15,0.15}{{$10^{0}$}}}}
\fontsize{6}{0}
\selectfont\put(207,27.7676){\makebox(0,0)[t]{\textcolor[rgb]{0.15,0.15,0.15}{{$10^{1}$}}}}
\fontsize{6}{0}
\selectfont\put(258.667,27.7676){\makebox(0,0)[t]{\textcolor[rgb]{0.15,0.15,0.15}{{$10^{2}$}}}}
\fontsize{6}{0}
\selectfont\put(310.333,27.7676){\makebox(0,0)[t]{\textcolor[rgb]{0.15,0.15,0.15}{{$10^{3}$}}}}
\fontsize{6}{0}
\selectfont\put(362,27.7676){\makebox(0,0)[t]{\textcolor[rgb]{0.15,0.15,0.15}{{$10^{4}$}}}}
\fontsize{6}{0}
\selectfont\put(48.5278,33){\makebox(0,0)[r]{\textcolor[rgb]{0.15,0.15,0.15}{{0}}}}
\fontsize{6}{0}
\selectfont\put(48.5278,44.2524){\makebox(0,0)[r]{\textcolor[rgb]{0.15,0.15,0.15}{{0.1}}}}
\fontsize{6}{0}
\selectfont\put(48.5278,55.5049){\makebox(0,0)[r]{\textcolor[rgb]{0.15,0.15,0.15}{{0.2}}}}
\fontsize{6}{0}
\selectfont\put(48.5278,66.7568){\makebox(0,0)[r]{\textcolor[rgb]{0.15,0.15,0.15}{{0.3}}}}
\fontsize{6}{0}
\selectfont\put(48.5278,78.0093){\makebox(0,0)[r]{\textcolor[rgb]{0.15,0.15,0.15}{{0.4}}}}
\fontsize{6}{0}
\selectfont\put(48.5278,89.2617){\makebox(0,0)[r]{\textcolor[rgb]{0.15,0.15,0.15}{{0.5}}}}
\fontsize{7}{0}
\selectfont\put(33.5278,61.1309){\rotatebox{90}{\makebox(0,0)[b]{\textcolor[rgb]{0.15,0.15,0.15}{{$|\textit{C}(jw)|\bar{\Delta}(w)$}}}}}
\fontsize{7}{0}
\selectfont\put(207,13.7676){\makebox(0,0)[t]{\textcolor[rgb]{0.15,0.15,0.15}{{Frequência $w$ (rad/s)}}}}
\end{picture}

    \end{minipage}
\end{figure}

Conclui-se a a partir do último gráfico da Figura
\ref{fig:desafio-2:resultado-questao-3-4} que o sistema em malha fechada é
estável, dentro de uma faixa de frequência, com a planta em diferentes pontos de
operação dado as incertezas dos modelos M1, M2, M3 e M4 e o controlador $C(s)$
projetado considerando o modelo nominal da planta $G_{n}(s)$. Esta conclusão
pode ser obtida pois

\begin{equation}
    \label{eq:desafio-2:principio-do-modelo-interno}
    \sup_{w}|\textit{C}(jw)|\bar{\Delta}(w) \le 1, \ 10^{-2} \leq \forall w \leq {10^{4}}.
\end{equation}

\subsubsection{Resposta em malha fechadas dos modelos M1, M2, M3 e M4}
\subsubsection{Resposta em malha fechada do modelo M1 com diferentes configurações}

\subsection{Conclusões}
(Concluir em que medida os resultados apresentam relação com a motivação.
Permitem ilustratar ou concluir algo sobre a motivação? )


\begin{figure}[!ht]
    \caption{Simulação do sistema de controle em malhada fechada para condições
    impostas no Cenário 1.}
    \vspace{-10pt}
    \hspace{-30pt}
    \label{fig:resultado-desafio2-questao5}
    \begin{minipage}{\linewidth}
        % Title: gl2ps_renderer figure
% Creator: GL2PS 1.4.0, (C) 1999-2017 C. Geuzaine
% For: Octave
% CreationDate: Sun Sep 26 17:47:58 2021
\setlength{\unitlength}{1pt}
\begin{picture}(0,0)
\includegraphics{images/challenge2/resultado-questao-5-inc}
\end{picture}%
\begin{picture}(400,250)(0,0)
\fontsize{6}{0}
\selectfont\put(52,141.404){\makebox(0,0)[t]{\textcolor[rgb]{0.15,0.15,0.15}{{0}}}}
\fontsize{6}{0}
\selectfont\put(103.675,141.404){\makebox(0,0)[t]{\textcolor[rgb]{0.15,0.15,0.15}{{10}}}}
\fontsize{6}{0}
\selectfont\put(155.351,141.404){\makebox(0,0)[t]{\textcolor[rgb]{0.15,0.15,0.15}{{20}}}}
\fontsize{6}{0}
\selectfont\put(207.026,141.404){\makebox(0,0)[t]{\textcolor[rgb]{0.15,0.15,0.15}{{30}}}}
\fontsize{6}{0}
\selectfont\put(258.701,141.404){\makebox(0,0)[t]{\textcolor[rgb]{0.15,0.15,0.15}{{40}}}}
\fontsize{6}{0}
\selectfont\put(310.376,141.404){\makebox(0,0)[t]{\textcolor[rgb]{0.15,0.15,0.15}{{50}}}}
\fontsize{6}{0}
\selectfont\put(48.5278,155.215){\makebox(0,0)[r]{\textcolor[rgb]{0.15,0.15,0.15}{{0}}}}
\fontsize{6}{0}
\selectfont\put(48.5278,176.659){\makebox(0,0)[r]{\textcolor[rgb]{0.15,0.15,0.15}{{0.5}}}}
\fontsize{6}{0}
\selectfont\put(48.5278,198.104){\makebox(0,0)[r]{\textcolor[rgb]{0.15,0.15,0.15}{{1}}}}
\fontsize{6}{0}
\selectfont\put(48.5278,219.548){\makebox(0,0)[r]{\textcolor[rgb]{0.15,0.15,0.15}{{1.5}}}}
\fontsize{7}{0}
\selectfont\put(33.5278,186.325){\rotatebox{90}{\makebox(0,0)[b]{\textcolor[rgb]{0.15,0.15,0.15}{{Saída $y_{i}(t)$}}}}}
\fontsize{7}{0}
\selectfont\put(207,130.404){\makebox(0,0)[t]{\textcolor[rgb]{0.15,0.15,0.15}{{Tempo (s)}}}}
\fontsize{6}{0}
\selectfont\put(332,210.139){\makebox(0,0)[l]{\textcolor[rgb]{0,0,0}{{$r(t)$}}}}
\fontsize{6}{0}
\selectfont\put(332,198.639){\makebox(0,0)[l]{\textcolor[rgb]{0,0,0}{{$y_{1}(t)$}}}}
\fontsize{6}{0}
\selectfont\put(332,186.638){\makebox(0,0)[l]{\textcolor[rgb]{0,0,0}{{$y_{2}(t)$}}}}
\fontsize{6}{0}
\selectfont\put(332,174.638){\makebox(0,0)[l]{\textcolor[rgb]{0,0,0}{{$y_{3}(t)$}}}}
\fontsize{6}{0}
\selectfont\put(332,162.638){\makebox(0,0)[l]{\textcolor[rgb]{0,0,0}{{$y_{4}(t)$}}}}
\fontsize{6}{0}
\selectfont\put(52,22.2671){\makebox(0,0)[t]{\textcolor[rgb]{0.15,0.15,0.15}{{0}}}}
\fontsize{6}{0}
\selectfont\put(103.675,22.2671){\makebox(0,0)[t]{\textcolor[rgb]{0.15,0.15,0.15}{{10}}}}
\fontsize{6}{0}
\selectfont\put(155.351,22.2671){\makebox(0,0)[t]{\textcolor[rgb]{0.15,0.15,0.15}{{20}}}}
\fontsize{6}{0}
\selectfont\put(207.026,22.2671){\makebox(0,0)[t]{\textcolor[rgb]{0.15,0.15,0.15}{{30}}}}
\fontsize{6}{0}
\selectfont\put(258.701,22.2671){\makebox(0,0)[t]{\textcolor[rgb]{0.15,0.15,0.15}{{40}}}}
\fontsize{6}{0}
\selectfont\put(310.376,22.2671){\makebox(0,0)[t]{\textcolor[rgb]{0.15,0.15,0.15}{{50}}}}
\fontsize{6}{0}
\selectfont\put(48.5278,34.7158){\makebox(0,0)[r]{\textcolor[rgb]{0.15,0.15,0.15}{{0}}}}
\fontsize{6}{0}
\selectfont\put(48.5278,52.7559){\makebox(0,0)[r]{\textcolor[rgb]{0.15,0.15,0.15}{{0.5}}}}
\fontsize{6}{0}
\selectfont\put(48.5278,70.7954){\makebox(0,0)[r]{\textcolor[rgb]{0.15,0.15,0.15}{{1}}}}
\fontsize{6}{0}
\selectfont\put(48.5278,88.8354){\makebox(0,0)[r]{\textcolor[rgb]{0.15,0.15,0.15}{{1.5}}}}
\fontsize{6}{0}
\selectfont\put(48.5278,106.875){\makebox(0,0)[r]{\textcolor[rgb]{0.15,0.15,0.15}{{2}}}}
\fontsize{7}{0}
\selectfont\put(33.5278,67.1875){\rotatebox{90}{\makebox(0,0)[b]{\textcolor[rgb]{0.15,0.15,0.15}{{Sinal $u_{i}(t)$}}}}}
\fontsize{7}{0}
\selectfont\put(207,11.2671){\makebox(0,0)[t]{\textcolor[rgb]{0.15,0.15,0.15}{{Tempo (s)}}}}
\fontsize{6}{0}
\selectfont\put(332,76.0015){\makebox(0,0)[l]{\textcolor[rgb]{0,0,0}{{$u_{1}(t)$}}}}
\fontsize{6}{0}
\selectfont\put(332,65.001){\makebox(0,0)[l]{\textcolor[rgb]{0,0,0}{{$u_{2}(t)$}}}}
\fontsize{6}{0}
\selectfont\put(332,54.001){\makebox(0,0)[l]{\textcolor[rgb]{0,0,0}{{$u_{3}(t)$}}}}
\fontsize{6}{0}
\selectfont\put(332,43.0005){\makebox(0,0)[l]{\textcolor[rgb]{0,0,0}{{$u_{4}(t)$}}}}
\end{picture}

    \end{minipage}
\end{figure}

\begin{figure}[!ht]
    \caption{Simulação do sistema de controle em malhada fechada para condições
    impostas no Cenário 1.}
    \vspace{-10pt}
    \hspace{-30pt}
    \label{fig:resultado-desafio2-questao6}
    \begin{minipage}{\linewidth}
        % Title: gl2ps_renderer figure
% Creator: GL2PS 1.4.0, (C) 1999-2017 C. Geuzaine
% For: Octave
% CreationDate: Fri Oct  1 21:40:36 2021
\setlength{\unitlength}{1pt}
\begin{picture}(0,0)
\includegraphics{images/challenge2/resultado-questao-6-9-inc}
\end{picture}%
\begin{picture}(400,500)(0,0)
\fontsize{6}{0}
\selectfont\put(52,356.357){\makebox(0,0)[t]{\textcolor[rgb]{0.15,0.15,0.15}{{0}}}}
\fontsize{6}{0}
\selectfont\put(103.675,356.357){\makebox(0,0)[t]{\textcolor[rgb]{0.15,0.15,0.15}{{10}}}}
\fontsize{6}{0}
\selectfont\put(155.351,356.357){\makebox(0,0)[t]{\textcolor[rgb]{0.15,0.15,0.15}{{20}}}}
\fontsize{6}{0}
\selectfont\put(207.026,356.357){\makebox(0,0)[t]{\textcolor[rgb]{0.15,0.15,0.15}{{30}}}}
\fontsize{6}{0}
\selectfont\put(258.701,356.357){\makebox(0,0)[t]{\textcolor[rgb]{0.15,0.15,0.15}{{40}}}}
\fontsize{6}{0}
\selectfont\put(310.376,356.357){\makebox(0,0)[t]{\textcolor[rgb]{0.15,0.15,0.15}{{50}}}}
\fontsize{6}{0}
\selectfont\put(48.5278,374.133){\makebox(0,0)[r]{\textcolor[rgb]{0.15,0.15,0.15}{{0}}}}
\fontsize{6}{0}
\selectfont\put(48.5278,405.495){\makebox(0,0)[r]{\textcolor[rgb]{0.15,0.15,0.15}{{0.5}}}}
\fontsize{6}{0}
\selectfont\put(48.5278,436.856){\makebox(0,0)[r]{\textcolor[rgb]{0.15,0.15,0.15}{{1}}}}
\fontsize{7}{0}
\selectfont\put(33.5278,411.806){\rotatebox{90}{\makebox(0,0)[b]{\textcolor[rgb]{0.15,0.15,0.15}{{Saida $y(t)$}}}}}
\fontsize{7}{0}
\selectfont\put(207,345.357){\makebox(0,0)[t]{\textcolor[rgb]{0.15,0.15,0.15}{{Tempo (s)}}}}
\fontsize{6}{0}
\selectfont\put(263,436.09){\makebox(0,0)[l]{\textcolor[rgb]{0,0,0}{{$r_{f}(t)$}}}}
\fontsize{6}{0}
\selectfont\put(263,425.09){\makebox(0,0)[l]{\textcolor[rgb]{0,0,0}{{$r(t)$}}}}
\fontsize{6}{0}
\selectfont\put(263,413.589){\makebox(0,0)[l]{\textcolor[rgb]{0,0,0}{{$y_{m1}(t)$ com filtro de referência}}}}
\fontsize{6}{0}
\selectfont\put(263,401.589){\makebox(0,0)[l]{\textcolor[rgb]{0,0,0}{{$y_{m1}(t)$ com controlador I+P}}}}
\fontsize{6}{0}
\selectfont\put(263,389.589){\makebox(0,0)[l]{\textcolor[rgb]{0,0,0}{{$y_{m1}(t)$ com saturação de $u(t)$}}}}
\fontsize{6}{0}
\selectfont\put(263,377.589){\makebox(0,0)[l]{\textcolor[rgb]{0,0,0}{{$y_{m1}(t)$ com anti-windup}}}}
\fontsize{6}{0}
\selectfont\put(52,203.063){\makebox(0,0)[t]{\textcolor[rgb]{0.15,0.15,0.15}{{0}}}}
\fontsize{6}{0}
\selectfont\put(103.675,203.063){\makebox(0,0)[t]{\textcolor[rgb]{0.15,0.15,0.15}{{10}}}}
\fontsize{6}{0}
\selectfont\put(155.351,203.063){\makebox(0,0)[t]{\textcolor[rgb]{0.15,0.15,0.15}{{20}}}}
\fontsize{6}{0}
\selectfont\put(207.026,203.063){\makebox(0,0)[t]{\textcolor[rgb]{0.15,0.15,0.15}{{30}}}}
\fontsize{6}{0}
\selectfont\put(258.701,203.063){\makebox(0,0)[t]{\textcolor[rgb]{0.15,0.15,0.15}{{40}}}}
\fontsize{6}{0}
\selectfont\put(310.376,203.063){\makebox(0,0)[t]{\textcolor[rgb]{0.15,0.15,0.15}{{50}}}}
\fontsize{6}{0}
\selectfont\put(48.5278,234.411){\makebox(0,0)[r]{\textcolor[rgb]{0.15,0.15,0.15}{{0}}}}
\fontsize{6}{0}
\selectfont\put(48.5278,265.377){\makebox(0,0)[r]{\textcolor[rgb]{0.15,0.15,0.15}{{0.5}}}}
\fontsize{6}{0}
\selectfont\put(48.5278,296.344){\makebox(0,0)[r]{\textcolor[rgb]{0.15,0.15,0.15}{{1}}}}
\fontsize{7}{0}
\selectfont\put(33.5278,258.512){\rotatebox{90}{\makebox(0,0)[b]{\textcolor[rgb]{0.15,0.15,0.15}{{Erro $e(t)$}}}}}
\fontsize{7}{0}
\selectfont\put(207,192.063){\makebox(0,0)[t]{\textcolor[rgb]{0.15,0.15,0.15}{{Tempo (s)}}}}
\fontsize{6}{0}
\selectfont\put(267,258.795){\makebox(0,0)[l]{\textcolor[rgb]{0,0,0}{{$e_{m1}(t)$ com filtro de referência}}}}
\fontsize{6}{0}
\selectfont\put(267,247.795){\makebox(0,0)[l]{\textcolor[rgb]{0,0,0}{{$e_{m1}(t)$ com controlador I+P}}}}
\fontsize{6}{0}
\selectfont\put(267,236.295){\makebox(0,0)[l]{\textcolor[rgb]{0,0,0}{{$e_{m1}(t)$ com saturação}}}}
\fontsize{6}{0}
\selectfont\put(267,224.295){\makebox(0,0)[l]{\textcolor[rgb]{0,0,0}{{$e_{m1}(t)$ com anti-windup}}}}
\fontsize{6}{0}
\selectfont\put(52,49.769){\makebox(0,0)[t]{\textcolor[rgb]{0.15,0.15,0.15}{{0}}}}
\fontsize{6}{0}
\selectfont\put(103.675,49.769){\makebox(0,0)[t]{\textcolor[rgb]{0.15,0.15,0.15}{{10}}}}
\fontsize{6}{0}
\selectfont\put(155.351,49.769){\makebox(0,0)[t]{\textcolor[rgb]{0.15,0.15,0.15}{{20}}}}
\fontsize{6}{0}
\selectfont\put(207.026,49.769){\makebox(0,0)[t]{\textcolor[rgb]{0.15,0.15,0.15}{{30}}}}
\fontsize{6}{0}
\selectfont\put(258.701,49.769){\makebox(0,0)[t]{\textcolor[rgb]{0.15,0.15,0.15}{{40}}}}
\fontsize{6}{0}
\selectfont\put(310.376,49.769){\makebox(0,0)[t]{\textcolor[rgb]{0.15,0.15,0.15}{{50}}}}
\fontsize{6}{0}
\selectfont\put(48.5278,67.5547){\makebox(0,0)[r]{\textcolor[rgb]{0.15,0.15,0.15}{{0}}}}
\fontsize{6}{0}
\selectfont\put(48.5278,80.1089){\makebox(0,0)[r]{\textcolor[rgb]{0.15,0.15,0.15}{{0.2}}}}
\fontsize{6}{0}
\selectfont\put(48.5278,92.6636){\makebox(0,0)[r]{\textcolor[rgb]{0.15,0.15,0.15}{{0.4}}}}
\fontsize{6}{0}
\selectfont\put(48.5278,105.218){\makebox(0,0)[r]{\textcolor[rgb]{0.15,0.15,0.15}{{0.6}}}}
\fontsize{6}{0}
\selectfont\put(48.5278,117.772){\makebox(0,0)[r]{\textcolor[rgb]{0.15,0.15,0.15}{{0.8}}}}
\fontsize{6}{0}
\selectfont\put(48.5278,130.327){\makebox(0,0)[r]{\textcolor[rgb]{0.15,0.15,0.15}{{1}}}}
\fontsize{6}{0}
\selectfont\put(48.5278,142.882){\makebox(0,0)[r]{\textcolor[rgb]{0.15,0.15,0.15}{{1.2}}}}
\fontsize{7}{0}
\selectfont\put(33.5278,105.218){\rotatebox{90}{\makebox(0,0)[b]{\textcolor[rgb]{0.15,0.15,0.15}{{Sinal de Controle $u(t)$}}}}}
\fontsize{7}{0}
\selectfont\put(207,38.769){\makebox(0,0)[t]{\textcolor[rgb]{0.15,0.15,0.15}{{Tempo (s)}}}}
\fontsize{6}{0}
\selectfont\put(267,105.501){\makebox(0,0)[l]{\textcolor[rgb]{0,0,0}{{$u_{m1}(t)$ com filtro de referência}}}}
\fontsize{6}{0}
\selectfont\put(267,94.501){\makebox(0,0)[l]{\textcolor[rgb]{0,0,0}{{$u_{m1}(t)$ com controlador I+P}}}}
\fontsize{6}{0}
\selectfont\put(267,83.001){\makebox(0,0)[l]{\textcolor[rgb]{0,0,0}{{$u_{m1}(t)$ com saturação}}}}
\fontsize{6}{0}
\selectfont\put(267,71.0005){\makebox(0,0)[l]{\textcolor[rgb]{0,0,0}{{$u_{m1}(t)$ com anti-windup}}}}
\end{picture}

    \end{minipage}
\end{figure}

\begin{figure}[!ht]
    \caption{Simulação do sistema de controle em malhada fechada para condições
    impostas no Cenário 1.}
    \vspace{-10pt}
    \hspace{-30pt}
    \label{fig:resultado-extra-desafio2-questao6}
    \begin{minipage}{\linewidth}
        % Title: gl2ps_renderer figure
% Creator: GL2PS 1.4.0, (C) 1999-2017 C. Geuzaine
% For: Octave
% CreationDate: Sat Sep 25 23:05:31 2021
\setlength{\unitlength}{1pt}
\begin{picture}(0,0)
\includegraphics{images/challenge2/resultado-extra-questao-6-9-inc}
\end{picture}%
\begin{picture}(400,500)(0,0)
\fontsize{6}{0}
\selectfont\put(52,49.7642){\makebox(0,0)[t]{\textcolor[rgb]{0.15,0.15,0.15}{{0}}}}
\fontsize{6}{0}
\selectfont\put(103.675,49.7642){\makebox(0,0)[t]{\textcolor[rgb]{0.15,0.15,0.15}{{10}}}}
\fontsize{6}{0}
\selectfont\put(155.351,49.7642){\makebox(0,0)[t]{\textcolor[rgb]{0.15,0.15,0.15}{{20}}}}
\fontsize{6}{0}
\selectfont\put(207.026,49.7642){\makebox(0,0)[t]{\textcolor[rgb]{0.15,0.15,0.15}{{30}}}}
\fontsize{6}{0}
\selectfont\put(258.701,49.7642){\makebox(0,0)[t]{\textcolor[rgb]{0.15,0.15,0.15}{{40}}}}
\fontsize{6}{0}
\selectfont\put(310.376,49.7642){\makebox(0,0)[t]{\textcolor[rgb]{0.15,0.15,0.15}{{50}}}}
\fontsize{6}{0}
\selectfont\put(48.5278,55){\makebox(0,0)[r]{\textcolor[rgb]{0.15,0.15,0.15}{{-0.2}}}}
\fontsize{6}{0}
\selectfont\put(48.5278,63.5454){\makebox(0,0)[r]{\textcolor[rgb]{0.15,0.15,0.15}{{0}}}}
\fontsize{6}{0}
\selectfont\put(48.5278,72.0908){\makebox(0,0)[r]{\textcolor[rgb]{0.15,0.15,0.15}{{0.2}}}}
\fontsize{6}{0}
\selectfont\put(48.5278,80.6362){\makebox(0,0)[r]{\textcolor[rgb]{0.15,0.15,0.15}{{0.4}}}}
\fontsize{6}{0}
\selectfont\put(48.5278,89.1812){\makebox(0,0)[r]{\textcolor[rgb]{0.15,0.15,0.15}{{0.6}}}}
\fontsize{6}{0}
\selectfont\put(48.5278,97.7266){\makebox(0,0)[r]{\textcolor[rgb]{0.15,0.15,0.15}{{0.8}}}}
\fontsize{6}{0}
\selectfont\put(48.5278,106.272){\makebox(0,0)[r]{\textcolor[rgb]{0.15,0.15,0.15}{{1}}}}
\fontsize{6}{0}
\selectfont\put(48.5278,114.817){\makebox(0,0)[r]{\textcolor[rgb]{0.15,0.15,0.15}{{1.2}}}}
\fontsize{6}{0}
\selectfont\put(48.5278,123.363){\makebox(0,0)[r]{\textcolor[rgb]{0.15,0.15,0.15}{{1.4}}}}
\fontsize{7}{0}
\selectfont\put(31.5278,89.1812){\rotatebox{90}{\makebox(0,0)[b]{\textcolor[rgb]{0.15,0.15,0.15}{{Valor no instante $t$}}}}}
\fontsize{7}{0}
\selectfont\put(207,38.7642){\makebox(0,0)[t]{\textcolor[rgb]{0.15,0.15,0.15}{{Tempo (s)}}}}
\fontsize{6}{0}
\selectfont\put(296,106.001){\makebox(0,0)[l]{\textcolor[rgb]{0,0,0}{{$|u_{d}(t) - u(t)|$}}}}
\fontsize{6}{0}
\selectfont\put(296,94.001){\makebox(0,0)[l]{\textcolor[rgb]{0,0,0}{{$u_{d}(t)$}}}}
\fontsize{6}{0}
\selectfont\put(296,82.501){\makebox(0,0)[l]{\textcolor[rgb]{0,0,0}{{$u(t)$}}}}
\fontsize{6}{0}
\selectfont\put(296,71.0005){\makebox(0,0)[l]{\textcolor[rgb]{0,0,0}{{anti-windup habilitado}}}}
\fontsize{6}{0}
\selectfont\put(52,388.426){\makebox(0,0)[t]{\textcolor[rgb]{0.15,0.15,0.15}{{0}}}}
\fontsize{6}{0}
\selectfont\put(103.675,388.426){\makebox(0,0)[t]{\textcolor[rgb]{0.15,0.15,0.15}{{10}}}}
\fontsize{6}{0}
\selectfont\put(155.351,388.426){\makebox(0,0)[t]{\textcolor[rgb]{0.15,0.15,0.15}{{20}}}}
\fontsize{6}{0}
\selectfont\put(207.026,388.426){\makebox(0,0)[t]{\textcolor[rgb]{0.15,0.15,0.15}{{30}}}}
\fontsize{6}{0}
\selectfont\put(258.701,388.426){\makebox(0,0)[t]{\textcolor[rgb]{0.15,0.15,0.15}{{40}}}}
\fontsize{6}{0}
\selectfont\put(310.376,388.426){\makebox(0,0)[t]{\textcolor[rgb]{0.15,0.15,0.15}{{50}}}}
\fontsize{6}{0}
\selectfont\put(48.5278,393.662){\makebox(0,0)[r]{\textcolor[rgb]{0.15,0.15,0.15}{{-0.2}}}}
\fontsize{6}{0}
\selectfont\put(48.5278,402.435){\makebox(0,0)[r]{\textcolor[rgb]{0.15,0.15,0.15}{{0}}}}
\fontsize{6}{0}
\selectfont\put(48.5278,411.208){\makebox(0,0)[r]{\textcolor[rgb]{0.15,0.15,0.15}{{0.2}}}}
\fontsize{6}{0}
\selectfont\put(48.5278,419.982){\makebox(0,0)[r]{\textcolor[rgb]{0.15,0.15,0.15}{{0.4}}}}
\fontsize{6}{0}
\selectfont\put(48.5278,428.756){\makebox(0,0)[r]{\textcolor[rgb]{0.15,0.15,0.15}{{0.6}}}}
\fontsize{6}{0}
\selectfont\put(48.5278,437.529){\makebox(0,0)[r]{\textcolor[rgb]{0.15,0.15,0.15}{{0.8}}}}
\fontsize{6}{0}
\selectfont\put(48.5278,446.303){\makebox(0,0)[r]{\textcolor[rgb]{0.15,0.15,0.15}{{1}}}}
\fontsize{6}{0}
\selectfont\put(48.5278,455.076){\makebox(0,0)[r]{\textcolor[rgb]{0.15,0.15,0.15}{{1.2}}}}
\fontsize{7}{0}
\selectfont\put(31.5278,427.843){\rotatebox{90}{\makebox(0,0)[b]{\textcolor[rgb]{0.15,0.15,0.15}{{Saida $y(t)$}}}}}
\fontsize{7}{0}
\selectfont\put(207,377.426){\makebox(0,0)[t]{\textcolor[rgb]{0.15,0.15,0.15}{{Tempo (s)}}}}
\fontsize{6}{0}
\selectfont\put(288,421.163){\makebox(0,0)[l]{\textcolor[rgb]{0,0,0}{{r(t)}}}}
\fontsize{6}{0}
\selectfont\put(288,409.662){\makebox(0,0)[l]{\textcolor[rgb]{0,0,0}{{$y_{1}(t)$ com anti windup}}}}
\fontsize{6}{0}
\selectfont\put(52,162.651){\makebox(0,0)[t]{\textcolor[rgb]{0.15,0.15,0.15}{{0}}}}
\fontsize{6}{0}
\selectfont\put(103.675,162.651){\makebox(0,0)[t]{\textcolor[rgb]{0.15,0.15,0.15}{{10}}}}
\fontsize{6}{0}
\selectfont\put(155.351,162.651){\makebox(0,0)[t]{\textcolor[rgb]{0.15,0.15,0.15}{{20}}}}
\fontsize{6}{0}
\selectfont\put(207.026,162.651){\makebox(0,0)[t]{\textcolor[rgb]{0.15,0.15,0.15}{{30}}}}
\fontsize{6}{0}
\selectfont\put(258.701,162.651){\makebox(0,0)[t]{\textcolor[rgb]{0.15,0.15,0.15}{{40}}}}
\fontsize{6}{0}
\selectfont\put(310.376,162.651){\makebox(0,0)[t]{\textcolor[rgb]{0.15,0.15,0.15}{{50}}}}
\fontsize{6}{0}
\selectfont\put(48.5278,167.887){\makebox(0,0)[r]{\textcolor[rgb]{0.15,0.15,0.15}{{-0.2}}}}
\fontsize{6}{0}
\selectfont\put(48.5278,179.281){\makebox(0,0)[r]{\textcolor[rgb]{0.15,0.15,0.15}{{0}}}}
\fontsize{6}{0}
\selectfont\put(48.5278,190.675){\makebox(0,0)[r]{\textcolor[rgb]{0.15,0.15,0.15}{{0.2}}}}
\fontsize{6}{0}
\selectfont\put(48.5278,202.068){\makebox(0,0)[r]{\textcolor[rgb]{0.15,0.15,0.15}{{0.4}}}}
\fontsize{6}{0}
\selectfont\put(48.5278,213.462){\makebox(0,0)[r]{\textcolor[rgb]{0.15,0.15,0.15}{{0.6}}}}
\fontsize{6}{0}
\selectfont\put(48.5278,224.856){\makebox(0,0)[r]{\textcolor[rgb]{0.15,0.15,0.15}{{0.8}}}}
\fontsize{6}{0}
\selectfont\put(48.5278,236.25){\makebox(0,0)[r]{\textcolor[rgb]{0.15,0.15,0.15}{{1}}}}
\fontsize{7}{0}
\selectfont\put(31.5278,202.068){\rotatebox{90}{\makebox(0,0)[b]{\textcolor[rgb]{0.15,0.15,0.15}{{Sinal de Controle $u(t)$}}}}}
\fontsize{7}{0}
\selectfont\put(207,151.651){\makebox(0,0)[t]{\textcolor[rgb]{0.15,0.15,0.15}{{Tempo (s)}}}}
\fontsize{6}{0}
\selectfont\put(288,183.888){\makebox(0,0)[l]{\textcolor[rgb]{0,0,0}{{$u_{1}(t)$ com anti windup}}}}
\fontsize{6}{0}
\selectfont\put(52,275.539){\makebox(0,0)[t]{\textcolor[rgb]{0.15,0.15,0.15}{{0}}}}
\fontsize{6}{0}
\selectfont\put(103.675,275.539){\makebox(0,0)[t]{\textcolor[rgb]{0.15,0.15,0.15}{{10}}}}
\fontsize{6}{0}
\selectfont\put(155.351,275.539){\makebox(0,0)[t]{\textcolor[rgb]{0.15,0.15,0.15}{{20}}}}
\fontsize{6}{0}
\selectfont\put(207.026,275.539){\makebox(0,0)[t]{\textcolor[rgb]{0.15,0.15,0.15}{{30}}}}
\fontsize{6}{0}
\selectfont\put(258.701,275.539){\makebox(0,0)[t]{\textcolor[rgb]{0.15,0.15,0.15}{{40}}}}
\fontsize{6}{0}
\selectfont\put(310.376,275.539){\makebox(0,0)[t]{\textcolor[rgb]{0.15,0.15,0.15}{{50}}}}
\fontsize{6}{0}
\selectfont\put(48.5278,287.722){\makebox(0,0)[r]{\textcolor[rgb]{0.15,0.15,0.15}{{-0.2}}}}
\fontsize{6}{0}
\selectfont\put(48.5278,296.496){\makebox(0,0)[r]{\textcolor[rgb]{0.15,0.15,0.15}{{0}}}}
\fontsize{6}{0}
\selectfont\put(48.5278,305.27){\makebox(0,0)[r]{\textcolor[rgb]{0.15,0.15,0.15}{{0.2}}}}
\fontsize{6}{0}
\selectfont\put(48.5278,314.043){\makebox(0,0)[r]{\textcolor[rgb]{0.15,0.15,0.15}{{0.4}}}}
\fontsize{6}{0}
\selectfont\put(48.5278,322.816){\makebox(0,0)[r]{\textcolor[rgb]{0.15,0.15,0.15}{{0.6}}}}
\fontsize{6}{0}
\selectfont\put(48.5278,331.59){\makebox(0,0)[r]{\textcolor[rgb]{0.15,0.15,0.15}{{0.8}}}}
\fontsize{6}{0}
\selectfont\put(48.5278,340.364){\makebox(0,0)[r]{\textcolor[rgb]{0.15,0.15,0.15}{{1}}}}
\fontsize{7}{0}
\selectfont\put(31.5278,314.956){\rotatebox{90}{\makebox(0,0)[b]{\textcolor[rgb]{0.15,0.15,0.15}{{Erro $e(t)$}}}}}
\fontsize{7}{0}
\selectfont\put(207,264.539){\makebox(0,0)[t]{\textcolor[rgb]{0.15,0.15,0.15}{{Tempo (s)}}}}
\fontsize{6}{0}
\selectfont\put(335,296.275){\makebox(0,0)[l]{\textcolor[rgb]{0,0,0}{{$e(t)$}}}}
\end{picture}

    \end{minipage}
\end{figure}
\section{Desafio III - Resposta em Frequência} 

\subsection{Motivação}
Este último desafio da série visa reduzir a abstração do critério de Nyquist,
evidenciando como ele pode ser aplicado no projeto de controladores no domínio
da frequência. Além disso, o uso da resposta em frequência junto com o critério
de Nyquist permite que os controladores sejam projetados apenas com a função de
transferência de malha aberta, sem precisar também do conceito de dominância
modal. Portanto, poder projetar controladores no domínio da frequência é mais
uma ferramenta que o projetista tem em mãos na hora de decidir qual abordagem
seguir na resolução de um determinado problema de controle. Os campos que se
beneficiam dessa abordagem são principalmente aqueles que envolvam eletrônica de
potência, como o controle de tensão na saída de conversores DC-DC regulados, e
vibrações. 

\subsection{Simulações realizadas}
As simulações realizadas focaram no design de compensadores de avanço, de atraso
e de avanço-atraso de fase.

Primeiro as simulações foram realizadas para analisar a
resposta em frequência de um compensador em avanço variando primeiro a folga de
fase utilizada no projeto; depois projetando o compensador com o requisito de
largura de banda de malha aberta; e no fim desta primeira parte, analisou o
compensador com diferentes ganhos.

A segunda etapa consistiu na análise no
domínio do tempo da resposta do sistema em malha fechada com os diferentes
compensadores projetados na primeira parte e também com a adição de um filtro de
referência.

Por fim, foi realizada novamente a simulação do sistema em malha fechada,
entretanto, com a adição de um compensador em atraso, ficando o compensador
final com a topologia de avanço-atraso. Também avaliou-se a resposta em malha
fechada com um filtro de referência. Adicionalmente, foi avaliada a resposta do
sistema para uma perturbação do tipo degrau na entrada da planta.

Também como nos dois desafios anteriores, este desafio teve como referência o
sistema em malha fechada ilustrado pela Figura
\ref{fig:diagrama-de-blocos-malha-fechada} e descrita no tópico
\ref{sec:desafio-1-simulacoes-realizadas}. A função de transferência da
planta/processo utilizada está definida na Equação \ref{eq:desafio-3:g-de-s}. O
tópico posterior descreve e contém discussões dos resultados obtidos.

\begin{equation}
    \label{eq:desafio-3:g-de-s}
    G(s) = \frac{0,5}{(s^2 + 0,6s +1)(0,1s + 1)}.
\end{equation}

\subsection{Resultados obtidos}
As simulações começaram com a definição de $\overline{K}$ tal que
$\overline{K}G(0) = 1$. Através de álgebra simples, chegou-se a $\overline{K} = 2$.
Com a definição desse valor de ganho, foi calculada a largura de banda de $P(s)
= \overline{K}G(s)$, cujo resultado é ilustrado na Figura
\ref{fig:desafio-3:questao-3}. A largura de banda $w_{b} \approx 1,45$ rad/s.
Vale ressaltar que como foi realizado computação numérica, os valores
serão aproximados devido a discretização da magnitude e fase no diagrama de bode.

\begin{figure}[!ht]
    \caption{Magnitude da resposta em frequência de $P(s)$
    com destaque na largura de banda.}
    \vspace{-10pt}
    \hspace{-30pt}
    \label{fig:desafio-3:questao-3}
    \begin{minipage}{\linewidth}
        % Title: gl2ps_renderer figure
% Creator: GL2PS 1.4.0, (C) 1999-2017 C. Geuzaine
% For: Octave
% CreationDate: Wed Oct  6 21:32:31 2021
\setlength{\unitlength}{1pt}
\begin{picture}(0,0)
\includegraphics{images/challenge3/resultado-questao-2-inc}
\end{picture}%
\begin{picture}(400,120)(0,0)
\fontsize{6}{0}
\selectfont\put(52,23){\makebox(0,0)[t]{\textcolor[rgb]{0.15,0.15,0.15}{{$10^{-4}$}}}}
\fontsize{6}{0}
\selectfont\put(103.667,23){\makebox(0,0)[t]{\textcolor[rgb]{0.15,0.15,0.15}{{$10^{-3}$}}}}
\fontsize{6}{0}
\selectfont\put(155.333,23){\makebox(0,0)[t]{\textcolor[rgb]{0.15,0.15,0.15}{{$10^{-2}$}}}}
\fontsize{6}{0}
\selectfont\put(207,23){\makebox(0,0)[t]{\textcolor[rgb]{0.15,0.15,0.15}{{$10^{-1}$}}}}
\fontsize{6}{0}
\selectfont\put(258.667,23){\makebox(0,0)[t]{\textcolor[rgb]{0.15,0.15,0.15}{{$10^{0}$}}}}
\fontsize{6}{0}
\selectfont\put(310.333,23){\makebox(0,0)[t]{\textcolor[rgb]{0.15,0.15,0.15}{{$10^{1}$}}}}
\fontsize{6}{0}
\selectfont\put(362,23){\makebox(0,0)[t]{\textcolor[rgb]{0.15,0.15,0.15}{{$10^{2}$}}}}
\fontsize{6}{0}
\selectfont\put(48.5278,31.9443){\makebox(0,0)[r]{\textcolor[rgb]{0.15,0.15,0.15}{{-100}}}}
\fontsize{6}{0}
\selectfont\put(48.5278,46.1294){\makebox(0,0)[r]{\textcolor[rgb]{0.15,0.15,0.15}{{-80}}}}
\fontsize{6}{0}
\selectfont\put(48.5278,60.314){\makebox(0,0)[r]{\textcolor[rgb]{0.15,0.15,0.15}{{-60}}}}
\fontsize{6}{0}
\selectfont\put(48.5278,74.499){\makebox(0,0)[r]{\textcolor[rgb]{0.15,0.15,0.15}{{-40}}}}
\fontsize{6}{0}
\selectfont\put(48.5278,88.6841){\makebox(0,0)[r]{\textcolor[rgb]{0.15,0.15,0.15}{{-20}}}}
\fontsize{6}{0}
\selectfont\put(48.5278,102.869){\makebox(0,0)[r]{\textcolor[rgb]{0.15,0.15,0.15}{{0}}}}
\fontsize{7}{0}
\selectfont\put(207,9){\makebox(0,0)[t]{\textcolor[rgb]{0.15,0.15,0.15}{{Frequência [rad/s]}}}}
\fontsize{7}{0}
\selectfont\put(29.5278,69.0977){\rotatebox{90}{\makebox(0,0)[b]{\textcolor[rgb]{0.15,0.15,0.15}{{Magnitude [dB]}}}}}
\fontsize{6}{0}
\selectfont\put(269.878,100.742){\makebox(0,0)[l]{\textcolor[rgb]{0,0,0}{{$|P(w_{b} \approx  1.45)| \approx -3_{db}$}}}}
\fontsize{6}{0}
\selectfont\put(88.002,44.1963){\makebox(0,0)[l]{\textcolor[rgb]{0,0,0}{{$|P(jw)|$}}}}
\end{picture}

    \end{minipage}
\end{figure}

Em seguida foi projetado dois compensadores em avanço, dado pela Equação
\ref{eq:desafio-3:clinha-de-s}, tal que a margem de fase de $C(s)G(s)$ fosse
maior ou igual a $60^{\circ}$, a largura de banda de malha aberta fosse maior
que 2,5 rad/s dado $C(s) = {C}'(s)\overline{K}$ e $K_{c} = 1$. Um controlador
foi projetado considerando uma folga de $12^{\circ}$ e o outro para uma folga de
$24^{\circ}$. Os parâmetros do compensador foram calculados conforme as Equações
\ref{eq:desafio-3:phi-max}, \ref{eq:desafio-3:alpha} e \ref{eq:desafio-3:te}.

\begin{equation}
    \label{eq:desafio-3:clinha-de-s}
    {C}'(s) = K_{c}\frac{Ts + 1}{\alpha Ts + 1}
\end{equation}

\begin{equation}
    \label{eq:desafio-3:phi-max}
    \phi_{max} = mf_{d} - mf_{a} + folga,
\end{equation}
onde:

\begin{conditions*}
    \phi_{max} & a máxima fase que o compensador em avanço necessita ter para
    que o sistema em malha aberta possua a margem de fase desejada; \\
    mf_{d} & a margem de fase desejada; e \\
    mf_{a} & a margem de fase atual. \\
\end{conditions*}

\begin{equation}
    \label{eq:desafio-3:alpha}
    \alpha = \frac{1 - \sin(\phi_{max})}{1 + \sin(\phi_{max})},
\end{equation}

\begin{equation}
    \label{eq:desafio-3:te}
    T = \frac{1}{w_{\phi_{max}}\sqrt{\alpha}}
\end{equation}

Para o compensador com menor folga foi encontrado $\phi_{max} = 28,72^{\circ}$,
$\alpha = 0.35$, $T = 1,09$ e $w_{b} = 1,81$ rad/s. Já para o segundo
compensador $\phi_{max} = 40,72^{\circ}$, $\alpha = 0,21$, $T = 1,28$ e $w_{b} =
2,04$ rad/s. A resposta em frequência dos dois compensadores projetados é
expressa na Figura \ref{fig:desafio-3:questao-3-4-compensadores}.

\begin{figure}[!ht]
    \caption{Compensador em avanço ${C}'(s)$ projetado com folgas de
    $12^{\circ}$ e $24^{\circ}$.}
    \vspace{-10pt}
    \hspace{-30pt}
    \label{fig:desafio-3:questao-3-4-compensadores}
    \begin{minipage}{\linewidth}
        % Title: gl2ps_renderer figure
% Creator: GL2PS 1.4.0, (C) 1999-2017 C. Geuzaine
% For: Octave
% CreationDate: Wed Oct  6 21:11:28 2021
\setlength{\unitlength}{1pt}
\begin{picture}(0,0)
\includegraphics{images/challenge3/resultado-questao-3-4-compensadores-inc}
\end{picture}%
\begin{picture}(400,270)(0,0)
\fontsize{6}{0}
\selectfont\put(52,153.188){\makebox(0,0)[t]{\textcolor[rgb]{0.15,0.15,0.15}{{$10^{-4}$}}}}
\fontsize{6}{0}
\selectfont\put(103.667,153.188){\makebox(0,0)[t]{\textcolor[rgb]{0.15,0.15,0.15}{{$10^{-3}$}}}}
\fontsize{6}{0}
\selectfont\put(155.333,153.188){\makebox(0,0)[t]{\textcolor[rgb]{0.15,0.15,0.15}{{$10^{-2}$}}}}
\fontsize{6}{0}
\selectfont\put(207,153.188){\makebox(0,0)[t]{\textcolor[rgb]{0.15,0.15,0.15}{{$10^{-1}$}}}}
\fontsize{6}{0}
\selectfont\put(258.667,153.188){\makebox(0,0)[t]{\textcolor[rgb]{0.15,0.15,0.15}{{$10^{0}$}}}}
\fontsize{6}{0}
\selectfont\put(310.333,153.188){\makebox(0,0)[t]{\textcolor[rgb]{0.15,0.15,0.15}{{$10^{1}$}}}}
\fontsize{6}{0}
\selectfont\put(362,153.188){\makebox(0,0)[t]{\textcolor[rgb]{0.15,0.15,0.15}{{$10^{2}$}}}}
\fontsize{6}{0}
\selectfont\put(48.5278,158.368){\makebox(0,0)[r]{\textcolor[rgb]{0.15,0.15,0.15}{{0}}}}
\fontsize{6}{0}
\selectfont\put(48.5278,170.729){\makebox(0,0)[r]{\textcolor[rgb]{0.15,0.15,0.15}{{2}}}}
\fontsize{6}{0}
\selectfont\put(48.5278,183.089){\makebox(0,0)[r]{\textcolor[rgb]{0.15,0.15,0.15}{{4}}}}
\fontsize{6}{0}
\selectfont\put(48.5278,195.45){\makebox(0,0)[r]{\textcolor[rgb]{0.15,0.15,0.15}{{6}}}}
\fontsize{6}{0}
\selectfont\put(48.5278,207.811){\makebox(0,0)[r]{\textcolor[rgb]{0.15,0.15,0.15}{{8}}}}
\fontsize{6}{0}
\selectfont\put(48.5278,220.172){\makebox(0,0)[r]{\textcolor[rgb]{0.15,0.15,0.15}{{10}}}}
\fontsize{6}{0}
\selectfont\put(48.5278,232.532){\makebox(0,0)[r]{\textcolor[rgb]{0.15,0.15,0.15}{{12}}}}
\fontsize{6}{0}
\selectfont\put(48.5278,244.893){\makebox(0,0)[r]{\textcolor[rgb]{0.15,0.15,0.15}{{14}}}}
\fontsize{7}{0}
\selectfont\put(207,139.188){\makebox(0,0)[t]{\textcolor[rgb]{0.15,0.15,0.15}{{Frequência [rad/s]}}}}
\fontsize{7}{0}
\selectfont\put(35.5278,201.631){\rotatebox{90}{\makebox(0,0)[b]{\textcolor[rgb]{0.15,0.15,0.15}{{Magnitude [db]}}}}}
\fontsize{6}{0}
\selectfont\put(88.002,228.893){\makebox(0,0)[l]{\textcolor[rgb]{0,0,0}{{${C}'(jw)$ p/ folga $= 12^{\circ}$}}}}
\fontsize{6}{0}
\selectfont\put(88.002,216.892){\makebox(0,0)[l]{\textcolor[rgb]{0,0,0}{{${C}'(jw)$ p/ folga $= 24^{\circ}$}}}}
\fontsize{6}{0}
\selectfont\put(52,24.5205){\makebox(0,0)[t]{\textcolor[rgb]{0.15,0.15,0.15}{{$10^{-4}$}}}}
\fontsize{6}{0}
\selectfont\put(103.667,24.5205){\makebox(0,0)[t]{\textcolor[rgb]{0.15,0.15,0.15}{{$10^{-3}$}}}}
\fontsize{6}{0}
\selectfont\put(155.333,24.5205){\makebox(0,0)[t]{\textcolor[rgb]{0.15,0.15,0.15}{{$10^{-2}$}}}}
\fontsize{6}{0}
\selectfont\put(207,24.5205){\makebox(0,0)[t]{\textcolor[rgb]{0.15,0.15,0.15}{{$10^{-1}$}}}}
\fontsize{6}{0}
\selectfont\put(258.667,24.5205){\makebox(0,0)[t]{\textcolor[rgb]{0.15,0.15,0.15}{{$10^{0}$}}}}
\fontsize{6}{0}
\selectfont\put(310.333,24.5205){\makebox(0,0)[t]{\textcolor[rgb]{0.15,0.15,0.15}{{$10^{1}$}}}}
\fontsize{6}{0}
\selectfont\put(362,24.5205){\makebox(0,0)[t]{\textcolor[rgb]{0.15,0.15,0.15}{{$10^{2}$}}}}
\fontsize{6}{0}
\selectfont\put(48.5278,29.7002){\makebox(0,0)[r]{\textcolor[rgb]{0.15,0.15,0.15}{{0}}}}
\fontsize{6}{0}
\selectfont\put(48.5278,47.0049){\makebox(0,0)[r]{\textcolor[rgb]{0.15,0.15,0.15}{{10}}}}
\fontsize{6}{0}
\selectfont\put(48.5278,64.3101){\makebox(0,0)[r]{\textcolor[rgb]{0.15,0.15,0.15}{{20}}}}
\fontsize{6}{0}
\selectfont\put(48.5278,81.6152){\makebox(0,0)[r]{\textcolor[rgb]{0.15,0.15,0.15}{{30}}}}
\fontsize{6}{0}
\selectfont\put(48.5278,98.9199){\makebox(0,0)[r]{\textcolor[rgb]{0.15,0.15,0.15}{{40}}}}
\fontsize{6}{0}
\selectfont\put(48.5278,116.225){\makebox(0,0)[r]{\textcolor[rgb]{0.15,0.15,0.15}{{50}}}}
\fontsize{7}{0}
\selectfont\put(207,10.5205){\makebox(0,0)[t]{\textcolor[rgb]{0.15,0.15,0.15}{{Frequência [rad/s]}}}}
\fontsize{7}{0}
\selectfont\put(35.5278,72.9624){\rotatebox{90}{\makebox(0,0)[b]{\textcolor[rgb]{0.15,0.15,0.15}{{Fase [graus]}}}}}
\fontsize{6}{0}
\selectfont\put(88.002,57.7007){\makebox(0,0)[l]{\textcolor[rgb]{0,0,0}{{${C}'(jw)$ p/ folga $= 12^{\circ}$}}}}
\fontsize{6}{0}
\selectfont\put(88.002,45.7007){\makebox(0,0)[l]{\textcolor[rgb]{0,0,0}{{${C}'(jw)$ p/ folga $= 24^{\circ}$}}}}
\end{picture}

    \end{minipage}
\end{figure}

Na Figura \ref{fig:desafio-3:questao-3-4-compensadores} é possível visualizar
que o controlador projetado com maior folga foi o que obteve maior pico de fase,
condizente com a Equação \ref{eq:desafio-3:phi-max}. O compensador com folga de
$24^{\circ}$ também possui maior ganho na frequência de pico de fase. Para
avaliar a contribuição dos dois controladores em $G(s)$, foi obtida a resposta
em frequência $C(s)G(s)$ conforme Figura
\ref{fig:desafio-3:questao-3-4-malha-aberta}.

\begin{figure}[!ht]
    \caption{Resposta em frequência de ${C}'(s)\overline{K}G(s)$ com compensador
    em avanço ${C}'(s)$ projetado com folgas de $12^{\circ}$ e $24^{\circ}$.}
    \vspace{-10pt}
    \hspace{-30pt}
    \label{fig:desafio-3:questao-3-4-malha-aberta}
    \begin{minipage}{\linewidth}
        % Title: gl2ps_renderer figure
% Creator: GL2PS 1.4.0, (C) 1999-2017 C. Geuzaine
% For: Octave
% CreationDate: Wed Oct  6 21:53:16 2021
\setlength{\unitlength}{1pt}
\begin{picture}(0,0)
\includegraphics{images/challenge3/resultado-questao-3-4-malha-aberta-inc}
\end{picture}%
\begin{picture}(400,270)(0,0)
\fontsize{6}{0}
\selectfont\put(52,153.188){\makebox(0,0)[t]{\textcolor[rgb]{0.15,0.15,0.15}{{$10^{-4}$}}}}
\fontsize{6}{0}
\selectfont\put(103.667,153.188){\makebox(0,0)[t]{\textcolor[rgb]{0.15,0.15,0.15}{{$10^{-3}$}}}}
\fontsize{6}{0}
\selectfont\put(155.333,153.188){\makebox(0,0)[t]{\textcolor[rgb]{0.15,0.15,0.15}{{$10^{-2}$}}}}
\fontsize{6}{0}
\selectfont\put(207,153.188){\makebox(0,0)[t]{\textcolor[rgb]{0.15,0.15,0.15}{{$10^{-1}$}}}}
\fontsize{6}{0}
\selectfont\put(258.667,153.188){\makebox(0,0)[t]{\textcolor[rgb]{0.15,0.15,0.15}{{$10^{0}$}}}}
\fontsize{6}{0}
\selectfont\put(310.333,153.188){\makebox(0,0)[t]{\textcolor[rgb]{0.15,0.15,0.15}{{$10^{1}$}}}}
\fontsize{6}{0}
\selectfont\put(362,153.188){\makebox(0,0)[t]{\textcolor[rgb]{0.15,0.15,0.15}{{$10^{2}$}}}}
\fontsize{6}{0}
\selectfont\put(48.5278,158.368){\makebox(0,0)[r]{\textcolor[rgb]{0.15,0.15,0.15}{{-120}}}}
\fontsize{6}{0}
\selectfont\put(48.5278,170.729){\makebox(0,0)[r]{\textcolor[rgb]{0.15,0.15,0.15}{{-100}}}}
\fontsize{6}{0}
\selectfont\put(48.5278,183.089){\makebox(0,0)[r]{\textcolor[rgb]{0.15,0.15,0.15}{{-80}}}}
\fontsize{6}{0}
\selectfont\put(48.5278,195.45){\makebox(0,0)[r]{\textcolor[rgb]{0.15,0.15,0.15}{{-60}}}}
\fontsize{6}{0}
\selectfont\put(48.5278,207.811){\makebox(0,0)[r]{\textcolor[rgb]{0.15,0.15,0.15}{{-40}}}}
\fontsize{6}{0}
\selectfont\put(48.5278,220.172){\makebox(0,0)[r]{\textcolor[rgb]{0.15,0.15,0.15}{{-20}}}}
\fontsize{6}{0}
\selectfont\put(48.5278,232.532){\makebox(0,0)[r]{\textcolor[rgb]{0.15,0.15,0.15}{{0}}}}
\fontsize{6}{0}
\selectfont\put(48.5278,244.893){\makebox(0,0)[r]{\textcolor[rgb]{0.15,0.15,0.15}{{20}}}}
\fontsize{7}{0}
\selectfont\put(207,139.188){\makebox(0,0)[t]{\textcolor[rgb]{0.15,0.15,0.15}{{Frequência [rad/s]}}}}
\fontsize{7}{0}
\selectfont\put(29.5278,201.631){\rotatebox{90}{\makebox(0,0)[b]{\textcolor[rgb]{0.15,0.15,0.15}{{Ganho [db]}}}}}
\fontsize{6}{0}
\selectfont\put(88.002,198.369){\makebox(0,0)[l]{\textcolor[rgb]{0,0,0}{{$\overline{K}G(jw)$}}}}
\fontsize{6}{0}
\selectfont\put(88.002,186.369){\makebox(0,0)[l]{\textcolor[rgb]{0,0,0}{{$C(jw)G(jw)$ p/ folga $= 12^{\circ}$}}}}
\fontsize{6}{0}
\selectfont\put(88.002,174.369){\makebox(0,0)[l]{\textcolor[rgb]{0,0,0}{{$C(jw)G(jw)$ p/ folga $= 24^{\circ}$}}}}
\fontsize{6}{0}
\selectfont\put(52,24.5205){\makebox(0,0)[t]{\textcolor[rgb]{0.15,0.15,0.15}{{$10^{-4}$}}}}
\fontsize{6}{0}
\selectfont\put(103.667,24.5205){\makebox(0,0)[t]{\textcolor[rgb]{0.15,0.15,0.15}{{$10^{-3}$}}}}
\fontsize{6}{0}
\selectfont\put(155.333,24.5205){\makebox(0,0)[t]{\textcolor[rgb]{0.15,0.15,0.15}{{$10^{-2}$}}}}
\fontsize{6}{0}
\selectfont\put(207,24.5205){\makebox(0,0)[t]{\textcolor[rgb]{0.15,0.15,0.15}{{$10^{-1}$}}}}
\fontsize{6}{0}
\selectfont\put(258.667,24.5205){\makebox(0,0)[t]{\textcolor[rgb]{0.15,0.15,0.15}{{$10^{0}$}}}}
\fontsize{6}{0}
\selectfont\put(310.333,24.5205){\makebox(0,0)[t]{\textcolor[rgb]{0.15,0.15,0.15}{{$10^{1}$}}}}
\fontsize{6}{0}
\selectfont\put(362,24.5205){\makebox(0,0)[t]{\textcolor[rgb]{0.15,0.15,0.15}{{$10^{2}$}}}}
\fontsize{6}{0}
\selectfont\put(48.5278,29.7002){\makebox(0,0)[r]{\textcolor[rgb]{0.15,0.15,0.15}{{-300}}}}
\fontsize{6}{0}
\selectfont\put(48.5278,42.0605){\makebox(0,0)[r]{\textcolor[rgb]{0.15,0.15,0.15}{{-250}}}}
\fontsize{6}{0}
\selectfont\put(48.5278,54.4214){\makebox(0,0)[r]{\textcolor[rgb]{0.15,0.15,0.15}{{-200}}}}
\fontsize{6}{0}
\selectfont\put(48.5278,66.7822){\makebox(0,0)[r]{\textcolor[rgb]{0.15,0.15,0.15}{{-150}}}}
\fontsize{6}{0}
\selectfont\put(48.5278,79.1431){\makebox(0,0)[r]{\textcolor[rgb]{0.15,0.15,0.15}{{-100}}}}
\fontsize{6}{0}
\selectfont\put(48.5278,91.5034){\makebox(0,0)[r]{\textcolor[rgb]{0.15,0.15,0.15}{{-50}}}}
\fontsize{6}{0}
\selectfont\put(48.5278,103.864){\makebox(0,0)[r]{\textcolor[rgb]{0.15,0.15,0.15}{{0}}}}
\fontsize{6}{0}
\selectfont\put(48.5278,116.225){\makebox(0,0)[r]{\textcolor[rgb]{0.15,0.15,0.15}{{50}}}}
\fontsize{7}{0}
\selectfont\put(207,10.5205){\makebox(0,0)[t]{\textcolor[rgb]{0.15,0.15,0.15}{{Frequência [rad/s]}}}}
\fontsize{7}{0}
\selectfont\put(29.5278,72.9624){\rotatebox{90}{\makebox(0,0)[b]{\textcolor[rgb]{0.15,0.15,0.15}{{Fase [graus]}}}}}
\fontsize{6}{0}
\selectfont\put(88.002,69.7012){\makebox(0,0)[l]{\textcolor[rgb]{0,0,0}{{$\overline{K}G(jw)$}}}}
\fontsize{6}{0}
\selectfont\put(88.002,57.7012){\makebox(0,0)[l]{\textcolor[rgb]{0,0,0}{{$C(jw)G(jw)$ p/ folga $= 12^{\circ}$}}}}
\fontsize{6}{0}
\selectfont\put(88.002,45.7007){\makebox(0,0)[l]{\textcolor[rgb]{0,0,0}{{$C(jw)G(jw)$ p/ folga $= 24^{\circ}$}}}}
\end{picture}

    \end{minipage}
\end{figure}

Como realçado nas linhas verticais do gráfico de fase na Figura
\ref{fig:desafio-3:questao-3-4-malha-aberta}, o sistema em malha aberta
$\overline{K}G(s)$ é o que possui menor margem de fase ($M_{f} =
43,28^{\circ}$), conforme esperado. A margem de fase é melhorada com a adição do
compensador projetado com a folga de $12^{\circ}$, atingindo uma $M_{f} =
53,70^{\circ}$, mas que ainda possui uma diferença considerável da margem de
fase desejada ($60^{\circ}$). Por outro lado, o compensador projetado com uma
folga de $24^{\circ}$ contribuiu com maior fase e portanto a margem de fase de
$C(s)G(s)$ atingiu $59,44^{\circ}$, muito mais próxima ao valor desejado. A
melhor contribuição do segundo projeto pode ser explicado devido as incertezas
no projeto de compensador no domínio da frequência. É incerto em que valor de
frequência ocorrerá o pico de fase do compensador quando ele é colocado junto
com a planta/processo. O pico pode ficar tanto antes quanto depois do ponto de
frequência desejável. Dessa forma, utilizando uma folga de $24^{\circ}$ permite
que caso o pico de fase do compensador ocorra antes do valor de frequência
desejado, o compensador ainda contribua com um valor alto de fase para o sistema
em malha aberta. Além disso, caso o pico ocorra após a frequência desejada, o
projeto vai ser ainda mais conservador, já que a margem de fase vai ser superior
a desejada. Portanto, o projeto usando uma folda de $24^{\circ}$ acaba sendo
mais conservador se comparado ao de $12^{\circ}$.

Embora o segundo compensador tenha coontribuido efetivamente para o requisito de
margem de fase, o mesmo não foi obtido para largura de banda de malha aberta.
Nessa linha, o déficit do projeto usando a folga de $12^{\circ}$ é evidente já
que ele também não contribuiu para o sistema em malha aberta para atingir a
largura de banda requerida. Diante disso, visando aumentar a largura de banda de
$C(s)G(s)$, foi calculado o valor de $K_{c}$ tal que $w_{b} = 2,5$ rad/s. O
valor de ganho do compensador foi calculado conforme Equação
\ref{eq:desafio3:calculo-de-kc}, obtendo assim $K_c = 1,43$.

\begin{equation}
    \label{eq:desafio3:calculo-de-kc}
    K_{c} = \left. \frac{1}{\sqrt{2}(|{C}'(jw)\overline{K}G(jw)|)} \right |_{w = 2,5}
\end{equation}

A inserção do ganho no compensador de avanço desloca todo os valores de
magnitude para a direita. Como consequência a frequência de cruzamento de ganho
também é deslocada para direita. Entretanto, a fase não é deslocada na inserção
de um ganho estático. A consequência é que a margem de fase do sistema em
malha aberta é reduzida, que para esse caso à um valor de
$49,49^{\circ}$. Este fato fica evidente quando mostrado de forma gráfica na
Figura \ref{fig:desafio-3:questao-5-6-malha-aberta}.

\begin{figure}[ht!]
    \caption{Comparação da resposta em frequência de ${C}'(s)\overline{K}G(s)$ com
    diferentes valores de ganho estático do compensador e largura de banda em
    malha aberta.}
    \vspace{-10pt}
    \hspace{-30pt}
    \label{fig:desafio-3:questao-5-6-malha-aberta}
    \begin{minipage}{\linewidth}
        % Title: gl2ps_renderer figure
% Creator: GL2PS 1.4.0, (C) 1999-2017 C. Geuzaine
% For: Octave
% CreationDate: Wed Oct  6 23:40:48 2021
\setlength{\unitlength}{1pt}
\begin{picture}(0,0)
\includegraphics{images/challenge3/resultado-questao-5-6-malha-aberta-inc}
\end{picture}%
\begin{picture}(400,270)(0,0)
\fontsize{6}{0}
\selectfont\put(52,153.188){\makebox(0,0)[t]{\textcolor[rgb]{0.15,0.15,0.15}{{$10^{-4}$}}}}
\fontsize{6}{0}
\selectfont\put(103.667,153.188){\makebox(0,0)[t]{\textcolor[rgb]{0.15,0.15,0.15}{{$10^{-3}$}}}}
\fontsize{6}{0}
\selectfont\put(155.333,153.188){\makebox(0,0)[t]{\textcolor[rgb]{0.15,0.15,0.15}{{$10^{-2}$}}}}
\fontsize{6}{0}
\selectfont\put(207,153.188){\makebox(0,0)[t]{\textcolor[rgb]{0.15,0.15,0.15}{{$10^{-1}$}}}}
\fontsize{6}{0}
\selectfont\put(258.667,153.188){\makebox(0,0)[t]{\textcolor[rgb]{0.15,0.15,0.15}{{$10^{0}$}}}}
\fontsize{6}{0}
\selectfont\put(310.333,153.188){\makebox(0,0)[t]{\textcolor[rgb]{0.15,0.15,0.15}{{$10^{1}$}}}}
\fontsize{6}{0}
\selectfont\put(362,153.188){\makebox(0,0)[t]{\textcolor[rgb]{0.15,0.15,0.15}{{$10^{2}$}}}}
\fontsize{6}{0}
\selectfont\put(48.5278,158.368){\makebox(0,0)[r]{\textcolor[rgb]{0.15,0.15,0.15}{{-100}}}}
\fontsize{6}{0}
\selectfont\put(48.5278,172.789){\makebox(0,0)[r]{\textcolor[rgb]{0.15,0.15,0.15}{{-80}}}}
\fontsize{6}{0}
\selectfont\put(48.5278,187.21){\makebox(0,0)[r]{\textcolor[rgb]{0.15,0.15,0.15}{{-60}}}}
\fontsize{6}{0}
\selectfont\put(48.5278,201.631){\makebox(0,0)[r]{\textcolor[rgb]{0.15,0.15,0.15}{{-40}}}}
\fontsize{6}{0}
\selectfont\put(48.5278,216.051){\makebox(0,0)[r]{\textcolor[rgb]{0.15,0.15,0.15}{{-20}}}}
\fontsize{6}{0}
\selectfont\put(48.5278,230.472){\makebox(0,0)[r]{\textcolor[rgb]{0.15,0.15,0.15}{{0}}}}
\fontsize{6}{0}
\selectfont\put(48.5278,244.893){\makebox(0,0)[r]{\textcolor[rgb]{0.15,0.15,0.15}{{20}}}}
\fontsize{7}{0}
\selectfont\put(207,139.188){\makebox(0,0)[t]{\textcolor[rgb]{0.15,0.15,0.15}{{Frequência [rad/s]}}}}
\fontsize{7}{0}
\selectfont\put(29.5278,201.631){\rotatebox{90}{\makebox(0,0)[b]{\textcolor[rgb]{0.15,0.15,0.15}{{Ganho [db]}}}}}
\fontsize{6}{0}
\selectfont\put(88.002,198.369){\makebox(0,0)[l]{\textcolor[rgb]{0,0,0}{{$C(jw)G(jw)$ p/ folga $= 24^{\circ}$}}}}
\fontsize{6}{0}
\selectfont\put(88.002,186.369){\makebox(0,0)[l]{\textcolor[rgb]{0,0,0}{{${C}'(jw)\overline{K}G(jw)$ c/ $w_{b}$ = 2.5}}}}
\fontsize{6}{0}
\selectfont\put(88.002,174.369){\makebox(0,0)[l]{\textcolor[rgb]{0,0,0}{{${C}'(jw)\overline{K}G(jw)$ c/ $K_{c}$ fixo}}}}
\fontsize{6}{0}
\selectfont\put(52,24.5205){\makebox(0,0)[t]{\textcolor[rgb]{0.15,0.15,0.15}{{$10^{-4}$}}}}
\fontsize{6}{0}
\selectfont\put(103.667,24.5205){\makebox(0,0)[t]{\textcolor[rgb]{0.15,0.15,0.15}{{$10^{-3}$}}}}
\fontsize{6}{0}
\selectfont\put(155.333,24.5205){\makebox(0,0)[t]{\textcolor[rgb]{0.15,0.15,0.15}{{$10^{-2}$}}}}
\fontsize{6}{0}
\selectfont\put(207,24.5205){\makebox(0,0)[t]{\textcolor[rgb]{0.15,0.15,0.15}{{$10^{-1}$}}}}
\fontsize{6}{0}
\selectfont\put(258.667,24.5205){\makebox(0,0)[t]{\textcolor[rgb]{0.15,0.15,0.15}{{$10^{0}$}}}}
\fontsize{6}{0}
\selectfont\put(310.333,24.5205){\makebox(0,0)[t]{\textcolor[rgb]{0.15,0.15,0.15}{{$10^{1}$}}}}
\fontsize{6}{0}
\selectfont\put(362,24.5205){\makebox(0,0)[t]{\textcolor[rgb]{0.15,0.15,0.15}{{$10^{2}$}}}}
\fontsize{6}{0}
\selectfont\put(48.5278,29.7002){\makebox(0,0)[r]{\textcolor[rgb]{0.15,0.15,0.15}{{-300}}}}
\fontsize{6}{0}
\selectfont\put(48.5278,42.0605){\makebox(0,0)[r]{\textcolor[rgb]{0.15,0.15,0.15}{{-250}}}}
\fontsize{6}{0}
\selectfont\put(48.5278,54.4214){\makebox(0,0)[r]{\textcolor[rgb]{0.15,0.15,0.15}{{-200}}}}
\fontsize{6}{0}
\selectfont\put(48.5278,66.7822){\makebox(0,0)[r]{\textcolor[rgb]{0.15,0.15,0.15}{{-150}}}}
\fontsize{6}{0}
\selectfont\put(48.5278,79.1431){\makebox(0,0)[r]{\textcolor[rgb]{0.15,0.15,0.15}{{-100}}}}
\fontsize{6}{0}
\selectfont\put(48.5278,91.5034){\makebox(0,0)[r]{\textcolor[rgb]{0.15,0.15,0.15}{{-50}}}}
\fontsize{6}{0}
\selectfont\put(48.5278,103.864){\makebox(0,0)[r]{\textcolor[rgb]{0.15,0.15,0.15}{{0}}}}
\fontsize{6}{0}
\selectfont\put(48.5278,116.225){\makebox(0,0)[r]{\textcolor[rgb]{0.15,0.15,0.15}{{50}}}}
\fontsize{7}{0}
\selectfont\put(207,10.5205){\makebox(0,0)[t]{\textcolor[rgb]{0.15,0.15,0.15}{{Frequência [rad/s]}}}}
\fontsize{7}{0}
\selectfont\put(29.5278,72.9624){\rotatebox{90}{\makebox(0,0)[b]{\textcolor[rgb]{0.15,0.15,0.15}{{Fase [graus]}}}}}
\fontsize{6}{0}
\selectfont\put(88.002,69.7012){\makebox(0,0)[l]{\textcolor[rgb]{0,0,0}{{$C(jw)G(jw)$ p/ folga $= 24^{\circ}$}}}}
\fontsize{6}{0}
\selectfont\put(88.002,57.7012){\makebox(0,0)[l]{\textcolor[rgb]{0,0,0}{{${C}'(jw)\overline{K}G(jw)$ c/ $w_{b}$ = 2.5}}}}
\fontsize{6}{0}
\selectfont\put(88.002,45.7007){\makebox(0,0)[l]{\textcolor[rgb]{0,0,0}{{${C}'(jw)\overline{K}G(jw)$ c/ $K_{c}$ fixo}}}}
\end{picture}

    \end{minipage}
\end{figure}

Como já evidenciado na Figura \ref{fig:desafio-3:questao-5-6-malha-aberta}, a
margem de fase foi corrigida para próximo do valor desejado novamente. A
correção se deu projetando um novo compensador em avanço fixando o valor de
$K_{c}$, isto é, projetando ${C}'(s)$ considerando que a planta é
$K_{c}\overline{K}G(s)$. O projeto foi realizado também para uma margem de fase
desejada de $60^{\circ}$ com folga de $24^{\circ}$. Após correção, a margem de
fase obtida foi de $59,88^{\circ}$, superior até à obtida com o compensar em
avanço com ganho estático unitário. Já a largura de banda teve um acréscimo,
ficando igual a 2,83 rad/s. O compensador ao final tem os parâmetros $\phi_{max}
= 54,30^{\circ}$, $\alpha = 0,10$ e $T = 1,36$, conforme Equação
\ref{eq:desafio3:compensador-com-kc-diferente-de-um}.

\begin{equation}
    \label{eq:desafio3:compensador-com-kc-diferente-de-um}
    C'(s) = 1,43\frac{1,36s + 1}{0,14s + 1}
\end{equation}

A fim de avaliar a resposta em malha fechada para os dois compensadores, foi
simulada a saída da planta no domínio do tempo. O resultado obtido é apresentado
graficamente na Figura \ref{fig:desafio-3:questao-7-dominio-do-tempo}.

\begin{figure}[ht!]
    \caption{Resposta da planta em malha fechada no domínio do tempo para
    diferentes ganhos do compensador em avanço.}
    \vspace{-10pt}
    \hspace{-30pt}
    \label{fig:desafio-3:questao-7-dominio-do-tempo}
    \begin{minipage}{\linewidth}
        % Title: gl2ps_renderer figure
% Creator: GL2PS 1.4.0, (C) 1999-2017 C. Geuzaine
% For: Octave
% CreationDate: Thu Oct  7 00:38:30 2021
\setlength{\unitlength}{1pt}
\begin{picture}(0,0)
\includegraphics{images/challenge3/resultado-questao-7-resposta-no-tempo-inc}
\end{picture}%
\begin{picture}(400,150)(0,0)
\fontsize{6}{0}
\selectfont\put(52,20){\makebox(0,0)[t]{\textcolor[rgb]{0.15,0.15,0.15}{{0}}}}
\fontsize{6}{0}
\selectfont\put(114,20){\makebox(0,0)[t]{\textcolor[rgb]{0.15,0.15,0.15}{{10}}}}
\fontsize{6}{0}
\selectfont\put(176,20){\makebox(0,0)[t]{\textcolor[rgb]{0.15,0.15,0.15}{{20}}}}
\fontsize{6}{0}
\selectfont\put(238,20){\makebox(0,0)[t]{\textcolor[rgb]{0.15,0.15,0.15}{{30}}}}
\fontsize{6}{0}
\selectfont\put(300,20){\makebox(0,0)[t]{\textcolor[rgb]{0.15,0.15,0.15}{{40}}}}
\fontsize{6}{0}
\selectfont\put(362,20){\makebox(0,0)[t]{\textcolor[rgb]{0.15,0.15,0.15}{{50}}}}
\fontsize{6}{0}
\selectfont\put(48.5278,25.188){\makebox(0,0)[r]{\textcolor[rgb]{0.15,0.15,0.15}{{-0.2}}}}
\fontsize{6}{0}
\selectfont\put(48.5278,41.4111){\makebox(0,0)[r]{\textcolor[rgb]{0.15,0.15,0.15}{{0}}}}
\fontsize{6}{0}
\selectfont\put(48.5278,57.6343){\makebox(0,0)[r]{\textcolor[rgb]{0.15,0.15,0.15}{{0.2}}}}
\fontsize{6}{0}
\selectfont\put(48.5278,73.8574){\makebox(0,0)[r]{\textcolor[rgb]{0.15,0.15,0.15}{{0.4}}}}
\fontsize{6}{0}
\selectfont\put(48.5278,90.0806){\makebox(0,0)[r]{\textcolor[rgb]{0.15,0.15,0.15}{{0.6}}}}
\fontsize{6}{0}
\selectfont\put(48.5278,106.304){\makebox(0,0)[r]{\textcolor[rgb]{0.15,0.15,0.15}{{0.8}}}}
\fontsize{6}{0}
\selectfont\put(48.5278,122.527){\makebox(0,0)[r]{\textcolor[rgb]{0.15,0.15,0.15}{{1}}}}
\fontsize{7}{0}
\selectfont\put(207,9){\makebox(0,0)[t]{\textcolor[rgb]{0.15,0.15,0.15}{{Tempo(s)}}}}
\fontsize{7}{0}
\selectfont\put(31.5278,81.9692){\rotatebox{90}{\makebox(0,0)[b]{\textcolor[rgb]{0.15,0.15,0.15}{{Saída $y(t)$}}}}}
\fontsize{6}{0}
\selectfont\put(294,64.689){\makebox(0,0)[l]{\textcolor[rgb]{0,0,0}{{$r(t)$}}}}
\fontsize{6}{0}
\selectfont\put(294,53.189){\makebox(0,0)[l]{\textcolor[rgb]{0,0,0}{{$y(t)$ p/ $K_{c} = 1$}}}}
\fontsize{6}{0}
\selectfont\put(294,41.189){\makebox(0,0)[l]{\textcolor[rgb]{0,0,0}{{$y(t)$ p/ $K_{c} = 1.43$}}}}
\end{picture}

    \end{minipage}
\end{figure}

Observa-se na Figura \ref{fig:desafio-3:questao-7-dominio-do-tempo} que ambos os
compensadores possuem uma resposta em regime permanente com erro não nulo.
Diante do diagrama de magnitude da Figura
\ref{fig:desafio-3:questao-5-6-malha-aberta}, nota-se que o ganho em baixas
frequências é menor quando $K_c = 1$ e um pouco maior para $K_c = 1,43$.
Exatamente essa diferença que explica o erro não nulo em regime permanente para
$y(t)$ em ambos os casos e também porque o erro é menor para o sistema com
compensador de maior ganho estático. Esta interpretação fica mais palpável
quando aplicado ao teorema do valor final em que para uma referência do tipo
degrau

\begin{equation}
    \label{eq:desafio3-teorema-do-valor-final}
    \lim_{t \rightarrow \infty }e(t)=\lim_{s \rightarrow 0 }sE(s)=\frac{1}{1+K_v},
\end{equation}
onde:

\begin{conditions*}
    K_v & igual $\lim_{s \rightarrow 0 }C(0)G(0)$.
\end{conditions*}

Assim, para $K_c = 1$ temos que $e(\infty) = 0,5$ e $y(\infty) = 0,5$. Já para
$K_c = 1,43$ $e(\infty) = 0,41$ e $y(\infty) = 0,59$, condizente com o gráfico.
Uma outra interpretação é da resposta do regime transitório. Como para o sistema
com compensador de ganho unitário a margem de fase é menor, consequentemente
o fator de amortecimento $\xi$ também é menor, decorrendo em um maior
\textit{overshoot} da resposta em malha fechada do sistema para uma referência
do tipo degrau.

\subsection{Conclusões}
(Concluir em que medida os resultados apresentam relação com a motivação.
Permitem ilustratar ou concluir algo sobre a motivação? )


\addcontentsline{toc}{section}{Referências}
\bibliographystyle{apalike}
\bibliography{references.bib}

\end{document}
