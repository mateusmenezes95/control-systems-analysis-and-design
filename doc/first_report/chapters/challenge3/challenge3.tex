\section{Desafio III - Resposta em Frequência}

\subsection{Motivação}
Este último desafio da série visa reduzir a abstração do critério de Nyquist,
evidenciando como ele pode ser aplicado no projeto de controladores no domínio
da frequência. Além disso, o uso da resposta em frequência junto com o critério
de Nyquist permite que os controladores sejam projetados apenas com a função de
transferência de malha aberta, sem precisar também do conceito de dominância
modal. Portanto, poder projetar controladores no domínio da frequência é mais
uma ferramenta que o projetista tem em mãos na hora de decidir qual abordagem
seguir na resolução de um determinado problema de controle. Os campos que se
beneficiam dessa abordagem são principalmente aqueles que envolvam eletrônica de
potência, como o controle de tensão na saída de conversores DC-DC regulados
conforme exemplo na Figura \ref{fig:desafio3:dc-dc-datasheet}, e vibrações.

\begin{figure}[h]
	\centering
	\caption{Recorte do Datasheet de um conversor DC-DC isolado e regulado.}
	\label{fig:desafio3:dc-dc-datasheet}
	\includegraphics[width=\textwidth]{images/challenge3/datasheet-screenshot.png}
\end{figure}

\subsection{Simulações realizadas}
As simulações realizadas focaram no design de compensadores de avanço, de atraso
e de avanço-atraso de fase.

Primeiro as simulações foram realizadas para analisar a
resposta em frequência de um compensador em avanço variando primeiro a folga de
fase utilizada no projeto; depois projetando o compensador com o requisito de
largura de banda de malha aberta; e no fim desta primeira parte, analisou o
compensador com diferentes ganhos.

A segunda etapa consistiu na análise no
domínio do tempo da resposta do sistema em malha fechada com os diferentes
compensadores projetados na primeira parte e também com a adição de um filtro de
referência.

Por fim, foi realizada novamente a simulação do sistema em malha fechada,
entretanto, com a adição de um compensador em atraso, ficando o compensador
final com a topologia de avanço-atraso. Também avaliou-se a resposta em malha
fechada com um filtro de referência. Adicionalmente, foi avaliada a resposta do
sistema para uma perturbação do tipo degrau na entrada da planta.

Também como nos dois desafios anteriores, este desafio teve como referência o
sistema em malha fechada ilustrado pela Figura
\ref{fig:diagrama-de-blocos-malha-fechada} e descrita no tópico
\ref{sec:desafio-1-simulacoes-realizadas}. A função de transferência da
planta/processo utilizada está definida na Equação \ref{eq:desafio-3:g-de-s}. O
tópico posterior descreve e contém discussões dos resultados obtidos.

\begin{equation}
    \label{eq:desafio-3:g-de-s}
    G(s) = \frac{0,5}{(s^2 + 0,6s +1)(0,1s + 1)}.
\end{equation}

\subsection{Resultados obtidos}
As simulações começaram com a definição de $\overline{K}$ tal que
$\overline{K}G(0) = 1$. Através de álgebra simples, chegou-se a $\overline{K} = 2$.
Com a definição desse valor de ganho, foi calculada a largura de banda de $P(s)
= \overline{K}G(s)$, cujo resultado é ilustrado na Figura
\ref{fig:desafio-3:questao-3}. A largura de banda $w_{b} \approx 1,45$ rad/s.
Vale ressaltar que como foi realizado computação numérica, os valores
serão aproximados devido a discretização da magnitude e fase no diagrama de bode.

\begin{figure}[!ht]
    \caption{Magnitude da resposta em frequência de $P(s)$
    com destaque para largura de banda.}
    \vspace{-10pt}
    \hspace{-30pt}
    \label{fig:desafio-3:questao-3}
    \begin{minipage}{\linewidth}
        % Title: gl2ps_renderer figure
% Creator: GL2PS 1.4.0, (C) 1999-2017 C. Geuzaine
% For: Octave
% CreationDate: Wed Oct  6 21:32:31 2021
\setlength{\unitlength}{1pt}
\begin{picture}(0,0)
\includegraphics{images/challenge3/resultado-questao-2-inc}
\end{picture}%
\begin{picture}(400,120)(0,0)
\fontsize{6}{0}
\selectfont\put(52,23){\makebox(0,0)[t]{\textcolor[rgb]{0.15,0.15,0.15}{{$10^{-4}$}}}}
\fontsize{6}{0}
\selectfont\put(103.667,23){\makebox(0,0)[t]{\textcolor[rgb]{0.15,0.15,0.15}{{$10^{-3}$}}}}
\fontsize{6}{0}
\selectfont\put(155.333,23){\makebox(0,0)[t]{\textcolor[rgb]{0.15,0.15,0.15}{{$10^{-2}$}}}}
\fontsize{6}{0}
\selectfont\put(207,23){\makebox(0,0)[t]{\textcolor[rgb]{0.15,0.15,0.15}{{$10^{-1}$}}}}
\fontsize{6}{0}
\selectfont\put(258.667,23){\makebox(0,0)[t]{\textcolor[rgb]{0.15,0.15,0.15}{{$10^{0}$}}}}
\fontsize{6}{0}
\selectfont\put(310.333,23){\makebox(0,0)[t]{\textcolor[rgb]{0.15,0.15,0.15}{{$10^{1}$}}}}
\fontsize{6}{0}
\selectfont\put(362,23){\makebox(0,0)[t]{\textcolor[rgb]{0.15,0.15,0.15}{{$10^{2}$}}}}
\fontsize{6}{0}
\selectfont\put(48.5278,31.9443){\makebox(0,0)[r]{\textcolor[rgb]{0.15,0.15,0.15}{{-100}}}}
\fontsize{6}{0}
\selectfont\put(48.5278,46.1294){\makebox(0,0)[r]{\textcolor[rgb]{0.15,0.15,0.15}{{-80}}}}
\fontsize{6}{0}
\selectfont\put(48.5278,60.314){\makebox(0,0)[r]{\textcolor[rgb]{0.15,0.15,0.15}{{-60}}}}
\fontsize{6}{0}
\selectfont\put(48.5278,74.499){\makebox(0,0)[r]{\textcolor[rgb]{0.15,0.15,0.15}{{-40}}}}
\fontsize{6}{0}
\selectfont\put(48.5278,88.6841){\makebox(0,0)[r]{\textcolor[rgb]{0.15,0.15,0.15}{{-20}}}}
\fontsize{6}{0}
\selectfont\put(48.5278,102.869){\makebox(0,0)[r]{\textcolor[rgb]{0.15,0.15,0.15}{{0}}}}
\fontsize{7}{0}
\selectfont\put(207,9){\makebox(0,0)[t]{\textcolor[rgb]{0.15,0.15,0.15}{{Frequência [rad/s]}}}}
\fontsize{7}{0}
\selectfont\put(29.5278,69.0977){\rotatebox{90}{\makebox(0,0)[b]{\textcolor[rgb]{0.15,0.15,0.15}{{Magnitude [dB]}}}}}
\fontsize{6}{0}
\selectfont\put(269.878,100.742){\makebox(0,0)[l]{\textcolor[rgb]{0,0,0}{{$|P(w_{b} \approx  1.45)| \approx -3_{db}$}}}}
\fontsize{6}{0}
\selectfont\put(88.002,44.1963){\makebox(0,0)[l]{\textcolor[rgb]{0,0,0}{{$|P(jw)|$}}}}
\end{picture}

    \end{minipage}
\end{figure}

Em seguida foi projetado dois compensadores em avanço, dado pela Equação
\ref{eq:desafio-3:clinha-de-s}, tal que a margem de fase de $C(s)G(s)$ fosse
maior ou igual a $60^{\circ}$ e a largura de banda de malha aberta fosse maior
que 2,5 rad/s dados $C(s) = {C}'(s)\overline{K}$ e $K_{c} = 1$. Um controlador
foi projetado considerando uma folga de $12^{\circ}$ e o outro $24^{\circ}$. Os
parâmetros do compensador foram calculados conforme as Equações
\ref{eq:desafio-3:phi-max}, \ref{eq:desafio-3:alpha} e \ref{eq:desafio-3:te}.

\begin{equation}
    \label{eq:desafio-3:clinha-de-s}
    {C}'(s) = K_{c}\frac{Ts + 1}{\alpha Ts + 1}
\end{equation}

\begin{equation}
    \label{eq:desafio-3:phi-max}
    \phi_{max} = mf_{d} - mf_{a} + folga,
\end{equation}
onde:

\begin{conditions*}
    \phi_{max} & a máxima fase que o compensador em avanço necessita ter para
    que o sistema em malha aberta possua a margem de fase desejada; \\
    mf_{d} & a margem de fase desejada; e \\
    mf_{a} & a margem de fase atual. \\
\end{conditions*}

\begin{equation}
    \label{eq:desafio-3:alpha}
    \alpha = \frac{1 - \sin(\phi_{max})}{1 + \sin(\phi_{max})},
\end{equation}

\begin{equation}
    \label{eq:desafio-3:te}
    T = \frac{1}{w_{\phi_{max}}\sqrt{\alpha}}
\end{equation}

Para o compensador com menor folga foi encontrado $\phi_{max} = 28,72^{\circ}$,
$\alpha = 0.35$, $T = 1,09$ e $w_{b} = 1,81$ rad/s. Já para o segundo
compensador $\phi_{max} = 40,72^{\circ}$, $\alpha = 0,21$, $T = 1,28$ e $w_{b} =
2,04$ rad/s. A resposta em frequência dos dois compensadores projetados é
expressa na Figura \ref{fig:desafio-3:questao-3-4-compensadores}.

\begin{figure}[!ht]
    \caption{Compensador em avanço ${C}'(s)$ projetado com folgas de
    $12^{\circ}$ e $24^{\circ}$.}
    \vspace{-10pt}
    \hspace{-30pt}
    \label{fig:desafio-3:questao-3-4-compensadores}
    \begin{minipage}{\linewidth}
        % Title: gl2ps_renderer figure
% Creator: GL2PS 1.4.0, (C) 1999-2017 C. Geuzaine
% For: Octave
% CreationDate: Wed Oct  6 21:11:28 2021
\setlength{\unitlength}{1pt}
\begin{picture}(0,0)
\includegraphics{images/challenge3/resultado-questao-3-4-compensadores-inc}
\end{picture}%
\begin{picture}(400,270)(0,0)
\fontsize{6}{0}
\selectfont\put(52,153.188){\makebox(0,0)[t]{\textcolor[rgb]{0.15,0.15,0.15}{{$10^{-4}$}}}}
\fontsize{6}{0}
\selectfont\put(103.667,153.188){\makebox(0,0)[t]{\textcolor[rgb]{0.15,0.15,0.15}{{$10^{-3}$}}}}
\fontsize{6}{0}
\selectfont\put(155.333,153.188){\makebox(0,0)[t]{\textcolor[rgb]{0.15,0.15,0.15}{{$10^{-2}$}}}}
\fontsize{6}{0}
\selectfont\put(207,153.188){\makebox(0,0)[t]{\textcolor[rgb]{0.15,0.15,0.15}{{$10^{-1}$}}}}
\fontsize{6}{0}
\selectfont\put(258.667,153.188){\makebox(0,0)[t]{\textcolor[rgb]{0.15,0.15,0.15}{{$10^{0}$}}}}
\fontsize{6}{0}
\selectfont\put(310.333,153.188){\makebox(0,0)[t]{\textcolor[rgb]{0.15,0.15,0.15}{{$10^{1}$}}}}
\fontsize{6}{0}
\selectfont\put(362,153.188){\makebox(0,0)[t]{\textcolor[rgb]{0.15,0.15,0.15}{{$10^{2}$}}}}
\fontsize{6}{0}
\selectfont\put(48.5278,158.368){\makebox(0,0)[r]{\textcolor[rgb]{0.15,0.15,0.15}{{0}}}}
\fontsize{6}{0}
\selectfont\put(48.5278,170.729){\makebox(0,0)[r]{\textcolor[rgb]{0.15,0.15,0.15}{{2}}}}
\fontsize{6}{0}
\selectfont\put(48.5278,183.089){\makebox(0,0)[r]{\textcolor[rgb]{0.15,0.15,0.15}{{4}}}}
\fontsize{6}{0}
\selectfont\put(48.5278,195.45){\makebox(0,0)[r]{\textcolor[rgb]{0.15,0.15,0.15}{{6}}}}
\fontsize{6}{0}
\selectfont\put(48.5278,207.811){\makebox(0,0)[r]{\textcolor[rgb]{0.15,0.15,0.15}{{8}}}}
\fontsize{6}{0}
\selectfont\put(48.5278,220.172){\makebox(0,0)[r]{\textcolor[rgb]{0.15,0.15,0.15}{{10}}}}
\fontsize{6}{0}
\selectfont\put(48.5278,232.532){\makebox(0,0)[r]{\textcolor[rgb]{0.15,0.15,0.15}{{12}}}}
\fontsize{6}{0}
\selectfont\put(48.5278,244.893){\makebox(0,0)[r]{\textcolor[rgb]{0.15,0.15,0.15}{{14}}}}
\fontsize{7}{0}
\selectfont\put(207,139.188){\makebox(0,0)[t]{\textcolor[rgb]{0.15,0.15,0.15}{{Frequência [rad/s]}}}}
\fontsize{7}{0}
\selectfont\put(35.5278,201.631){\rotatebox{90}{\makebox(0,0)[b]{\textcolor[rgb]{0.15,0.15,0.15}{{Magnitude [db]}}}}}
\fontsize{6}{0}
\selectfont\put(88.002,228.893){\makebox(0,0)[l]{\textcolor[rgb]{0,0,0}{{${C}'(jw)$ p/ folga $= 12^{\circ}$}}}}
\fontsize{6}{0}
\selectfont\put(88.002,216.892){\makebox(0,0)[l]{\textcolor[rgb]{0,0,0}{{${C}'(jw)$ p/ folga $= 24^{\circ}$}}}}
\fontsize{6}{0}
\selectfont\put(52,24.5205){\makebox(0,0)[t]{\textcolor[rgb]{0.15,0.15,0.15}{{$10^{-4}$}}}}
\fontsize{6}{0}
\selectfont\put(103.667,24.5205){\makebox(0,0)[t]{\textcolor[rgb]{0.15,0.15,0.15}{{$10^{-3}$}}}}
\fontsize{6}{0}
\selectfont\put(155.333,24.5205){\makebox(0,0)[t]{\textcolor[rgb]{0.15,0.15,0.15}{{$10^{-2}$}}}}
\fontsize{6}{0}
\selectfont\put(207,24.5205){\makebox(0,0)[t]{\textcolor[rgb]{0.15,0.15,0.15}{{$10^{-1}$}}}}
\fontsize{6}{0}
\selectfont\put(258.667,24.5205){\makebox(0,0)[t]{\textcolor[rgb]{0.15,0.15,0.15}{{$10^{0}$}}}}
\fontsize{6}{0}
\selectfont\put(310.333,24.5205){\makebox(0,0)[t]{\textcolor[rgb]{0.15,0.15,0.15}{{$10^{1}$}}}}
\fontsize{6}{0}
\selectfont\put(362,24.5205){\makebox(0,0)[t]{\textcolor[rgb]{0.15,0.15,0.15}{{$10^{2}$}}}}
\fontsize{6}{0}
\selectfont\put(48.5278,29.7002){\makebox(0,0)[r]{\textcolor[rgb]{0.15,0.15,0.15}{{0}}}}
\fontsize{6}{0}
\selectfont\put(48.5278,47.0049){\makebox(0,0)[r]{\textcolor[rgb]{0.15,0.15,0.15}{{10}}}}
\fontsize{6}{0}
\selectfont\put(48.5278,64.3101){\makebox(0,0)[r]{\textcolor[rgb]{0.15,0.15,0.15}{{20}}}}
\fontsize{6}{0}
\selectfont\put(48.5278,81.6152){\makebox(0,0)[r]{\textcolor[rgb]{0.15,0.15,0.15}{{30}}}}
\fontsize{6}{0}
\selectfont\put(48.5278,98.9199){\makebox(0,0)[r]{\textcolor[rgb]{0.15,0.15,0.15}{{40}}}}
\fontsize{6}{0}
\selectfont\put(48.5278,116.225){\makebox(0,0)[r]{\textcolor[rgb]{0.15,0.15,0.15}{{50}}}}
\fontsize{7}{0}
\selectfont\put(207,10.5205){\makebox(0,0)[t]{\textcolor[rgb]{0.15,0.15,0.15}{{Frequência [rad/s]}}}}
\fontsize{7}{0}
\selectfont\put(35.5278,72.9624){\rotatebox{90}{\makebox(0,0)[b]{\textcolor[rgb]{0.15,0.15,0.15}{{Fase [graus]}}}}}
\fontsize{6}{0}
\selectfont\put(88.002,57.7007){\makebox(0,0)[l]{\textcolor[rgb]{0,0,0}{{${C}'(jw)$ p/ folga $= 12^{\circ}$}}}}
\fontsize{6}{0}
\selectfont\put(88.002,45.7007){\makebox(0,0)[l]{\textcolor[rgb]{0,0,0}{{${C}'(jw)$ p/ folga $= 24^{\circ}$}}}}
\end{picture}

    \end{minipage}
\end{figure}

Na Figura \ref{fig:desafio-3:questao-3-4-compensadores} é possível visualizar
que o controlador projetado com maior folga foi o que obteve maior pico de fase,
condizente com a Equação \ref{eq:desafio-3:phi-max}. O compensador com folga de
$24^{\circ}$ também possui maior ganho na frequência de pico de fase. Para
avaliar a contribuição dos dois controladores em $G(s)$, foi obtida a resposta
em frequência $C(s)G(s)$ conforme Figura
\ref{fig:desafio-3:questao-3-4-malha-aberta}.

\begin{figure}[!ht]
    \caption{Resposta em frequência de ${C}'(s)\overline{K}G(s)$ com compensador
    em avanço ${C}'(s)$ projetado com folgas de $12^{\circ}$ e $24^{\circ}$.}
    \vspace{-10pt}
    \hspace{-30pt}
    \label{fig:desafio-3:questao-3-4-malha-aberta}
    \begin{minipage}{\linewidth}
        % Title: gl2ps_renderer figure
% Creator: GL2PS 1.4.0, (C) 1999-2017 C. Geuzaine
% For: Octave
% CreationDate: Wed Oct  6 21:53:16 2021
\setlength{\unitlength}{1pt}
\begin{picture}(0,0)
\includegraphics{images/challenge3/resultado-questao-3-4-malha-aberta-inc}
\end{picture}%
\begin{picture}(400,270)(0,0)
\fontsize{6}{0}
\selectfont\put(52,153.188){\makebox(0,0)[t]{\textcolor[rgb]{0.15,0.15,0.15}{{$10^{-4}$}}}}
\fontsize{6}{0}
\selectfont\put(103.667,153.188){\makebox(0,0)[t]{\textcolor[rgb]{0.15,0.15,0.15}{{$10^{-3}$}}}}
\fontsize{6}{0}
\selectfont\put(155.333,153.188){\makebox(0,0)[t]{\textcolor[rgb]{0.15,0.15,0.15}{{$10^{-2}$}}}}
\fontsize{6}{0}
\selectfont\put(207,153.188){\makebox(0,0)[t]{\textcolor[rgb]{0.15,0.15,0.15}{{$10^{-1}$}}}}
\fontsize{6}{0}
\selectfont\put(258.667,153.188){\makebox(0,0)[t]{\textcolor[rgb]{0.15,0.15,0.15}{{$10^{0}$}}}}
\fontsize{6}{0}
\selectfont\put(310.333,153.188){\makebox(0,0)[t]{\textcolor[rgb]{0.15,0.15,0.15}{{$10^{1}$}}}}
\fontsize{6}{0}
\selectfont\put(362,153.188){\makebox(0,0)[t]{\textcolor[rgb]{0.15,0.15,0.15}{{$10^{2}$}}}}
\fontsize{6}{0}
\selectfont\put(48.5278,158.368){\makebox(0,0)[r]{\textcolor[rgb]{0.15,0.15,0.15}{{-120}}}}
\fontsize{6}{0}
\selectfont\put(48.5278,170.729){\makebox(0,0)[r]{\textcolor[rgb]{0.15,0.15,0.15}{{-100}}}}
\fontsize{6}{0}
\selectfont\put(48.5278,183.089){\makebox(0,0)[r]{\textcolor[rgb]{0.15,0.15,0.15}{{-80}}}}
\fontsize{6}{0}
\selectfont\put(48.5278,195.45){\makebox(0,0)[r]{\textcolor[rgb]{0.15,0.15,0.15}{{-60}}}}
\fontsize{6}{0}
\selectfont\put(48.5278,207.811){\makebox(0,0)[r]{\textcolor[rgb]{0.15,0.15,0.15}{{-40}}}}
\fontsize{6}{0}
\selectfont\put(48.5278,220.172){\makebox(0,0)[r]{\textcolor[rgb]{0.15,0.15,0.15}{{-20}}}}
\fontsize{6}{0}
\selectfont\put(48.5278,232.532){\makebox(0,0)[r]{\textcolor[rgb]{0.15,0.15,0.15}{{0}}}}
\fontsize{6}{0}
\selectfont\put(48.5278,244.893){\makebox(0,0)[r]{\textcolor[rgb]{0.15,0.15,0.15}{{20}}}}
\fontsize{7}{0}
\selectfont\put(207,139.188){\makebox(0,0)[t]{\textcolor[rgb]{0.15,0.15,0.15}{{Frequência [rad/s]}}}}
\fontsize{7}{0}
\selectfont\put(29.5278,201.631){\rotatebox{90}{\makebox(0,0)[b]{\textcolor[rgb]{0.15,0.15,0.15}{{Ganho [db]}}}}}
\fontsize{6}{0}
\selectfont\put(88.002,198.369){\makebox(0,0)[l]{\textcolor[rgb]{0,0,0}{{$\overline{K}G(jw)$}}}}
\fontsize{6}{0}
\selectfont\put(88.002,186.369){\makebox(0,0)[l]{\textcolor[rgb]{0,0,0}{{$C(jw)G(jw)$ p/ folga $= 12^{\circ}$}}}}
\fontsize{6}{0}
\selectfont\put(88.002,174.369){\makebox(0,0)[l]{\textcolor[rgb]{0,0,0}{{$C(jw)G(jw)$ p/ folga $= 24^{\circ}$}}}}
\fontsize{6}{0}
\selectfont\put(52,24.5205){\makebox(0,0)[t]{\textcolor[rgb]{0.15,0.15,0.15}{{$10^{-4}$}}}}
\fontsize{6}{0}
\selectfont\put(103.667,24.5205){\makebox(0,0)[t]{\textcolor[rgb]{0.15,0.15,0.15}{{$10^{-3}$}}}}
\fontsize{6}{0}
\selectfont\put(155.333,24.5205){\makebox(0,0)[t]{\textcolor[rgb]{0.15,0.15,0.15}{{$10^{-2}$}}}}
\fontsize{6}{0}
\selectfont\put(207,24.5205){\makebox(0,0)[t]{\textcolor[rgb]{0.15,0.15,0.15}{{$10^{-1}$}}}}
\fontsize{6}{0}
\selectfont\put(258.667,24.5205){\makebox(0,0)[t]{\textcolor[rgb]{0.15,0.15,0.15}{{$10^{0}$}}}}
\fontsize{6}{0}
\selectfont\put(310.333,24.5205){\makebox(0,0)[t]{\textcolor[rgb]{0.15,0.15,0.15}{{$10^{1}$}}}}
\fontsize{6}{0}
\selectfont\put(362,24.5205){\makebox(0,0)[t]{\textcolor[rgb]{0.15,0.15,0.15}{{$10^{2}$}}}}
\fontsize{6}{0}
\selectfont\put(48.5278,29.7002){\makebox(0,0)[r]{\textcolor[rgb]{0.15,0.15,0.15}{{-300}}}}
\fontsize{6}{0}
\selectfont\put(48.5278,42.0605){\makebox(0,0)[r]{\textcolor[rgb]{0.15,0.15,0.15}{{-250}}}}
\fontsize{6}{0}
\selectfont\put(48.5278,54.4214){\makebox(0,0)[r]{\textcolor[rgb]{0.15,0.15,0.15}{{-200}}}}
\fontsize{6}{0}
\selectfont\put(48.5278,66.7822){\makebox(0,0)[r]{\textcolor[rgb]{0.15,0.15,0.15}{{-150}}}}
\fontsize{6}{0}
\selectfont\put(48.5278,79.1431){\makebox(0,0)[r]{\textcolor[rgb]{0.15,0.15,0.15}{{-100}}}}
\fontsize{6}{0}
\selectfont\put(48.5278,91.5034){\makebox(0,0)[r]{\textcolor[rgb]{0.15,0.15,0.15}{{-50}}}}
\fontsize{6}{0}
\selectfont\put(48.5278,103.864){\makebox(0,0)[r]{\textcolor[rgb]{0.15,0.15,0.15}{{0}}}}
\fontsize{6}{0}
\selectfont\put(48.5278,116.225){\makebox(0,0)[r]{\textcolor[rgb]{0.15,0.15,0.15}{{50}}}}
\fontsize{7}{0}
\selectfont\put(207,10.5205){\makebox(0,0)[t]{\textcolor[rgb]{0.15,0.15,0.15}{{Frequência [rad/s]}}}}
\fontsize{7}{0}
\selectfont\put(29.5278,72.9624){\rotatebox{90}{\makebox(0,0)[b]{\textcolor[rgb]{0.15,0.15,0.15}{{Fase [graus]}}}}}
\fontsize{6}{0}
\selectfont\put(88.002,69.7012){\makebox(0,0)[l]{\textcolor[rgb]{0,0,0}{{$\overline{K}G(jw)$}}}}
\fontsize{6}{0}
\selectfont\put(88.002,57.7012){\makebox(0,0)[l]{\textcolor[rgb]{0,0,0}{{$C(jw)G(jw)$ p/ folga $= 12^{\circ}$}}}}
\fontsize{6}{0}
\selectfont\put(88.002,45.7007){\makebox(0,0)[l]{\textcolor[rgb]{0,0,0}{{$C(jw)G(jw)$ p/ folga $= 24^{\circ}$}}}}
\end{picture}

    \end{minipage}
\end{figure}

Como realçado nas linhas verticais do gráfico de fase na Figura
\ref{fig:desafio-3:questao-3-4-malha-aberta}, o sistema em malha aberta
$\overline{K}G(s)$ é o que possui menor margem de fase ($M_{f} =
43,28^{\circ}$), conforme esperado. A margem de fase é melhorada com a adição do
compensador projetado com a folga de $12^{\circ}$, atingindo uma $M_{f} =
53,70^{\circ}$, mas que ainda possui uma diferença considerável da margem de
fase desejada ($60^{\circ}$). Por outro lado, o compensador projetado com uma
folga de $24^{\circ}$ contribuiu com maior fase e portanto a margem de fase de
$C(s)G(s)$ atingiu $59,44^{\circ}$, muito mais próxima ao valor desejado. A
melhor contribuição do segundo projeto pode ser explicado devido as incertezas
no projeto de compensador no domínio da frequência. É incerto em que valor de
frequência ocorrerá o pico de fase do compensador quando ele é colocado junto
com a planta/processo. O pico pode ficar tanto antes quanto depois do ponto de
frequência desejável. Dessa forma, utilizando uma folga de $24^{\circ}$ permite
que caso o pico de fase do compensador ocorra antes do valor de frequência
desejado, o compensador ainda contribua com um valor alto de fase para o sistema
em malha aberta. Além disso, caso o pico ocorra após a frequência desejada, o
projeto vai ser ainda mais conservador, já que a margem de fase vai ser superior
a desejada. Portanto, o projeto usando uma folga de $24^{\circ}$ acaba sendo
mais conservador se comparado ao de $12^{\circ}$.

Embora o segundo compensador tenha coontribuido efetivamente para o requisito de
margem de fase, o mesmo não foi obtido para largura de banda de malha aberta.
Nessa linha, o déficit do projeto usando a folga de $12^{\circ}$ é evidente já
que ele também não contribuiu para o sistema em malha aberta para atingir a
largura de banda requerida. Diante disso, visando aumentar a largura de banda de
$C(s)G(s)$, foi calculado o valor de $K_{c}$ tal que $w_{b} = 2,5$ rad/s. O
valor de ganho do compensador foi calculado conforme Equação
\ref{eq:desafio3:calculo-de-kc}, obtendo assim $K_c = 1,43$.

\begin{equation}
    \label{eq:desafio3:calculo-de-kc}
    K_{c} = \left. \frac{1}{\sqrt{2}(|{C}'(jw)\overline{K}G(jw)|)} \right |_{w = 2,5}
\end{equation}

A inserção do ganho no compensador de avanço desloca todo os valores de
magnitude para a direita. Como consequência a frequência de cruzamento de ganho
também é deslocada para direita. Entretanto, a fase não é deslocada na inserção
de um ganho estático. A consequência é que a margem de fase do sistema em
malha aberta é reduzida, que para esse caso à um valor de
$49,49^{\circ}$. Este fato fica evidente quando mostrado de forma gráfica na
Figura \ref{fig:desafio-3:questao-5-6-malha-aberta}.

\begin{figure}[ht!]
    \caption{Comparação da resposta em frequência de ${C}'(s)\overline{K}G(s)$ com
    diferentes valores de ganho estático do compensador e largura de banda em
    malha aberta.}
    \vspace{-10pt}
    \hspace{-30pt}
    \label{fig:desafio-3:questao-5-6-malha-aberta}
    \begin{minipage}{\linewidth}
        % Title: gl2ps_renderer figure
% Creator: GL2PS 1.4.0, (C) 1999-2017 C. Geuzaine
% For: Octave
% CreationDate: Wed Oct  6 23:40:48 2021
\setlength{\unitlength}{1pt}
\begin{picture}(0,0)
\includegraphics{images/challenge3/resultado-questao-5-6-malha-aberta-inc}
\end{picture}%
\begin{picture}(400,270)(0,0)
\fontsize{6}{0}
\selectfont\put(52,153.188){\makebox(0,0)[t]{\textcolor[rgb]{0.15,0.15,0.15}{{$10^{-4}$}}}}
\fontsize{6}{0}
\selectfont\put(103.667,153.188){\makebox(0,0)[t]{\textcolor[rgb]{0.15,0.15,0.15}{{$10^{-3}$}}}}
\fontsize{6}{0}
\selectfont\put(155.333,153.188){\makebox(0,0)[t]{\textcolor[rgb]{0.15,0.15,0.15}{{$10^{-2}$}}}}
\fontsize{6}{0}
\selectfont\put(207,153.188){\makebox(0,0)[t]{\textcolor[rgb]{0.15,0.15,0.15}{{$10^{-1}$}}}}
\fontsize{6}{0}
\selectfont\put(258.667,153.188){\makebox(0,0)[t]{\textcolor[rgb]{0.15,0.15,0.15}{{$10^{0}$}}}}
\fontsize{6}{0}
\selectfont\put(310.333,153.188){\makebox(0,0)[t]{\textcolor[rgb]{0.15,0.15,0.15}{{$10^{1}$}}}}
\fontsize{6}{0}
\selectfont\put(362,153.188){\makebox(0,0)[t]{\textcolor[rgb]{0.15,0.15,0.15}{{$10^{2}$}}}}
\fontsize{6}{0}
\selectfont\put(48.5278,158.368){\makebox(0,0)[r]{\textcolor[rgb]{0.15,0.15,0.15}{{-100}}}}
\fontsize{6}{0}
\selectfont\put(48.5278,172.789){\makebox(0,0)[r]{\textcolor[rgb]{0.15,0.15,0.15}{{-80}}}}
\fontsize{6}{0}
\selectfont\put(48.5278,187.21){\makebox(0,0)[r]{\textcolor[rgb]{0.15,0.15,0.15}{{-60}}}}
\fontsize{6}{0}
\selectfont\put(48.5278,201.631){\makebox(0,0)[r]{\textcolor[rgb]{0.15,0.15,0.15}{{-40}}}}
\fontsize{6}{0}
\selectfont\put(48.5278,216.051){\makebox(0,0)[r]{\textcolor[rgb]{0.15,0.15,0.15}{{-20}}}}
\fontsize{6}{0}
\selectfont\put(48.5278,230.472){\makebox(0,0)[r]{\textcolor[rgb]{0.15,0.15,0.15}{{0}}}}
\fontsize{6}{0}
\selectfont\put(48.5278,244.893){\makebox(0,0)[r]{\textcolor[rgb]{0.15,0.15,0.15}{{20}}}}
\fontsize{7}{0}
\selectfont\put(207,139.188){\makebox(0,0)[t]{\textcolor[rgb]{0.15,0.15,0.15}{{Frequência [rad/s]}}}}
\fontsize{7}{0}
\selectfont\put(29.5278,201.631){\rotatebox{90}{\makebox(0,0)[b]{\textcolor[rgb]{0.15,0.15,0.15}{{Ganho [db]}}}}}
\fontsize{6}{0}
\selectfont\put(88.002,198.369){\makebox(0,0)[l]{\textcolor[rgb]{0,0,0}{{$C(jw)G(jw)$ p/ folga $= 24^{\circ}$}}}}
\fontsize{6}{0}
\selectfont\put(88.002,186.369){\makebox(0,0)[l]{\textcolor[rgb]{0,0,0}{{${C}'(jw)\overline{K}G(jw)$ c/ $w_{b}$ = 2.5}}}}
\fontsize{6}{0}
\selectfont\put(88.002,174.369){\makebox(0,0)[l]{\textcolor[rgb]{0,0,0}{{${C}'(jw)\overline{K}G(jw)$ c/ $K_{c}$ fixo}}}}
\fontsize{6}{0}
\selectfont\put(52,24.5205){\makebox(0,0)[t]{\textcolor[rgb]{0.15,0.15,0.15}{{$10^{-4}$}}}}
\fontsize{6}{0}
\selectfont\put(103.667,24.5205){\makebox(0,0)[t]{\textcolor[rgb]{0.15,0.15,0.15}{{$10^{-3}$}}}}
\fontsize{6}{0}
\selectfont\put(155.333,24.5205){\makebox(0,0)[t]{\textcolor[rgb]{0.15,0.15,0.15}{{$10^{-2}$}}}}
\fontsize{6}{0}
\selectfont\put(207,24.5205){\makebox(0,0)[t]{\textcolor[rgb]{0.15,0.15,0.15}{{$10^{-1}$}}}}
\fontsize{6}{0}
\selectfont\put(258.667,24.5205){\makebox(0,0)[t]{\textcolor[rgb]{0.15,0.15,0.15}{{$10^{0}$}}}}
\fontsize{6}{0}
\selectfont\put(310.333,24.5205){\makebox(0,0)[t]{\textcolor[rgb]{0.15,0.15,0.15}{{$10^{1}$}}}}
\fontsize{6}{0}
\selectfont\put(362,24.5205){\makebox(0,0)[t]{\textcolor[rgb]{0.15,0.15,0.15}{{$10^{2}$}}}}
\fontsize{6}{0}
\selectfont\put(48.5278,29.7002){\makebox(0,0)[r]{\textcolor[rgb]{0.15,0.15,0.15}{{-300}}}}
\fontsize{6}{0}
\selectfont\put(48.5278,42.0605){\makebox(0,0)[r]{\textcolor[rgb]{0.15,0.15,0.15}{{-250}}}}
\fontsize{6}{0}
\selectfont\put(48.5278,54.4214){\makebox(0,0)[r]{\textcolor[rgb]{0.15,0.15,0.15}{{-200}}}}
\fontsize{6}{0}
\selectfont\put(48.5278,66.7822){\makebox(0,0)[r]{\textcolor[rgb]{0.15,0.15,0.15}{{-150}}}}
\fontsize{6}{0}
\selectfont\put(48.5278,79.1431){\makebox(0,0)[r]{\textcolor[rgb]{0.15,0.15,0.15}{{-100}}}}
\fontsize{6}{0}
\selectfont\put(48.5278,91.5034){\makebox(0,0)[r]{\textcolor[rgb]{0.15,0.15,0.15}{{-50}}}}
\fontsize{6}{0}
\selectfont\put(48.5278,103.864){\makebox(0,0)[r]{\textcolor[rgb]{0.15,0.15,0.15}{{0}}}}
\fontsize{6}{0}
\selectfont\put(48.5278,116.225){\makebox(0,0)[r]{\textcolor[rgb]{0.15,0.15,0.15}{{50}}}}
\fontsize{7}{0}
\selectfont\put(207,10.5205){\makebox(0,0)[t]{\textcolor[rgb]{0.15,0.15,0.15}{{Frequência [rad/s]}}}}
\fontsize{7}{0}
\selectfont\put(29.5278,72.9624){\rotatebox{90}{\makebox(0,0)[b]{\textcolor[rgb]{0.15,0.15,0.15}{{Fase [graus]}}}}}
\fontsize{6}{0}
\selectfont\put(88.002,69.7012){\makebox(0,0)[l]{\textcolor[rgb]{0,0,0}{{$C(jw)G(jw)$ p/ folga $= 24^{\circ}$}}}}
\fontsize{6}{0}
\selectfont\put(88.002,57.7012){\makebox(0,0)[l]{\textcolor[rgb]{0,0,0}{{${C}'(jw)\overline{K}G(jw)$ c/ $w_{b}$ = 2.5}}}}
\fontsize{6}{0}
\selectfont\put(88.002,45.7007){\makebox(0,0)[l]{\textcolor[rgb]{0,0,0}{{${C}'(jw)\overline{K}G(jw)$ c/ $K_{c}$ fixo}}}}
\end{picture}

    \end{minipage}
\end{figure}

Como já evidenciado na Figura \ref{fig:desafio-3:questao-5-6-malha-aberta}, a
margem de fase foi corrigida para próximo do valor desejado novamente. A
correção se deu projetando um novo compensador em avanço fixando o valor de
$K_{c}$, isto é, projetando ${C}'(s)$ considerando que a planta é
$K_{c}\overline{K}G(s)$. O projeto foi realizado também para uma margem de fase
desejada de $60^{\circ}$ com folga de $24^{\circ}$. Após correção, a margem de
fase obtida foi de $59,88^{\circ}$, superior até à obtida com o compensador em
avanço com ganho estático unitário. Já a largura de banda teve um acréscimo,
ficando igual a 2,83 rad/s. O compensador ao final tem os parâmetros $\phi_{max}
= 54,30^{\circ}$, $\alpha = 0,10$ e $T = 1,36$, conforme Equação
\ref{eq:desafio3:compensador-com-kc-diferente-de-um}.

\begin{equation}
    \label{eq:desafio3:compensador-com-kc-diferente-de-um}
    C'(s) = 1,43\frac{1,36s + 1}{0,14s + 1}
\end{equation}

A fim de avaliar a resposta em malha fechada para os dois compensadores, foi
simulada a saída da planta no domínio do tempo. O resultado obtido é apresentado
graficamente na Figura \ref{fig:desafio-3:questao-7-dominio-do-tempo}.

\begin{figure}[ht!]
    \caption{Resposta da planta em malha fechada no domínio do tempo para
    diferentes ganhos do compensador em avanço.}
    \vspace{-10pt}
    \hspace{-30pt}
    \label{fig:desafio-3:questao-7-dominio-do-tempo}
    \begin{minipage}{\linewidth}
        % Title: gl2ps_renderer figure
% Creator: GL2PS 1.4.0, (C) 1999-2017 C. Geuzaine
% For: Octave
% CreationDate: Thu Oct  7 00:38:30 2021
\setlength{\unitlength}{1pt}
\begin{picture}(0,0)
\includegraphics{images/challenge3/resultado-questao-7-resposta-no-tempo-inc}
\end{picture}%
\begin{picture}(400,150)(0,0)
\fontsize{6}{0}
\selectfont\put(52,20){\makebox(0,0)[t]{\textcolor[rgb]{0.15,0.15,0.15}{{0}}}}
\fontsize{6}{0}
\selectfont\put(114,20){\makebox(0,0)[t]{\textcolor[rgb]{0.15,0.15,0.15}{{10}}}}
\fontsize{6}{0}
\selectfont\put(176,20){\makebox(0,0)[t]{\textcolor[rgb]{0.15,0.15,0.15}{{20}}}}
\fontsize{6}{0}
\selectfont\put(238,20){\makebox(0,0)[t]{\textcolor[rgb]{0.15,0.15,0.15}{{30}}}}
\fontsize{6}{0}
\selectfont\put(300,20){\makebox(0,0)[t]{\textcolor[rgb]{0.15,0.15,0.15}{{40}}}}
\fontsize{6}{0}
\selectfont\put(362,20){\makebox(0,0)[t]{\textcolor[rgb]{0.15,0.15,0.15}{{50}}}}
\fontsize{6}{0}
\selectfont\put(48.5278,25.188){\makebox(0,0)[r]{\textcolor[rgb]{0.15,0.15,0.15}{{-0.2}}}}
\fontsize{6}{0}
\selectfont\put(48.5278,41.4111){\makebox(0,0)[r]{\textcolor[rgb]{0.15,0.15,0.15}{{0}}}}
\fontsize{6}{0}
\selectfont\put(48.5278,57.6343){\makebox(0,0)[r]{\textcolor[rgb]{0.15,0.15,0.15}{{0.2}}}}
\fontsize{6}{0}
\selectfont\put(48.5278,73.8574){\makebox(0,0)[r]{\textcolor[rgb]{0.15,0.15,0.15}{{0.4}}}}
\fontsize{6}{0}
\selectfont\put(48.5278,90.0806){\makebox(0,0)[r]{\textcolor[rgb]{0.15,0.15,0.15}{{0.6}}}}
\fontsize{6}{0}
\selectfont\put(48.5278,106.304){\makebox(0,0)[r]{\textcolor[rgb]{0.15,0.15,0.15}{{0.8}}}}
\fontsize{6}{0}
\selectfont\put(48.5278,122.527){\makebox(0,0)[r]{\textcolor[rgb]{0.15,0.15,0.15}{{1}}}}
\fontsize{7}{0}
\selectfont\put(207,9){\makebox(0,0)[t]{\textcolor[rgb]{0.15,0.15,0.15}{{Tempo(s)}}}}
\fontsize{7}{0}
\selectfont\put(31.5278,81.9692){\rotatebox{90}{\makebox(0,0)[b]{\textcolor[rgb]{0.15,0.15,0.15}{{Saída $y(t)$}}}}}
\fontsize{6}{0}
\selectfont\put(294,64.689){\makebox(0,0)[l]{\textcolor[rgb]{0,0,0}{{$r(t)$}}}}
\fontsize{6}{0}
\selectfont\put(294,53.189){\makebox(0,0)[l]{\textcolor[rgb]{0,0,0}{{$y(t)$ p/ $K_{c} = 1$}}}}
\fontsize{6}{0}
\selectfont\put(294,41.189){\makebox(0,0)[l]{\textcolor[rgb]{0,0,0}{{$y(t)$ p/ $K_{c} = 1.43$}}}}
\end{picture}

    \end{minipage}
\end{figure}

Observa-se na Figura \ref{fig:desafio-3:questao-7-dominio-do-tempo} que ambos os
compensadores possuem uma resposta em regime permanente com erro não nulo.
Diante do diagrama de magnitude da Figura
\ref{fig:desafio-3:questao-5-6-malha-aberta}, nota-se que o ganho em baixas
frequências é menor quando $K_c = 1$ e um pouco maior para $K_c = 1,43$.
Exatamente essa diferença que explica o erro não nulo em regime permanente para
$y(t)$ em ambos os casos e também porque o erro é menor para o sistema com
compensador de maior ganho estático. Esta interpretação fica mais palpável
quando aplicado ao Teorema do valor Final em que para uma referência do tipo
degrau

\begin{equation}
    \label{eq:desafio3-teorema-do-valor-final}
    \lim_{t \rightarrow \infty }e(t)=\lim_{s \rightarrow 0 }sE(s)=\frac{1}{1+K_v},
\end{equation}
onde:

\begin{conditions*}
    K_v & igual $\lim_{s \rightarrow 0 }C(0)G(0)$.
\end{conditions*}

Assim, para $K_c = 1$ temos que $e(\infty) = 0,5$ e $y(\infty) = 0,5$. Já para
$K_c = 1,43$ $e(\infty) = 0,41$ e $y(\infty) = 0,59$, condizente com o gráfico.
Uma outra interpretação é da resposta do regime transitório. Como para o sistema
com compensador de ganho unitário a margem de fase é menor, consequentemente
o fator de amortecimento $\xi$ também é menor, decorrendo em um maior
\textit{overshoot} da resposta em malha fechada do sistema para uma referência
do tipo degrau.

Objetivando melhorar a resposta em regime permanente, foi adicionado um
compensador em atraso para melhorar o ganho do sistema em malha aberta nas
baixas frequências. O compensador em atraso projetado é governado pela Equação
\ref{eq:desafio3:compensador-em-atraso}. Para manter os requisitos anteriores, o
compensador em avanço foi mantido, tornando o controlador final com a topologia
de avanço-atraso. Dessa forma o valor de $\beta$ foi calculado para $C(0)G(0) =
5$, em que $C(s) = C'(s)\overline{K}C_{at}$, conforme Equação
\ref{eq:desafio3:beta} e $T_{at} = 5T$.

\begin{equation}
    \label{eq:desafio3:compensador-em-atraso}
    C_{at} = \frac{\beta (T_{at}s + 1)}{\beta T_{at}s+1}
\end{equation}

\begin{equation}
    \label{eq:desafio3:beta}
    \beta = \frac{C(0)G(0)}{C'(0)\overline{K}G(0)} = \frac{5}{C'(0)\overline{K}G(0)}
\end{equation}

Portanto, para o sistem com o compensador em avanço com $K_c = 1$, foi
encontrado $\beta = 5,00$ e $T_{at} = 6,40$. Já para $K_c = 1,43$, chegou-se aos
valores de $\beta = 3,49$ e $T_{at} = 6,82$. Assim, os dois compensadores em
atraso resultaram nas Equações
\ref{eq:desafio3:compensador-em-atraso-para-kc-unitario} e
\ref{eq:desafio3:compensador-em-atraso-para-kc-nao-unitario}, para os respectivos
ganhos do compensador em avanço. A resposta em frequência de ambos
compensadores em atraso bem como com os controladores avanço-atraso é expressa na
Figura \ref{fig:desafio-3:resposta-em-frequencia-dos-compensadores}.

\begin{equation}
    \label{eq:desafio3:compensador-em-atraso-para-kc-unitario}
    C_{at} = \frac{5,00 (6,40s + 1)}{32s+1}
\end{equation}

\begin{equation}
    \label{eq:desafio3:compensador-em-atraso-para-kc-nao-unitario}
    C_{at} = \frac{3,87 (6,82s + 1)}{23,78s+1}
\end{equation}

\begin{figure}[ht!]
    \caption{Resposta em frequência dos compensadores em atraso e avanço-atraso
    para diferentes valores de $K_c$.}
    \vspace{-10pt}
    \hspace{-30pt}
    \label{fig:desafio-3:resposta-em-frequencia-dos-compensadores}
    \begin{minipage}{\linewidth}
        % Title: gl2ps_renderer figure
% Creator: GL2PS 1.4.0, (C) 1999-2017 C. Geuzaine
% For: Octave
% CreationDate: Thu Oct  7 22:32:19 2021
\setlength{\unitlength}{1pt}
\begin{picture}(0,0)
\includegraphics{images/challenge3/resultado-questao-8-9-compensadores-inc}
\end{picture}%
\begin{picture}(400,220)(0,0)
\fontsize{6}{0}
\selectfont\put(52,127.841){\makebox(0,0)[t]{\textcolor[rgb]{0.15,0.15,0.15}{{$10^{-4}$}}}}
\fontsize{6}{0}
\selectfont\put(86.333,127.841){\makebox(0,0)[t]{\textcolor[rgb]{0.15,0.15,0.15}{{$10^{-3}$}}}}
\fontsize{6}{0}
\selectfont\put(120.666,127.841){\makebox(0,0)[t]{\textcolor[rgb]{0.15,0.15,0.15}{{$10^{-2}$}}}}
\fontsize{6}{0}
\selectfont\put(154.999,127.841){\makebox(0,0)[t]{\textcolor[rgb]{0.15,0.15,0.15}{{$10^{-1}$}}}}
\fontsize{6}{0}
\selectfont\put(189.332,127.841){\makebox(0,0)[t]{\textcolor[rgb]{0.15,0.15,0.15}{{$10^{0}$}}}}
\fontsize{6}{0}
\selectfont\put(223.665,127.841){\makebox(0,0)[t]{\textcolor[rgb]{0.15,0.15,0.15}{{$10^{1}$}}}}
\fontsize{6}{0}
\selectfont\put(257.997,127.841){\makebox(0,0)[t]{\textcolor[rgb]{0.15,0.15,0.15}{{$10^{2}$}}}}
\fontsize{6}{0}
\selectfont\put(48.5278,133.022){\makebox(0,0)[r]{\textcolor[rgb]{0.15,0.15,0.15}{{-5}}}}
\fontsize{6}{0}
\selectfont\put(48.5278,142.261){\makebox(0,0)[r]{\textcolor[rgb]{0.15,0.15,0.15}{{0}}}}
\fontsize{6}{0}
\selectfont\put(48.5278,151.499){\makebox(0,0)[r]{\textcolor[rgb]{0.15,0.15,0.15}{{5}}}}
\fontsize{6}{0}
\selectfont\put(48.5278,160.737){\makebox(0,0)[r]{\textcolor[rgb]{0.15,0.15,0.15}{{10}}}}
\fontsize{6}{0}
\selectfont\put(48.5278,169.976){\makebox(0,0)[r]{\textcolor[rgb]{0.15,0.15,0.15}{{15}}}}
\fontsize{6}{0}
\selectfont\put(48.5278,179.214){\makebox(0,0)[r]{\textcolor[rgb]{0.15,0.15,0.15}{{20}}}}
\fontsize{6}{0}
\selectfont\put(48.5278,188.452){\makebox(0,0)[r]{\textcolor[rgb]{0.15,0.15,0.15}{{25}}}}
\fontsize{6}{0}
\selectfont\put(48.5278,197.691){\makebox(0,0)[r]{\textcolor[rgb]{0.15,0.15,0.15}{{30}}}}
\fontsize{7}{0}
\selectfont\put(154.999,113.841){\makebox(0,0)[t]{\textcolor[rgb]{0.15,0.15,0.15}{{Frequência [rad/s]}}}}
\fontsize{7}{0}
\selectfont\put(35.5278,165.356){\rotatebox{90}{\makebox(0,0)[b]{\textcolor[rgb]{0.15,0.15,0.15}{{Ganho [db]}}}}}
\fontsize{6}{0}
\selectfont\put(294,189.69){\makebox(0,0)[l]{\textcolor[rgb]{0,0,0}{{$C_{at}(jw)$ p/ $K_{c} = 1$}}}}
\fontsize{6}{0}
\selectfont\put(294,177.69){\makebox(0,0)[l]{\textcolor[rgb]{0,0,0}{{$C_{at}(jw)$ p/ $K_{c} = 1,43$}}}}
\fontsize{6}{0}
\selectfont\put(294,165.69){\makebox(0,0)[l]{\textcolor[rgb]{0,0,0}{{$C'(jw)\overline{K}C_{at}$ p/ $K_c = 1$}}}}
\fontsize{6}{0}
\selectfont\put(294,153.69){\makebox(0,0)[l]{\textcolor[rgb]{0,0,0}{{$C'(jw)\overline{K}C_{at}$ p/ $K_c = 1,43$}}}}
\fontsize{6}{0}
\selectfont\put(52,23){\makebox(0,0)[t]{\textcolor[rgb]{0.15,0.15,0.15}{{$10^{-4}$}}}}
\fontsize{6}{0}
\selectfont\put(86.333,23){\makebox(0,0)[t]{\textcolor[rgb]{0.15,0.15,0.15}{{$10^{-3}$}}}}
\fontsize{6}{0}
\selectfont\put(120.666,23){\makebox(0,0)[t]{\textcolor[rgb]{0.15,0.15,0.15}{{$10^{-2}$}}}}
\fontsize{6}{0}
\selectfont\put(154.999,23){\makebox(0,0)[t]{\textcolor[rgb]{0.15,0.15,0.15}{{$10^{-1}$}}}}
\fontsize{6}{0}
\selectfont\put(189.332,23){\makebox(0,0)[t]{\textcolor[rgb]{0.15,0.15,0.15}{{$10^{0}$}}}}
\fontsize{6}{0}
\selectfont\put(223.665,23){\makebox(0,0)[t]{\textcolor[rgb]{0.15,0.15,0.15}{{$10^{1}$}}}}
\fontsize{6}{0}
\selectfont\put(257.997,23){\makebox(0,0)[t]{\textcolor[rgb]{0.15,0.15,0.15}{{$10^{2}$}}}}
\fontsize{6}{0}
\selectfont\put(48.5278,28.1816){\makebox(0,0)[r]{\textcolor[rgb]{0.15,0.15,0.15}{{-60}}}}
\fontsize{6}{0}
\selectfont\put(48.5278,38.9595){\makebox(0,0)[r]{\textcolor[rgb]{0.15,0.15,0.15}{{-40}}}}
\fontsize{6}{0}
\selectfont\put(48.5278,49.7378){\makebox(0,0)[r]{\textcolor[rgb]{0.15,0.15,0.15}{{-20}}}}
\fontsize{6}{0}
\selectfont\put(48.5278,60.5156){\makebox(0,0)[r]{\textcolor[rgb]{0.15,0.15,0.15}{{0}}}}
\fontsize{6}{0}
\selectfont\put(48.5278,71.2939){\makebox(0,0)[r]{\textcolor[rgb]{0.15,0.15,0.15}{{20}}}}
\fontsize{6}{0}
\selectfont\put(48.5278,82.0718){\makebox(0,0)[r]{\textcolor[rgb]{0.15,0.15,0.15}{{40}}}}
\fontsize{6}{0}
\selectfont\put(48.5278,92.8501){\makebox(0,0)[r]{\textcolor[rgb]{0.15,0.15,0.15}{{60}}}}
\fontsize{7}{0}
\selectfont\put(154.999,9){\makebox(0,0)[t]{\textcolor[rgb]{0.15,0.15,0.15}{{Frequência [rad/s]}}}}
\fontsize{7}{0}
\selectfont\put(33.5278,60.5156){\rotatebox{90}{\makebox(0,0)[b]{\textcolor[rgb]{0.15,0.15,0.15}{{Fase [graus]}}}}}
\fontsize{6}{0}
\selectfont\put(294,84.8496){\makebox(0,0)[l]{\textcolor[rgb]{0,0,0}{{$C_{at}(jw)$ p/ $K_{c} = 1$}}}}
\fontsize{6}{0}
\selectfont\put(294,72.8496){\makebox(0,0)[l]{\textcolor[rgb]{0,0,0}{{$C_{at}(jw)$ p/ $K_{c} = 1,43$}}}}
\fontsize{6}{0}
\selectfont\put(294,60.8491){\makebox(0,0)[l]{\textcolor[rgb]{0,0,0}{{$C'(jw)\overline{K}C_{at}$ p/ $K_c = 1$}}}}
\fontsize{6}{0}
\selectfont\put(294,48.8491){\makebox(0,0)[l]{\textcolor[rgb]{0,0,0}{{$C'(jw)\overline{K}C_{at}$ p/ $K_c = 1,43$}}}}
\end{picture}

    \end{minipage}
\end{figure}

Como pode ser analisado na Figura
\ref{fig:desafio-3:resposta-em-frequencia-dos-compensadores}, o compensador em
atraso cumpriu seu objetivo contribuindo com um alto ganho nas baixas frequências
ao mesmo tempo que não interferiu nas altas frequências. É interessante notar
que embora os compensadores em atraso aprensentem ganhos diferentes nas baixas
frequências, quando compostos com o compensador em avaço a diferença desaparece.
Isto está em concordância com a teoria, pois apesar dos valores de $\beta$
serem diferentes, ambos compensadores em atraso foram projetados para ter o
mesmo ganho estático $C(0)G(0)$ e, portanto, o ganho para baixas frequências
serão iguais, divergindo com o aumento da frequência.

O fato observado do ganho em baixas frequências fica mais evidente quando
analisado a resposta em frequência de $C(s)G(s)$ para os dois compensadores
avanço-atraso na Figura
\ref{fig:desafio-3:resposta-em-frequencia-da-malha-aberta}. Percebe-se que os
valores de ganho dos dois compensadores avanço-atraso são próximos até 0,1
rad/s. Vale constatar também a não interferência nas altas frequências. Como
pode ser visualizado na Figura
\ref{fig:desafio-3:resposta-em-frequencia-da-malha-aberta}, a margem de fase
teve uma leve modificação, não sendo possível a ver a diferença devido a quase
completa sobreposição das linhas indicadoras da margem de fase.

\begin{figure}[H]
    \caption{Resposta em frequência do sistema de malha aberta com compensadores
    em atraso e avanço-atraso para diferentes valores de $K_c$.}
    \vspace{-10pt}
    \hspace{-30pt}
    \label{fig:desafio-3:resposta-em-frequencia-da-malha-aberta}
    \begin{minipage}{\linewidth}
        % Title: gl2ps_renderer figure
% Creator: GL2PS 1.4.0, (C) 1999-2017 C. Geuzaine
% For: Octave
% CreationDate: Thu Oct  7 22:31:32 2021
\setlength{\unitlength}{1pt}
\begin{picture}(0,0)
\includegraphics{images/challenge3/resultado-questao-8-9-malha-aberta-inc}
\end{picture}%
\begin{picture}(400,270)(0,0)
\fontsize{6}{0}
\selectfont\put(52,153.188){\makebox(0,0)[t]{\textcolor[rgb]{0.15,0.15,0.15}{{$10^{-4}$}}}}
\fontsize{6}{0}
\selectfont\put(84.9995,153.188){\makebox(0,0)[t]{\textcolor[rgb]{0.15,0.15,0.15}{{$10^{-3}$}}}}
\fontsize{6}{0}
\selectfont\put(117.999,153.188){\makebox(0,0)[t]{\textcolor[rgb]{0.15,0.15,0.15}{{$10^{-2}$}}}}
\fontsize{6}{0}
\selectfont\put(150.999,153.188){\makebox(0,0)[t]{\textcolor[rgb]{0.15,0.15,0.15}{{$10^{-1}$}}}}
\fontsize{6}{0}
\selectfont\put(183.998,153.188){\makebox(0,0)[t]{\textcolor[rgb]{0.15,0.15,0.15}{{$10^{0}$}}}}
\fontsize{6}{0}
\selectfont\put(216.998,153.188){\makebox(0,0)[t]{\textcolor[rgb]{0.15,0.15,0.15}{{$10^{1}$}}}}
\fontsize{6}{0}
\selectfont\put(249.997,153.188){\makebox(0,0)[t]{\textcolor[rgb]{0.15,0.15,0.15}{{$10^{2}$}}}}
\fontsize{6}{0}
\selectfont\put(48.5278,158.368){\makebox(0,0)[r]{\textcolor[rgb]{0.15,0.15,0.15}{{-100}}}}
\fontsize{6}{0}
\selectfont\put(48.5278,172.789){\makebox(0,0)[r]{\textcolor[rgb]{0.15,0.15,0.15}{{-80}}}}
\fontsize{6}{0}
\selectfont\put(48.5278,187.21){\makebox(0,0)[r]{\textcolor[rgb]{0.15,0.15,0.15}{{-60}}}}
\fontsize{6}{0}
\selectfont\put(48.5278,201.631){\makebox(0,0)[r]{\textcolor[rgb]{0.15,0.15,0.15}{{-40}}}}
\fontsize{6}{0}
\selectfont\put(48.5278,216.051){\makebox(0,0)[r]{\textcolor[rgb]{0.15,0.15,0.15}{{-20}}}}
\fontsize{6}{0}
\selectfont\put(48.5278,230.472){\makebox(0,0)[r]{\textcolor[rgb]{0.15,0.15,0.15}{{0}}}}
\fontsize{6}{0}
\selectfont\put(48.5278,244.893){\makebox(0,0)[r]{\textcolor[rgb]{0.15,0.15,0.15}{{20}}}}
\fontsize{7}{0}
\selectfont\put(150.999,139.188){\makebox(0,0)[t]{\textcolor[rgb]{0.15,0.15,0.15}{{Frequência [rad/s]}}}}
\fontsize{7}{0}
\selectfont\put(29.5278,201.631){\rotatebox{90}{\makebox(0,0)[b]{\textcolor[rgb]{0.15,0.15,0.15}{{Ganho [db]}}}}}
\fontsize{6}{0}
\selectfont\put(286,236.893){\makebox(0,0)[l]{\textcolor[rgb]{0,0,0}{{$C(jw)G(jw)$ p/ $K_c = 1$}}}}
\fontsize{6}{0}
\selectfont\put(286,224.893){\makebox(0,0)[l]{\textcolor[rgb]{0,0,0}{{$C(jw)G(jw)$ p/ $K_c = 1,43$}}}}
\fontsize{6}{0}
\selectfont\put(286,212.893){\makebox(0,0)[l]{\textcolor[rgb]{0,0,0}{{$C'(jw)\overline{K}G(jw)$ p/ $K_c = 1$}}}}
\fontsize{6}{0}
\selectfont\put(286,200.892){\makebox(0,0)[l]{\textcolor[rgb]{0,0,0}{{$C'(jw)\overline{K}G(jw)$ p/ $K_c = 1,43$}}}}
\fontsize{6}{0}
\selectfont\put(52,24.5205){\makebox(0,0)[t]{\textcolor[rgb]{0.15,0.15,0.15}{{$10^{-4}$}}}}
\fontsize{6}{0}
\selectfont\put(84.9995,24.5205){\makebox(0,0)[t]{\textcolor[rgb]{0.15,0.15,0.15}{{$10^{-3}$}}}}
\fontsize{6}{0}
\selectfont\put(117.999,24.5205){\makebox(0,0)[t]{\textcolor[rgb]{0.15,0.15,0.15}{{$10^{-2}$}}}}
\fontsize{6}{0}
\selectfont\put(150.999,24.5205){\makebox(0,0)[t]{\textcolor[rgb]{0.15,0.15,0.15}{{$10^{-1}$}}}}
\fontsize{6}{0}
\selectfont\put(183.998,24.5205){\makebox(0,0)[t]{\textcolor[rgb]{0.15,0.15,0.15}{{$10^{0}$}}}}
\fontsize{6}{0}
\selectfont\put(216.998,24.5205){\makebox(0,0)[t]{\textcolor[rgb]{0.15,0.15,0.15}{{$10^{1}$}}}}
\fontsize{6}{0}
\selectfont\put(249.997,24.5205){\makebox(0,0)[t]{\textcolor[rgb]{0.15,0.15,0.15}{{$10^{2}$}}}}
\fontsize{6}{0}
\selectfont\put(48.5278,29.7002){\makebox(0,0)[r]{\textcolor[rgb]{0.15,0.15,0.15}{{-300}}}}
\fontsize{6}{0}
\selectfont\put(48.5278,42.0605){\makebox(0,0)[r]{\textcolor[rgb]{0.15,0.15,0.15}{{-250}}}}
\fontsize{6}{0}
\selectfont\put(48.5278,54.4214){\makebox(0,0)[r]{\textcolor[rgb]{0.15,0.15,0.15}{{-200}}}}
\fontsize{6}{0}
\selectfont\put(48.5278,66.7822){\makebox(0,0)[r]{\textcolor[rgb]{0.15,0.15,0.15}{{-150}}}}
\fontsize{6}{0}
\selectfont\put(48.5278,79.1431){\makebox(0,0)[r]{\textcolor[rgb]{0.15,0.15,0.15}{{-100}}}}
\fontsize{6}{0}
\selectfont\put(48.5278,91.5034){\makebox(0,0)[r]{\textcolor[rgb]{0.15,0.15,0.15}{{-50}}}}
\fontsize{6}{0}
\selectfont\put(48.5278,103.864){\makebox(0,0)[r]{\textcolor[rgb]{0.15,0.15,0.15}{{0}}}}
\fontsize{6}{0}
\selectfont\put(48.5278,116.225){\makebox(0,0)[r]{\textcolor[rgb]{0.15,0.15,0.15}{{50}}}}
\fontsize{7}{0}
\selectfont\put(150.999,10.5205){\makebox(0,0)[t]{\textcolor[rgb]{0.15,0.15,0.15}{{Frequência [rad/s]}}}}
\fontsize{7}{0}
\selectfont\put(29.5278,72.9624){\rotatebox{90}{\makebox(0,0)[b]{\textcolor[rgb]{0.15,0.15,0.15}{{Fase [graus]}}}}}
\fontsize{6}{0}
\selectfont\put(286,108.225){\makebox(0,0)[l]{\textcolor[rgb]{0,0,0}{{$C(jw)G(jw)$ p/ $K_c = 1$}}}}
\fontsize{6}{0}
\selectfont\put(286,96.2246){\makebox(0,0)[l]{\textcolor[rgb]{0,0,0}{{$C(jw)G(jw)$ p/ $K_c = 1,43$}}}}
\fontsize{6}{0}
\selectfont\put(286,84.2241){\makebox(0,0)[l]{\textcolor[rgb]{0,0,0}{{$C'(jw)\overline{K}G(jw)$ p/ $K_c = 1$}}}}
\fontsize{6}{0}
\selectfont\put(286,72.2241){\makebox(0,0)[l]{\textcolor[rgb]{0,0,0}{{$C'(jw)\overline{K}G(jw)$ p/ $K_c = 1,43$}}}}
\end{picture}

    \end{minipage}
\end{figure}

Por fim, foi analisado também a resposta em malha fechada no domínio do tempo
para os compensadores avanço-atraso, fazendo uma comparação deles com os
compensadores de somente avanço. O resultado obtido é expresso na Figura
\ref{fig:desafio-3:resposta-dominio-do-tempo-dos-compensadores-avanco-atraso}.

\begin{figure}[H]
    \caption{Resposta no domínio do tempo do sistema de malha fechada com
    compensadores avanço-atraso comparado aos compensadores de somente avanço
    para diferentes valores de $K_c$.}
    \vspace{-10pt}
    \hspace{-30pt}
    \label{fig:desafio-3:resposta-dominio-do-tempo-dos-compensadores-avanco-atraso}
    \begin{minipage}{\linewidth}
        % Title: gl2ps_renderer figure
% Creator: GL2PS 1.4.0, (C) 1999-2017 C. Geuzaine
% For: Octave
% CreationDate: Fri Oct  8 00:18:29 2021
\setlength{\unitlength}{1pt}
\begin{picture}(0,0)
\includegraphics{images/challenge3/resultado-questao-8-9-resposta-no-tempo-inc}
\end{picture}%
\begin{picture}(400,250)(0,0)
\fontsize{6}{0}
\selectfont\put(52,141.404){\makebox(0,0)[t]{\textcolor[rgb]{0.15,0.15,0.15}{{0}}}}
\fontsize{6}{0}
\selectfont\put(106,141.404){\makebox(0,0)[t]{\textcolor[rgb]{0.15,0.15,0.15}{{50}}}}
\fontsize{6}{0}
\selectfont\put(159.999,141.404){\makebox(0,0)[t]{\textcolor[rgb]{0.15,0.15,0.15}{{100}}}}
\fontsize{6}{0}
\selectfont\put(213.998,141.404){\makebox(0,0)[t]{\textcolor[rgb]{0.15,0.15,0.15}{{150}}}}
\fontsize{6}{0}
\selectfont\put(267.997,141.404){\makebox(0,0)[t]{\textcolor[rgb]{0.15,0.15,0.15}{{200}}}}
\fontsize{6}{0}
\selectfont\put(48.5278,146.637){\makebox(0,0)[r]{\textcolor[rgb]{0.15,0.15,0.15}{{-0.2}}}}
\fontsize{6}{0}
\selectfont\put(48.5278,157.258){\makebox(0,0)[r]{\textcolor[rgb]{0.15,0.15,0.15}{{0}}}}
\fontsize{6}{0}
\selectfont\put(48.5278,167.88){\makebox(0,0)[r]{\textcolor[rgb]{0.15,0.15,0.15}{{0.2}}}}
\fontsize{6}{0}
\selectfont\put(48.5278,178.501){\makebox(0,0)[r]{\textcolor[rgb]{0.15,0.15,0.15}{{0.4}}}}
\fontsize{6}{0}
\selectfont\put(48.5278,189.122){\makebox(0,0)[r]{\textcolor[rgb]{0.15,0.15,0.15}{{0.6}}}}
\fontsize{6}{0}
\selectfont\put(48.5278,199.744){\makebox(0,0)[r]{\textcolor[rgb]{0.15,0.15,0.15}{{0.8}}}}
\fontsize{6}{0}
\selectfont\put(48.5278,210.365){\makebox(0,0)[r]{\textcolor[rgb]{0.15,0.15,0.15}{{1}}}}
\fontsize{6}{0}
\selectfont\put(48.5278,220.986){\makebox(0,0)[r]{\textcolor[rgb]{0.15,0.15,0.15}{{1.2}}}}
\fontsize{7}{0}
\selectfont\put(31.5278,186.325){\rotatebox{90}{\makebox(0,0)[b]{\textcolor[rgb]{0.15,0.15,0.15}{{Saída $y(t)$}}}}}
\fontsize{7}{0}
\selectfont\put(159.999,130.404){\makebox(0,0)[t]{\textcolor[rgb]{0.15,0.15,0.15}{{Tempo(jw)}}}}
\fontsize{6}{0}
\selectfont\put(304,218.512){\makebox(0,0)[l]{\textcolor[rgb]{0,0,0}{{$r(t)$}}}}
\fontsize{6}{0}
\selectfont\put(304,207.012){\makebox(0,0)[l]{\textcolor[rgb]{0,0,0}{{$y(t)$ c/ $C_{at}$ e p/ $K_{c} = 1$}}}}
\fontsize{6}{0}
\selectfont\put(304,195.011){\makebox(0,0)[l]{\textcolor[rgb]{0,0,0}{{$y(t)$ c/ $C_{at}$ e p/ $K_{c} = 1.43$}}}}
\fontsize{6}{0}
\selectfont\put(304,183.011){\makebox(0,0)[l]{\textcolor[rgb]{0,0,0}{{$y(t)$  s/ $C_{at}$ e p/ $K_{c} = 1$}}}}
\fontsize{6}{0}
\selectfont\put(304,171.011){\makebox(0,0)[l]{\textcolor[rgb]{0,0,0}{{$y(t)$ s/ $C_{at}$ e p/ $K_{c} = 1.43$}}}}
\fontsize{6}{0}
\selectfont\put(52,22.2671){\makebox(0,0)[t]{\textcolor[rgb]{0.15,0.15,0.15}{{0}}}}
\fontsize{6}{0}
\selectfont\put(106,22.2671){\makebox(0,0)[t]{\textcolor[rgb]{0.15,0.15,0.15}{{50}}}}
\fontsize{6}{0}
\selectfont\put(159.999,22.2671){\makebox(0,0)[t]{\textcolor[rgb]{0.15,0.15,0.15}{{100}}}}
\fontsize{6}{0}
\selectfont\put(213.998,22.2671){\makebox(0,0)[t]{\textcolor[rgb]{0.15,0.15,0.15}{{150}}}}
\fontsize{6}{0}
\selectfont\put(267.997,22.2671){\makebox(0,0)[t]{\textcolor[rgb]{0.15,0.15,0.15}{{200}}}}
\fontsize{6}{0}
\selectfont\put(48.5278,32.5264){\makebox(0,0)[r]{\textcolor[rgb]{0.15,0.15,0.15}{{-0.2}}}}
\fontsize{6}{0}
\selectfont\put(48.5278,43.1475){\makebox(0,0)[r]{\textcolor[rgb]{0.15,0.15,0.15}{{0}}}}
\fontsize{6}{0}
\selectfont\put(48.5278,53.7686){\makebox(0,0)[r]{\textcolor[rgb]{0.15,0.15,0.15}{{0.2}}}}
\fontsize{6}{0}
\selectfont\put(48.5278,64.3901){\makebox(0,0)[r]{\textcolor[rgb]{0.15,0.15,0.15}{{0.4}}}}
\fontsize{6}{0}
\selectfont\put(48.5278,75.0112){\makebox(0,0)[r]{\textcolor[rgb]{0.15,0.15,0.15}{{0.6}}}}
\fontsize{6}{0}
\selectfont\put(48.5278,85.6323){\makebox(0,0)[r]{\textcolor[rgb]{0.15,0.15,0.15}{{0.8}}}}
\fontsize{6}{0}
\selectfont\put(48.5278,96.2539){\makebox(0,0)[r]{\textcolor[rgb]{0.15,0.15,0.15}{{1}}}}
\fontsize{7}{0}
\selectfont\put(31.5278,67.1875){\rotatebox{90}{\makebox(0,0)[b]{\textcolor[rgb]{0.15,0.15,0.15}{{Erro $e(t)$}}}}}
\fontsize{7}{0}
\selectfont\put(159.999,11.2671){\makebox(0,0)[t]{\textcolor[rgb]{0.15,0.15,0.15}{{Tempo(jw)}}}}
\fontsize{6}{0}
\selectfont\put(304,98.8745){\makebox(0,0)[l]{\textcolor[rgb]{0,0,0}{{$e(t)$ c/ $C_{at}$ e p/ $K_{c} = 1$}}}}
\fontsize{6}{0}
\selectfont\put(304,86.8745){\makebox(0,0)[l]{\textcolor[rgb]{0,0,0}{{$e(t)$ c/ $C_{at}$ e p/ $K_{c} = 1.43$}}}}
\fontsize{6}{0}
\selectfont\put(304,74.8745){\makebox(0,0)[l]{\textcolor[rgb]{0,0,0}{{$e(t)$ s/ $C_{at}$ e p/ $K_{c} = 1$}}}}
\fontsize{6}{0}
\selectfont\put(304,62.874){\makebox(0,0)[l]{\textcolor[rgb]{0,0,0}{{$e(t)$ s/ $C_{at}$ e p/ $K_{c} = 1.43$}}}}
\end{picture}

    \end{minipage}
\end{figure}

Conforme esperado, nota-se na Figura
\ref{fig:desafio-3:resposta-dominio-do-tempo-dos-compensadores-avanco-atraso}.
que o regime permanente melhorou de forma igualitária para os dois compensadores
de avanço atraso. A igualidade é devido ao ganho estático fornecido pelo
compensador em atraso, cujo para ambos os casos $C(0)G(0) = 5$. O mesmo cenário
de melhoria e igualidade acontece para o regime permanente na ocorrência de uma
perturbação na entrada do processo. Para comprovar isto, basta analisar
novamente o Teorema do valor Final para $y(t)$ dado $q_u(t)$ do tipo degrau. O
valor de $y(t)$ em regime permanente é calculado conforme a Equação
\ref{eq:desafio3-teorema-do-valor-com-perturbacao-na-entrada}. Assim, para ambos
valores de $K_c$ e considerando uma perturbação na entrada do tipo degrau,
tem-se que $e(\infty) = 0,08$ e $y(\infty) = 0,92$. 

\begin{equation}
    \label{eq:desafio3-teorema-do-valor-com-perturbacao-na-entrada}
    \lim_{t \rightarrow \infty }e(t)=\lim_{s \rightarrow 0 }s
    \left [ \frac{G(s)}{1+C(s)G(s)}Q_u(s)  \right ]
    = \frac{G(0)}{1+K_v},
\end{equation}

\subsection{Conclusões}
Em suma, os resultados apresentados evidenciaram de forma muito clara e objetiva
a utilização do critério de Nyquist para projeto de controladores. A evidência
fica clara principalmente quando foi observado as margens de fase se alterando
quando os parâmetros dos compensadores eram também modificados. Fica notório
também que em nenhum momento foi necessário utilizar o conceito de dominância
modal para projetar os compensadores. Por fim, foi possível observar a
flexibilidade que o projeto no domínio da frequência oferece ao projetista,
sendo possível analisar os resultados ainda com o sistema em malha aberta.
