\section{Desafio III - Resposta em Frequência} 

\subsection{Motivação}
Este último desafio da série visa reduzir a abstração do critério de Nyquist,
evidenciando como ele pode ser aplicado no projeto de controladores no domínio
da frequência. Além disso, o uso da resposta em frequência junto com o critério
de Nyquist permite que os controladores sejam projetados apenas com a função de
transferência de malha aberta, sem precisar também do conceito de dominância
modal. Portanto, poder projetar controladores no domínio da frequência é mais
uma ferramenta que o projetista tem em mãos na hora de decidir qual abordagem
seguir na resolução de um determinado problema de controle. Os campos que se
beneficiam dessa abordagem são principalmente aqueles que envolvam eletrônica de
potência, como o controle de tensão na saída de conversores DC-DC regulados, e
vibrações. 

\subsection{Simulações realizadas}
As simulações realizadas focaram no design de compensadores de avanço, de atraso
e de avanço-atraso de fase.

Primeiro as simulações foram realizadas para analisar a
resposta em frequência de um compensador em avanço variando primeiro a folga de
fase utilizada no projeto; depois projetando o compensador com o requisito de
largura de banda de malha aberta; e no fim desta primeira parte, analisou o
compensador com diferentes ganhos.

A segunda etapa consistiu na análise no
domínio do tempo da resposta do sistema em malha fechada com os diferentes
compensadores projetados na primeira parte e também com a adição de um filtro de
referência.

Por fim, foi realizada novamente a simulação do sistema em malha fechada,
entretanto, com a adição de um compensador em atraso, ficando o compensador
final com a topologia de avanço-atraso. Também avaliou-se a resposta em malha
fechada com um filtro de referência. Adicionalmente, foi avaliada a resposta do
sistema para uma perturbação do tipo degrau na entrada da planta.

Também como nos dois desafios anteriores, este desafio teve como referência o
sistema em malha fechada ilustrado pela Figura
\ref{fig:diagrama-de-blocos-malha-fechada} e descrita no tópico
\ref{sec:desafio-1-simulacoes-realizadas}. A função de transferência da
planta/processo utilizada está definida na Equação \ref{eq:desafio-3:g-de-s}. O
tópico posterior descreve e contém discussões dos resultados obtidos.

\begin{equation}
    \label{eq:desafio-3:g-de-s}
    G(s) = \frac{0,5}{(s^2 + 0,6s +1)(0,1s + 1)}.
\end{equation}

\subsection{Resultados obtidos}
As simulações começaram com a definição de $\overline{K}$ tal que
$\overline{K}G(0) = 1$. Através de álgebra simples, achou-se $\overline{K} = 2$.
Com a definição desse valor de ganho, foi calculada a largura de banda de $P(s)
= \overline{K}G(s)$, cujo resultado é ilustrado na Figura
\ref{fig:desafio-3:questao-3}. A largura de banda $w_{b} \approx 1,45$ rad/s.
Vale ressaltar que como foi realizado computação numérica, os valores
serão aproximados devido a discretização da magnitude e fase no diagrama de bode.

\begin{figure}[!ht]
    \caption{Magnitude da resposta em frequência de $P(s)$
    com destaque na largura de banda.}
    \vspace{-10pt}
    \hspace{-30pt}
    \label{fig:desafio-3:questao-3}
    \begin{minipage}{\linewidth}
        % Title: gl2ps_renderer figure
% Creator: GL2PS 1.4.0, (C) 1999-2017 C. Geuzaine
% For: Octave
% CreationDate: Wed Oct  6 21:32:31 2021
\setlength{\unitlength}{1pt}
\begin{picture}(0,0)
\includegraphics{images/challenge3/resultado-questao-2-inc}
\end{picture}%
\begin{picture}(400,120)(0,0)
\fontsize{6}{0}
\selectfont\put(52,23){\makebox(0,0)[t]{\textcolor[rgb]{0.15,0.15,0.15}{{$10^{-4}$}}}}
\fontsize{6}{0}
\selectfont\put(103.667,23){\makebox(0,0)[t]{\textcolor[rgb]{0.15,0.15,0.15}{{$10^{-3}$}}}}
\fontsize{6}{0}
\selectfont\put(155.333,23){\makebox(0,0)[t]{\textcolor[rgb]{0.15,0.15,0.15}{{$10^{-2}$}}}}
\fontsize{6}{0}
\selectfont\put(207,23){\makebox(0,0)[t]{\textcolor[rgb]{0.15,0.15,0.15}{{$10^{-1}$}}}}
\fontsize{6}{0}
\selectfont\put(258.667,23){\makebox(0,0)[t]{\textcolor[rgb]{0.15,0.15,0.15}{{$10^{0}$}}}}
\fontsize{6}{0}
\selectfont\put(310.333,23){\makebox(0,0)[t]{\textcolor[rgb]{0.15,0.15,0.15}{{$10^{1}$}}}}
\fontsize{6}{0}
\selectfont\put(362,23){\makebox(0,0)[t]{\textcolor[rgb]{0.15,0.15,0.15}{{$10^{2}$}}}}
\fontsize{6}{0}
\selectfont\put(48.5278,31.9443){\makebox(0,0)[r]{\textcolor[rgb]{0.15,0.15,0.15}{{-100}}}}
\fontsize{6}{0}
\selectfont\put(48.5278,46.1294){\makebox(0,0)[r]{\textcolor[rgb]{0.15,0.15,0.15}{{-80}}}}
\fontsize{6}{0}
\selectfont\put(48.5278,60.314){\makebox(0,0)[r]{\textcolor[rgb]{0.15,0.15,0.15}{{-60}}}}
\fontsize{6}{0}
\selectfont\put(48.5278,74.499){\makebox(0,0)[r]{\textcolor[rgb]{0.15,0.15,0.15}{{-40}}}}
\fontsize{6}{0}
\selectfont\put(48.5278,88.6841){\makebox(0,0)[r]{\textcolor[rgb]{0.15,0.15,0.15}{{-20}}}}
\fontsize{6}{0}
\selectfont\put(48.5278,102.869){\makebox(0,0)[r]{\textcolor[rgb]{0.15,0.15,0.15}{{0}}}}
\fontsize{7}{0}
\selectfont\put(207,9){\makebox(0,0)[t]{\textcolor[rgb]{0.15,0.15,0.15}{{Frequência [rad/s]}}}}
\fontsize{7}{0}
\selectfont\put(29.5278,69.0977){\rotatebox{90}{\makebox(0,0)[b]{\textcolor[rgb]{0.15,0.15,0.15}{{Magnitude [dB]}}}}}
\fontsize{6}{0}
\selectfont\put(269.878,100.742){\makebox(0,0)[l]{\textcolor[rgb]{0,0,0}{{$|P(w_{b} \approx  1.45)| \approx -3_{db}$}}}}
\fontsize{6}{0}
\selectfont\put(88.002,44.1963){\makebox(0,0)[l]{\textcolor[rgb]{0,0,0}{{$|P(jw)|$}}}}
\end{picture}

    \end{minipage}
\end{figure}

\begin{figure}[!ht]
    \caption{Magnitude da resposta em frequência de $P(s)$
    com destaque na largura de banda.}
    \vspace{-10pt}
    \hspace{-30pt}
    \label{fig:desafio-3:questao-3-4}
    \begin{minipage}{\linewidth}
        % Title: gl2ps_renderer figure
% Creator: GL2PS 1.4.0, (C) 1999-2017 C. Geuzaine
% For: Octave
% CreationDate: Wed Oct  6 21:11:28 2021
\setlength{\unitlength}{1pt}
\begin{picture}(0,0)
\includegraphics{images/challenge3/resultado-questao-3-4-compensadores-inc}
\end{picture}%
\begin{picture}(400,270)(0,0)
\fontsize{6}{0}
\selectfont\put(52,153.188){\makebox(0,0)[t]{\textcolor[rgb]{0.15,0.15,0.15}{{$10^{-4}$}}}}
\fontsize{6}{0}
\selectfont\put(103.667,153.188){\makebox(0,0)[t]{\textcolor[rgb]{0.15,0.15,0.15}{{$10^{-3}$}}}}
\fontsize{6}{0}
\selectfont\put(155.333,153.188){\makebox(0,0)[t]{\textcolor[rgb]{0.15,0.15,0.15}{{$10^{-2}$}}}}
\fontsize{6}{0}
\selectfont\put(207,153.188){\makebox(0,0)[t]{\textcolor[rgb]{0.15,0.15,0.15}{{$10^{-1}$}}}}
\fontsize{6}{0}
\selectfont\put(258.667,153.188){\makebox(0,0)[t]{\textcolor[rgb]{0.15,0.15,0.15}{{$10^{0}$}}}}
\fontsize{6}{0}
\selectfont\put(310.333,153.188){\makebox(0,0)[t]{\textcolor[rgb]{0.15,0.15,0.15}{{$10^{1}$}}}}
\fontsize{6}{0}
\selectfont\put(362,153.188){\makebox(0,0)[t]{\textcolor[rgb]{0.15,0.15,0.15}{{$10^{2}$}}}}
\fontsize{6}{0}
\selectfont\put(48.5278,158.368){\makebox(0,0)[r]{\textcolor[rgb]{0.15,0.15,0.15}{{0}}}}
\fontsize{6}{0}
\selectfont\put(48.5278,170.729){\makebox(0,0)[r]{\textcolor[rgb]{0.15,0.15,0.15}{{2}}}}
\fontsize{6}{0}
\selectfont\put(48.5278,183.089){\makebox(0,0)[r]{\textcolor[rgb]{0.15,0.15,0.15}{{4}}}}
\fontsize{6}{0}
\selectfont\put(48.5278,195.45){\makebox(0,0)[r]{\textcolor[rgb]{0.15,0.15,0.15}{{6}}}}
\fontsize{6}{0}
\selectfont\put(48.5278,207.811){\makebox(0,0)[r]{\textcolor[rgb]{0.15,0.15,0.15}{{8}}}}
\fontsize{6}{0}
\selectfont\put(48.5278,220.172){\makebox(0,0)[r]{\textcolor[rgb]{0.15,0.15,0.15}{{10}}}}
\fontsize{6}{0}
\selectfont\put(48.5278,232.532){\makebox(0,0)[r]{\textcolor[rgb]{0.15,0.15,0.15}{{12}}}}
\fontsize{6}{0}
\selectfont\put(48.5278,244.893){\makebox(0,0)[r]{\textcolor[rgb]{0.15,0.15,0.15}{{14}}}}
\fontsize{7}{0}
\selectfont\put(207,139.188){\makebox(0,0)[t]{\textcolor[rgb]{0.15,0.15,0.15}{{Frequência [rad/s]}}}}
\fontsize{7}{0}
\selectfont\put(35.5278,201.631){\rotatebox{90}{\makebox(0,0)[b]{\textcolor[rgb]{0.15,0.15,0.15}{{Magnitude [db]}}}}}
\fontsize{6}{0}
\selectfont\put(88.002,228.893){\makebox(0,0)[l]{\textcolor[rgb]{0,0,0}{{${C}'(jw)$ p/ folga $= 12^{\circ}$}}}}
\fontsize{6}{0}
\selectfont\put(88.002,216.892){\makebox(0,0)[l]{\textcolor[rgb]{0,0,0}{{${C}'(jw)$ p/ folga $= 24^{\circ}$}}}}
\fontsize{6}{0}
\selectfont\put(52,24.5205){\makebox(0,0)[t]{\textcolor[rgb]{0.15,0.15,0.15}{{$10^{-4}$}}}}
\fontsize{6}{0}
\selectfont\put(103.667,24.5205){\makebox(0,0)[t]{\textcolor[rgb]{0.15,0.15,0.15}{{$10^{-3}$}}}}
\fontsize{6}{0}
\selectfont\put(155.333,24.5205){\makebox(0,0)[t]{\textcolor[rgb]{0.15,0.15,0.15}{{$10^{-2}$}}}}
\fontsize{6}{0}
\selectfont\put(207,24.5205){\makebox(0,0)[t]{\textcolor[rgb]{0.15,0.15,0.15}{{$10^{-1}$}}}}
\fontsize{6}{0}
\selectfont\put(258.667,24.5205){\makebox(0,0)[t]{\textcolor[rgb]{0.15,0.15,0.15}{{$10^{0}$}}}}
\fontsize{6}{0}
\selectfont\put(310.333,24.5205){\makebox(0,0)[t]{\textcolor[rgb]{0.15,0.15,0.15}{{$10^{1}$}}}}
\fontsize{6}{0}
\selectfont\put(362,24.5205){\makebox(0,0)[t]{\textcolor[rgb]{0.15,0.15,0.15}{{$10^{2}$}}}}
\fontsize{6}{0}
\selectfont\put(48.5278,29.7002){\makebox(0,0)[r]{\textcolor[rgb]{0.15,0.15,0.15}{{0}}}}
\fontsize{6}{0}
\selectfont\put(48.5278,47.0049){\makebox(0,0)[r]{\textcolor[rgb]{0.15,0.15,0.15}{{10}}}}
\fontsize{6}{0}
\selectfont\put(48.5278,64.3101){\makebox(0,0)[r]{\textcolor[rgb]{0.15,0.15,0.15}{{20}}}}
\fontsize{6}{0}
\selectfont\put(48.5278,81.6152){\makebox(0,0)[r]{\textcolor[rgb]{0.15,0.15,0.15}{{30}}}}
\fontsize{6}{0}
\selectfont\put(48.5278,98.9199){\makebox(0,0)[r]{\textcolor[rgb]{0.15,0.15,0.15}{{40}}}}
\fontsize{6}{0}
\selectfont\put(48.5278,116.225){\makebox(0,0)[r]{\textcolor[rgb]{0.15,0.15,0.15}{{50}}}}
\fontsize{7}{0}
\selectfont\put(207,10.5205){\makebox(0,0)[t]{\textcolor[rgb]{0.15,0.15,0.15}{{Frequência [rad/s]}}}}
\fontsize{7}{0}
\selectfont\put(35.5278,72.9624){\rotatebox{90}{\makebox(0,0)[b]{\textcolor[rgb]{0.15,0.15,0.15}{{Fase [graus]}}}}}
\fontsize{6}{0}
\selectfont\put(88.002,57.7007){\makebox(0,0)[l]{\textcolor[rgb]{0,0,0}{{${C}'(jw)$ p/ folga $= 12^{\circ}$}}}}
\fontsize{6}{0}
\selectfont\put(88.002,45.7007){\makebox(0,0)[l]{\textcolor[rgb]{0,0,0}{{${C}'(jw)$ p/ folga $= 24^{\circ}$}}}}
\end{picture}

    \end{minipage}
\end{figure}

\begin{figure}[!ht]
    \caption{Magnitude da resposta em frequência de $P(s)$
    com destaque na largura de banda.}
    \vspace{-10pt}
    \hspace{-30pt}
    \label{fig:desafio-3:questao-3-4}
    \begin{minipage}{\linewidth}
        % Title: gl2ps_renderer figure
% Creator: GL2PS 1.4.0, (C) 1999-2017 C. Geuzaine
% For: Octave
% CreationDate: Wed Oct  6 21:53:16 2021
\setlength{\unitlength}{1pt}
\begin{picture}(0,0)
\includegraphics{images/challenge3/resultado-questao-3-4-malha-aberta-inc}
\end{picture}%
\begin{picture}(400,270)(0,0)
\fontsize{6}{0}
\selectfont\put(52,153.188){\makebox(0,0)[t]{\textcolor[rgb]{0.15,0.15,0.15}{{$10^{-4}$}}}}
\fontsize{6}{0}
\selectfont\put(103.667,153.188){\makebox(0,0)[t]{\textcolor[rgb]{0.15,0.15,0.15}{{$10^{-3}$}}}}
\fontsize{6}{0}
\selectfont\put(155.333,153.188){\makebox(0,0)[t]{\textcolor[rgb]{0.15,0.15,0.15}{{$10^{-2}$}}}}
\fontsize{6}{0}
\selectfont\put(207,153.188){\makebox(0,0)[t]{\textcolor[rgb]{0.15,0.15,0.15}{{$10^{-1}$}}}}
\fontsize{6}{0}
\selectfont\put(258.667,153.188){\makebox(0,0)[t]{\textcolor[rgb]{0.15,0.15,0.15}{{$10^{0}$}}}}
\fontsize{6}{0}
\selectfont\put(310.333,153.188){\makebox(0,0)[t]{\textcolor[rgb]{0.15,0.15,0.15}{{$10^{1}$}}}}
\fontsize{6}{0}
\selectfont\put(362,153.188){\makebox(0,0)[t]{\textcolor[rgb]{0.15,0.15,0.15}{{$10^{2}$}}}}
\fontsize{6}{0}
\selectfont\put(48.5278,158.368){\makebox(0,0)[r]{\textcolor[rgb]{0.15,0.15,0.15}{{-120}}}}
\fontsize{6}{0}
\selectfont\put(48.5278,170.729){\makebox(0,0)[r]{\textcolor[rgb]{0.15,0.15,0.15}{{-100}}}}
\fontsize{6}{0}
\selectfont\put(48.5278,183.089){\makebox(0,0)[r]{\textcolor[rgb]{0.15,0.15,0.15}{{-80}}}}
\fontsize{6}{0}
\selectfont\put(48.5278,195.45){\makebox(0,0)[r]{\textcolor[rgb]{0.15,0.15,0.15}{{-60}}}}
\fontsize{6}{0}
\selectfont\put(48.5278,207.811){\makebox(0,0)[r]{\textcolor[rgb]{0.15,0.15,0.15}{{-40}}}}
\fontsize{6}{0}
\selectfont\put(48.5278,220.172){\makebox(0,0)[r]{\textcolor[rgb]{0.15,0.15,0.15}{{-20}}}}
\fontsize{6}{0}
\selectfont\put(48.5278,232.532){\makebox(0,0)[r]{\textcolor[rgb]{0.15,0.15,0.15}{{0}}}}
\fontsize{6}{0}
\selectfont\put(48.5278,244.893){\makebox(0,0)[r]{\textcolor[rgb]{0.15,0.15,0.15}{{20}}}}
\fontsize{7}{0}
\selectfont\put(207,139.188){\makebox(0,0)[t]{\textcolor[rgb]{0.15,0.15,0.15}{{Frequência [rad/s]}}}}
\fontsize{7}{0}
\selectfont\put(29.5278,201.631){\rotatebox{90}{\makebox(0,0)[b]{\textcolor[rgb]{0.15,0.15,0.15}{{Ganho [db]}}}}}
\fontsize{6}{0}
\selectfont\put(88.002,198.369){\makebox(0,0)[l]{\textcolor[rgb]{0,0,0}{{$\overline{K}G(jw)$}}}}
\fontsize{6}{0}
\selectfont\put(88.002,186.369){\makebox(0,0)[l]{\textcolor[rgb]{0,0,0}{{$C(jw)G(jw)$ p/ folga $= 12^{\circ}$}}}}
\fontsize{6}{0}
\selectfont\put(88.002,174.369){\makebox(0,0)[l]{\textcolor[rgb]{0,0,0}{{$C(jw)G(jw)$ p/ folga $= 24^{\circ}$}}}}
\fontsize{6}{0}
\selectfont\put(52,24.5205){\makebox(0,0)[t]{\textcolor[rgb]{0.15,0.15,0.15}{{$10^{-4}$}}}}
\fontsize{6}{0}
\selectfont\put(103.667,24.5205){\makebox(0,0)[t]{\textcolor[rgb]{0.15,0.15,0.15}{{$10^{-3}$}}}}
\fontsize{6}{0}
\selectfont\put(155.333,24.5205){\makebox(0,0)[t]{\textcolor[rgb]{0.15,0.15,0.15}{{$10^{-2}$}}}}
\fontsize{6}{0}
\selectfont\put(207,24.5205){\makebox(0,0)[t]{\textcolor[rgb]{0.15,0.15,0.15}{{$10^{-1}$}}}}
\fontsize{6}{0}
\selectfont\put(258.667,24.5205){\makebox(0,0)[t]{\textcolor[rgb]{0.15,0.15,0.15}{{$10^{0}$}}}}
\fontsize{6}{0}
\selectfont\put(310.333,24.5205){\makebox(0,0)[t]{\textcolor[rgb]{0.15,0.15,0.15}{{$10^{1}$}}}}
\fontsize{6}{0}
\selectfont\put(362,24.5205){\makebox(0,0)[t]{\textcolor[rgb]{0.15,0.15,0.15}{{$10^{2}$}}}}
\fontsize{6}{0}
\selectfont\put(48.5278,29.7002){\makebox(0,0)[r]{\textcolor[rgb]{0.15,0.15,0.15}{{-300}}}}
\fontsize{6}{0}
\selectfont\put(48.5278,42.0605){\makebox(0,0)[r]{\textcolor[rgb]{0.15,0.15,0.15}{{-250}}}}
\fontsize{6}{0}
\selectfont\put(48.5278,54.4214){\makebox(0,0)[r]{\textcolor[rgb]{0.15,0.15,0.15}{{-200}}}}
\fontsize{6}{0}
\selectfont\put(48.5278,66.7822){\makebox(0,0)[r]{\textcolor[rgb]{0.15,0.15,0.15}{{-150}}}}
\fontsize{6}{0}
\selectfont\put(48.5278,79.1431){\makebox(0,0)[r]{\textcolor[rgb]{0.15,0.15,0.15}{{-100}}}}
\fontsize{6}{0}
\selectfont\put(48.5278,91.5034){\makebox(0,0)[r]{\textcolor[rgb]{0.15,0.15,0.15}{{-50}}}}
\fontsize{6}{0}
\selectfont\put(48.5278,103.864){\makebox(0,0)[r]{\textcolor[rgb]{0.15,0.15,0.15}{{0}}}}
\fontsize{6}{0}
\selectfont\put(48.5278,116.225){\makebox(0,0)[r]{\textcolor[rgb]{0.15,0.15,0.15}{{50}}}}
\fontsize{7}{0}
\selectfont\put(207,10.5205){\makebox(0,0)[t]{\textcolor[rgb]{0.15,0.15,0.15}{{Frequência [rad/s]}}}}
\fontsize{7}{0}
\selectfont\put(29.5278,72.9624){\rotatebox{90}{\makebox(0,0)[b]{\textcolor[rgb]{0.15,0.15,0.15}{{Fase [graus]}}}}}
\fontsize{6}{0}
\selectfont\put(88.002,69.7012){\makebox(0,0)[l]{\textcolor[rgb]{0,0,0}{{$\overline{K}G(jw)$}}}}
\fontsize{6}{0}
\selectfont\put(88.002,57.7012){\makebox(0,0)[l]{\textcolor[rgb]{0,0,0}{{$C(jw)G(jw)$ p/ folga $= 12^{\circ}$}}}}
\fontsize{6}{0}
\selectfont\put(88.002,45.7007){\makebox(0,0)[l]{\textcolor[rgb]{0,0,0}{{$C(jw)G(jw)$ p/ folga $= 24^{\circ}$}}}}
\end{picture}

    \end{minipage}
\end{figure}

\subsection{Conclusões}
(Concluir em que medida os resultados apresentam relação com a motivação.
Permitem ilustratar ou concluir algo sobre a motivação? )


\begin{figure}[!ht]
    \caption{Resposta ao degrau para os modelos M1, M2, M3 e M4.}
    \vspace{-10pt}
    \hspace{-30pt}
    \label{fig:desafio2:questao5}
    \begin{minipage}{\linewidth}
        \input{images/challenge3/resultado-questao-5-6-leadcs.tex}
    \end{minipage}
\end{figure}

\begin{figure}[!ht]
    \caption{Resposta ao degrau para os modelos M1, M2, M3 e M4.}
    \vspace{-10pt}
    \hspace{-30pt}
    \label{fig:desafio2:questao-7}
    \begin{minipage}{\linewidth}
        % Title: gl2ps_renderer figure
% Creator: GL2PS 1.4.0, (C) 1999-2017 C. Geuzaine
% For: Octave
% CreationDate: Thu Nov 25 18:14:01 2021
\setlength{\unitlength}{1pt}
\begin{picture}(0,0)
\includegraphics{chapters/challenge6/images/resultado-questao-7-inc}
\end{picture}%
\begin{picture}(400,250)(0,0)
\fontsize{6}{0}
\selectfont\put(52,141.404){\makebox(0,0)[t]{\textcolor[rgb]{0.15,0.15,0.15}{{0}}}}
\fontsize{6}{0}
\selectfont\put(70.4365,141.404){\makebox(0,0)[t]{\textcolor[rgb]{0.15,0.15,0.15}{{20}}}}
\fontsize{6}{0}
\selectfont\put(88.873,141.404){\makebox(0,0)[t]{\textcolor[rgb]{0.15,0.15,0.15}{{40}}}}
\fontsize{6}{0}
\selectfont\put(107.309,141.404){\makebox(0,0)[t]{\textcolor[rgb]{0.15,0.15,0.15}{{60}}}}
\fontsize{6}{0}
\selectfont\put(125.746,141.404){\makebox(0,0)[t]{\textcolor[rgb]{0.15,0.15,0.15}{{80}}}}
\fontsize{6}{0}
\selectfont\put(144.182,141.404){\makebox(0,0)[t]{\textcolor[rgb]{0.15,0.15,0.15}{{100}}}}
\fontsize{6}{0}
\selectfont\put(162.619,141.404){\makebox(0,0)[t]{\textcolor[rgb]{0.15,0.15,0.15}{{120}}}}
\fontsize{6}{0}
\selectfont\put(181.055,141.404){\makebox(0,0)[t]{\textcolor[rgb]{0.15,0.15,0.15}{{140}}}}
\fontsize{6}{0}
\selectfont\put(48.5205,153.193){\makebox(0,0)[r]{\textcolor[rgb]{0.15,0.15,0.15}{{0}}}}
\fontsize{6}{0}
\selectfont\put(48.5205,169.582){\makebox(0,0)[r]{\textcolor[rgb]{0.15,0.15,0.15}{{0.5}}}}
\fontsize{6}{0}
\selectfont\put(48.5205,185.97){\makebox(0,0)[r]{\textcolor[rgb]{0.15,0.15,0.15}{{1}}}}
\fontsize{6}{0}
\selectfont\put(48.5205,202.359){\makebox(0,0)[r]{\textcolor[rgb]{0.15,0.15,0.15}{{1.5}}}}
\fontsize{6}{0}
\selectfont\put(48.5205,218.748){\makebox(0,0)[r]{\textcolor[rgb]{0.15,0.15,0.15}{{2}}}}
\fontsize{7}{0}
\selectfont\put(33.5205,186.325){\rotatebox{90}{\makebox(0,0)[b]{\textcolor[rgb]{0.15,0.15,0.15}{{Saida $y_1(t)$}}}}}
\fontsize{7}{0}
\selectfont\put(116.643,130.404){\makebox(0,0)[t]{\textcolor[rgb]{0.15,0.15,0.15}{{Tempo (s)}}}}
\fontsize{6}{0}
\selectfont\put(232.715,141.404){\makebox(0,0)[t]{\textcolor[rgb]{0.15,0.15,0.15}{{0}}}}
\fontsize{6}{0}
\selectfont\put(251.151,141.404){\makebox(0,0)[t]{\textcolor[rgb]{0.15,0.15,0.15}{{20}}}}
\fontsize{6}{0}
\selectfont\put(269.587,141.404){\makebox(0,0)[t]{\textcolor[rgb]{0.15,0.15,0.15}{{40}}}}
\fontsize{6}{0}
\selectfont\put(288.024,141.404){\makebox(0,0)[t]{\textcolor[rgb]{0.15,0.15,0.15}{{60}}}}
\fontsize{6}{0}
\selectfont\put(306.46,141.404){\makebox(0,0)[t]{\textcolor[rgb]{0.15,0.15,0.15}{{80}}}}
\fontsize{6}{0}
\selectfont\put(324.896,141.404){\makebox(0,0)[t]{\textcolor[rgb]{0.15,0.15,0.15}{{100}}}}
\fontsize{6}{0}
\selectfont\put(343.333,141.404){\makebox(0,0)[t]{\textcolor[rgb]{0.15,0.15,0.15}{{120}}}}
\fontsize{6}{0}
\selectfont\put(361.77,141.404){\makebox(0,0)[t]{\textcolor[rgb]{0.15,0.15,0.15}{{140}}}}
\fontsize{6}{0}
\selectfont\put(229.235,146.637){\makebox(0,0)[r]{\textcolor[rgb]{0.15,0.15,0.15}{{-0.2}}}}
\fontsize{6}{0}
\selectfont\put(229.235,157.638){\makebox(0,0)[r]{\textcolor[rgb]{0.15,0.15,0.15}{{0}}}}
\fontsize{6}{0}
\selectfont\put(229.235,168.638){\makebox(0,0)[r]{\textcolor[rgb]{0.15,0.15,0.15}{{0.2}}}}
\fontsize{6}{0}
\selectfont\put(229.235,179.638){\makebox(0,0)[r]{\textcolor[rgb]{0.15,0.15,0.15}{{0.4}}}}
\fontsize{6}{0}
\selectfont\put(229.235,190.638){\makebox(0,0)[r]{\textcolor[rgb]{0.15,0.15,0.15}{{0.6}}}}
\fontsize{6}{0}
\selectfont\put(229.235,201.639){\makebox(0,0)[r]{\textcolor[rgb]{0.15,0.15,0.15}{{0.8}}}}
\fontsize{6}{0}
\selectfont\put(229.235,212.639){\makebox(0,0)[r]{\textcolor[rgb]{0.15,0.15,0.15}{{1}}}}
\fontsize{6}{0}
\selectfont\put(229.235,223.639){\makebox(0,0)[r]{\textcolor[rgb]{0.15,0.15,0.15}{{1.2}}}}
\fontsize{7}{0}
\selectfont\put(212.235,186.325){\rotatebox{90}{\makebox(0,0)[b]{\textcolor[rgb]{0.15,0.15,0.15}{{Saida $y_2(t)$}}}}}
\fontsize{7}{0}
\selectfont\put(297.357,130.404){\makebox(0,0)[t]{\textcolor[rgb]{0.15,0.15,0.15}{{Tempo (s)}}}}
\fontsize{6}{0}
\selectfont\put(52,22.2671){\makebox(0,0)[t]{\textcolor[rgb]{0.15,0.15,0.15}{{0}}}}
\fontsize{6}{0}
\selectfont\put(70.4365,22.2671){\makebox(0,0)[t]{\textcolor[rgb]{0.15,0.15,0.15}{{20}}}}
\fontsize{6}{0}
\selectfont\put(88.873,22.2671){\makebox(0,0)[t]{\textcolor[rgb]{0.15,0.15,0.15}{{40}}}}
\fontsize{6}{0}
\selectfont\put(107.309,22.2671){\makebox(0,0)[t]{\textcolor[rgb]{0.15,0.15,0.15}{{60}}}}
\fontsize{6}{0}
\selectfont\put(125.746,22.2671){\makebox(0,0)[t]{\textcolor[rgb]{0.15,0.15,0.15}{{80}}}}
\fontsize{6}{0}
\selectfont\put(144.182,22.2671){\makebox(0,0)[t]{\textcolor[rgb]{0.15,0.15,0.15}{{100}}}}
\fontsize{6}{0}
\selectfont\put(162.619,22.2671){\makebox(0,0)[t]{\textcolor[rgb]{0.15,0.15,0.15}{{120}}}}
\fontsize{6}{0}
\selectfont\put(181.055,22.2671){\makebox(0,0)[t]{\textcolor[rgb]{0.15,0.15,0.15}{{140}}}}
\fontsize{6}{0}
\selectfont\put(48.5205,36.0054){\makebox(0,0)[r]{\textcolor[rgb]{0.15,0.15,0.15}{{0}}}}
\fontsize{6}{0}
\selectfont\put(48.5205,57.269){\makebox(0,0)[r]{\textcolor[rgb]{0.15,0.15,0.15}{{0.5}}}}
\fontsize{6}{0}
\selectfont\put(48.5205,78.5322){\makebox(0,0)[r]{\textcolor[rgb]{0.15,0.15,0.15}{{1}}}}
\fontsize{6}{0}
\selectfont\put(48.5205,99.7959){\makebox(0,0)[r]{\textcolor[rgb]{0.15,0.15,0.15}{{1.5}}}}
\fontsize{7}{0}
\selectfont\put(33.5205,67.1875){\rotatebox{90}{\makebox(0,0)[b]{\textcolor[rgb]{0.15,0.15,0.15}{{Sinal de Controle $u_1(t)$}}}}}
\fontsize{7}{0}
\selectfont\put(116.643,11.2671){\makebox(0,0)[t]{\textcolor[rgb]{0.15,0.15,0.15}{{Tempo (s)}}}}
\fontsize{6}{0}
\selectfont\put(232.715,22.2671){\makebox(0,0)[t]{\textcolor[rgb]{0.15,0.15,0.15}{{0}}}}
\fontsize{6}{0}
\selectfont\put(251.151,22.2671){\makebox(0,0)[t]{\textcolor[rgb]{0.15,0.15,0.15}{{20}}}}
\fontsize{6}{0}
\selectfont\put(269.587,22.2671){\makebox(0,0)[t]{\textcolor[rgb]{0.15,0.15,0.15}{{40}}}}
\fontsize{6}{0}
\selectfont\put(288.024,22.2671){\makebox(0,0)[t]{\textcolor[rgb]{0.15,0.15,0.15}{{60}}}}
\fontsize{6}{0}
\selectfont\put(306.46,22.2671){\makebox(0,0)[t]{\textcolor[rgb]{0.15,0.15,0.15}{{80}}}}
\fontsize{6}{0}
\selectfont\put(324.896,22.2671){\makebox(0,0)[t]{\textcolor[rgb]{0.15,0.15,0.15}{{100}}}}
\fontsize{6}{0}
\selectfont\put(343.333,22.2671){\makebox(0,0)[t]{\textcolor[rgb]{0.15,0.15,0.15}{{120}}}}
\fontsize{6}{0}
\selectfont\put(361.77,22.2671){\makebox(0,0)[t]{\textcolor[rgb]{0.15,0.15,0.15}{{140}}}}
\fontsize{6}{0}
\selectfont\put(229.235,27.5){\makebox(0,0)[r]{\textcolor[rgb]{0.15,0.15,0.15}{{-0.2}}}}
\fontsize{6}{0}
\selectfont\put(229.235,42.8911){\makebox(0,0)[r]{\textcolor[rgb]{0.15,0.15,0.15}{{0}}}}
\fontsize{6}{0}
\selectfont\put(229.235,58.2817){\makebox(0,0)[r]{\textcolor[rgb]{0.15,0.15,0.15}{{0.2}}}}
\fontsize{6}{0}
\selectfont\put(229.235,73.6729){\makebox(0,0)[r]{\textcolor[rgb]{0.15,0.15,0.15}{{0.4}}}}
\fontsize{6}{0}
\selectfont\put(229.235,89.064){\makebox(0,0)[r]{\textcolor[rgb]{0.15,0.15,0.15}{{0.6}}}}
\fontsize{6}{0}
\selectfont\put(229.235,104.455){\makebox(0,0)[r]{\textcolor[rgb]{0.15,0.15,0.15}{{0.8}}}}
\fontsize{7}{0}
\selectfont\put(212.235,67.1875){\rotatebox{90}{\makebox(0,0)[b]{\textcolor[rgb]{0.15,0.15,0.15}{{Sinal de Controle $u_2(t)$}}}}}
\fontsize{7}{0}
\selectfont\put(297.357,11.2671){\makebox(0,0)[t]{\textcolor[rgb]{0.15,0.15,0.15}{{Tempo (s)}}}}
\fontsize{6}{0}
\selectfont\put(299.842,238.48){\makebox(0,0)[l]{\textcolor[rgb]{0,0,0}{{equilibrio}}}}
\end{picture}

    \end{minipage}
\end{figure}