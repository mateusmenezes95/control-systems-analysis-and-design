\section{Desafio II - Teorema do Pequeno Ganho e Anti-Windup} 
 
\subsection{Motivação}
Este dasafio foi dividido em duas partes. A primeira visa a análise de
incertezas na modelagem de sistemas via Teorema do Pequeno Ganho e Teoria de
Controle Robusto. Já a segunda, em série com o resultado da primeira parte,
analisa o comportamento da saída de um modelo de planta, considerando um ponto
de operação específico, e sinal de controle para um sistema em malha fechada na
presença de saturação do sinal de controle.

A análise de incertezas é de suma importância na avaliação de modelagem de
plantas via alguma técnica específica, como resposta ao degrau ou resposta em
frequência. Esta modelagem é realizada principalmente quando não se tem o modelo
teórico. Desta forma é preciso realizar a modelagem com a planta real, obtendo
assim um modelo nominal da planta. Assim, torna-se necessário analisar qual o
efeito das incertezas oriundas da modelagem, para garantir a estabilidade do
sistema em malha fechada sendo controlado por um controlador projetado
utilizando o modelo nominal da planta. Esta análise de robustez é obtida via
Teorema do Pequeno Ganho e Teoria da Instabilidade Interna (não contemplada
neste desafio).

Além da análise de robustez, é ainda necessário analisar o sistema em malha
fechada obtido através do modelo nominal em um ponto específico de operação na
presença de saturação do sinal de controle. Esta análise também é de extrema
relevância na teoria e aplicação de controle, pois os sistemas de controle
reais tem limitações, como a energia finita de atuadores, que caso não analisado
corretamente, podem surgir no sistema de controle implementado comportamentos
indesejáveis. Um exemplo comum de saturação é nas malhas de controle de vazão,
em que válvulas na linha do processo tem um limiar de abertura além da presença
de zonas mortas de operação.

\subsection{Simulações realizadas}
O sistema de controle em malha fechada utilizado nas simulações do segundo
desafio é o mesmo do desafio 1 e está descrito no tópico
\ref{sec:desafio-1-simulacoes-realizadas} e ilustrado em diagrama de blocos na
Figura \ref{fig:diagrama-de-blocos-malha-fechada}. O modelo $G(s)$ utilizado nas
simulações é dado por

\begin{equation}
    \label{eq:modelo-g-de-s-utilizado-no-desafio-2}
    G(s) = \frac{K_{i}e^{-sL_{i}}}{T_{i}s + 1}.
\end{equation}

O modelo descrito na Equação \ref{eq:modelo-g-de-s-utilizado-no-desafio-2} foi
levantado a partir de 4 pontos de operação distintos, gerando 4 modelos: M1, M2,
M3 e M4. Os parâmetros obtidos para cada modelo $i$ foram os seguintes: $L_{i} =
[0,9; 0,7; 0,6; 0,4]$, $K_{i} = [1,3; 0,9; 1,2; 0,8]$ e $\tau_{i} = [1,2; 1,1;
0,8; 0,9]$.

Para o projeto do controlador utilizou o modelo nominal de $G(s)$ obtido
através das médias $L_{n}$, $K_{n}$ e $\tau_{n}$ dos parâmetros $L_{i}$, $K_{i}$
e $\tau_{i}$, respectivamente. Assim, obteve-se $G{n}(s)$ dado por

\begin{equation}
    \label{eq:modelo-gn-de-s-utilizado-no-desafio-2}
    G_{n}(s) = \frac{1,05e^{-0,65s}}{s + 1}.
\end{equation}

O controlador utilizado nas simulações foi o PI de Skogestad para um sistema de
primeira ordem com atraso, dado por

\begin{equation}
    \label{eq:desafio2:controlador-de-skogestad}
    C(s) = K_{c}\frac{sT_{i} + 1}{sT_{i}} = \frac{0,8282(s + 1)}{s}
\end{equation}
em que $K_{c} = \tau_{n}/[K_{n}(\tau_{c} + L_{n})]$, $T_{i} = min(\tau_{n},
4(\tau_{c} + L_{n}))$ e $\tau_{c} = 0,5$. 

A partir destes modelos foram realizadas 3 simulações:
\begin{enumerate}
    \item Geração das incertezas multiplicativas dos 4 modelos e análise de
    robustez via Teorema do Pequeno Ganho;
    \item Resposta em malha fechada dos 4 modelos, dado o controlador definido
    na Equação \ref{eq:desafio2:controlador-de-skogestad}, com uma entrada de
    referência $r(t)$ em degrau unitário e uma perturbação na entrada $q_{u}(t)$
    também em degrau;
    \item Resposta do modelo M1 em malha fechada com diferentes configurações:
    com filtro de referência $F(s)$, com controlador I+P, com controlador I+P e
    saturação do sinal de controle e com os dois itens anteriores mais a ação de
    anti-windup.
\end{enumerate}

O tópico seguinte irar descrever os resultados obtidos para os 3 itens acima.

\subsection{Resultados obtidos}
\subsubsection{Geração de incertezas multiplicativas}
\label{subsub:geracao-de-incertezas-multiplicativas}
Para cada modelo $G_{i}(s)$ com $i \in [1,4]$, calculou-se o módulo da incerteza
multiplicativa conforme Equação \ref{eq:desafio-2-calculo-das-incertezas}. O
resultado obtido está ilustrado na Figura
\ref{fig:desafio-2-resultado-questao-1}.

\begin{equation}
    \label{eq:desafio-2-calculo-das-incertezas}
    \Delta_{i}(w) = \left | \frac{G_{n}(jw) - G_{i}(jw)}{G_{n}(jw)} \right |,
    10^{-2} \leq w \leq 10^{4}.
\end{equation}

\begin{figure}[!ht]
    \caption{Módulo das incertezas multiplicativas de cada modelo $i$ em
    função da frequência $w$.}
    \vspace{-10pt}
    \hspace{-30pt}
    \label{fig:desafio-2-resultado-questao-1}
    \begin{minipage}{\linewidth}
        % Title: gl2ps_renderer figure
% Creator: GL2PS 1.4.0, (C) 1999-2017 C. Geuzaine
% For: Octave
% CreationDate: Sun Sep 26 17:17:39 2021
\setlength{\unitlength}{1pt}
\begin{picture}(0,0)
\includegraphics{images/challenge2/resultado-questao-1-inc}
\end{picture}%
\begin{picture}(400,300)(0,0)
\fontsize{6}{0}
\selectfont\put(48.5278,240.483){\makebox(0,0)[r]{\textcolor[rgb]{0.15,0.15,0.15}{{0.5}}}}
\fontsize{6}{0}
\selectfont\put(48.5278,248.666){\makebox(0,0)[r]{\textcolor[rgb]{0.15,0.15,0.15}{{1}}}}
\fontsize{6}{0}
\selectfont\put(48.5278,256.849){\makebox(0,0)[r]{\textcolor[rgb]{0.15,0.15,0.15}{{1.5}}}}
\fontsize{6}{0}
\selectfont\put(48.5278,265.032){\makebox(0,0)[r]{\textcolor[rgb]{0.15,0.15,0.15}{{2}}}}
\fontsize{6}{0}
\selectfont\put(48.5278,273.215){\makebox(0,0)[r]{\textcolor[rgb]{0.15,0.15,0.15}{{2.5}}}}
\fontsize{7}{0}
\selectfont\put(33.5278,254.706){\rotatebox{90}{\makebox(0,0)[b]{\textcolor[rgb]{0.15,0.15,0.15}{{$\Delta_{1}(jw)$}}}}}
\fontsize{6}{0}
\selectfont\put(48.5278,172.75){\makebox(0,0)[r]{\textcolor[rgb]{0.15,0.15,0.15}{{0.5}}}}
\fontsize{6}{0}
\selectfont\put(48.5278,180.934){\makebox(0,0)[r]{\textcolor[rgb]{0.15,0.15,0.15}{{1}}}}
\fontsize{6}{0}
\selectfont\put(48.5278,189.116){\makebox(0,0)[r]{\textcolor[rgb]{0.15,0.15,0.15}{{1.5}}}}
\fontsize{6}{0}
\selectfont\put(48.5278,197.299){\makebox(0,0)[r]{\textcolor[rgb]{0.15,0.15,0.15}{{2}}}}
\fontsize{6}{0}
\selectfont\put(48.5278,205.482){\makebox(0,0)[r]{\textcolor[rgb]{0.15,0.15,0.15}{{2.5}}}}
\fontsize{7}{0}
\selectfont\put(33.5278,186.974){\rotatebox{90}{\makebox(0,0)[b]{\textcolor[rgb]{0.15,0.15,0.15}{{$\Delta_{2}(jw)$}}}}}
\fontsize{6}{0}
\selectfont\put(48.5278,105.018){\makebox(0,0)[r]{\textcolor[rgb]{0.15,0.15,0.15}{{0.5}}}}
\fontsize{6}{0}
\selectfont\put(48.5278,113.201){\makebox(0,0)[r]{\textcolor[rgb]{0.15,0.15,0.15}{{1}}}}
\fontsize{6}{0}
\selectfont\put(48.5278,121.384){\makebox(0,0)[r]{\textcolor[rgb]{0.15,0.15,0.15}{{1.5}}}}
\fontsize{6}{0}
\selectfont\put(48.5278,129.567){\makebox(0,0)[r]{\textcolor[rgb]{0.15,0.15,0.15}{{2}}}}
\fontsize{6}{0}
\selectfont\put(48.5278,137.75){\makebox(0,0)[r]{\textcolor[rgb]{0.15,0.15,0.15}{{2.5}}}}
\fontsize{7}{0}
\selectfont\put(33.5278,119.241){\rotatebox{90}{\makebox(0,0)[b]{\textcolor[rgb]{0.15,0.15,0.15}{{$\Delta_{3}(jw)$}}}}}
\fontsize{6}{0}
\selectfont\put(52,27.7896){\makebox(0,0)[t]{\textcolor[rgb]{0.15,0.15,0.15}{{$10^{-2}$}}}}
\fontsize{6}{0}
\selectfont\put(103.667,27.7896){\makebox(0,0)[t]{\textcolor[rgb]{0.15,0.15,0.15}{{$10^{-1}$}}}}
\fontsize{6}{0}
\selectfont\put(155.333,27.7896){\makebox(0,0)[t]{\textcolor[rgb]{0.15,0.15,0.15}{{$10^{0}$}}}}
\fontsize{6}{0}
\selectfont\put(207,27.7896){\makebox(0,0)[t]{\textcolor[rgb]{0.15,0.15,0.15}{{$10^{1}$}}}}
\fontsize{6}{0}
\selectfont\put(258.667,27.7896){\makebox(0,0)[t]{\textcolor[rgb]{0.15,0.15,0.15}{{$10^{2}$}}}}
\fontsize{6}{0}
\selectfont\put(310.333,27.7896){\makebox(0,0)[t]{\textcolor[rgb]{0.15,0.15,0.15}{{$10^{3}$}}}}
\fontsize{6}{0}
\selectfont\put(362,27.7896){\makebox(0,0)[t]{\textcolor[rgb]{0.15,0.15,0.15}{{$10^{4}$}}}}
\fontsize{6}{0}
\selectfont\put(48.5278,37.2856){\makebox(0,0)[r]{\textcolor[rgb]{0.15,0.15,0.15}{{0.5}}}}
\fontsize{6}{0}
\selectfont\put(48.5278,45.4688){\makebox(0,0)[r]{\textcolor[rgb]{0.15,0.15,0.15}{{1}}}}
\fontsize{6}{0}
\selectfont\put(48.5278,53.6519){\makebox(0,0)[r]{\textcolor[rgb]{0.15,0.15,0.15}{{1.5}}}}
\fontsize{6}{0}
\selectfont\put(48.5278,61.835){\makebox(0,0)[r]{\textcolor[rgb]{0.15,0.15,0.15}{{2}}}}
\fontsize{6}{0}
\selectfont\put(48.5278,70.0176){\makebox(0,0)[r]{\textcolor[rgb]{0.15,0.15,0.15}{{2.5}}}}
\fontsize{7}{0}
\selectfont\put(33.5278,51.5088){\rotatebox{90}{\makebox(0,0)[b]{\textcolor[rgb]{0.15,0.15,0.15}{{$\Delta_{4}(jw)$}}}}}
\fontsize{7}{0}
\selectfont\put(207,13.7896){\makebox(0,0)[t]{\textcolor[rgb]{0.15,0.15,0.15}{{Frequência $w$ (rad/s)}}}}
\end{picture}

    \end{minipage}
\end{figure}

Para verificar se o sistema em malha fechada com o modelo nominal $G_{n}(s)$ é
estável, calculou-se o módulo da complementar de sensibidade, Equação
\ref{eq:desafio-2:modulo-da-complementar-de-sensibilidade}, que foi multiplicado
pelo limitante superior das incertezas multiplicativas definido pela Equação
\ref{eq:desafio-2:limitante-superior-das-incertezas-multiplicativas}.

\begin{equation}
    \label{eq:desafio-2:modulo-da-complementar-de-sensibilidade}
    |\textit{C}(jw)| = \left | \frac{C(jw)G_{n}(jw)}{1 + C(jw)G_{n}(jw)} \right |
\end{equation}

\begin{equation}
    \label{eq:desafio-2:limitante-superior-das-incertezas-multiplicativas}
    \bar{\Delta}(w[k]) = min(\Delta_{1}(w[k]), \Delta_{2}(w[k]), \Delta_{3}(w[k]),
    \Delta_{4}(w[k])). 
\end{equation}
em que $w[k]$ é um vetor de 10000 pontos em escala logarítimica com $10^{-2}
\leq w \leq 10^{4}$ rad/s.

O resultado obtido está ilustrado na Figura
\ref{fig:desafio-2:resultado-questao-3-4}, em que é possível visualizar o
limitante superior das incertezas multiplicativas $\bar{\Delta}(w)$, o módulo da
função complementar de sensibidade $|\textit{C}(jw)|$ e o resultado da
multiplicação dos dois valores.

\begin{figure}[!ht]
    \caption{Resultado da análise das incertezas multiplicativas.}
    \vspace{-10pt}
    \hspace{-30pt}
    \label{fig:desafio-2:resultado-questao-3-4}
    \begin{minipage}{\linewidth}
        % Title: gl2ps_renderer figure
% Creator: GL2PS 1.4.0, (C) 1999-2017 C. Geuzaine
% For: Octave
% CreationDate: Sun Sep 26 17:19:09 2021
\setlength{\unitlength}{1pt}
\begin{picture}(0,0)
\includegraphics{images/challenge2/resultado-questao-3-4-inc}
\end{picture}%
\begin{picture}(400,300)(0,0)
\fontsize{6}{0}
\selectfont\put(48.5278,216.953){\makebox(0,0)[r]{\textcolor[rgb]{0.15,0.15,0.15}{{0}}}}
\fontsize{6}{0}
\selectfont\put(48.5278,228.205){\makebox(0,0)[r]{\textcolor[rgb]{0.15,0.15,0.15}{{0.5}}}}
\fontsize{6}{0}
\selectfont\put(48.5278,239.458){\makebox(0,0)[r]{\textcolor[rgb]{0.15,0.15,0.15}{{1}}}}
\fontsize{6}{0}
\selectfont\put(48.5278,250.71){\makebox(0,0)[r]{\textcolor[rgb]{0.15,0.15,0.15}{{1.5}}}}
\fontsize{6}{0}
\selectfont\put(48.5278,261.962){\makebox(0,0)[r]{\textcolor[rgb]{0.15,0.15,0.15}{{2}}}}
\fontsize{6}{0}
\selectfont\put(48.5278,273.215){\makebox(0,0)[r]{\textcolor[rgb]{0.15,0.15,0.15}{{2.5}}}}
\fontsize{7}{0}
\selectfont\put(33.5278,245.083){\rotatebox{90}{\makebox(0,0)[b]{\textcolor[rgb]{0.15,0.15,0.15}{{$\bar{\Delta}(w)$}}}}}
\fontsize{6}{0}
\selectfont\put(48.5278,124.977){\makebox(0,0)[r]{\textcolor[rgb]{0.15,0.15,0.15}{{0}}}}
\fontsize{6}{0}
\selectfont\put(48.5278,133.014){\makebox(0,0)[r]{\textcolor[rgb]{0.15,0.15,0.15}{{0.2}}}}
\fontsize{6}{0}
\selectfont\put(48.5278,141.051){\makebox(0,0)[r]{\textcolor[rgb]{0.15,0.15,0.15}{{0.4}}}}
\fontsize{6}{0}
\selectfont\put(48.5278,149.088){\makebox(0,0)[r]{\textcolor[rgb]{0.15,0.15,0.15}{{0.6}}}}
\fontsize{6}{0}
\selectfont\put(48.5278,157.126){\makebox(0,0)[r]{\textcolor[rgb]{0.15,0.15,0.15}{{0.8}}}}
\fontsize{6}{0}
\selectfont\put(48.5278,165.164){\makebox(0,0)[r]{\textcolor[rgb]{0.15,0.15,0.15}{{1}}}}
\fontsize{6}{0}
\selectfont\put(48.5278,173.201){\makebox(0,0)[r]{\textcolor[rgb]{0.15,0.15,0.15}{{1.2}}}}
\fontsize{6}{0}
\selectfont\put(48.5278,181.238){\makebox(0,0)[r]{\textcolor[rgb]{0.15,0.15,0.15}{{1.4}}}}
\fontsize{7}{0}
\selectfont\put(33.5278,153.107){\rotatebox{90}{\makebox(0,0)[b]{\textcolor[rgb]{0.15,0.15,0.15}{{$|\textit{C}(jw)|$}}}}}
\fontsize{6}{0}
\selectfont\put(52,27.7676){\makebox(0,0)[t]{\textcolor[rgb]{0.15,0.15,0.15}{{$10^{-2}$}}}}
\fontsize{6}{0}
\selectfont\put(103.667,27.7676){\makebox(0,0)[t]{\textcolor[rgb]{0.15,0.15,0.15}{{$10^{-1}$}}}}
\fontsize{6}{0}
\selectfont\put(155.333,27.7676){\makebox(0,0)[t]{\textcolor[rgb]{0.15,0.15,0.15}{{$10^{0}$}}}}
\fontsize{6}{0}
\selectfont\put(207,27.7676){\makebox(0,0)[t]{\textcolor[rgb]{0.15,0.15,0.15}{{$10^{1}$}}}}
\fontsize{6}{0}
\selectfont\put(258.667,27.7676){\makebox(0,0)[t]{\textcolor[rgb]{0.15,0.15,0.15}{{$10^{2}$}}}}
\fontsize{6}{0}
\selectfont\put(310.333,27.7676){\makebox(0,0)[t]{\textcolor[rgb]{0.15,0.15,0.15}{{$10^{3}$}}}}
\fontsize{6}{0}
\selectfont\put(362,27.7676){\makebox(0,0)[t]{\textcolor[rgb]{0.15,0.15,0.15}{{$10^{4}$}}}}
\fontsize{6}{0}
\selectfont\put(48.5278,33){\makebox(0,0)[r]{\textcolor[rgb]{0.15,0.15,0.15}{{0}}}}
\fontsize{6}{0}
\selectfont\put(48.5278,44.2524){\makebox(0,0)[r]{\textcolor[rgb]{0.15,0.15,0.15}{{0.1}}}}
\fontsize{6}{0}
\selectfont\put(48.5278,55.5049){\makebox(0,0)[r]{\textcolor[rgb]{0.15,0.15,0.15}{{0.2}}}}
\fontsize{6}{0}
\selectfont\put(48.5278,66.7568){\makebox(0,0)[r]{\textcolor[rgb]{0.15,0.15,0.15}{{0.3}}}}
\fontsize{6}{0}
\selectfont\put(48.5278,78.0093){\makebox(0,0)[r]{\textcolor[rgb]{0.15,0.15,0.15}{{0.4}}}}
\fontsize{6}{0}
\selectfont\put(48.5278,89.2617){\makebox(0,0)[r]{\textcolor[rgb]{0.15,0.15,0.15}{{0.5}}}}
\fontsize{7}{0}
\selectfont\put(33.5278,61.1309){\rotatebox{90}{\makebox(0,0)[b]{\textcolor[rgb]{0.15,0.15,0.15}{{$|\textit{C}(jw)|\bar{\Delta}(w)$}}}}}
\fontsize{7}{0}
\selectfont\put(207,13.7676){\makebox(0,0)[t]{\textcolor[rgb]{0.15,0.15,0.15}{{Frequência $w$ (rad/s)}}}}
\end{picture}

    \end{minipage}
\end{figure}

Conclui-se a a partir do último gráfico da Figura
\ref{fig:desafio-2:resultado-questao-3-4} que o sistema em malha fechada é
estável, dentro de uma faixa de frequência, com a planta em diferentes pontos de
operação dado as incertezas dos modelos M1, M2, M3 e M4 e o controlador $C(s)$
projetado considerando o modelo nominal da planta $G_{n}(s)$. Esta conclusão
pode ser obtida pois

\begin{equation}
    \label{eq:desafio-2:principio-do-modelo-interno}
    \sup_{w}|\textit{C}(jw)|\bar{\Delta}(w) \le 1, \ 10^{-2} \leq \forall w \leq {10^{4}}.
\end{equation}

\subsubsection{Resposta em malha fechadas dos modelos M1, M2, M3 e M4}
\label{subsub:desafio2:resposta-em-malha-fechadas-dos-modelos-m1-m2-m3-e-m4}
Para reforçar a conclusão de robustez obtida conforme discutida no tópico
\ref{subsub:geracao-de-incertezas-multiplicativas}, simulou-se a resposta ao
degrau dos 4 modelos M1, M2, M3 e M4, cuja as funções de transferência são as
Equações \ref{eq:desafio2:modelo-g1}, \ref{eq:desafio2:modelo-g2},
\ref{eq:desafio2:modelo-g3} e \ref{eq:desafio2:modelo-g4}, respectivamente.

\begin{equation}
    \label{eq:desafio2:modelo-g1}
    G_{1}(s) = \frac{1,3e^{-0,9s}}{1,2s + 1},
\end{equation}

\begin{equation}
    \label{eq:desafio2:modelo-g2}
    G_{2}(s) = \frac{0,9e^{-0,7s}}{1,1s + 1},
\end{equation}

\begin{equation}
    \label{eq:desafio2:modelo-g3}
    G_{3}(s) = \frac{1,2e^{-0,6s}}{0,8s + 1},
\end{equation}

\begin{equation}
    \label{eq:desafio2:modelo-g4}
    G_{4}(s) = \frac{0,8e^{-0,4s}}{0,9s + 1}.
\end{equation}

Como esperado, a Figura \ref{fig:desafio2:questao5} demonstra que todos os
sistemas são estáveis quando a malha é fechada utilizando o controlador
projetado com o modelo nominal $G_{n}(s)$ do processo.

\begin{figure}[!ht]
    \caption{Resposta ao degrau para os modelos M1, M2, M3 e M4.}
    \vspace{-10pt}
    \hspace{-30pt}
    \label{fig:desafio2:questao5}
    \begin{minipage}{\linewidth}
        % Title: gl2ps_renderer figure
% Creator: GL2PS 1.4.0, (C) 1999-2017 C. Geuzaine
% For: Octave
% CreationDate: Sun Sep 26 17:47:58 2021
\setlength{\unitlength}{1pt}
\begin{picture}(0,0)
\includegraphics{images/challenge2/resultado-questao-5-inc}
\end{picture}%
\begin{picture}(400,250)(0,0)
\fontsize{6}{0}
\selectfont\put(52,141.404){\makebox(0,0)[t]{\textcolor[rgb]{0.15,0.15,0.15}{{0}}}}
\fontsize{6}{0}
\selectfont\put(103.675,141.404){\makebox(0,0)[t]{\textcolor[rgb]{0.15,0.15,0.15}{{10}}}}
\fontsize{6}{0}
\selectfont\put(155.351,141.404){\makebox(0,0)[t]{\textcolor[rgb]{0.15,0.15,0.15}{{20}}}}
\fontsize{6}{0}
\selectfont\put(207.026,141.404){\makebox(0,0)[t]{\textcolor[rgb]{0.15,0.15,0.15}{{30}}}}
\fontsize{6}{0}
\selectfont\put(258.701,141.404){\makebox(0,0)[t]{\textcolor[rgb]{0.15,0.15,0.15}{{40}}}}
\fontsize{6}{0}
\selectfont\put(310.376,141.404){\makebox(0,0)[t]{\textcolor[rgb]{0.15,0.15,0.15}{{50}}}}
\fontsize{6}{0}
\selectfont\put(48.5278,155.215){\makebox(0,0)[r]{\textcolor[rgb]{0.15,0.15,0.15}{{0}}}}
\fontsize{6}{0}
\selectfont\put(48.5278,176.659){\makebox(0,0)[r]{\textcolor[rgb]{0.15,0.15,0.15}{{0.5}}}}
\fontsize{6}{0}
\selectfont\put(48.5278,198.104){\makebox(0,0)[r]{\textcolor[rgb]{0.15,0.15,0.15}{{1}}}}
\fontsize{6}{0}
\selectfont\put(48.5278,219.548){\makebox(0,0)[r]{\textcolor[rgb]{0.15,0.15,0.15}{{1.5}}}}
\fontsize{7}{0}
\selectfont\put(33.5278,186.325){\rotatebox{90}{\makebox(0,0)[b]{\textcolor[rgb]{0.15,0.15,0.15}{{Saída $y_{i}(t)$}}}}}
\fontsize{7}{0}
\selectfont\put(207,130.404){\makebox(0,0)[t]{\textcolor[rgb]{0.15,0.15,0.15}{{Tempo (s)}}}}
\fontsize{6}{0}
\selectfont\put(332,210.139){\makebox(0,0)[l]{\textcolor[rgb]{0,0,0}{{$r(t)$}}}}
\fontsize{6}{0}
\selectfont\put(332,198.639){\makebox(0,0)[l]{\textcolor[rgb]{0,0,0}{{$y_{1}(t)$}}}}
\fontsize{6}{0}
\selectfont\put(332,186.638){\makebox(0,0)[l]{\textcolor[rgb]{0,0,0}{{$y_{2}(t)$}}}}
\fontsize{6}{0}
\selectfont\put(332,174.638){\makebox(0,0)[l]{\textcolor[rgb]{0,0,0}{{$y_{3}(t)$}}}}
\fontsize{6}{0}
\selectfont\put(332,162.638){\makebox(0,0)[l]{\textcolor[rgb]{0,0,0}{{$y_{4}(t)$}}}}
\fontsize{6}{0}
\selectfont\put(52,22.2671){\makebox(0,0)[t]{\textcolor[rgb]{0.15,0.15,0.15}{{0}}}}
\fontsize{6}{0}
\selectfont\put(103.675,22.2671){\makebox(0,0)[t]{\textcolor[rgb]{0.15,0.15,0.15}{{10}}}}
\fontsize{6}{0}
\selectfont\put(155.351,22.2671){\makebox(0,0)[t]{\textcolor[rgb]{0.15,0.15,0.15}{{20}}}}
\fontsize{6}{0}
\selectfont\put(207.026,22.2671){\makebox(0,0)[t]{\textcolor[rgb]{0.15,0.15,0.15}{{30}}}}
\fontsize{6}{0}
\selectfont\put(258.701,22.2671){\makebox(0,0)[t]{\textcolor[rgb]{0.15,0.15,0.15}{{40}}}}
\fontsize{6}{0}
\selectfont\put(310.376,22.2671){\makebox(0,0)[t]{\textcolor[rgb]{0.15,0.15,0.15}{{50}}}}
\fontsize{6}{0}
\selectfont\put(48.5278,34.7158){\makebox(0,0)[r]{\textcolor[rgb]{0.15,0.15,0.15}{{0}}}}
\fontsize{6}{0}
\selectfont\put(48.5278,52.7559){\makebox(0,0)[r]{\textcolor[rgb]{0.15,0.15,0.15}{{0.5}}}}
\fontsize{6}{0}
\selectfont\put(48.5278,70.7954){\makebox(0,0)[r]{\textcolor[rgb]{0.15,0.15,0.15}{{1}}}}
\fontsize{6}{0}
\selectfont\put(48.5278,88.8354){\makebox(0,0)[r]{\textcolor[rgb]{0.15,0.15,0.15}{{1.5}}}}
\fontsize{6}{0}
\selectfont\put(48.5278,106.875){\makebox(0,0)[r]{\textcolor[rgb]{0.15,0.15,0.15}{{2}}}}
\fontsize{7}{0}
\selectfont\put(33.5278,67.1875){\rotatebox{90}{\makebox(0,0)[b]{\textcolor[rgb]{0.15,0.15,0.15}{{Sinal $u_{i}(t)$}}}}}
\fontsize{7}{0}
\selectfont\put(207,11.2671){\makebox(0,0)[t]{\textcolor[rgb]{0.15,0.15,0.15}{{Tempo (s)}}}}
\fontsize{6}{0}
\selectfont\put(332,76.0015){\makebox(0,0)[l]{\textcolor[rgb]{0,0,0}{{$u_{1}(t)$}}}}
\fontsize{6}{0}
\selectfont\put(332,65.001){\makebox(0,0)[l]{\textcolor[rgb]{0,0,0}{{$u_{2}(t)$}}}}
\fontsize{6}{0}
\selectfont\put(332,54.001){\makebox(0,0)[l]{\textcolor[rgb]{0,0,0}{{$u_{3}(t)$}}}}
\fontsize{6}{0}
\selectfont\put(332,43.0005){\makebox(0,0)[l]{\textcolor[rgb]{0,0,0}{{$u_{4}(t)$}}}}
\end{picture}

    \end{minipage}
\end{figure}

É possível observar também na Figura \ref{fig:desafio2:questao5} uma diferença
das respostas ao degrau dos modelos M1 e M4. O primeiro responde com um alto
overshoot e oscilação na resposta transitória. Já o segundo se comporta próximo
a um sistema superamortecido. Com base na resposta em frequência, pode se
analisar tal diferença entre os modelos. Na resposta em frequência, o atraso
$e^{-sL_{i}}$ insere no sistema uma diminuição de fase de formar linear, pois
$\arg\angle e^{-jwL_{i}} = L_{i}w$. Dessa forma, como $L_{1} > L_{4}$, a fase em
altas frequência do sistema do modelo M1 é maior, consequentemente menor margem
de fase e mais próximo do eixo $jw$ os polos do sistema em malha fechada estão.
Como o degrau é rico em altas frequências, o modelo com menor margem de fase
tende a ter maior oscilação e, consequentemente, um maior overshoot. O que é
observado no modelo M1.

\subsubsection{Resposta em malha fechada do modelo M1 com diferentes configurações}
Conforme discutido no final do tópico
\ref{subsub:desafio2:resposta-em-malha-fechadas-dos-modelos-m1-m2-m3-e-m4}, o
modelo M1 apresenta uma resposta transitória ao degrau com alto
\textit{overshoot} e oscilação. Desta forma, este modelo foi simulado em
diferentes configurações do sistema de controle em malha fechada. Simulou-se o
sistema em malha fechada à resposta ao degrau e perturbação na entrada do tipo
degrau com: filtro de referência $F(s)$ de primeira ordem; utilizando
controlador I+P; com controlador I+P e saturação do sinal de controle; e, por
fim, com os dois itens anteriores mais a ação de \textit{anti-windup}.

O primeiro resultado a ser observado, Figura
\ref{fig:desafio2:resultado-questao-6-7}, é a resposta do sistema em malha
fechada com a utilização do filtro de referência comparado ao uso do controlador
I+P.

\begin{figure}[!ht]
    \caption{Simulação do sistema de controle em malhada fechada com filtro de
    referência e com controlador I+P.}
    \vspace{-10pt}
    \hspace{-30pt}
    \label{fig:desafio2:resultado-questao-6-7}
    \begin{minipage}{\linewidth}
        % Title: gl2ps_renderer figure
% Creator: GL2PS 1.4.0, (C) 1999-2017 C. Geuzaine
% For: Octave
% CreationDate: Sat Oct  2 20:17:41 2021
\setlength{\unitlength}{1pt}
\begin{picture}(0,0)
\includegraphics{images/challenge2/resultado-questao-6-7-inc}
\end{picture}%
\begin{picture}(400,350)(0,0)
\fontsize{6}{0}
\selectfont\put(52,247.88){\makebox(0,0)[t]{\textcolor[rgb]{0.15,0.15,0.15}{{0}}}}
\fontsize{6}{0}
\selectfont\put(103.667,247.88){\makebox(0,0)[t]{\textcolor[rgb]{0.15,0.15,0.15}{{10}}}}
\fontsize{6}{0}
\selectfont\put(155.333,247.88){\makebox(0,0)[t]{\textcolor[rgb]{0.15,0.15,0.15}{{20}}}}
\fontsize{6}{0}
\selectfont\put(207,247.88){\makebox(0,0)[t]{\textcolor[rgb]{0.15,0.15,0.15}{{30}}}}
\fontsize{6}{0}
\selectfont\put(258.667,247.88){\makebox(0,0)[t]{\textcolor[rgb]{0.15,0.15,0.15}{{40}}}}
\fontsize{6}{0}
\selectfont\put(310.333,247.88){\makebox(0,0)[t]{\textcolor[rgb]{0.15,0.15,0.15}{{50}}}}
\fontsize{6}{0}
\selectfont\put(362,247.88){\makebox(0,0)[t]{\textcolor[rgb]{0.15,0.15,0.15}{{60}}}}
\fontsize{6}{0}
\selectfont\put(48.5278,261.518){\makebox(0,0)[r]{\textcolor[rgb]{0.15,0.15,0.15}{{0}}}}
\fontsize{6}{0}
\selectfont\put(48.5278,282.535){\makebox(0,0)[r]{\textcolor[rgb]{0.15,0.15,0.15}{{0.5}}}}
\fontsize{6}{0}
\selectfont\put(48.5278,303.551){\makebox(0,0)[r]{\textcolor[rgb]{0.15,0.15,0.15}{{1}}}}
\fontsize{7}{0}
\selectfont\put(33.5278,286.764){\rotatebox{90}{\makebox(0,0)[b]{\textcolor[rgb]{0.15,0.15,0.15}{{Saida $y(t)$}}}}}
\fontsize{7}{0}
\selectfont\put(207,236.88){\makebox(0,0)[t]{\textcolor[rgb]{0.15,0.15,0.15}{{Tempo (s)}}}}
\fontsize{6}{0}
\selectfont\put(267,303.613){\makebox(0,0)[l]{\textcolor[rgb]{0,0,0}{{$r_{f}(t)$}}}}
\fontsize{6}{0}
\selectfont\put(267,292.613){\makebox(0,0)[l]{\textcolor[rgb]{0,0,0}{{$r(t)$}}}}
\fontsize{6}{0}
\selectfont\put(267,281.113){\makebox(0,0)[l]{\textcolor[rgb]{0,0,0}{{$y_{m1}(t)$ com filtro de referência}}}}
\fontsize{6}{0}
\selectfont\put(267,269.112){\makebox(0,0)[l]{\textcolor[rgb]{0,0,0}{{$y_{m1}(t)$ com controlador I+P}}}}
\fontsize{6}{0}
\selectfont\put(52,140.574){\makebox(0,0)[t]{\textcolor[rgb]{0.15,0.15,0.15}{{0}}}}
\fontsize{6}{0}
\selectfont\put(103.667,140.574){\makebox(0,0)[t]{\textcolor[rgb]{0.15,0.15,0.15}{{10}}}}
\fontsize{6}{0}
\selectfont\put(155.333,140.574){\makebox(0,0)[t]{\textcolor[rgb]{0.15,0.15,0.15}{{20}}}}
\fontsize{6}{0}
\selectfont\put(207,140.574){\makebox(0,0)[t]{\textcolor[rgb]{0.15,0.15,0.15}{{30}}}}
\fontsize{6}{0}
\selectfont\put(258.667,140.574){\makebox(0,0)[t]{\textcolor[rgb]{0.15,0.15,0.15}{{40}}}}
\fontsize{6}{0}
\selectfont\put(310.333,140.574){\makebox(0,0)[t]{\textcolor[rgb]{0.15,0.15,0.15}{{50}}}}
\fontsize{6}{0}
\selectfont\put(362,140.574){\makebox(0,0)[t]{\textcolor[rgb]{0.15,0.15,0.15}{{60}}}}
\fontsize{6}{0}
\selectfont\put(48.5278,163.308){\makebox(0,0)[r]{\textcolor[rgb]{0.15,0.15,0.15}{{0}}}}
\fontsize{6}{0}
\selectfont\put(48.5278,184.059){\makebox(0,0)[r]{\textcolor[rgb]{0.15,0.15,0.15}{{0.5}}}}
\fontsize{6}{0}
\selectfont\put(48.5278,204.811){\makebox(0,0)[r]{\textcolor[rgb]{0.15,0.15,0.15}{{1}}}}
\fontsize{7}{0}
\selectfont\put(33.5278,179.458){\rotatebox{90}{\makebox(0,0)[b]{\textcolor[rgb]{0.15,0.15,0.15}{{Erro $e(t)$}}}}}
\fontsize{7}{0}
\selectfont\put(207,129.574){\makebox(0,0)[t]{\textcolor[rgb]{0.15,0.15,0.15}{{Tempo (s)}}}}
\fontsize{6}{0}
\selectfont\put(267,172.307){\makebox(0,0)[l]{\textcolor[rgb]{0,0,0}{{$e_{m1}(t)$ com filtro de referência}}}}
\fontsize{6}{0}
\selectfont\put(267,161.307){\makebox(0,0)[l]{\textcolor[rgb]{0,0,0}{{$e_{m1}(t)$ com controlador I+P}}}}
\fontsize{6}{0}
\selectfont\put(52,33.2681){\makebox(0,0)[t]{\textcolor[rgb]{0.15,0.15,0.15}{{0}}}}
\fontsize{6}{0}
\selectfont\put(103.667,33.2681){\makebox(0,0)[t]{\textcolor[rgb]{0.15,0.15,0.15}{{10}}}}
\fontsize{6}{0}
\selectfont\put(155.333,33.2681){\makebox(0,0)[t]{\textcolor[rgb]{0.15,0.15,0.15}{{20}}}}
\fontsize{6}{0}
\selectfont\put(207,33.2681){\makebox(0,0)[t]{\textcolor[rgb]{0.15,0.15,0.15}{{30}}}}
\fontsize{6}{0}
\selectfont\put(258.667,33.2681){\makebox(0,0)[t]{\textcolor[rgb]{0.15,0.15,0.15}{{40}}}}
\fontsize{6}{0}
\selectfont\put(310.333,33.2681){\makebox(0,0)[t]{\textcolor[rgb]{0.15,0.15,0.15}{{50}}}}
\fontsize{6}{0}
\selectfont\put(362,33.2681){\makebox(0,0)[t]{\textcolor[rgb]{0.15,0.15,0.15}{{60}}}}
\fontsize{6}{0}
\selectfont\put(48.5278,46.9131){\makebox(0,0)[r]{\textcolor[rgb]{0.15,0.15,0.15}{{0}}}}
\fontsize{6}{0}
\selectfont\put(48.5278,67.9463){\makebox(0,0)[r]{\textcolor[rgb]{0.15,0.15,0.15}{{0.5}}}}
\fontsize{6}{0}
\selectfont\put(48.5278,88.979){\makebox(0,0)[r]{\textcolor[rgb]{0.15,0.15,0.15}{{1}}}}
\fontsize{7}{0}
\selectfont\put(33.5278,72.1528){\rotatebox{90}{\makebox(0,0)[b]{\textcolor[rgb]{0.15,0.15,0.15}{{Sinal de Controle $u(t)$}}}}}
\fontsize{7}{0}
\selectfont\put(207,22.2681){\makebox(0,0)[t]{\textcolor[rgb]{0.15,0.15,0.15}{{Tempo (s)}}}}
\fontsize{6}{0}
\selectfont\put(267,65.001){\makebox(0,0)[l]{\textcolor[rgb]{0,0,0}{{$u_{m1}(t)$ com filtro de referência}}}}
\fontsize{6}{0}
\selectfont\put(267,54.0005){\makebox(0,0)[l]{\textcolor[rgb]{0,0,0}{{$u_{m1}(t)$ com controlador I+P}}}}
\end{picture}

    \end{minipage}
\end{figure}

Um resultado peculiar observado no primeiro gráfico da Figura
\ref{fig:desafio2:resultado-questao-6-7} é que a resposta do sistema em malha
fechada com o filtro de referência e com o controlador I+P são aproximadamente
iguais. A mesma conclusão pode ser obtida do gráfico contendo os sinais de
controle. De forma algébrica é possível provar que o sinal de controle $U(s)$ é
igual dado as duas configurações. Temos que para o sistema em malha fechada com
um filtro de referência de primeira ordem dado por $F(s) = 1/({T_{i}(s) + 1})$,

\begin{equation}
    \label{eq:desafio2:sinal-de-controle-com-filtro-de-referencia}
    \begin{split}
        U_{f}(s) &= C(s)E(s) \\
                   &= C(s)[F(s)R(s) - Y(s)] \\
                   &= \frac{K_{c}\cancel Z_{c}}{P_{c}}\frac{1}{\cancel P_{f}}R(s) - [K_{c} + C_{I}(s)]Y(s) \\
                   &= C_{I}(s)R(s) - [K_{c} + C_{I}(s)]Y(s)
    \end{split}
\end{equation}
onde:

\begin{conditions*}
    U_{f}(s) & o sinal de controle dado uma referência $R(s)$ filtrada por $F(s)$;  \\
    C(s) & a função de transferência do controlador de Skogestad (Equação
    \ref{eq:desafio2:controlador-de-skogestad});  \\
    K_{c} & o ganho do controlador $C(s)$;  \\
    Z_{c} & o zero do controlador $C(s)$;  \\
    P_{c} & o polo do controlador $C(s)$;  \\
    P_{f} & o polo do filtro de referência $F(s)$; e  \\
    C_{I}(s) & a porção integradora de $C(s)$ dada por $\frac{K_{c}}{T_{i}s}$.
\end{conditions*}

Para o controlador I+P a análise é parecida e temos que

\begin{equation}
    \label{eq:desafio2:sinal-de-controle-com-controlador-i-mais-p}
    \begin{split}
        U_{I+P}(s) &= C_{I}(s)E(s) - K_{c}Y(s) \\
                   &= C_{I}(s)[R(s) - Y(s)] - K_{c}Y(s) \\
                   &= C_{I}(s)R(s) - [K_{c} + C_{I}(s)]Y(s)
    \end{split}
\end{equation}

onde:

\begin{conditions*}
    U_{I+P}(s) & a Transformada de Laplace do sinal de controle do controlador I+P.
\end{conditions*}

Portanto, observa-se através das deduções realizadas nas Equações
\ref{eq:desafio2:sinal-de-controle-com-filtro-de-referencia} e
\ref{eq:desafio2:sinal-de-controle-com-controlador-i-mais-p} que o sinal de
controle é equivalente para as duas configurações analisadas e, desta forma, é
ratificado o que foi observado na Figura
\ref{fig:desafio2:resultado-questao-6-7}. Entretanto, observa-se que o erro do
sistema não se comporta da mesma forma. Fica notório no gráfico de erro a função
do filtro de referência na suavização da referência. Percebe-se também uma
diminuição do \textit{overshoot} e da oscilação na resposta do sistema. Esta
diminuição pode ser analisada novamente com base na resposta em frequência, pois
o uso de um filtro de referência remove significativamente as altas frequências
do sinal de erro, o que inibe a amplificação de sinais mais próximos do eixo
$jw$ pelo controlador $C(s)$.

\begin{figure}[!ht]
    \caption{Resposta, erro e sinal de controle do sistema em malha fechada com
    controlador I+P na presença de saturação do sinal de controle.}
    \vspace{-10pt}
    \hspace{-30pt}
    \label{fig:desafio2:resultado-questao-8}
    \begin{minipage}{\linewidth}
        % Title: gl2ps_renderer figure
% Creator: GL2PS 1.4.0, (C) 1999-2017 C. Geuzaine
% For: Octave
% CreationDate: Sat Oct  2 20:56:22 2021
\setlength{\unitlength}{1pt}
\begin{picture}(0,0)
\includegraphics{images/challenge2/resultado-questao-8-inc}
\end{picture}%
\begin{picture}(400,300)(0,0)
\fontsize{6}{0}
\selectfont\put(52,211.72){\makebox(0,0)[t]{\textcolor[rgb]{0.15,0.15,0.15}{{0}}}}
\fontsize{6}{0}
\selectfont\put(103.667,211.72){\makebox(0,0)[t]{\textcolor[rgb]{0.15,0.15,0.15}{{10}}}}
\fontsize{6}{0}
\selectfont\put(155.333,211.72){\makebox(0,0)[t]{\textcolor[rgb]{0.15,0.15,0.15}{{20}}}}
\fontsize{6}{0}
\selectfont\put(207,211.72){\makebox(0,0)[t]{\textcolor[rgb]{0.15,0.15,0.15}{{30}}}}
\fontsize{6}{0}
\selectfont\put(258.667,211.72){\makebox(0,0)[t]{\textcolor[rgb]{0.15,0.15,0.15}{{40}}}}
\fontsize{6}{0}
\selectfont\put(310.333,211.72){\makebox(0,0)[t]{\textcolor[rgb]{0.15,0.15,0.15}{{50}}}}
\fontsize{6}{0}
\selectfont\put(362,211.72){\makebox(0,0)[t]{\textcolor[rgb]{0.15,0.15,0.15}{{60}}}}
\fontsize{6}{0}
\selectfont\put(48.5278,216.953){\makebox(0,0)[r]{\textcolor[rgb]{0.15,0.15,0.15}{{-0.2}}}}
\fontsize{6}{0}
\selectfont\put(48.5278,224.99){\makebox(0,0)[r]{\textcolor[rgb]{0.15,0.15,0.15}{{0}}}}
\fontsize{6}{0}
\selectfont\put(48.5278,233.028){\makebox(0,0)[r]{\textcolor[rgb]{0.15,0.15,0.15}{{0.2}}}}
\fontsize{6}{0}
\selectfont\put(48.5278,241.065){\makebox(0,0)[r]{\textcolor[rgb]{0.15,0.15,0.15}{{0.4}}}}
\fontsize{6}{0}
\selectfont\put(48.5278,249.103){\makebox(0,0)[r]{\textcolor[rgb]{0.15,0.15,0.15}{{0.6}}}}
\fontsize{6}{0}
\selectfont\put(48.5278,257.14){\makebox(0,0)[r]{\textcolor[rgb]{0.15,0.15,0.15}{{0.8}}}}
\fontsize{6}{0}
\selectfont\put(48.5278,265.177){\makebox(0,0)[r]{\textcolor[rgb]{0.15,0.15,0.15}{{1}}}}
\fontsize{6}{0}
\selectfont\put(48.5278,273.215){\makebox(0,0)[r]{\textcolor[rgb]{0.15,0.15,0.15}{{1.2}}}}
\fontsize{7}{0}
\selectfont\put(207,200.72){\makebox(0,0)[t]{\textcolor[rgb]{0.15,0.15,0.15}{{Tempo (s)}}}}
\fontsize{7}{0}
\selectfont\put(31.5278,245.083){\rotatebox{90}{\makebox(0,0)[b]{\textcolor[rgb]{0.15,0.15,0.15}{{Saida $y(t)$}}}}}
\fontsize{6}{0}
\selectfont\put(64.2969,220.168){\makebox(0,0)[l]{\textcolor[rgb]{0,0,0}{{$t_{1}$}}}}
\fontsize{6}{0}
\selectfont\put(82.1216,220.168){\makebox(0,0)[l]{\textcolor[rgb]{0,0,0}{{$t_{2}$}}}}
\fontsize{6}{0}
\selectfont\put(146.499,220.168){\makebox(0,0)[l]{\textcolor[rgb]{0,0,0}{{$t_{3}$}}}}
\fontsize{6}{0}
\selectfont\put(157.348,220.168){\makebox(0,0)[l]{\textcolor[rgb]{0,0,0}{{$t_{4}$}}}}
\fontsize{6}{0}
\selectfont\put(263,232.954){\makebox(0,0)[l]{\textcolor[rgb]{0,0,0}{{$y_{m1}(t)$ com saturação de $u(t)$}}}}
\fontsize{6}{0}
\selectfont\put(52,119.744){\makebox(0,0)[t]{\textcolor[rgb]{0.15,0.15,0.15}{{0}}}}
\fontsize{6}{0}
\selectfont\put(103.667,119.744){\makebox(0,0)[t]{\textcolor[rgb]{0.15,0.15,0.15}{{10}}}}
\fontsize{6}{0}
\selectfont\put(155.333,119.744){\makebox(0,0)[t]{\textcolor[rgb]{0.15,0.15,0.15}{{20}}}}
\fontsize{6}{0}
\selectfont\put(207,119.744){\makebox(0,0)[t]{\textcolor[rgb]{0.15,0.15,0.15}{{30}}}}
\fontsize{6}{0}
\selectfont\put(258.667,119.744){\makebox(0,0)[t]{\textcolor[rgb]{0.15,0.15,0.15}{{40}}}}
\fontsize{6}{0}
\selectfont\put(310.333,119.744){\makebox(0,0)[t]{\textcolor[rgb]{0.15,0.15,0.15}{{50}}}}
\fontsize{6}{0}
\selectfont\put(362,119.744){\makebox(0,0)[t]{\textcolor[rgb]{0.15,0.15,0.15}{{60}}}}
\fontsize{6}{0}
\selectfont\put(48.5278,130.72){\makebox(0,0)[r]{\textcolor[rgb]{0.15,0.15,0.15}{{-0.2}}}}
\fontsize{6}{0}
\selectfont\put(48.5278,137.937){\makebox(0,0)[r]{\textcolor[rgb]{0.15,0.15,0.15}{{0}}}}
\fontsize{6}{0}
\selectfont\put(48.5278,145.154){\makebox(0,0)[r]{\textcolor[rgb]{0.15,0.15,0.15}{{0.2}}}}
\fontsize{6}{0}
\selectfont\put(48.5278,152.371){\makebox(0,0)[r]{\textcolor[rgb]{0.15,0.15,0.15}{{0.4}}}}
\fontsize{6}{0}
\selectfont\put(48.5278,159.587){\makebox(0,0)[r]{\textcolor[rgb]{0.15,0.15,0.15}{{0.6}}}}
\fontsize{6}{0}
\selectfont\put(48.5278,166.804){\makebox(0,0)[r]{\textcolor[rgb]{0.15,0.15,0.15}{{0.8}}}}
\fontsize{6}{0}
\selectfont\put(48.5278,174.021){\makebox(0,0)[r]{\textcolor[rgb]{0.15,0.15,0.15}{{1}}}}
\fontsize{7}{0}
\selectfont\put(31.5278,153.107){\rotatebox{90}{\makebox(0,0)[b]{\textcolor[rgb]{0.15,0.15,0.15}{{Erro $e(t)$}}}}}
\fontsize{7}{0}
\selectfont\put(207,108.744){\makebox(0,0)[t]{\textcolor[rgb]{0.15,0.15,0.15}{{Tempo (s)}}}}
\fontsize{6}{0}
\selectfont\put(64.2969,127.863){\makebox(0,0)[l]{\textcolor[rgb]{0,0,0}{{$t_{1}$}}}}
\fontsize{6}{0}
\selectfont\put(82.1216,127.863){\makebox(0,0)[l]{\textcolor[rgb]{0,0,0}{{$t_{2}$}}}}
\fontsize{6}{0}
\selectfont\put(146.499,127.863){\makebox(0,0)[l]{\textcolor[rgb]{0,0,0}{{$t_{3}$}}}}
\fontsize{6}{0}
\selectfont\put(157.348,127.863){\makebox(0,0)[l]{\textcolor[rgb]{0,0,0}{{$t_{4}$}}}}
\fontsize{6}{0}
\selectfont\put(288,140.977){\makebox(0,0)[l]{\textcolor[rgb]{0,0,0}{{$e_{m1}(t)$ com saturação}}}}
\fontsize{6}{0}
\selectfont\put(52,27.7676){\makebox(0,0)[t]{\textcolor[rgb]{0.15,0.15,0.15}{{0}}}}
\fontsize{6}{0}
\selectfont\put(103.667,27.7676){\makebox(0,0)[t]{\textcolor[rgb]{0.15,0.15,0.15}{{10}}}}
\fontsize{6}{0}
\selectfont\put(155.333,27.7676){\makebox(0,0)[t]{\textcolor[rgb]{0.15,0.15,0.15}{{20}}}}
\fontsize{6}{0}
\selectfont\put(207,27.7676){\makebox(0,0)[t]{\textcolor[rgb]{0.15,0.15,0.15}{{30}}}}
\fontsize{6}{0}
\selectfont\put(258.667,27.7676){\makebox(0,0)[t]{\textcolor[rgb]{0.15,0.15,0.15}{{40}}}}
\fontsize{6}{0}
\selectfont\put(310.333,27.7676){\makebox(0,0)[t]{\textcolor[rgb]{0.15,0.15,0.15}{{50}}}}
\fontsize{6}{0}
\selectfont\put(362,27.7676){\makebox(0,0)[t]{\textcolor[rgb]{0.15,0.15,0.15}{{60}}}}
\fontsize{6}{0}
\selectfont\put(48.5278,33){\makebox(0,0)[r]{\textcolor[rgb]{0.15,0.15,0.15}{{-0.2}}}}
\fontsize{6}{0}
\selectfont\put(48.5278,42.377){\makebox(0,0)[r]{\textcolor[rgb]{0.15,0.15,0.15}{{0}}}}
\fontsize{6}{0}
\selectfont\put(48.5278,51.7539){\makebox(0,0)[r]{\textcolor[rgb]{0.15,0.15,0.15}{{0.2}}}}
\fontsize{6}{0}
\selectfont\put(48.5278,61.1309){\makebox(0,0)[r]{\textcolor[rgb]{0.15,0.15,0.15}{{0.4}}}}
\fontsize{6}{0}
\selectfont\put(48.5278,70.5078){\makebox(0,0)[r]{\textcolor[rgb]{0.15,0.15,0.15}{{0.6}}}}
\fontsize{6}{0}
\selectfont\put(48.5278,79.8848){\makebox(0,0)[r]{\textcolor[rgb]{0.15,0.15,0.15}{{0.8}}}}
\fontsize{6}{0}
\selectfont\put(48.5278,89.2617){\makebox(0,0)[r]{\textcolor[rgb]{0.15,0.15,0.15}{{1}}}}
\fontsize{7}{0}
\selectfont\put(31.5278,61.1309){\rotatebox{90}{\makebox(0,0)[b]{\textcolor[rgb]{0.15,0.15,0.15}{{Sinal de Controle $u(t)$}}}}}
\fontsize{7}{0}
\selectfont\put(207,16.7676){\makebox(0,0)[t]{\textcolor[rgb]{0.15,0.15,0.15}{{Tempo (s)}}}}
\fontsize{6}{0}
\selectfont\put(64.2969,36.751){\makebox(0,0)[l]{\textcolor[rgb]{0,0,0}{{$t_{1}$}}}}
\fontsize{6}{0}
\selectfont\put(82.1216,36.751){\makebox(0,0)[l]{\textcolor[rgb]{0,0,0}{{$t_{2}$}}}}
\fontsize{6}{0}
\selectfont\put(146.499,36.751){\makebox(0,0)[l]{\textcolor[rgb]{0,0,0}{{$t_{3}$}}}}
\fontsize{6}{0}
\selectfont\put(157.348,36.751){\makebox(0,0)[l]{\textcolor[rgb]{0,0,0}{{$t_{4}$}}}}
\fontsize{6}{0}
\selectfont\put(288,49.0005){\makebox(0,0)[l]{\textcolor[rgb]{0,0,0}{{$u_{m1}(t)$ com saturação}}}}
\end{picture}

    \end{minipage}
\end{figure}

A Figura \ref{fig:desafio2:resultado-questao-8} permite analisar o que acontece
com o mesmo sistema em malha fechada quando existe uma não linearidade no sinal
de controle devido ao efeito de saturação do sinal, como observado no último
gráfico. Observa-se que a resposta do sistema não apresente um
\textit{overshoot}, pelo contrário, ela se comporta próxima a resposta de um
sistema de primeira ordem. Antes de analisar graficamemte a consequência da
saturação do sinal, é necessário entender algebricamente o controlador I+P. No
domínio $s$ temos que 

\begin{equation}
    U_{I+P}(s) = \frac{K_{c}}{sTi}E(s) - K_{c}Y(s),
\end{equation}
cuja sua versão no domínio do tempo é

\begin{equation}
    \label{eq:desafio2:sinal-de-controle-i-mais-p-no-tempo}
    u_{I+P}(t) = \pounds( U_{I+P}(s)) = \frac{K_{c}}{T_{i}}\int_{0}^{\tau}e(\tau)d\tau - K_{c}y(t).
\end{equation}

Voltando então à Figura \ref{fig:desafio2:resultado-questao-8}, nota-se que o
sinal de controle satura em $t = t_{1}$, pois a parcela da integração da Equação
\ref{eq:desafio2:sinal-de-controle-i-mais-p-no-tempo} é maior que parcela
proporcional neste intervalo de tempo. Vale salientar que o atraso no sistema
contribui de maneira significativa com a saturação do sinal de controle, pois o
erro permanece no seu valor máximo durante o período de atraso. A parcela da
integração continua crescendo após $t = t_{1}$, pois $e(t) > 0$ para $t_{1} \le
t < t_{2}$. Para $t_{2} \le t \le t_{3}$, $e(t) < 0$, desacumulando dessa forma
a parcela da integração, sendo $t = t_{3}$ o momento que a integração da parcela
negativa do erro juntamente com a parte proporcional do controlador faz o sinal
de controle sair da zona de saturação. Percebe-se porém que a desacumulação
acontece de forma muito mais lenta que a acumulação, fazendo com que o sistema
leve um longo tempo ($t_{2} \le t \le t_{4}$) até chegar em erro em regime
permanente nulo. A depender do processo, este longo intervalo de não seguimento
de referência trás consequências para o processo e, consequentemente, torna-se
necessário a sua mitigação. Uma das formas de mitigar o problema da saturação do
sinal de controle é o uso da ação de \textit{anti-windup}. Portanto, para
analisar o seu efeito no problema acima descrito, foi implementada a ação de
\textit{anti-windup} e o resultado pode ser examinado na Figura
\ref{fig:desafio2:resultado-questao-9}.

\begin{figure}[!ht]
    \caption{Resposta, erro e sinal de controle do sistema em malha fechada com
    controlador I+P na presença de saturação do sinal de controle com ação
    \textit{anti-windup}.}
    \vspace{-10pt}
    \hspace{-30pt}
    \label{fig:desafio2:resultado-questao-9}
    \begin{minipage}{\linewidth}
        % Title: gl2ps_renderer figure
% Creator: GL2PS 1.4.0, (C) 1999-2017 C. Geuzaine
% For: Octave
% CreationDate: Thu Nov 25 17:46:16 2021
\setlength{\unitlength}{1pt}
\begin{picture}(0,0)
\includegraphics{chapters/challenge6/images/resultado-questao-9-inc}
\end{picture}%
\begin{picture}(400,250)(0,0)
\fontsize{6}{0}
\selectfont\put(52,141.404){\makebox(0,0)[t]{\textcolor[rgb]{0.15,0.15,0.15}{{0}}}}
\fontsize{6}{0}
\selectfont\put(70.4365,141.404){\makebox(0,0)[t]{\textcolor[rgb]{0.15,0.15,0.15}{{20}}}}
\fontsize{6}{0}
\selectfont\put(88.873,141.404){\makebox(0,0)[t]{\textcolor[rgb]{0.15,0.15,0.15}{{40}}}}
\fontsize{6}{0}
\selectfont\put(107.309,141.404){\makebox(0,0)[t]{\textcolor[rgb]{0.15,0.15,0.15}{{60}}}}
\fontsize{6}{0}
\selectfont\put(125.746,141.404){\makebox(0,0)[t]{\textcolor[rgb]{0.15,0.15,0.15}{{80}}}}
\fontsize{6}{0}
\selectfont\put(144.182,141.404){\makebox(0,0)[t]{\textcolor[rgb]{0.15,0.15,0.15}{{100}}}}
\fontsize{6}{0}
\selectfont\put(162.619,141.404){\makebox(0,0)[t]{\textcolor[rgb]{0.15,0.15,0.15}{{120}}}}
\fontsize{6}{0}
\selectfont\put(181.055,141.404){\makebox(0,0)[t]{\textcolor[rgb]{0.15,0.15,0.15}{{140}}}}
\fontsize{6}{0}
\selectfont\put(48.5205,151.661){\makebox(0,0)[r]{\textcolor[rgb]{0.15,0.15,0.15}{{0}}}}
\fontsize{6}{0}
\selectfont\put(48.5205,164.221){\makebox(0,0)[r]{\textcolor[rgb]{0.15,0.15,0.15}{{0.5}}}}
\fontsize{6}{0}
\selectfont\put(48.5205,176.78){\makebox(0,0)[r]{\textcolor[rgb]{0.15,0.15,0.15}{{1}}}}
\fontsize{6}{0}
\selectfont\put(48.5205,189.34){\makebox(0,0)[r]{\textcolor[rgb]{0.15,0.15,0.15}{{1.5}}}}
\fontsize{6}{0}
\selectfont\put(48.5205,201.899){\makebox(0,0)[r]{\textcolor[rgb]{0.15,0.15,0.15}{{2}}}}
\fontsize{6}{0}
\selectfont\put(48.5205,214.459){\makebox(0,0)[r]{\textcolor[rgb]{0.15,0.15,0.15}{{2.5}}}}
\fontsize{7}{0}
\selectfont\put(33.5205,186.325){\rotatebox{90}{\makebox(0,0)[b]{\textcolor[rgb]{0.15,0.15,0.15}{{Saida $y_1(t)$}}}}}
\fontsize{7}{0}
\selectfont\put(116.643,130.404){\makebox(0,0)[t]{\textcolor[rgb]{0.15,0.15,0.15}{{Tempo (s)}}}}
\fontsize{6}{0}
\selectfont\put(232.715,141.404){\makebox(0,0)[t]{\textcolor[rgb]{0.15,0.15,0.15}{{0}}}}
\fontsize{6}{0}
\selectfont\put(251.151,141.404){\makebox(0,0)[t]{\textcolor[rgb]{0.15,0.15,0.15}{{20}}}}
\fontsize{6}{0}
\selectfont\put(269.587,141.404){\makebox(0,0)[t]{\textcolor[rgb]{0.15,0.15,0.15}{{40}}}}
\fontsize{6}{0}
\selectfont\put(288.024,141.404){\makebox(0,0)[t]{\textcolor[rgb]{0.15,0.15,0.15}{{60}}}}
\fontsize{6}{0}
\selectfont\put(306.46,141.404){\makebox(0,0)[t]{\textcolor[rgb]{0.15,0.15,0.15}{{80}}}}
\fontsize{6}{0}
\selectfont\put(324.896,141.404){\makebox(0,0)[t]{\textcolor[rgb]{0.15,0.15,0.15}{{100}}}}
\fontsize{6}{0}
\selectfont\put(343.333,141.404){\makebox(0,0)[t]{\textcolor[rgb]{0.15,0.15,0.15}{{120}}}}
\fontsize{6}{0}
\selectfont\put(361.77,141.404){\makebox(0,0)[t]{\textcolor[rgb]{0.15,0.15,0.15}{{140}}}}
\fontsize{6}{0}
\selectfont\put(229.235,154.496){\makebox(0,0)[r]{\textcolor[rgb]{0.15,0.15,0.15}{{0}}}}
\fontsize{6}{0}
\selectfont\put(229.235,174.144){\makebox(0,0)[r]{\textcolor[rgb]{0.15,0.15,0.15}{{0.5}}}}
\fontsize{6}{0}
\selectfont\put(229.235,193.792){\makebox(0,0)[r]{\textcolor[rgb]{0.15,0.15,0.15}{{1}}}}
\fontsize{6}{0}
\selectfont\put(229.235,213.44){\makebox(0,0)[r]{\textcolor[rgb]{0.15,0.15,0.15}{{1.5}}}}
\fontsize{7}{0}
\selectfont\put(214.235,186.325){\rotatebox{90}{\makebox(0,0)[b]{\textcolor[rgb]{0.15,0.15,0.15}{{Saida $y_2(t)$}}}}}
\fontsize{7}{0}
\selectfont\put(297.357,130.404){\makebox(0,0)[t]{\textcolor[rgb]{0.15,0.15,0.15}{{Tempo (s)}}}}
\fontsize{6}{0}
\selectfont\put(52,22.2671){\makebox(0,0)[t]{\textcolor[rgb]{0.15,0.15,0.15}{{0}}}}
\fontsize{6}{0}
\selectfont\put(70.4365,22.2671){\makebox(0,0)[t]{\textcolor[rgb]{0.15,0.15,0.15}{{20}}}}
\fontsize{6}{0}
\selectfont\put(88.873,22.2671){\makebox(0,0)[t]{\textcolor[rgb]{0.15,0.15,0.15}{{40}}}}
\fontsize{6}{0}
\selectfont\put(107.309,22.2671){\makebox(0,0)[t]{\textcolor[rgb]{0.15,0.15,0.15}{{60}}}}
\fontsize{6}{0}
\selectfont\put(125.746,22.2671){\makebox(0,0)[t]{\textcolor[rgb]{0.15,0.15,0.15}{{80}}}}
\fontsize{6}{0}
\selectfont\put(144.182,22.2671){\makebox(0,0)[t]{\textcolor[rgb]{0.15,0.15,0.15}{{100}}}}
\fontsize{6}{0}
\selectfont\put(162.619,22.2671){\makebox(0,0)[t]{\textcolor[rgb]{0.15,0.15,0.15}{{120}}}}
\fontsize{6}{0}
\selectfont\put(181.055,22.2671){\makebox(0,0)[t]{\textcolor[rgb]{0.15,0.15,0.15}{{140}}}}
\fontsize{6}{0}
\selectfont\put(48.5205,36.0054){\makebox(0,0)[r]{\textcolor[rgb]{0.15,0.15,0.15}{{0}}}}
\fontsize{6}{0}
\selectfont\put(48.5205,57.269){\makebox(0,0)[r]{\textcolor[rgb]{0.15,0.15,0.15}{{0.5}}}}
\fontsize{6}{0}
\selectfont\put(48.5205,78.5322){\makebox(0,0)[r]{\textcolor[rgb]{0.15,0.15,0.15}{{1}}}}
\fontsize{6}{0}
\selectfont\put(48.5205,99.7959){\makebox(0,0)[r]{\textcolor[rgb]{0.15,0.15,0.15}{{1.5}}}}
\fontsize{7}{0}
\selectfont\put(33.5205,67.1875){\rotatebox{90}{\makebox(0,0)[b]{\textcolor[rgb]{0.15,0.15,0.15}{{Sinal de Controle $u_1(t)$}}}}}
\fontsize{7}{0}
\selectfont\put(116.643,11.2671){\makebox(0,0)[t]{\textcolor[rgb]{0.15,0.15,0.15}{{Tempo (s)}}}}
\fontsize{6}{0}
\selectfont\put(232.715,22.2671){\makebox(0,0)[t]{\textcolor[rgb]{0.15,0.15,0.15}{{0}}}}
\fontsize{6}{0}
\selectfont\put(251.151,22.2671){\makebox(0,0)[t]{\textcolor[rgb]{0.15,0.15,0.15}{{20}}}}
\fontsize{6}{0}
\selectfont\put(269.587,22.2671){\makebox(0,0)[t]{\textcolor[rgb]{0.15,0.15,0.15}{{40}}}}
\fontsize{6}{0}
\selectfont\put(288.024,22.2671){\makebox(0,0)[t]{\textcolor[rgb]{0.15,0.15,0.15}{{60}}}}
\fontsize{6}{0}
\selectfont\put(306.46,22.2671){\makebox(0,0)[t]{\textcolor[rgb]{0.15,0.15,0.15}{{80}}}}
\fontsize{6}{0}
\selectfont\put(324.896,22.2671){\makebox(0,0)[t]{\textcolor[rgb]{0.15,0.15,0.15}{{100}}}}
\fontsize{6}{0}
\selectfont\put(343.333,22.2671){\makebox(0,0)[t]{\textcolor[rgb]{0.15,0.15,0.15}{{120}}}}
\fontsize{6}{0}
\selectfont\put(361.77,22.2671){\makebox(0,0)[t]{\textcolor[rgb]{0.15,0.15,0.15}{{140}}}}
\fontsize{6}{0}
\selectfont\put(229.235,28.3999){\makebox(0,0)[r]{\textcolor[rgb]{0.15,0.15,0.15}{{-0.2}}}}
\fontsize{6}{0}
\selectfont\put(229.235,43.6162){\makebox(0,0)[r]{\textcolor[rgb]{0.15,0.15,0.15}{{0}}}}
\fontsize{6}{0}
\selectfont\put(229.235,58.833){\makebox(0,0)[r]{\textcolor[rgb]{0.15,0.15,0.15}{{0.2}}}}
\fontsize{6}{0}
\selectfont\put(229.235,74.0493){\makebox(0,0)[r]{\textcolor[rgb]{0.15,0.15,0.15}{{0.4}}}}
\fontsize{6}{0}
\selectfont\put(229.235,89.2661){\makebox(0,0)[r]{\textcolor[rgb]{0.15,0.15,0.15}{{0.6}}}}
\fontsize{6}{0}
\selectfont\put(229.235,104.482){\makebox(0,0)[r]{\textcolor[rgb]{0.15,0.15,0.15}{{0.8}}}}
\fontsize{7}{0}
\selectfont\put(212.235,67.1875){\rotatebox{90}{\makebox(0,0)[b]{\textcolor[rgb]{0.15,0.15,0.15}{{Sinal de Controle $u_2(t)$}}}}}
\fontsize{7}{0}
\selectfont\put(297.357,11.2671){\makebox(0,0)[t]{\textcolor[rgb]{0.15,0.15,0.15}{{Tempo (s)}}}}
\fontsize{6}{0}
\selectfont\put(128.001,242.023){\makebox(0,0)[l]{\textcolor[rgb]{0,0,0}{{referencia}}}}
\fontsize{6}{0}
\selectfont\put(185.003,242.023){\makebox(0,0)[l]{\textcolor[rgb]{0,0,0}{{sem acao integral}}}}
\fontsize{6}{0}
\selectfont\put(262.005,242.023){\makebox(0,0)[l]{\textcolor[rgb]{0,0,0}{{com acao integral}}}}
\end{picture}

    \end{minipage}
\end{figure}

Analisando o efeito \textit{anti-windup} na Figura
\ref{fig:desafio2:resultado-questao-9}, percebe-se que a resposta do sistema
alcança o erro nulo mais rápido que o sistema com saturação sem a ação
\textit{anti-windup}. No último gráfico da figura é mostrado a depuração da ação
\textit{anti-windup}. Verifica-se que a ação ocorre sempre que o sinal de
controle é saturado. Para o controlador em questão foi imposto o valor zero a
entrada da ação integradora do controlador sempre que $|u_{d}(t) - u(t)| >
10^{-5}$ em que $u_{d}(t)$ é o sinal de controle na saída do controlador I+P e
$u(t)$ é o sinal efeitvo aplicado a planta, que nesse caso pode estar saturado
ou não. A ação de zerar a parcela integral, analisando algebricamente a Equação
\ref{eq:desafio2:sinal-de-controle-i-mais-p-no-tempo}, permite inferir que o erro
não é mais acumulado por um longo período de tempo caso a saturação ocorra,
inibindo desta forma a lentidão do ínicio do seguimento de referencia da
resposta do sistema em malha fechada.

Por fim, as 4 respostas das 4 diferentes configurações foram inseridas no mesmo
gráfico a título de comparação, conforme Figura
\ref{fig:desafio2:resultado-questao-6-9}. Diante das respostas é possível
verificar que as configurações com filtro de referência e controlador I+P sem
saturação a primeira vista apresentam um melhor regime transitório. Entretanto,
é possível observar que o tempo de acomodação nestas duas configurações é
próximo para configuração com ação \textit{anti-windup}. É possível concluir
também que a depender do requisito de tempo de acomodação (5\%, 2\%, 1\% etc),
mesmo o sistema com a saturação do sinal de controle tem um tempo de acomodação
próximo aos demais. Não menos importante, nota-se que todas as configurações
rejeitaram a perturbação na entrada da planta. Isto acontece pois ambos os
controladores têm ação integral. Assim, de acordo com o Princípio do Modelo
Interno, as perturbações do tipo degrau serão sempre rejeitadas pois sempre
existirá a ação integradora no caminho de realimentação entre a saída aa planta
e a perturbação da entrada.

\begin{figure}[!ht]
    \caption{Simulação do sistema de controle em malhada fechada para diferentes
    configurações.}
    \vspace{-10pt}
    \hspace{-30pt}
    \label{fig:desafio2:resultado-questao-6-9}
    \begin{minipage}{\linewidth}
        % Title: gl2ps_renderer figure
% Creator: GL2PS 1.4.0, (C) 1999-2017 C. Geuzaine
% For: Octave
% CreationDate: Fri Oct  1 21:40:36 2021
\setlength{\unitlength}{1pt}
\begin{picture}(0,0)
\includegraphics{images/challenge2/resultado-questao-6-9-inc}
\end{picture}%
\begin{picture}(400,500)(0,0)
\fontsize{6}{0}
\selectfont\put(52,356.357){\makebox(0,0)[t]{\textcolor[rgb]{0.15,0.15,0.15}{{0}}}}
\fontsize{6}{0}
\selectfont\put(103.675,356.357){\makebox(0,0)[t]{\textcolor[rgb]{0.15,0.15,0.15}{{10}}}}
\fontsize{6}{0}
\selectfont\put(155.351,356.357){\makebox(0,0)[t]{\textcolor[rgb]{0.15,0.15,0.15}{{20}}}}
\fontsize{6}{0}
\selectfont\put(207.026,356.357){\makebox(0,0)[t]{\textcolor[rgb]{0.15,0.15,0.15}{{30}}}}
\fontsize{6}{0}
\selectfont\put(258.701,356.357){\makebox(0,0)[t]{\textcolor[rgb]{0.15,0.15,0.15}{{40}}}}
\fontsize{6}{0}
\selectfont\put(310.376,356.357){\makebox(0,0)[t]{\textcolor[rgb]{0.15,0.15,0.15}{{50}}}}
\fontsize{6}{0}
\selectfont\put(48.5278,374.133){\makebox(0,0)[r]{\textcolor[rgb]{0.15,0.15,0.15}{{0}}}}
\fontsize{6}{0}
\selectfont\put(48.5278,405.495){\makebox(0,0)[r]{\textcolor[rgb]{0.15,0.15,0.15}{{0.5}}}}
\fontsize{6}{0}
\selectfont\put(48.5278,436.856){\makebox(0,0)[r]{\textcolor[rgb]{0.15,0.15,0.15}{{1}}}}
\fontsize{7}{0}
\selectfont\put(33.5278,411.806){\rotatebox{90}{\makebox(0,0)[b]{\textcolor[rgb]{0.15,0.15,0.15}{{Saida $y(t)$}}}}}
\fontsize{7}{0}
\selectfont\put(207,345.357){\makebox(0,0)[t]{\textcolor[rgb]{0.15,0.15,0.15}{{Tempo (s)}}}}
\fontsize{6}{0}
\selectfont\put(263,436.09){\makebox(0,0)[l]{\textcolor[rgb]{0,0,0}{{$r_{f}(t)$}}}}
\fontsize{6}{0}
\selectfont\put(263,425.09){\makebox(0,0)[l]{\textcolor[rgb]{0,0,0}{{$r(t)$}}}}
\fontsize{6}{0}
\selectfont\put(263,413.589){\makebox(0,0)[l]{\textcolor[rgb]{0,0,0}{{$y_{m1}(t)$ com filtro de referência}}}}
\fontsize{6}{0}
\selectfont\put(263,401.589){\makebox(0,0)[l]{\textcolor[rgb]{0,0,0}{{$y_{m1}(t)$ com controlador I+P}}}}
\fontsize{6}{0}
\selectfont\put(263,389.589){\makebox(0,0)[l]{\textcolor[rgb]{0,0,0}{{$y_{m1}(t)$ com saturação de $u(t)$}}}}
\fontsize{6}{0}
\selectfont\put(263,377.589){\makebox(0,0)[l]{\textcolor[rgb]{0,0,0}{{$y_{m1}(t)$ com anti-windup}}}}
\fontsize{6}{0}
\selectfont\put(52,203.063){\makebox(0,0)[t]{\textcolor[rgb]{0.15,0.15,0.15}{{0}}}}
\fontsize{6}{0}
\selectfont\put(103.675,203.063){\makebox(0,0)[t]{\textcolor[rgb]{0.15,0.15,0.15}{{10}}}}
\fontsize{6}{0}
\selectfont\put(155.351,203.063){\makebox(0,0)[t]{\textcolor[rgb]{0.15,0.15,0.15}{{20}}}}
\fontsize{6}{0}
\selectfont\put(207.026,203.063){\makebox(0,0)[t]{\textcolor[rgb]{0.15,0.15,0.15}{{30}}}}
\fontsize{6}{0}
\selectfont\put(258.701,203.063){\makebox(0,0)[t]{\textcolor[rgb]{0.15,0.15,0.15}{{40}}}}
\fontsize{6}{0}
\selectfont\put(310.376,203.063){\makebox(0,0)[t]{\textcolor[rgb]{0.15,0.15,0.15}{{50}}}}
\fontsize{6}{0}
\selectfont\put(48.5278,234.411){\makebox(0,0)[r]{\textcolor[rgb]{0.15,0.15,0.15}{{0}}}}
\fontsize{6}{0}
\selectfont\put(48.5278,265.377){\makebox(0,0)[r]{\textcolor[rgb]{0.15,0.15,0.15}{{0.5}}}}
\fontsize{6}{0}
\selectfont\put(48.5278,296.344){\makebox(0,0)[r]{\textcolor[rgb]{0.15,0.15,0.15}{{1}}}}
\fontsize{7}{0}
\selectfont\put(33.5278,258.512){\rotatebox{90}{\makebox(0,0)[b]{\textcolor[rgb]{0.15,0.15,0.15}{{Erro $e(t)$}}}}}
\fontsize{7}{0}
\selectfont\put(207,192.063){\makebox(0,0)[t]{\textcolor[rgb]{0.15,0.15,0.15}{{Tempo (s)}}}}
\fontsize{6}{0}
\selectfont\put(267,258.795){\makebox(0,0)[l]{\textcolor[rgb]{0,0,0}{{$e_{m1}(t)$ com filtro de referência}}}}
\fontsize{6}{0}
\selectfont\put(267,247.795){\makebox(0,0)[l]{\textcolor[rgb]{0,0,0}{{$e_{m1}(t)$ com controlador I+P}}}}
\fontsize{6}{0}
\selectfont\put(267,236.295){\makebox(0,0)[l]{\textcolor[rgb]{0,0,0}{{$e_{m1}(t)$ com saturação}}}}
\fontsize{6}{0}
\selectfont\put(267,224.295){\makebox(0,0)[l]{\textcolor[rgb]{0,0,0}{{$e_{m1}(t)$ com anti-windup}}}}
\fontsize{6}{0}
\selectfont\put(52,49.769){\makebox(0,0)[t]{\textcolor[rgb]{0.15,0.15,0.15}{{0}}}}
\fontsize{6}{0}
\selectfont\put(103.675,49.769){\makebox(0,0)[t]{\textcolor[rgb]{0.15,0.15,0.15}{{10}}}}
\fontsize{6}{0}
\selectfont\put(155.351,49.769){\makebox(0,0)[t]{\textcolor[rgb]{0.15,0.15,0.15}{{20}}}}
\fontsize{6}{0}
\selectfont\put(207.026,49.769){\makebox(0,0)[t]{\textcolor[rgb]{0.15,0.15,0.15}{{30}}}}
\fontsize{6}{0}
\selectfont\put(258.701,49.769){\makebox(0,0)[t]{\textcolor[rgb]{0.15,0.15,0.15}{{40}}}}
\fontsize{6}{0}
\selectfont\put(310.376,49.769){\makebox(0,0)[t]{\textcolor[rgb]{0.15,0.15,0.15}{{50}}}}
\fontsize{6}{0}
\selectfont\put(48.5278,67.5547){\makebox(0,0)[r]{\textcolor[rgb]{0.15,0.15,0.15}{{0}}}}
\fontsize{6}{0}
\selectfont\put(48.5278,80.1089){\makebox(0,0)[r]{\textcolor[rgb]{0.15,0.15,0.15}{{0.2}}}}
\fontsize{6}{0}
\selectfont\put(48.5278,92.6636){\makebox(0,0)[r]{\textcolor[rgb]{0.15,0.15,0.15}{{0.4}}}}
\fontsize{6}{0}
\selectfont\put(48.5278,105.218){\makebox(0,0)[r]{\textcolor[rgb]{0.15,0.15,0.15}{{0.6}}}}
\fontsize{6}{0}
\selectfont\put(48.5278,117.772){\makebox(0,0)[r]{\textcolor[rgb]{0.15,0.15,0.15}{{0.8}}}}
\fontsize{6}{0}
\selectfont\put(48.5278,130.327){\makebox(0,0)[r]{\textcolor[rgb]{0.15,0.15,0.15}{{1}}}}
\fontsize{6}{0}
\selectfont\put(48.5278,142.882){\makebox(0,0)[r]{\textcolor[rgb]{0.15,0.15,0.15}{{1.2}}}}
\fontsize{7}{0}
\selectfont\put(33.5278,105.218){\rotatebox{90}{\makebox(0,0)[b]{\textcolor[rgb]{0.15,0.15,0.15}{{Sinal de Controle $u(t)$}}}}}
\fontsize{7}{0}
\selectfont\put(207,38.769){\makebox(0,0)[t]{\textcolor[rgb]{0.15,0.15,0.15}{{Tempo (s)}}}}
\fontsize{6}{0}
\selectfont\put(267,105.501){\makebox(0,0)[l]{\textcolor[rgb]{0,0,0}{{$u_{m1}(t)$ com filtro de referência}}}}
\fontsize{6}{0}
\selectfont\put(267,94.501){\makebox(0,0)[l]{\textcolor[rgb]{0,0,0}{{$u_{m1}(t)$ com controlador I+P}}}}
\fontsize{6}{0}
\selectfont\put(267,83.001){\makebox(0,0)[l]{\textcolor[rgb]{0,0,0}{{$u_{m1}(t)$ com saturação}}}}
\fontsize{6}{0}
\selectfont\put(267,71.0005){\makebox(0,0)[l]{\textcolor[rgb]{0,0,0}{{$u_{m1}(t)$ com anti-windup}}}}
\end{picture}

    \end{minipage}
\end{figure}

\subsection{Conclusões}
Os resultados apresentados nas subseções acima permitem concluir que o sistema
em malha fechada é estável para os 4 modelos da planta real que foram obtidos em
diferentes pontos de operação. Esta conclusão foi alcançada pois o controlador
projetado utilizando o modelo nominal, obtido da média paramétrica dos 4
modelos, não violou a regra do Teorema do Pequeno Ganho. Esta afirmação é
proveniente tanto da análise da resposta em frequência quanto da simulação de um
dos modelos em diferentes configurações do sistema.

Portanto, é notório a importância da aplicabilidade da análise da resposta em
frequência da planta para analisar sua robustez e estabilidade.
