Neste cenário a função de transferência $G(s)$ da planta difere dos três
anteriores e não mais possui um integrador. Este cenário também não impõe
dinâmica para o filtro de referência $F(s)$. Entretanto, ainda é mantido um
integrador no controlador $C(s)$. Desta forma, devido ao Princípio do Modelo
Interno, já é possível concluir que haverá seguimento de referência e rejeição a
perturbações na entrada e saída para sinais do tipo degrau, já que aparece um
polo na origem no caminho de realimentação para todas as 3 funções de
transferência. Partindo desta conclusão, para entradas do tipo degrau, basta
definir os parâmetros do controlador para que os requisitos de regime
transitório sejam satisfeitos.

Todavia, este cenário apresenta um novo tipo de perturbação: senoidal. Também
com ajuda do Princípio do Modelo Interno pode-se concluir que o controlador
utilizado nos dois últimos cenários não será suficiente para rejeitar
perturbações deste tipo. Assim, neste cenário é analisado a resposta do sistema
em malha fechada para perturbações do tipo senoidal dado um controlador cuja os
polos da função de transferência são iguais aos dos sinais de perturbação. Pelo
princípio do modelo interno já se conclui que as perturbações senoidais tanto na
entrada quanto na saída serão rejeitadas.

Começa-se a análise do cenário considerando

\begin{equation}
    \label{eq:gdes-cenario4}
    G(s) = \frac{2}{s + 1,5}, 
\end{equation}
e
\begin{equation}
    \label{eq:cdes-cenario4}
    C(s) = K\frac{s + z}{s}.
\end{equation}

Definindo $z = 1,5$, temos que

\begin{equation}
    \label{eq:y2r-comk-cenario4}
    \frac{Y(s)}{R(s)} = \frac{2K}{s + 2K}.
\end{equation}
que trata-se de uma função de transferência de primeira ordem em que $\tau =
\frac{1}{2K}$. Assim, para $t_{s_{2\%}} = 4\tau = 2s$, chega-se a $K = 1$.
Portanto

\begin{equation}
    \label{eq:y2r-solucionado-cenario4-a-b}
    \frac{Y(s)}{R(s)} = \frac{2}{s + 2},
\end{equation}

\begin{equation}
    \label{eq:y2qy-solucionado-cenario4-a-b}
    \frac{Y(s)}{Q_{y}(s)} = \frac{s}{s + 2},
\end{equation}

\begin{equation}
    \label{eq:y2qu-solucionado-cenario4-a-b}
    \frac{Y(s)}{Q_{u}(s)} = \frac{2s}{s^2 + 3,5s + 3}.
\end{equation}

Aplicando as entradas $r(t) = \mathds{1}(t - 2)$, $q_{y}(t) = -0,2\mathds{1}(t -
15)$ e $q_{u}(t) = -0,2\mathds{1}(t - 25)$ ao sistema de malha fechada
caracterizado pelas funções de transferência das Equações
\ref{eq:y2r-solucionado-cenario4-a-b}, \ref{eq:y2qy-solucionado-cenario4-a-b} e
\ref{eq:y2qu-solucionado-cenario4-a-b}, obtemos os resultados conforme Figura
\ref{fig:resultado-cenario4-a}.

\begin{figure}[!ht]
    \caption{Simulação do sistema de controle em malha fechada dada as funções
    de transferência das Equações
    \ref{eq:y2r-solucionado-cenario4-a-b}, \ref{eq:y2qy-solucionado-cenario4-a-b} e
    \ref{eq:y2qu-solucionado-cenario4-a-b} para perturbações do tipo degrau.}
    \vspace{-10pt}
    \hspace{-30pt}
    \label{fig:resultado-cenario4-a}
    \begin{minipage}{\linewidth}
        \input{images/challenge1/resultado-cenario4-a.tex}
    \end{minipage}
\end{figure}

Como esperado, nota-se na Figura \ref{fig:resultado-cenario4-a} que
houve rejeição de perturbações e seguimento de referência. Simulou-se também o
mesmo sistema em malha fechada para as perturbações $q_{y}(t) =
0,2sen(2t)\mathds{1}(t - 20)$ e $q_{u}(t) = 0,2sen(2t)\mathds{1}(t - 40)$. O
resultado obtido é ilustrado na Figura \ref{fig:resultado-cenario4-b}.
Como também esperado, já que nem o controlador nem a planta possuem polos em $s
= \pm j2$, o sistema não rejeita perturbações do tipo senoidal.

\begin{figure}[!ht]
    \caption{Simulação do sistema de controle em malha fechada dada as funções
    de transferência das Equações
    \ref{eq:y2r-solucionado-cenario4-a-b}, \ref{eq:y2qy-solucionado-cenario4-a-b} e
    \ref{eq:y2qu-solucionado-cenario4-a-b} para perturbações do tipo senoidal.}
    \vspace{-10pt}
    \hspace{-30pt}
    \label{fig:resultado-cenario4-b}
    \begin{minipage}{\linewidth}
        \input{images/challenge1/resultado-cenario4-b.tex}
    \end{minipage}
\end{figure}

Com o objetivo então de fazer com que o sistema com realimentação negativa
rejeite perturbações persistentes do tipo senoidal, modificou-se o tal que a
função de transferência dele é dada pela Equação
\ref{eq:controlador-com-polos-imaginarios} e função de transferência em malha
aberta do sistema dada pela Equação
\ref{eq:ft-malha-aberta-com-controlador-complexo}.

\begin{equation}
    \label{eq:controlador-com-polos-imaginarios}
    C(s) = K\left(\frac{s + z}{s}\right)\left(\frac{s^2 + 0,5s + 1.8^2}{s^2 + 2^2}\right).
\end{equation}

\begin{equation}
    \label{eq:ft-malha-aberta-com-controlador-complexo}
    C(s)G(s) = K\left(\frac{s + z}{s}\right)
    \left(\frac{s^2 + 0,5s + 1.8^2}{s^2 + 2^2}\right)
    \left(\frac{2}{s+1,5}\right).
\end{equation}

Para a definição do ganho $K$ foi utilizado o traçado \textit{Root Locus}. Após
alguns traçados realizados com a ajuda da função \textit{rlocus} do Octave,
percebeu-se que os polos $s = 0$ e $s = 1.5$ em malha aberta convergem
rapidamente para os zeros complexos conjugados do controlador $C(s)$ a medida
que o ganho $K$ aumenta. Partindo desse ponto, escolheu-se $z$ tal que a
convergência mencionada fosse rápida, mas também que a resposta do sistema em
malha fechada ao degrau não tivesse um \textit{overshoot} tão alto. Depois de
alguns testes, verificou-se que alocar $z$ 10 vezes mais afastado que o polo da
planta oferecia uma resposta ao degrau satisfatória para um valor de ganho tal
que satisfizesse o critério de tempo de acomodação $t_{s_{2\%}} = 2s$. Este
ganho foi obtido considerando o cancelamento dos zeros complexos conjugados do
controlador. Desta forma, foi feita uma aproximação da função em malha fechada
para uma função de segunda ordem e, então, utilizou-se a fórmula $t_{s_{2\%}} =
\frac{4}{\xi w_{n}}$ para chegar a um valor de ganho $K = 1,8$. Assim chegou-se
as seguintes funções de transferência:

\begin{equation}
    \label{eq:y2r-solucionado-cenario4-c}
    \frac{Y(s)}{R(s)} = \frac{3,6s^3 + 55,8s^2 + 38,66s + 175}
    {s^4 + 5,1s^3 + 59,8s^2 + 44,66s + 175},
\end{equation}

\begin{equation}
    \label{eq:y2qy-solucionado-cenario4-c}
    \frac{Y(s)}{Q_{y}(s)} = \frac{s(s^3 + 1.5s^2 + 4s + 6)}
    {s^4 + 5,1s^3 + 59,8s^2 + 44,66s + 175},
\end{equation}

\begin{equation}
    \label{eq:y2qu-solucionado-cenario4-c}
    \frac{Y(s)}{Q_{u}(s)} = \frac{ 2s(s^2 + 4)}
    {s^4 + 5,1s^3 + 59,8s^2 + 44,66s + 175}.
\end{equation}

O resultado da simulação é demonstrado na Figura
\ref{fig:resultado-cenario4-c}. Observa-se que o resultado obtido está
de acordo com o Princípio do Modelo Interno, pois os polos complexos inseridos
no controlador com a mesma frequência dos sinais periódicos das perturbações
foram responsáveis por as rejeitarem e, então, o erro em regime estacionário
para $y(t)$ em relação a $r(t)$ foi nulo. Entretanto, vale salientar que a
sintonia do controlador impôs uma outra característica na resposta do sistema de
malha fechada. Os zeros inseridos fizeram aumentar o \textit{overshoot} do sistema
impondo uma resposta muito agressiva ao sinal de controle $u(t)$. Para alguns
cenários, como o controle de vazão de um processo industrial por uma válvula
servo atuada, este tipo de sinal de controle pode trazer danos mecânicos ao
longo prazo, decorrendo em maior custos com manunenteções e paradas do processo
se for o caso.

\begin{figure}[!ht]
    \caption{Simulação do sistema de controle em malha fechada dada as funções
    de transferência das Equações
    \ref{eq:y2r-solucionado-cenario4-c}, \ref{eq:y2qy-solucionado-cenario4-c} e
    \ref{eq:y2qu-solucionado-cenario4-c} para perturbações do tipo senoidal.}
    \vspace{-10pt}
    \hspace{-30pt}
    \label{fig:resultado-cenario4-c}
    \begin{minipage}{\linewidth}
        % Title: gl2ps_renderer figure
% Creator: GL2PS 1.4.0, (C) 1999-2017 C. Geuzaine
% For: Octave
% CreationDate: Sat Sep 18 21:09:06 2021
\setlength{\unitlength}{1pt}
\begin{picture}(0,0)
\includegraphics{images/challenge1/resultado-cenario4-c-inc}
\end{picture}%
\begin{picture}(400,250)(0,0)
\fontsize{6}{0}
\selectfont\put(52,141.404){\makebox(0,0)[t]{\textcolor[rgb]{0.15,0.15,0.15}{{0}}}}
\fontsize{6}{0}
\selectfont\put(73.5474,141.404){\makebox(0,0)[t]{\textcolor[rgb]{0.15,0.15,0.15}{{10}}}}
\fontsize{6}{0}
\selectfont\put(95.0952,141.404){\makebox(0,0)[t]{\textcolor[rgb]{0.15,0.15,0.15}{{20}}}}
\fontsize{6}{0}
\selectfont\put(116.643,141.404){\makebox(0,0)[t]{\textcolor[rgb]{0.15,0.15,0.15}{{30}}}}
\fontsize{6}{0}
\selectfont\put(138.19,141.404){\makebox(0,0)[t]{\textcolor[rgb]{0.15,0.15,0.15}{{40}}}}
\fontsize{6}{0}
\selectfont\put(159.738,141.404){\makebox(0,0)[t]{\textcolor[rgb]{0.15,0.15,0.15}{{50}}}}
\fontsize{6}{0}
\selectfont\put(181.285,141.404){\makebox(0,0)[t]{\textcolor[rgb]{0.15,0.15,0.15}{{60}}}}
\fontsize{6}{0}
\selectfont\put(48.5205,155.488){\makebox(0,0)[r]{\textcolor[rgb]{0.15,0.15,0.15}{{0}}}}
\fontsize{6}{0}
\selectfont\put(48.5205,177.614){\makebox(0,0)[r]{\textcolor[rgb]{0.15,0.15,0.15}{{0.5}}}}
\fontsize{6}{0}
\selectfont\put(48.5205,199.741){\makebox(0,0)[r]{\textcolor[rgb]{0.15,0.15,0.15}{{1}}}}
\fontsize{6}{0}
\selectfont\put(48.5205,221.868){\makebox(0,0)[r]{\textcolor[rgb]{0.15,0.15,0.15}{{1.5}}}}
\fontsize{7}{0}
\selectfont\put(33.5205,186.325){\rotatebox{90}{\makebox(0,0)[b]{\textcolor[rgb]{0.15,0.15,0.15}{{Saída $y(t)$}}}}}
\fontsize{7}{0}
\selectfont\put(116.643,130.404){\makebox(0,0)[t]{\textcolor[rgb]{0.15,0.15,0.15}{{Tempo (s)}}}}
\fontsize{6}{0}
\selectfont\put(160.285,174.138){\makebox(0,0)[l]{\textcolor[rgb]{0,0,0}{{r(t)}}}}
\fontsize{6}{0}
\selectfont\put(160.285,162.638){\makebox(0,0)[l]{\textcolor[rgb]{0,0,0}{{y(t)}}}}
\fontsize{6}{0}
\selectfont\put(52,22.2671){\makebox(0,0)[t]{\textcolor[rgb]{0.15,0.15,0.15}{{0}}}}
\fontsize{6}{0}
\selectfont\put(73.5474,22.2671){\makebox(0,0)[t]{\textcolor[rgb]{0.15,0.15,0.15}{{10}}}}
\fontsize{6}{0}
\selectfont\put(95.0952,22.2671){\makebox(0,0)[t]{\textcolor[rgb]{0.15,0.15,0.15}{{20}}}}
\fontsize{6}{0}
\selectfont\put(116.643,22.2671){\makebox(0,0)[t]{\textcolor[rgb]{0.15,0.15,0.15}{{30}}}}
\fontsize{6}{0}
\selectfont\put(138.19,22.2671){\makebox(0,0)[t]{\textcolor[rgb]{0.15,0.15,0.15}{{40}}}}
\fontsize{6}{0}
\selectfont\put(159.738,22.2671){\makebox(0,0)[t]{\textcolor[rgb]{0.15,0.15,0.15}{{50}}}}
\fontsize{6}{0}
\selectfont\put(181.285,22.2671){\makebox(0,0)[t]{\textcolor[rgb]{0.15,0.15,0.15}{{60}}}}
\fontsize{6}{0}
\selectfont\put(48.5205,35.3989){\makebox(0,0)[r]{\textcolor[rgb]{0.15,0.15,0.15}{{0}}}}
\fontsize{6}{0}
\selectfont\put(48.5205,56.2959){\makebox(0,0)[r]{\textcolor[rgb]{0.15,0.15,0.15}{{1}}}}
\fontsize{6}{0}
\selectfont\put(48.5205,77.1929){\makebox(0,0)[r]{\textcolor[rgb]{0.15,0.15,0.15}{{2}}}}
\fontsize{6}{0}
\selectfont\put(48.5205,98.0903){\makebox(0,0)[r]{\textcolor[rgb]{0.15,0.15,0.15}{{3}}}}
\fontsize{7}{0}
\selectfont\put(39.5205,67.1875){\rotatebox{90}{\makebox(0,0)[b]{\textcolor[rgb]{0.15,0.15,0.15}{{Sinal de Controle $u(t)$}}}}}
\fontsize{7}{0}
\selectfont\put(116.643,11.2671){\makebox(0,0)[t]{\textcolor[rgb]{0.15,0.15,0.15}{{Tempo (s)}}}}
\fontsize{6}{0}
\selectfont\put(232.715,141.404){\makebox(0,0)[t]{\textcolor[rgb]{0.15,0.15,0.15}{{0}}}}
\fontsize{6}{0}
\selectfont\put(254.262,141.404){\makebox(0,0)[t]{\textcolor[rgb]{0.15,0.15,0.15}{{10}}}}
\fontsize{6}{0}
\selectfont\put(275.81,141.404){\makebox(0,0)[t]{\textcolor[rgb]{0.15,0.15,0.15}{{20}}}}
\fontsize{6}{0}
\selectfont\put(297.357,141.404){\makebox(0,0)[t]{\textcolor[rgb]{0.15,0.15,0.15}{{30}}}}
\fontsize{6}{0}
\selectfont\put(318.905,141.404){\makebox(0,0)[t]{\textcolor[rgb]{0.15,0.15,0.15}{{40}}}}
\fontsize{6}{0}
\selectfont\put(340.453,141.404){\makebox(0,0)[t]{\textcolor[rgb]{0.15,0.15,0.15}{{50}}}}
\fontsize{6}{0}
\selectfont\put(362,141.404){\makebox(0,0)[t]{\textcolor[rgb]{0.15,0.15,0.15}{{60}}}}
\fontsize{6}{0}
\selectfont\put(229.235,166.481){\makebox(0,0)[r]{\textcolor[rgb]{0.15,0.15,0.15}{{-0.2}}}}
\fontsize{6}{0}
\selectfont\put(229.235,186.325){\makebox(0,0)[r]{\textcolor[rgb]{0.15,0.15,0.15}{{0}}}}
\fontsize{6}{0}
\selectfont\put(229.235,206.168){\makebox(0,0)[r]{\textcolor[rgb]{0.15,0.15,0.15}{{0.2}}}}
\fontsize{7}{0}
\selectfont\put(212.235,186.325){\rotatebox{90}{\makebox(0,0)[b]{\textcolor[rgb]{0.15,0.15,0.15}{{Perturbação na Saída $q_{y}(t)$}}}}}
\fontsize{7}{0}
\selectfont\put(297.357,130.404){\makebox(0,0)[t]{\textcolor[rgb]{0.15,0.15,0.15}{{Tempo (s)}}}}
\fontsize{6}{0}
\selectfont\put(232.715,22.2671){\makebox(0,0)[t]{\textcolor[rgb]{0.15,0.15,0.15}{{0}}}}
\fontsize{6}{0}
\selectfont\put(254.262,22.2671){\makebox(0,0)[t]{\textcolor[rgb]{0.15,0.15,0.15}{{10}}}}
\fontsize{6}{0}
\selectfont\put(275.81,22.2671){\makebox(0,0)[t]{\textcolor[rgb]{0.15,0.15,0.15}{{20}}}}
\fontsize{6}{0}
\selectfont\put(297.357,22.2671){\makebox(0,0)[t]{\textcolor[rgb]{0.15,0.15,0.15}{{30}}}}
\fontsize{6}{0}
\selectfont\put(318.905,22.2671){\makebox(0,0)[t]{\textcolor[rgb]{0.15,0.15,0.15}{{40}}}}
\fontsize{6}{0}
\selectfont\put(340.453,22.2671){\makebox(0,0)[t]{\textcolor[rgb]{0.15,0.15,0.15}{{50}}}}
\fontsize{6}{0}
\selectfont\put(362,22.2671){\makebox(0,0)[t]{\textcolor[rgb]{0.15,0.15,0.15}{{60}}}}
\fontsize{6}{0}
\selectfont\put(229.235,47.3438){\makebox(0,0)[r]{\textcolor[rgb]{0.15,0.15,0.15}{{-0.2}}}}
\fontsize{6}{0}
\selectfont\put(229.235,67.1875){\makebox(0,0)[r]{\textcolor[rgb]{0.15,0.15,0.15}{{0}}}}
\fontsize{6}{0}
\selectfont\put(229.235,87.0312){\makebox(0,0)[r]{\textcolor[rgb]{0.15,0.15,0.15}{{0.2}}}}
\fontsize{7}{0}
\selectfont\put(212.235,67.1875){\rotatebox{90}{\makebox(0,0)[b]{\textcolor[rgb]{0.15,0.15,0.15}{{Perturbação na Entrada $q_{u}(t)$}}}}}
\fontsize{7}{0}
\selectfont\put(297.357,11.2671){\makebox(0,0)[t]{\textcolor[rgb]{0.15,0.15,0.15}{{Tempo (s)}}}}
\end{picture}

    \end{minipage}
\end{figure}