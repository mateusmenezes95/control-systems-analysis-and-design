Para a definição do ganho $K$ foi utilizado o traçado \textit{Root Locus}. Após
alguns traçados realizados com a ajuda da função \textit{rlocus} do Octave,
percebeu-se que os polos $s = 0$ e $s = 1.5$ em malha aberta convergem
rapidamente para os zeros complexos conjugados do controlador $C$ a medida que o
ganho $K$ aumenta. Partindo desse ponto, escolheu-se $z$ tal que a convergência
mencionada fosse rápida, mas também que a resposta do sistema em malha fechada
ao degrau não tivesse um \textit{overshoot} tão alto. Depois de alguns testes,
verificou-se que alocar $z$ 10 vezes mais afastado que o polo da planta oferecia
uma resposta ao degrau satisfatória para um valor de ganho tal que satisfizesse
o critério de tempo de acomodação $t_{s_{2\%}} = 2s$. Este ganho foi obtido
considerando o cancelamento dos zeros complexos conjugados do controlador. Desta
forma, foi feita uma aproximação da função em malha fechada para uma função de
segunda ordem e, então, utilizou-se a fórmula $t_{s_{2\%}} = \frac{4}{\xi
w_{n}}$ para chegar a um valor de ganho $K = 1,8$.