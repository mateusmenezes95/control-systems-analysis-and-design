Para $G(s)$, $C(s)$ e $F(s)$ dados, temos que

\begin{equation}
    \label{eq:y2r-comk-cenario1}
    \frac{Y(s)}{R(s)} = \frac{2K}{s + 2K}
\end{equation}

Como trata-se de um sistema de primeira ordem, temos também que $t_{s_{2\%}} =
4\tau$. Aplicando esta equação em \ref{eq:y2r-comk-cenario1}, obtemos $K = 0.5$.
Assim chegamos as seguintes funções de transferência:

\begin{equation}
    \label{eq:y2r-cenario1}
    \frac{Y(s)}{R(s)} = \frac{1}{s + 1},
\end{equation}

\begin{equation}
    \label{eq:y2qy-cenario1}
    \frac{Y(s)}{Q_{y}(s)} = \frac{s}{s + 1},
\end{equation}

\begin{equation}
    \label{eq:y2qu-cenario1}
    \frac{Y(s)}{Q_{u}(s)} = \frac{2}{s + 1}.
\end{equation}

Partindo da equação \ref{eq:saida-do-sistema} obteve-se a saída $y(t)$ para as
entradas $r(t) = \mathds{1}(t - 2)$, $q_{y}(t) = -0,2\mathds{1}(t - 15)$
$q_{u}(t) = -0,2\mathds{1}(t - 25)$. O mesmo princípio e propriedade, Equação
\ref{eq:sinal-de-controle}, foi usado para obter o sinal de controle $u(t)$
dadas as entradas deslocadas no tempo. O resultado obtido da simulação para o
cenário 1 pode ser visualizado na Figura \ref{fig:resultado-desafio1-cenario1}.

\begin{figure}[!ht]
    \caption{Simulação do sistema de controle em malhada fechada para condições
    impostas no Cenário 1.}
    \vspace{-10pt}
    \hspace{-30pt}
    \label{fig:resultado-desafio1-cenario1}
    \begin{minipage}{\linewidth}
        % Title: gl2ps_renderer figure
% Creator: GL2PS 1.4.0, (C) 1999-2017 C. Geuzaine
% For: Octave
% CreationDate: Sun Sep 12 03:32:01 2021
\setlength{\unitlength}{1pt}
\begin{picture}(0,0)
\includegraphics{images/resultado-desafio1-cenario1-inc}
\end{picture}%
\begin{picture}(400,250)(0,0)
\fontsize{6}{0}
\selectfont\put(52,141.404){\makebox(0,0)[t]{\textcolor[rgb]{0.15,0.15,0.15}{{0}}}}
\fontsize{6}{0}
\selectfont\put(77.8569,141.404){\makebox(0,0)[t]{\textcolor[rgb]{0.15,0.15,0.15}{{10}}}}
\fontsize{6}{0}
\selectfont\put(103.714,141.404){\makebox(0,0)[t]{\textcolor[rgb]{0.15,0.15,0.15}{{20}}}}
\fontsize{6}{0}
\selectfont\put(129.571,141.404){\makebox(0,0)[t]{\textcolor[rgb]{0.15,0.15,0.15}{{30}}}}
\fontsize{6}{0}
\selectfont\put(155.428,141.404){\makebox(0,0)[t]{\textcolor[rgb]{0.15,0.15,0.15}{{40}}}}
\fontsize{6}{0}
\selectfont\put(181.285,141.404){\makebox(0,0)[t]{\textcolor[rgb]{0.15,0.15,0.15}{{50}}}}
\fontsize{6}{0}
\selectfont\put(48.5205,146.637){\makebox(0,0)[r]{\textcolor[rgb]{0.15,0.15,0.15}{{-0.2}}}}
\fontsize{6}{0}
\selectfont\put(48.5205,157.977){\makebox(0,0)[r]{\textcolor[rgb]{0.15,0.15,0.15}{{0}}}}
\fontsize{6}{0}
\selectfont\put(48.5205,169.316){\makebox(0,0)[r]{\textcolor[rgb]{0.15,0.15,0.15}{{0.2}}}}
\fontsize{6}{0}
\selectfont\put(48.5205,180.655){\makebox(0,0)[r]{\textcolor[rgb]{0.15,0.15,0.15}{{0.4}}}}
\fontsize{6}{0}
\selectfont\put(48.5205,191.995){\makebox(0,0)[r]{\textcolor[rgb]{0.15,0.15,0.15}{{0.6}}}}
\fontsize{6}{0}
\selectfont\put(48.5205,203.333){\makebox(0,0)[r]{\textcolor[rgb]{0.15,0.15,0.15}{{0.8}}}}
\fontsize{6}{0}
\selectfont\put(48.5205,214.673){\makebox(0,0)[r]{\textcolor[rgb]{0.15,0.15,0.15}{{1}}}}
\fontsize{7}{0}
\selectfont\put(31.5205,186.325){\rotatebox{90}{\makebox(0,0)[b]{\textcolor[rgb]{0.15,0.15,0.15}{{Saída}}}}}
\fontsize{7}{0}
\selectfont\put(116.643,130.404){\makebox(0,0)[t]{\textcolor[rgb]{0.15,0.15,0.15}{{Tempo (s)}}}}
\fontsize{6}{0}
\selectfont\put(160.285,174.138){\makebox(0,0)[l]{\textcolor[rgb]{0,0,0}{{r(t)}}}}
\fontsize{6}{0}
\selectfont\put(160.285,162.638){\makebox(0,0)[l]{\textcolor[rgb]{0,0,0}{{y(t)}}}}
\fontsize{6}{0}
\selectfont\put(52,22.2671){\makebox(0,0)[t]{\textcolor[rgb]{0.15,0.15,0.15}{{0}}}}
\fontsize{6}{0}
\selectfont\put(77.8569,22.2671){\makebox(0,0)[t]{\textcolor[rgb]{0.15,0.15,0.15}{{10}}}}
\fontsize{6}{0}
\selectfont\put(103.714,22.2671){\makebox(0,0)[t]{\textcolor[rgb]{0.15,0.15,0.15}{{20}}}}
\fontsize{6}{0}
\selectfont\put(129.571,22.2671){\makebox(0,0)[t]{\textcolor[rgb]{0.15,0.15,0.15}{{30}}}}
\fontsize{6}{0}
\selectfont\put(155.428,22.2671){\makebox(0,0)[t]{\textcolor[rgb]{0.15,0.15,0.15}{{40}}}}
\fontsize{6}{0}
\selectfont\put(181.285,22.2671){\makebox(0,0)[t]{\textcolor[rgb]{0.15,0.15,0.15}{{50}}}}
\fontsize{6}{0}
\selectfont\put(48.5205,27.5){\makebox(0,0)[r]{\textcolor[rgb]{0.15,0.15,0.15}{{-0.2}}}}
\fontsize{6}{0}
\selectfont\put(48.5205,45.188){\makebox(0,0)[r]{\textcolor[rgb]{0.15,0.15,0.15}{{0}}}}
\fontsize{6}{0}
\selectfont\put(48.5205,62.8755){\makebox(0,0)[r]{\textcolor[rgb]{0.15,0.15,0.15}{{0.2}}}}
\fontsize{6}{0}
\selectfont\put(48.5205,80.5635){\makebox(0,0)[r]{\textcolor[rgb]{0.15,0.15,0.15}{{0.4}}}}
\fontsize{6}{0}
\selectfont\put(48.5205,98.2515){\makebox(0,0)[r]{\textcolor[rgb]{0.15,0.15,0.15}{{0.6}}}}
\fontsize{7}{0}
\selectfont\put(31.5205,67.1875){\rotatebox{90}{\makebox(0,0)[b]{\textcolor[rgb]{0.15,0.15,0.15}{{Sinal de Controle}}}}}
\fontsize{7}{0}
\selectfont\put(116.643,11.2671){\makebox(0,0)[t]{\textcolor[rgb]{0.15,0.15,0.15}{{Tempo (s)}}}}
\fontsize{6}{0}
\selectfont\put(232.715,141.404){\makebox(0,0)[t]{\textcolor[rgb]{0.15,0.15,0.15}{{0}}}}
\fontsize{6}{0}
\selectfont\put(258.572,141.404){\makebox(0,0)[t]{\textcolor[rgb]{0.15,0.15,0.15}{{10}}}}
\fontsize{6}{0}
\selectfont\put(284.429,141.404){\makebox(0,0)[t]{\textcolor[rgb]{0.15,0.15,0.15}{{20}}}}
\fontsize{6}{0}
\selectfont\put(310.286,141.404){\makebox(0,0)[t]{\textcolor[rgb]{0.15,0.15,0.15}{{30}}}}
\fontsize{6}{0}
\selectfont\put(336.143,141.404){\makebox(0,0)[t]{\textcolor[rgb]{0.15,0.15,0.15}{{40}}}}
\fontsize{6}{0}
\selectfont\put(362,141.404){\makebox(0,0)[t]{\textcolor[rgb]{0.15,0.15,0.15}{{50}}}}
\fontsize{6}{0}
\selectfont\put(229.235,146.637){\makebox(0,0)[r]{\textcolor[rgb]{0.15,0.15,0.15}{{-0.4}}}}
\fontsize{6}{0}
\selectfont\put(229.235,159.866){\makebox(0,0)[r]{\textcolor[rgb]{0.15,0.15,0.15}{{-0.3}}}}
\fontsize{6}{0}
\selectfont\put(229.235,173.096){\makebox(0,0)[r]{\textcolor[rgb]{0.15,0.15,0.15}{{-0.2}}}}
\fontsize{6}{0}
\selectfont\put(229.235,186.325){\makebox(0,0)[r]{\textcolor[rgb]{0.15,0.15,0.15}{{-0.1}}}}
\fontsize{6}{0}
\selectfont\put(229.235,199.554){\makebox(0,0)[r]{\textcolor[rgb]{0.15,0.15,0.15}{{0}}}}
\fontsize{6}{0}
\selectfont\put(229.235,212.783){\makebox(0,0)[r]{\textcolor[rgb]{0.15,0.15,0.15}{{0.1}}}}
\fontsize{6}{0}
\selectfont\put(229.235,226.012){\makebox(0,0)[r]{\textcolor[rgb]{0.15,0.15,0.15}{{0.2}}}}
\fontsize{7}{0}
\selectfont\put(212.235,186.325){\rotatebox{90}{\makebox(0,0)[b]{\textcolor[rgb]{0.15,0.15,0.15}{{Perturbação - Saída}}}}}
\fontsize{7}{0}
\selectfont\put(297.357,130.404){\makebox(0,0)[t]{\textcolor[rgb]{0.15,0.15,0.15}{{Tempo (s)}}}}
\fontsize{6}{0}
\selectfont\put(232.715,22.2671){\makebox(0,0)[t]{\textcolor[rgb]{0.15,0.15,0.15}{{0}}}}
\fontsize{6}{0}
\selectfont\put(258.572,22.2671){\makebox(0,0)[t]{\textcolor[rgb]{0.15,0.15,0.15}{{10}}}}
\fontsize{6}{0}
\selectfont\put(284.429,22.2671){\makebox(0,0)[t]{\textcolor[rgb]{0.15,0.15,0.15}{{20}}}}
\fontsize{6}{0}
\selectfont\put(310.286,22.2671){\makebox(0,0)[t]{\textcolor[rgb]{0.15,0.15,0.15}{{30}}}}
\fontsize{6}{0}
\selectfont\put(336.143,22.2671){\makebox(0,0)[t]{\textcolor[rgb]{0.15,0.15,0.15}{{40}}}}
\fontsize{6}{0}
\selectfont\put(362,22.2671){\makebox(0,0)[t]{\textcolor[rgb]{0.15,0.15,0.15}{{50}}}}
\fontsize{6}{0}
\selectfont\put(229.235,27.5){\makebox(0,0)[r]{\textcolor[rgb]{0.15,0.15,0.15}{{-0.4}}}}
\fontsize{6}{0}
\selectfont\put(229.235,40.729){\makebox(0,0)[r]{\textcolor[rgb]{0.15,0.15,0.15}{{-0.3}}}}
\fontsize{6}{0}
\selectfont\put(229.235,53.9585){\makebox(0,0)[r]{\textcolor[rgb]{0.15,0.15,0.15}{{-0.2}}}}
\fontsize{6}{0}
\selectfont\put(229.235,67.1875){\makebox(0,0)[r]{\textcolor[rgb]{0.15,0.15,0.15}{{-0.1}}}}
\fontsize{6}{0}
\selectfont\put(229.235,80.4165){\makebox(0,0)[r]{\textcolor[rgb]{0.15,0.15,0.15}{{0}}}}
\fontsize{6}{0}
\selectfont\put(229.235,93.646){\makebox(0,0)[r]{\textcolor[rgb]{0.15,0.15,0.15}{{0.1}}}}
\fontsize{6}{0}
\selectfont\put(229.235,106.875){\makebox(0,0)[r]{\textcolor[rgb]{0.15,0.15,0.15}{{0.2}}}}
\fontsize{7}{0}
\selectfont\put(212.235,67.1875){\rotatebox{90}{\makebox(0,0)[b]{\textcolor[rgb]{0.15,0.15,0.15}{{Perturbação - Entrada}}}}}
\fontsize{7}{0}
\selectfont\put(297.357,11.2671){\makebox(0,0)[t]{\textcolor[rgb]{0.15,0.15,0.15}{{Tempo (s)}}}}
\end{picture}

    \end{minipage}
\end{figure}

É possível concluir pela Figura \ref{fig:resultado-desafio1-cenario1} que o
controlador permitiu que a saída seguisse a referência. Observa-se também que o
sistema rejeitou a perturbação na saída da malha de controle, o que é coerente
já que seguimento de referência é equivalente a rejeição na saída como é
demonstrado nas Equações \ref{eq:y2r-tf} e \ref{eq:e2r-tf}. Entretanto, o
sistema não conseguiu rejeitar a perturbação na entrada. Estas conclusões
podem ser analisadas de forma conceitual pelo princípio do modelo interno. Como
a planta possui um polo na origem, que é o mesmo da perturbação de saída, a
perturbação é rejeitada. Porém, para a rejeição da perturbação na entrada é
preciso que ou o zero da planta ou o polo do controlador coincidam com o(s)
polo(s) da perturbações de entrada, o que não acontece para este cenário e,
portanto, não é possível rejeitar a perturbação de entrada.

Para enfatizar a conclusão acima, é aplicado o teorema do valor final para a
Equação \ref{eq:y2qu-cenario1} conforme Equação \ref{eq:limite-y2qu}.
Constata-se então um valor não nulo na saída $y(t)$ dado uma perturbação na
entrada $q_{u}(t) = -0,2\mathds{1}(t - 25)$.

\begin{equation}
    \label{eq:limite-y2qu}
    \lim_{s \rightarrow 0}s\left ( \frac{2}{s + 1}\frac{-0,2e^{-25s}}{s} \right ) = -0,4
\end{equation}