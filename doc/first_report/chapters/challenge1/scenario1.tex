Para $G(s)$, $C(s)$ e $F(s)$ dados, temos que

\begin{equation}
    \label{eq:y2r-comk-cenario1}
    \frac{Y(s)}{R(s)} = \frac{2K}{s + 2K}
\end{equation}

Como trata-se de um sistema de primeira ordem, temos também que $t_{s_{2\%}} =
4\tau$. Aplicando esta equação em \ref{eq:y2r-comk-cenario1}, obtemos $K = 0.5$.
Assim chegamos as seguintes funções de transferência:

\begin{equation}
    \label{eq:y2r-cenario1}
    \frac{Y(s)}{R(s)} = \frac{1}{s + 1},
\end{equation}

\begin{equation}
    \label{eq:y2qy-cenario1}
    \frac{Y(s)}{Q_{y}(s)} = \frac{s}{s + 1},
\end{equation}

\begin{equation}
    \label{eq:y2qu-cenario1}
    \frac{Y(s)}{Q_{u}(s)} = \frac{2}{s + 1}.
\end{equation}

A partir do princípio da superposição e da propriedade de linearidade pode se
obter a saída total do sistema através das somas das respostas individuais da
referência e das perturbações, isto é,

\begin{equation}
    \label{eq:saida-do-sistema}
    y(t) = y_{r}(t) + y_{q_{y}}(t) + y_{q_{u}}(t).
\end{equation}

Partindo da equação \ref{eq:saida-do-sistema} obteve-se a saída $y(t)$ para as
entradas $r(t) = \mathds{1}(t - 2)$, $q_{y}(t) = -0,2\mathds{1}(t - 15)$
$q_{u}(t) = -0,2\mathds{1}(t - 25)$. O mesmo princípio e propriedade foi usado
para obter o sinal de controle $u(t)$ dadas as entradas deslocadas no tempo. O
resultado obtido da simulação para o cenário 1 pode ser visualizado na Figura 
\ref{fig:resultado-desafio1-cenario1}.

\begin{figure}[!ht]
    \caption{Simulação do sistema de controle em malhada fechada para condições
    impostas no Cenário 1.}
    \vspace{-10pt}
    \hspace{-30pt}
    \label{fig:resultado-desafio1-cenario1}
    \begin{minipage}{\linewidth}
        \input{images/resultado-desafio1-cenario1.tex}
    \end{minipage}
\end{figure}

É possível concluir pela Figura \ref{fig:resultado-desafio1-cenario1} que o
controlador permitiu que a saída seguisse a referência. Observa-se também que o
sistema rejeitou a perturbação na saída da malha de controle, o que é coerente
já que seguimento de referência é equivalente a rejeição na saída como é
demonstrado nas Equações \ref{eq:y2r-tf} e \ref{eq:e2r-tf}. Entretanto, o
sistema não conseguiu rejeitar uma perturbação na entrada. Essas conclusões
podem ser analisadas de forma conceitual pelo princípio do modelo interno. Como
a planta possui um polo na origem, que é o mesmo da perturbação de saída, a
perturbação é rejeitada. Porém, para a rejeição da perturbação na entrada é
preciso que ou o zero da planta ou o polo do controlador coincidam com o(s)
polo(s) da perturbações de entrada, o que não acontece para este cenário e,
portanto, não é possível rejeitar a perturbação de entrada.

Para enfatizar a conclusão acima, é aplicado o teorema do valor final para a
Equação \ref{eq:y2qu-cenario1} conforme Equação \ref{eq:limite-y2qu}.
Constata-se então um valor não nulo na saída $y(t)$ dado uma perturbação na
entrada $q_{u}(t) = -0,2\mathds{1}(t - 25)$.

\begin{equation}
    \label{eq:limite-y2qu}
    \lim_{s \rightarrow 0}s\left ( \frac{2}{s + 1}\frac{-0,2e^{-25s}}{s} \right ) = -0,4
\end{equation}