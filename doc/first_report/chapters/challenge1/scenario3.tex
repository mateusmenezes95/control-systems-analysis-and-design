Este cenário utiliza os mesmos parâmetros do controlador, planta e filtro de
referência definidos no Cenário 2 (subseção \ref{subsub:cenario2}). Portanto, a
função de transferência $\frac{Y(s)}{R(s)}$ está definida na Equação
\ref{eq:y2r-comkz-cenario2}, que reescrita se torna 

\begin{equation}
    \label{eq:y2r-comkz-cenario3}
    \frac{Y(s)}{R(s)} = F(s)\frac{2K(s + z)}{s^2 + 2K(s + z)}.
\end{equation}

Considerando

\begin{equation}
    \label{eq:y2r-com-polos-iguais-cenario3}
    \frac{Y'(s)}{R'(s)} = \frac{1}{(\tau s + 1)^2}
\end{equation}
temos que para o tempo de acomodação $t_{s_{2\%}} \approx 6\tau$. Para
$t_{s_{2\%}} = 2s$, isto resulta em $\tau = 0.5$. Desta forma, a função de
transferência em malha fechada se torna

\begin{equation}
    \label{eq:y2r-semkz-cenario3}
    \frac{Y'(s)}{R'(s)} = \frac{1}{s^2 + 4s + 4}.
\end{equation}

Igualando \ref{eq:y2r-semkz-cenario3} a \ref{eq:y2r-comkz-cenario3}, obtém-se $K
= 2$ e $z = 1$. Para eliminar o zero da Equação \ref{eq:y2r-comkz-cenario3}, e
garantir que ele não interfira na resposta transitória do sistema em malha
fechada, opta-se se por definir $\tau_{d} = 1$ e $\tau_{n} = 0$ em $F(s)$. Assim,
temos que

\begin{equation}
    \label{eq:y2r-solucionado-cenario3}
    \frac{Y(s)}{R(s)} = \frac{4}{(s + 2)^2},
\end{equation}

\begin{equation}
    \label{eq:y2qy-solucionado-cenario3}
    \frac{Y(s)}{Q_{y}(s)} = \frac{s^2}{(s + 2)^2},
\end{equation}

\begin{equation}
    \label{eq:y2qu-solucionado-cenario3}
    \frac{Y(s)}{Q_{u}(s)} = \frac{2s}{(s + 2)^2}.
\end{equation}

A partir das equações acima, chegou-se ao resultado demonstrado no gráfico
superior esquerdo da Figura \ref{fig:resultado-desafio1-cenario3}.

\begin{figure}[!ht]
    \caption{Simulação do sistema de controle em malhada fechada para condições
    impostas no Cenário 2 para perturbações do tipo rampa.}
    \vspace{-10pt}
    \hspace{-30pt}
    \label{fig:resultado-desafio1-cenario3}
    \begin{minipage}{\linewidth}
        \input{images/resultado-desafio1-cenario3.tex}
    \end{minipage}
\end{figure}

Observa-se um comportamento semelhante ao exposto na subseção
\ref{subsub:cenario2} para entradas do tipo rampa. Como esperado, não houve
rejeição de perturbação na entrada da planta. Isto continua pois no caminho de
realimentação da função de transferência $\frac{Y(s)}{Q_{u}(s)}$ não há um
duplo integrador para eliminar a perturbação do tipo rampa. Isto é evidente já
que o controlador do Cenaŕio 2 para o Cenário 3 não foi modificado. Entretanto,
vale salientar que o erro em regime permanente diminuiu por um fator igual a
$4$ se comparado ao cenário anterior. A dimimuição se deu pois os valores de $K$
e $z$ dobraram em relação ao Cenário 2. Tal conclusão só pode ser evidenciada
através de simulação e/ou usando o Teorema do Valor Final.
