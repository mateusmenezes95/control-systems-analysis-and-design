Este cenário utiliza os mesmos parâmetros do controlador, planta e filtro de
referência definidos no Cenário 2 (subseção \ref{subsub:cenario2}). Portanto, a
função de transferência $\frac{Y(s)}{R(s)}$ está definida na Equação
\ref{eq:y2r-comkz-cenario2}, que reescrita se torna 

\begin{equation}
    \label{eq:y2r-comkz-cenario3}
    \frac{Y(s)}{R(s)} = F(s)\frac{2K(s + z)}{s^2 + 2K(s + z)}.
\end{equation}

Considerando

\begin{equation}
    \label{eq:y2r-com-polos-iguais-cenario3}
    \frac{Y'(s)}{R'(s)} = \frac{1}{(\tau s + 1)^2}
\end{equation}
temos que para o tempo de acomodação $t_{s_{2\%}} \approx 6\tau$. Para
$t_{s_{2\%}} = 2s$, isto resulta em $\tau = 0.5$. Desta forma, a função de
transferência em malha fechada se torna

\begin{equation}
    \label{eq:y2r-semkz-cenario3}
    \frac{Y'(s)}{R'(s)} = \frac{1}{s^2 + 4s + 4}.
\end{equation}

Igualando \ref{eq:y2r-semkz-cenario3} a \ref{eq:y2r-comkz-cenario3}, obtém-se $K
= 2$ e $z = 1$. Para eliminar o zero da Equação \ref{eq:y2r-comkz-cenario3}, e
garantir que ele não interfira na resposta transitória do sistema em malha
fechada, opta-se se por definir $\tau_{d} = 1$ e $\tau_{n} = 0$ em $F(s)$. Assim,
temos que

\begin{equation}
    \label{eq:y2r-solucionado-cenario3}
    \frac{Y(s)}{R(s)} = \frac{4}{(s + 2)^2},
\end{equation}

\begin{equation}
    \label{eq:y2qy-solucionado-cenario3}
    \frac{Y(s)}{Q_{y}(s)} = \frac{s^2}{(s + 2)^2},
\end{equation}

\begin{equation}
    \label{eq:y2qu-solucionado-cenario3}
    \frac{Y(s)}{Q_{u}(s)} = \frac{2s}{(s + 2)^2}.
\end{equation}

A partir das equações acima, chegou-se ao resultado demonstrado no gráfico
superior esquerdo da Figura \ref{fig:resultado-cenario3}.

\begin{figure}[!ht]
    \caption{Simulação do sistema de controle em malhada fechada para condições
    impostas no Cenário 2 para perturbações do tipo rampa.}
    \vspace{-10pt}
    \hspace{-30pt}
    \label{fig:resultado-cenario3}
    \begin{minipage}{\linewidth}
        % Title: gl2ps_renderer figure
% Creator: GL2PS 1.4.0, (C) 1999-2017 C. Geuzaine
% For: Octave
% CreationDate: Sat Sep 18 18:27:49 2021
\setlength{\unitlength}{1pt}
\begin{picture}(0,0)
\includegraphics{images/challenge1/resultado-cenario3-inc}
\end{picture}%
\begin{picture}(400,250)(0,0)
\fontsize{6}{0}
\selectfont\put(52,141.404){\makebox(0,0)[t]{\textcolor[rgb]{0.15,0.15,0.15}{{0}}}}
\fontsize{6}{0}
\selectfont\put(77.8569,141.404){\makebox(0,0)[t]{\textcolor[rgb]{0.15,0.15,0.15}{{10}}}}
\fontsize{6}{0}
\selectfont\put(103.714,141.404){\makebox(0,0)[t]{\textcolor[rgb]{0.15,0.15,0.15}{{20}}}}
\fontsize{6}{0}
\selectfont\put(129.571,141.404){\makebox(0,0)[t]{\textcolor[rgb]{0.15,0.15,0.15}{{30}}}}
\fontsize{6}{0}
\selectfont\put(155.428,141.404){\makebox(0,0)[t]{\textcolor[rgb]{0.15,0.15,0.15}{{40}}}}
\fontsize{6}{0}
\selectfont\put(181.285,141.404){\makebox(0,0)[t]{\textcolor[rgb]{0.15,0.15,0.15}{{50}}}}
\fontsize{6}{0}
\selectfont\put(48.5205,146.637){\makebox(0,0)[r]{\textcolor[rgb]{0.15,0.15,0.15}{{-0.2}}}}
\fontsize{6}{0}
\selectfont\put(48.5205,157.977){\makebox(0,0)[r]{\textcolor[rgb]{0.15,0.15,0.15}{{0}}}}
\fontsize{6}{0}
\selectfont\put(48.5205,169.316){\makebox(0,0)[r]{\textcolor[rgb]{0.15,0.15,0.15}{{0.2}}}}
\fontsize{6}{0}
\selectfont\put(48.5205,180.655){\makebox(0,0)[r]{\textcolor[rgb]{0.15,0.15,0.15}{{0.4}}}}
\fontsize{6}{0}
\selectfont\put(48.5205,191.994){\makebox(0,0)[r]{\textcolor[rgb]{0.15,0.15,0.15}{{0.6}}}}
\fontsize{6}{0}
\selectfont\put(48.5205,203.333){\makebox(0,0)[r]{\textcolor[rgb]{0.15,0.15,0.15}{{0.8}}}}
\fontsize{6}{0}
\selectfont\put(48.5205,214.673){\makebox(0,0)[r]{\textcolor[rgb]{0.15,0.15,0.15}{{1}}}}
\fontsize{7}{0}
\selectfont\put(31.5205,186.325){\rotatebox{90}{\makebox(0,0)[b]{\textcolor[rgb]{0.15,0.15,0.15}{{Saída $y(t)$}}}}}
\fontsize{7}{0}
\selectfont\put(116.643,130.404){\makebox(0,0)[t]{\textcolor[rgb]{0.15,0.15,0.15}{{Tempo (s)}}}}
\fontsize{6}{0}
\selectfont\put(160.285,174.138){\makebox(0,0)[l]{\textcolor[rgb]{0,0,0}{{r(t)}}}}
\fontsize{6}{0}
\selectfont\put(160.285,162.638){\makebox(0,0)[l]{\textcolor[rgb]{0,0,0}{{y(t)}}}}
\fontsize{6}{0}
\selectfont\put(52,22.2671){\makebox(0,0)[t]{\textcolor[rgb]{0.15,0.15,0.15}{{0}}}}
\fontsize{6}{0}
\selectfont\put(77.8569,22.2671){\makebox(0,0)[t]{\textcolor[rgb]{0.15,0.15,0.15}{{10}}}}
\fontsize{6}{0}
\selectfont\put(103.714,22.2671){\makebox(0,0)[t]{\textcolor[rgb]{0.15,0.15,0.15}{{20}}}}
\fontsize{6}{0}
\selectfont\put(129.571,22.2671){\makebox(0,0)[t]{\textcolor[rgb]{0.15,0.15,0.15}{{30}}}}
\fontsize{6}{0}
\selectfont\put(155.428,22.2671){\makebox(0,0)[t]{\textcolor[rgb]{0.15,0.15,0.15}{{40}}}}
\fontsize{6}{0}
\selectfont\put(181.285,22.2671){\makebox(0,0)[t]{\textcolor[rgb]{0.15,0.15,0.15}{{50}}}}
\fontsize{6}{0}
\selectfont\put(48.5205,30.3862){\makebox(0,0)[r]{\textcolor[rgb]{0.15,0.15,0.15}{{0}}}}
\fontsize{6}{0}
\selectfont\put(48.5205,44.8184){\makebox(0,0)[r]{\textcolor[rgb]{0.15,0.15,0.15}{{1}}}}
\fontsize{6}{0}
\selectfont\put(48.5205,59.25){\makebox(0,0)[r]{\textcolor[rgb]{0.15,0.15,0.15}{{2}}}}
\fontsize{6}{0}
\selectfont\put(48.5205,73.6816){\makebox(0,0)[r]{\textcolor[rgb]{0.15,0.15,0.15}{{3}}}}
\fontsize{6}{0}
\selectfont\put(48.5205,88.1138){\makebox(0,0)[r]{\textcolor[rgb]{0.15,0.15,0.15}{{4}}}}
\fontsize{6}{0}
\selectfont\put(48.5205,102.545){\makebox(0,0)[r]{\textcolor[rgb]{0.15,0.15,0.15}{{5}}}}
\fontsize{7}{0}
\selectfont\put(39.5205,67.1875){\rotatebox{90}{\makebox(0,0)[b]{\textcolor[rgb]{0.15,0.15,0.15}{{Sinal de Controle $u(t)$}}}}}
\fontsize{7}{0}
\selectfont\put(116.643,11.2671){\makebox(0,0)[t]{\textcolor[rgb]{0.15,0.15,0.15}{{Tempo (s)}}}}
\fontsize{6}{0}
\selectfont\put(232.715,141.404){\makebox(0,0)[t]{\textcolor[rgb]{0.15,0.15,0.15}{{0}}}}
\fontsize{6}{0}
\selectfont\put(258.572,141.404){\makebox(0,0)[t]{\textcolor[rgb]{0.15,0.15,0.15}{{10}}}}
\fontsize{6}{0}
\selectfont\put(284.429,141.404){\makebox(0,0)[t]{\textcolor[rgb]{0.15,0.15,0.15}{{20}}}}
\fontsize{6}{0}
\selectfont\put(310.286,141.404){\makebox(0,0)[t]{\textcolor[rgb]{0.15,0.15,0.15}{{30}}}}
\fontsize{6}{0}
\selectfont\put(336.143,141.404){\makebox(0,0)[t]{\textcolor[rgb]{0.15,0.15,0.15}{{40}}}}
\fontsize{6}{0}
\selectfont\put(362,141.404){\makebox(0,0)[t]{\textcolor[rgb]{0.15,0.15,0.15}{{50}}}}
\fontsize{6}{0}
\selectfont\put(229.235,159.509){\makebox(0,0)[r]{\textcolor[rgb]{0.15,0.15,0.15}{{-6}}}}
\fontsize{6}{0}
\selectfont\put(229.235,180.961){\makebox(0,0)[r]{\textcolor[rgb]{0.15,0.15,0.15}{{-4}}}}
\fontsize{6}{0}
\selectfont\put(229.235,202.414){\makebox(0,0)[r]{\textcolor[rgb]{0.15,0.15,0.15}{{-2}}}}
\fontsize{6}{0}
\selectfont\put(229.235,223.867){\makebox(0,0)[r]{\textcolor[rgb]{0.15,0.15,0.15}{{0}}}}
\fontsize{7}{0}
\selectfont\put(218.235,186.325){\rotatebox{90}{\makebox(0,0)[b]{\textcolor[rgb]{0.15,0.15,0.15}{{Perturbação na Saída $q_{y}(t)$}}}}}
\fontsize{7}{0}
\selectfont\put(297.357,130.404){\makebox(0,0)[t]{\textcolor[rgb]{0.15,0.15,0.15}{{Tempo (s)}}}}
\fontsize{6}{0}
\selectfont\put(232.715,22.2671){\makebox(0,0)[t]{\textcolor[rgb]{0.15,0.15,0.15}{{0}}}}
\fontsize{6}{0}
\selectfont\put(258.572,22.2671){\makebox(0,0)[t]{\textcolor[rgb]{0.15,0.15,0.15}{{10}}}}
\fontsize{6}{0}
\selectfont\put(284.429,22.2671){\makebox(0,0)[t]{\textcolor[rgb]{0.15,0.15,0.15}{{20}}}}
\fontsize{6}{0}
\selectfont\put(310.286,22.2671){\makebox(0,0)[t]{\textcolor[rgb]{0.15,0.15,0.15}{{30}}}}
\fontsize{6}{0}
\selectfont\put(336.143,22.2671){\makebox(0,0)[t]{\textcolor[rgb]{0.15,0.15,0.15}{{40}}}}
\fontsize{6}{0}
\selectfont\put(362,22.2671){\makebox(0,0)[t]{\textcolor[rgb]{0.15,0.15,0.15}{{50}}}}
\fontsize{6}{0}
\selectfont\put(229.235,30.4399){\makebox(0,0)[r]{\textcolor[rgb]{0.15,0.15,0.15}{{-5}}}}
\fontsize{6}{0}
\selectfont\put(229.235,45.1387){\makebox(0,0)[r]{\textcolor[rgb]{0.15,0.15,0.15}{{-4}}}}
\fontsize{6}{0}
\selectfont\put(229.235,59.8379){\makebox(0,0)[r]{\textcolor[rgb]{0.15,0.15,0.15}{{-3}}}}
\fontsize{6}{0}
\selectfont\put(229.235,74.5371){\makebox(0,0)[r]{\textcolor[rgb]{0.15,0.15,0.15}{{-2}}}}
\fontsize{6}{0}
\selectfont\put(229.235,89.2363){\makebox(0,0)[r]{\textcolor[rgb]{0.15,0.15,0.15}{{-1}}}}
\fontsize{6}{0}
\selectfont\put(229.235,103.935){\makebox(0,0)[r]{\textcolor[rgb]{0.15,0.15,0.15}{{0}}}}
\fontsize{7}{0}
\selectfont\put(218.235,67.1875){\rotatebox{90}{\makebox(0,0)[b]{\textcolor[rgb]{0.15,0.15,0.15}{{Perturbação na Entrada $q_{u}(t)$}}}}}
\fontsize{7}{0}
\selectfont\put(297.357,11.2671){\makebox(0,0)[t]{\textcolor[rgb]{0.15,0.15,0.15}{{Tempo (s)}}}}
\end{picture}

    \end{minipage}
\end{figure}

Observa-se um comportamento semelhante ao exposto na subseção
\ref{subsub:cenario2} para entradas do tipo rampa. Como esperado, não houve
rejeição de perturbação na entrada da planta. Isto continua pois no caminho de
realimentação da função de transferência $\frac{Y(s)}{Q_{u}(s)}$ não há um
duplo integrador para eliminar a perturbação do tipo rampa. Isto é evidente já
que o controlador do Cenaŕio 2 para o Cenário 3 não foi modificado. Entretanto,
vale salientar que o erro em regime permanente diminuiu por um fator igual a
$4$ se comparado ao cenário anterior. A dimimuição se deu pois os valores de $K$
e $z$ dobraram em relação ao Cenário 2. Tal conclusão só pode ser evidenciada
através de simulação e/ou usando o Teorema do Valor Final.
