Partindo de uma uma abordagem similar a realizada para o Cenário 1 conforme
subseção \ref{subsub:cenario1}, para $G(s)$, $C(s)$ e $F(s)$ dados, temos que

\begin{equation}
    \label{eq:y2r-comkz-cenario2}
    \frac{Y(s)}{R(s)} = F(s)\frac{2Ks + 2z}{s^2 + 2Ks + 2Kz}.
\end{equation}

Como requisito do problema de controle, foi imposto que as raízes em malha
fechada sejam dadas por $s^* = -1$. Igualando o denominador da Equação
\ref{eq:y2r-comkz-cenario2} ao polinônio $s^2 + 2s + 1$, chega-se aos valores
$K = 1$ e $z = 0,5$. Assim, a Equação \ref{eq:y2r-comkz-cenario2} torna-se 
a equação

\begin{equation}
    \label{eq:y2r-semkz-cenario2}
    \frac{Y(s)}{R(s)} = F(s)\frac{2s + 1}{s^2 + 2s + 1}
    = F(s)\frac{2s + 1}{(s+1)^2}.
\end{equation}

Observa-se na Equação \ref{eq:y2r-semkz-cenario2} que ela possui um zero em
$-0,5$, enquanto os dois polos estão localizados em $-1$. Este zero está na
região dominante e irá interferir significativamente na resposta transitória do
sistema em malha fechada. Portanto, para $F(s) = \frac{\tau_{n}s + 1}{\tau_{d}s
+ 1}$ e um requisito de $t_{s_{2\%}} = 6s$, definiu-se os zeros e polos do
filtro de referência $F(s)$ de modo que o zero dominante seja eliminado do
sistema em malha fechada e também que o requisito de acomodação seja satisfeito.
Assim, $\tau_{n} = 0$ e $\tau_{d} = 2$, resultando na Equação
\ref{eq:y2r-comfiltro-cenario2}.

\begin{equation}
    \label{eq:y2r-comfiltro-cenario2}
    \frac{Y(s)}{R(s)} = \frac{1}{(s+1)^2}.
\end{equation}

Constata-se na Equação \ref{eq:y2r-comfiltro-cenario2} que $\tau = 1$. Para um
sistema com duas raízes no SPE iguais, $t_{s_{2\%}} \approx 6\tau$ e, portanto,
o requisito de tempo de acomodação foi satisfeito.

Com $C(s)$ e $F(s)$ definidos, as funções de transferência das perturbações foram
calculadas conforme equações:

\begin{equation}
    \label{eq:y2qy-cenario2}
    \frac{Y(s)}{Q_{y}(s)} = \frac{s^2}{(s + 1)^2},
\end{equation}

\begin{equation}
    \label{eq:y2qu-cenario2}
    \frac{Y(s)}{Q_{u}(s)} = \frac{2s}{(s + 1)^2}.
\end{equation}

A partir do mesmo princípio e propriedade apresentado no Cenário 1, chegou-se a
saída do sistema em malha fechada $y(t)$ dadas as entradas
$r(t) = \mathds{1}(t - 2)$, $q_{y}(t) = -0,2\mathds{1}(t - 15)$ e
$q_{u}(t) = -0,2\mathds{1}(t - 25)$, iguais as usadas no Cenário 1. O resultado
da simulação pode ser analisado na Figura \ref{fig:resultado-cenario2-a}.

\begin{figure}[!ht]
    \caption{Simulação do sistema de controle em malhada fechada para condições
    impostas no Cenário 2 para perturbações do tipo degrau.}
    \vspace{-10pt}
    \hspace{-30pt}
    \label{fig:resultado-cenario2-a}
    \begin{minipage}{\linewidth}
        % Title: gl2ps_renderer figure
% Creator: GL2PS 1.4.0, (C) 1999-2017 C. Geuzaine
% For: Octave
% CreationDate: Sun Sep 12 21:00:59 2021
\setlength{\unitlength}{1pt}
\begin{picture}(0,0)
\includegraphics{images/challenge1/resultado-cenario2-a-inc}
\end{picture}%
\begin{picture}(400,250)(0,0)
\fontsize{6}{0}
\selectfont\put(52,141.404){\makebox(0,0)[t]{\textcolor[rgb]{0.15,0.15,0.15}{{0}}}}
\fontsize{6}{0}
\selectfont\put(77.8569,141.404){\makebox(0,0)[t]{\textcolor[rgb]{0.15,0.15,0.15}{{10}}}}
\fontsize{6}{0}
\selectfont\put(103.714,141.404){\makebox(0,0)[t]{\textcolor[rgb]{0.15,0.15,0.15}{{20}}}}
\fontsize{6}{0}
\selectfont\put(129.571,141.404){\makebox(0,0)[t]{\textcolor[rgb]{0.15,0.15,0.15}{{30}}}}
\fontsize{6}{0}
\selectfont\put(155.428,141.404){\makebox(0,0)[t]{\textcolor[rgb]{0.15,0.15,0.15}{{40}}}}
\fontsize{6}{0}
\selectfont\put(181.285,141.404){\makebox(0,0)[t]{\textcolor[rgb]{0.15,0.15,0.15}{{50}}}}
\fontsize{6}{0}
\selectfont\put(48.5205,146.637){\makebox(0,0)[r]{\textcolor[rgb]{0.15,0.15,0.15}{{-0.2}}}}
\fontsize{6}{0}
\selectfont\put(48.5205,157.761){\makebox(0,0)[r]{\textcolor[rgb]{0.15,0.15,0.15}{{0}}}}
\fontsize{6}{0}
\selectfont\put(48.5205,168.886){\makebox(0,0)[r]{\textcolor[rgb]{0.15,0.15,0.15}{{0.2}}}}
\fontsize{6}{0}
\selectfont\put(48.5205,180.01){\makebox(0,0)[r]{\textcolor[rgb]{0.15,0.15,0.15}{{0.4}}}}
\fontsize{6}{0}
\selectfont\put(48.5205,191.134){\makebox(0,0)[r]{\textcolor[rgb]{0.15,0.15,0.15}{{0.6}}}}
\fontsize{6}{0}
\selectfont\put(48.5205,202.258){\makebox(0,0)[r]{\textcolor[rgb]{0.15,0.15,0.15}{{0.8}}}}
\fontsize{6}{0}
\selectfont\put(48.5205,213.383){\makebox(0,0)[r]{\textcolor[rgb]{0.15,0.15,0.15}{{1}}}}
\fontsize{6}{0}
\selectfont\put(48.5205,224.507){\makebox(0,0)[r]{\textcolor[rgb]{0.15,0.15,0.15}{{1.2}}}}
\fontsize{7}{0}
\selectfont\put(31.5205,186.325){\rotatebox{90}{\makebox(0,0)[b]{\textcolor[rgb]{0.15,0.15,0.15}{{Saída $y(t)$}}}}}
\fontsize{7}{0}
\selectfont\put(116.643,130.404){\makebox(0,0)[t]{\textcolor[rgb]{0.15,0.15,0.15}{{Tempo (s)}}}}
\fontsize{6}{0}
\selectfont\put(160.285,174.138){\makebox(0,0)[l]{\textcolor[rgb]{0,0,0}{{r(t)}}}}
\fontsize{6}{0}
\selectfont\put(160.285,162.638){\makebox(0,0)[l]{\textcolor[rgb]{0,0,0}{{y(t)}}}}
\fontsize{6}{0}
\selectfont\put(52,22.2671){\makebox(0,0)[t]{\textcolor[rgb]{0.15,0.15,0.15}{{0}}}}
\fontsize{6}{0}
\selectfont\put(77.8569,22.2671){\makebox(0,0)[t]{\textcolor[rgb]{0.15,0.15,0.15}{{10}}}}
\fontsize{6}{0}
\selectfont\put(103.714,22.2671){\makebox(0,0)[t]{\textcolor[rgb]{0.15,0.15,0.15}{{20}}}}
\fontsize{6}{0}
\selectfont\put(129.571,22.2671){\makebox(0,0)[t]{\textcolor[rgb]{0.15,0.15,0.15}{{30}}}}
\fontsize{6}{0}
\selectfont\put(155.428,22.2671){\makebox(0,0)[t]{\textcolor[rgb]{0.15,0.15,0.15}{{40}}}}
\fontsize{6}{0}
\selectfont\put(181.285,22.2671){\makebox(0,0)[t]{\textcolor[rgb]{0.15,0.15,0.15}{{50}}}}
\fontsize{6}{0}
\selectfont\put(48.5205,28.125){\makebox(0,0)[r]{\textcolor[rgb]{0.15,0.15,0.15}{{-0.2}}}}
\fontsize{6}{0}
\selectfont\put(48.5205,40.6836){\makebox(0,0)[r]{\textcolor[rgb]{0.15,0.15,0.15}{{-0.1}}}}
\fontsize{6}{0}
\selectfont\put(48.5205,53.2422){\makebox(0,0)[r]{\textcolor[rgb]{0.15,0.15,0.15}{{0}}}}
\fontsize{6}{0}
\selectfont\put(48.5205,65.8008){\makebox(0,0)[r]{\textcolor[rgb]{0.15,0.15,0.15}{{0.1}}}}
\fontsize{6}{0}
\selectfont\put(48.5205,78.3594){\makebox(0,0)[r]{\textcolor[rgb]{0.15,0.15,0.15}{{0.2}}}}
\fontsize{6}{0}
\selectfont\put(48.5205,90.918){\makebox(0,0)[r]{\textcolor[rgb]{0.15,0.15,0.15}{{0.3}}}}
\fontsize{6}{0}
\selectfont\put(48.5205,103.477){\makebox(0,0)[r]{\textcolor[rgb]{0.15,0.15,0.15}{{0.4}}}}
\fontsize{7}{0}
\selectfont\put(31.5205,67.1875){\rotatebox{90}{\makebox(0,0)[b]{\textcolor[rgb]{0.15,0.15,0.15}{{Sinal de Controle $u(t)$}}}}}
\fontsize{7}{0}
\selectfont\put(116.643,11.2671){\makebox(0,0)[t]{\textcolor[rgb]{0.15,0.15,0.15}{{Tempo (s)}}}}
\fontsize{6}{0}
\selectfont\put(232.715,141.404){\makebox(0,0)[t]{\textcolor[rgb]{0.15,0.15,0.15}{{0}}}}
\fontsize{6}{0}
\selectfont\put(258.572,141.404){\makebox(0,0)[t]{\textcolor[rgb]{0.15,0.15,0.15}{{10}}}}
\fontsize{6}{0}
\selectfont\put(284.429,141.404){\makebox(0,0)[t]{\textcolor[rgb]{0.15,0.15,0.15}{{20}}}}
\fontsize{6}{0}
\selectfont\put(310.286,141.404){\makebox(0,0)[t]{\textcolor[rgb]{0.15,0.15,0.15}{{30}}}}
\fontsize{6}{0}
\selectfont\put(336.143,141.404){\makebox(0,0)[t]{\textcolor[rgb]{0.15,0.15,0.15}{{40}}}}
\fontsize{6}{0}
\selectfont\put(362,141.404){\makebox(0,0)[t]{\textcolor[rgb]{0.15,0.15,0.15}{{50}}}}
\fontsize{6}{0}
\selectfont\put(229.235,146.637){\makebox(0,0)[r]{\textcolor[rgb]{0.15,0.15,0.15}{{-0.4}}}}
\fontsize{6}{0}
\selectfont\put(229.235,159.866){\makebox(0,0)[r]{\textcolor[rgb]{0.15,0.15,0.15}{{-0.3}}}}
\fontsize{6}{0}
\selectfont\put(229.235,173.096){\makebox(0,0)[r]{\textcolor[rgb]{0.15,0.15,0.15}{{-0.2}}}}
\fontsize{6}{0}
\selectfont\put(229.235,186.325){\makebox(0,0)[r]{\textcolor[rgb]{0.15,0.15,0.15}{{-0.1}}}}
\fontsize{6}{0}
\selectfont\put(229.235,199.554){\makebox(0,0)[r]{\textcolor[rgb]{0.15,0.15,0.15}{{0}}}}
\fontsize{6}{0}
\selectfont\put(229.235,212.783){\makebox(0,0)[r]{\textcolor[rgb]{0.15,0.15,0.15}{{0.1}}}}
\fontsize{6}{0}
\selectfont\put(229.235,226.012){\makebox(0,0)[r]{\textcolor[rgb]{0.15,0.15,0.15}{{0.2}}}}
\fontsize{7}{0}
\selectfont\put(212.235,186.325){\rotatebox{90}{\makebox(0,0)[b]{\textcolor[rgb]{0.15,0.15,0.15}{{Perturbação na Saída $q_{y}(t)$}}}}}
\fontsize{7}{0}
\selectfont\put(297.357,130.404){\makebox(0,0)[t]{\textcolor[rgb]{0.15,0.15,0.15}{{Tempo (s)}}}}
\fontsize{6}{0}
\selectfont\put(232.715,22.2671){\makebox(0,0)[t]{\textcolor[rgb]{0.15,0.15,0.15}{{0}}}}
\fontsize{6}{0}
\selectfont\put(258.572,22.2671){\makebox(0,0)[t]{\textcolor[rgb]{0.15,0.15,0.15}{{10}}}}
\fontsize{6}{0}
\selectfont\put(284.429,22.2671){\makebox(0,0)[t]{\textcolor[rgb]{0.15,0.15,0.15}{{20}}}}
\fontsize{6}{0}
\selectfont\put(310.286,22.2671){\makebox(0,0)[t]{\textcolor[rgb]{0.15,0.15,0.15}{{30}}}}
\fontsize{6}{0}
\selectfont\put(336.143,22.2671){\makebox(0,0)[t]{\textcolor[rgb]{0.15,0.15,0.15}{{40}}}}
\fontsize{6}{0}
\selectfont\put(362,22.2671){\makebox(0,0)[t]{\textcolor[rgb]{0.15,0.15,0.15}{{50}}}}
\fontsize{6}{0}
\selectfont\put(229.235,27.5){\makebox(0,0)[r]{\textcolor[rgb]{0.15,0.15,0.15}{{-0.4}}}}
\fontsize{6}{0}
\selectfont\put(229.235,40.729){\makebox(0,0)[r]{\textcolor[rgb]{0.15,0.15,0.15}{{-0.3}}}}
\fontsize{6}{0}
\selectfont\put(229.235,53.9585){\makebox(0,0)[r]{\textcolor[rgb]{0.15,0.15,0.15}{{-0.2}}}}
\fontsize{6}{0}
\selectfont\put(229.235,67.1875){\makebox(0,0)[r]{\textcolor[rgb]{0.15,0.15,0.15}{{-0.1}}}}
\fontsize{6}{0}
\selectfont\put(229.235,80.4165){\makebox(0,0)[r]{\textcolor[rgb]{0.15,0.15,0.15}{{0}}}}
\fontsize{6}{0}
\selectfont\put(229.235,93.646){\makebox(0,0)[r]{\textcolor[rgb]{0.15,0.15,0.15}{{0.1}}}}
\fontsize{6}{0}
\selectfont\put(229.235,106.875){\makebox(0,0)[r]{\textcolor[rgb]{0.15,0.15,0.15}{{0.2}}}}
\fontsize{7}{0}
\selectfont\put(212.235,67.1875){\rotatebox{90}{\makebox(0,0)[b]{\textcolor[rgb]{0.15,0.15,0.15}{{Perturbação na Entrada $q_{u}(t)$}}}}}
\fontsize{7}{0}
\selectfont\put(297.357,11.2671){\makebox(0,0)[t]{\textcolor[rgb]{0.15,0.15,0.15}{{Tempo (s)}}}}
\end{picture}

    \end{minipage}
\end{figure}

Conclui-se da Figura \ref{fig:resultado-cenario2-a} que igualmente ao
Cenário 1, o sistema em malha fechada seguiu referência e rejeitou a perturbação
de saída. Isso é evidente já que avaliando conceitualmente a partir do Princípio
do Modelo Interno, observa-se que o polo de $R(s)$ e de $Q_{y}(s)$ aparecem
tanto em $C(s)$ quanto $G(s)$. Entretanto, diferente do primeiro cenário, o
sistema também foi capaz de rejeitar a perturbação na entrada. Esta rejeição
aconteceu pois o polo de $Q_{u}(s)$ aparece no caminho de realimentação da saída
para a perturbação de entrada, sendo nesse caso o cancelamento dos polos
realizados pelo polo de $C(s)$. Cabe ainda uma observação para o sistema em
malha fechada em questão. Nota-se uma suavidade no sinal de controle $u(t)$. Em
problemas de controle de motores elétricos com cargas de alta inércia acopladas
ao seu eixo, a suavidade resulta em menor esforço da máquina, tendo como
consequência redução de manutenções corretivas.

Afim de avaliar também a saída do sistema em malha fechada em regime
estacionário, foi considerada perturbações do tipo rampa conforme as equações
a seguir:

\begin{equation}
    \label{eq:qys-rampa-cenario2}
    Q_{y}(s) = -0,2\frac{e^{-15s}}{s^2},
\end{equation}

\begin{equation}
    \label{eq:qus-rampa-cenario2}
    Q_{u}(s) = -0,2\frac{e^{-25s}}{s^{2}}
\end{equation}

Observa-se nas Equações \ref{eq:qys-rampa-cenario2} e
\ref{eq:qus-rampa-cenario2} a presença de dois polos na origem. Utilizando o
Princípio do Modelo Interno pode se prever que a perturbação na saída $q_{y}(t)$
será rejeitada já que a junção dos polos do controlador e da planta aparecem no
caminho de realimentação da função de transferência $\frac{Y(s)}{Q_{y}(s)}$.
Entretanto, o mesmo não acontece para a perturbação na entrada, pois apenas o
polo de $C(s)$ na origem aparece no caminho de realimentação da função de
transferência $\frac{Y(s)}{Q_{u}(s)}$. Assim, espera-se um valor diferente da
referência $r(t)$ em regime estacionário para $y(t)$ dado uma perturbação na
entrada do tipo rampa.

A Figura \ref{fig:resultado-cenario2-b} apresenta os resultados para as
perturbações em rampa. Conforme esperado, o sistema em malha fechada não
rejeitou a perturbação na entrada. A saída $y(\infty) = 0,6$, sendo que o correto
era que $y(\infty) = r(\infty) = 1$.

\begin{figure}[!ht]
    \caption{Simulação do sistema de controle em malhada fechada para condições
    impostas no Cenário 2 para perturbações do tipo rampa.}
    \vspace{-10pt}
    \hspace{-30pt}
    \label{fig:resultado-cenario2-b}
    \begin{minipage}{\linewidth}
        % Title: gl2ps_renderer figure
% Creator: GL2PS 1.4.0, (C) 1999-2017 C. Geuzaine
% For: Octave
% CreationDate: Sun Sep 12 18:35:45 2021
\setlength{\unitlength}{1pt}
\begin{picture}(0,0)
\includegraphics{images/challenge1/resultado-cenario2-b-inc}
\end{picture}%
\begin{picture}(400,250)(0,0)
\fontsize{6}{0}
\selectfont\put(52,141.404){\makebox(0,0)[t]{\textcolor[rgb]{0.15,0.15,0.15}{{0}}}}
\fontsize{6}{0}
\selectfont\put(77.8569,141.404){\makebox(0,0)[t]{\textcolor[rgb]{0.15,0.15,0.15}{{10}}}}
\fontsize{6}{0}
\selectfont\put(103.714,141.404){\makebox(0,0)[t]{\textcolor[rgb]{0.15,0.15,0.15}{{20}}}}
\fontsize{6}{0}
\selectfont\put(129.571,141.404){\makebox(0,0)[t]{\textcolor[rgb]{0.15,0.15,0.15}{{30}}}}
\fontsize{6}{0}
\selectfont\put(155.428,141.404){\makebox(0,0)[t]{\textcolor[rgb]{0.15,0.15,0.15}{{40}}}}
\fontsize{6}{0}
\selectfont\put(181.285,141.404){\makebox(0,0)[t]{\textcolor[rgb]{0.15,0.15,0.15}{{50}}}}
\fontsize{6}{0}
\selectfont\put(48.5205,146.637){\makebox(0,0)[r]{\textcolor[rgb]{0.15,0.15,0.15}{{-0.2}}}}
\fontsize{6}{0}
\selectfont\put(48.5205,157.977){\makebox(0,0)[r]{\textcolor[rgb]{0.15,0.15,0.15}{{0}}}}
\fontsize{6}{0}
\selectfont\put(48.5205,169.316){\makebox(0,0)[r]{\textcolor[rgb]{0.15,0.15,0.15}{{0.2}}}}
\fontsize{6}{0}
\selectfont\put(48.5205,180.656){\makebox(0,0)[r]{\textcolor[rgb]{0.15,0.15,0.15}{{0.4}}}}
\fontsize{6}{0}
\selectfont\put(48.5205,191.995){\makebox(0,0)[r]{\textcolor[rgb]{0.15,0.15,0.15}{{0.6}}}}
\fontsize{6}{0}
\selectfont\put(48.5205,203.335){\makebox(0,0)[r]{\textcolor[rgb]{0.15,0.15,0.15}{{0.8}}}}
\fontsize{6}{0}
\selectfont\put(48.5205,214.674){\makebox(0,0)[r]{\textcolor[rgb]{0.15,0.15,0.15}{{1}}}}
\fontsize{7}{0}
\selectfont\put(31.5205,186.325){\rotatebox{90}{\makebox(0,0)[b]{\textcolor[rgb]{0.15,0.15,0.15}{{Saída $y(t)$}}}}}
\fontsize{7}{0}
\selectfont\put(116.643,130.404){\makebox(0,0)[t]{\textcolor[rgb]{0.15,0.15,0.15}{{Tempo (s)}}}}
\fontsize{6}{0}
\selectfont\put(160.285,174.138){\makebox(0,0)[l]{\textcolor[rgb]{0,0,0}{{r(t)}}}}
\fontsize{6}{0}
\selectfont\put(160.285,162.638){\makebox(0,0)[l]{\textcolor[rgb]{0,0,0}{{y(t)}}}}
\fontsize{6}{0}
\selectfont\put(52,22.2671){\makebox(0,0)[t]{\textcolor[rgb]{0.15,0.15,0.15}{{0}}}}
\fontsize{6}{0}
\selectfont\put(77.8569,22.2671){\makebox(0,0)[t]{\textcolor[rgb]{0.15,0.15,0.15}{{10}}}}
\fontsize{6}{0}
\selectfont\put(103.714,22.2671){\makebox(0,0)[t]{\textcolor[rgb]{0.15,0.15,0.15}{{20}}}}
\fontsize{6}{0}
\selectfont\put(129.571,22.2671){\makebox(0,0)[t]{\textcolor[rgb]{0.15,0.15,0.15}{{30}}}}
\fontsize{6}{0}
\selectfont\put(155.428,22.2671){\makebox(0,0)[t]{\textcolor[rgb]{0.15,0.15,0.15}{{40}}}}
\fontsize{6}{0}
\selectfont\put(181.285,22.2671){\makebox(0,0)[t]{\textcolor[rgb]{0.15,0.15,0.15}{{50}}}}
\fontsize{6}{0}
\selectfont\put(48.5205,30.3862){\makebox(0,0)[r]{\textcolor[rgb]{0.15,0.15,0.15}{{0}}}}
\fontsize{6}{0}
\selectfont\put(48.5205,44.8184){\makebox(0,0)[r]{\textcolor[rgb]{0.15,0.15,0.15}{{1}}}}
\fontsize{6}{0}
\selectfont\put(48.5205,59.25){\makebox(0,0)[r]{\textcolor[rgb]{0.15,0.15,0.15}{{2}}}}
\fontsize{6}{0}
\selectfont\put(48.5205,73.6816){\makebox(0,0)[r]{\textcolor[rgb]{0.15,0.15,0.15}{{3}}}}
\fontsize{6}{0}
\selectfont\put(48.5205,88.1138){\makebox(0,0)[r]{\textcolor[rgb]{0.15,0.15,0.15}{{4}}}}
\fontsize{6}{0}
\selectfont\put(48.5205,102.545){\makebox(0,0)[r]{\textcolor[rgb]{0.15,0.15,0.15}{{5}}}}
\fontsize{7}{0}
\selectfont\put(39.5205,67.1875){\rotatebox{90}{\makebox(0,0)[b]{\textcolor[rgb]{0.15,0.15,0.15}{{Sinal de Controle $u(t)$}}}}}
\fontsize{7}{0}
\selectfont\put(116.643,11.2671){\makebox(0,0)[t]{\textcolor[rgb]{0.15,0.15,0.15}{{Tempo (s)}}}}
\fontsize{6}{0}
\selectfont\put(232.715,141.404){\makebox(0,0)[t]{\textcolor[rgb]{0.15,0.15,0.15}{{0}}}}
\fontsize{6}{0}
\selectfont\put(258.572,141.404){\makebox(0,0)[t]{\textcolor[rgb]{0.15,0.15,0.15}{{10}}}}
\fontsize{6}{0}
\selectfont\put(284.429,141.404){\makebox(0,0)[t]{\textcolor[rgb]{0.15,0.15,0.15}{{20}}}}
\fontsize{6}{0}
\selectfont\put(310.286,141.404){\makebox(0,0)[t]{\textcolor[rgb]{0.15,0.15,0.15}{{30}}}}
\fontsize{6}{0}
\selectfont\put(336.143,141.404){\makebox(0,0)[t]{\textcolor[rgb]{0.15,0.15,0.15}{{40}}}}
\fontsize{6}{0}
\selectfont\put(362,141.404){\makebox(0,0)[t]{\textcolor[rgb]{0.15,0.15,0.15}{{50}}}}
\fontsize{6}{0}
\selectfont\put(229.235,159.509){\makebox(0,0)[r]{\textcolor[rgb]{0.15,0.15,0.15}{{-6}}}}
\fontsize{6}{0}
\selectfont\put(229.235,180.961){\makebox(0,0)[r]{\textcolor[rgb]{0.15,0.15,0.15}{{-4}}}}
\fontsize{6}{0}
\selectfont\put(229.235,202.414){\makebox(0,0)[r]{\textcolor[rgb]{0.15,0.15,0.15}{{-2}}}}
\fontsize{6}{0}
\selectfont\put(229.235,223.867){\makebox(0,0)[r]{\textcolor[rgb]{0.15,0.15,0.15}{{0}}}}
\fontsize{7}{0}
\selectfont\put(218.235,186.325){\rotatebox{90}{\makebox(0,0)[b]{\textcolor[rgb]{0.15,0.15,0.15}{{Perturbação na Saída $q_{y}(t)$}}}}}
\fontsize{7}{0}
\selectfont\put(297.357,130.404){\makebox(0,0)[t]{\textcolor[rgb]{0.15,0.15,0.15}{{Tempo (s)}}}}
\fontsize{6}{0}
\selectfont\put(232.715,22.2671){\makebox(0,0)[t]{\textcolor[rgb]{0.15,0.15,0.15}{{0}}}}
\fontsize{6}{0}
\selectfont\put(258.572,22.2671){\makebox(0,0)[t]{\textcolor[rgb]{0.15,0.15,0.15}{{10}}}}
\fontsize{6}{0}
\selectfont\put(284.429,22.2671){\makebox(0,0)[t]{\textcolor[rgb]{0.15,0.15,0.15}{{20}}}}
\fontsize{6}{0}
\selectfont\put(310.286,22.2671){\makebox(0,0)[t]{\textcolor[rgb]{0.15,0.15,0.15}{{30}}}}
\fontsize{6}{0}
\selectfont\put(336.143,22.2671){\makebox(0,0)[t]{\textcolor[rgb]{0.15,0.15,0.15}{{40}}}}
\fontsize{6}{0}
\selectfont\put(362,22.2671){\makebox(0,0)[t]{\textcolor[rgb]{0.15,0.15,0.15}{{50}}}}
\fontsize{6}{0}
\selectfont\put(229.235,30.4399){\makebox(0,0)[r]{\textcolor[rgb]{0.15,0.15,0.15}{{-5}}}}
\fontsize{6}{0}
\selectfont\put(229.235,45.1387){\makebox(0,0)[r]{\textcolor[rgb]{0.15,0.15,0.15}{{-4}}}}
\fontsize{6}{0}
\selectfont\put(229.235,59.8379){\makebox(0,0)[r]{\textcolor[rgb]{0.15,0.15,0.15}{{-3}}}}
\fontsize{6}{0}
\selectfont\put(229.235,74.5371){\makebox(0,0)[r]{\textcolor[rgb]{0.15,0.15,0.15}{{-2}}}}
\fontsize{6}{0}
\selectfont\put(229.235,89.2363){\makebox(0,0)[r]{\textcolor[rgb]{0.15,0.15,0.15}{{-1}}}}
\fontsize{6}{0}
\selectfont\put(229.235,103.935){\makebox(0,0)[r]{\textcolor[rgb]{0.15,0.15,0.15}{{0}}}}
\fontsize{7}{0}
\selectfont\put(218.235,67.1875){\rotatebox{90}{\makebox(0,0)[b]{\textcolor[rgb]{0.15,0.15,0.15}{{Perturbação na Entrada $q_{u}(t)$}}}}}
\fontsize{7}{0}
\selectfont\put(297.357,11.2671){\makebox(0,0)[t]{\textcolor[rgb]{0.15,0.15,0.15}{{Tempo (s)}}}}
\end{picture}

    \end{minipage}
\end{figure}